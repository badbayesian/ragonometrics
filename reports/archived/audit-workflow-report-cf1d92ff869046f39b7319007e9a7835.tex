% Options for packages loaded elsewhere
\PassOptionsToPackage{unicode}{hyperref}
\PassOptionsToPackage{hyphens}{url}
\documentclass[
]{article}
\usepackage{xcolor}
\usepackage{amsmath,amssymb}
\setcounter{secnumdepth}{-\maxdimen} % remove section numbering
\usepackage{iftex}
\ifPDFTeX
  \usepackage[T1]{fontenc}
  \usepackage[utf8]{inputenc}
  \usepackage{textcomp} % provide euro and other symbols
\else % if luatex or xetex
  \usepackage{unicode-math} % this also loads fontspec
  \defaultfontfeatures{Scale=MatchLowercase}
  \defaultfontfeatures[\rmfamily]{Ligatures=TeX,Scale=1}
\fi
\usepackage{lmodern}
\ifPDFTeX\else
  % xetex/luatex font selection
\fi
% Use upquote if available, for straight quotes in verbatim environments
\IfFileExists{upquote.sty}{\usepackage{upquote}}{}
\IfFileExists{microtype.sty}{% use microtype if available
  \usepackage[]{microtype}
  \UseMicrotypeSet[protrusion]{basicmath} % disable protrusion for tt fonts
}{}
\makeatletter
\@ifundefined{KOMAClassName}{% if non-KOMA class
  \IfFileExists{parskip.sty}{%
    \usepackage{parskip}
  }{% else
    \setlength{\parindent}{0pt}
    \setlength{\parskip}{6pt plus 2pt minus 1pt}}
}{% if KOMA class
  \KOMAoptions{parskip=half}}
\makeatother
\setlength{\emergencystretch}{3em} % prevent overfull lines
\providecommand{\tightlist}{%
  \setlength{\itemsep}{0pt}\setlength{\parskip}{0pt}}
\usepackage{bookmark}
\IfFileExists{xurl.sty}{\usepackage{xurl}}{} % add URL line breaks if available
\urlstyle{same}
\hypersetup{
  hidelinks,
  pdfcreator={LaTeX via pandoc}}

\author{}
\date{}

\begin{document}

\section{\texorpdfstring{Audit Report: Workflow \texttt{cf1d92ff869046f39b7319007e9a7835}}{Audit Report: Workflow cf1d92ff869046f39b7319007e9a7835}}\label{audit-report-workflow-cf1d92ff869046f39b7319007e9a7835}

\subsection{Overview}\label{overview}

\begin{itemize}
\tightlist
\item
  Source JSON: \texttt{reports\textbackslash{}workflow-report-cf1d92ff869046f39b7319007e9a7835.json}
\item
  Run ID: \texttt{cf1d92ff869046f39b7319007e9a7835}
\item
  Papers input: \texttt{papers\textbackslash{}NetworksElectoralCompetition.pdf}
\item
  Started at: \texttt{2026-02-15T19:29:20.818045+00:00}
\item
  Finished at: \texttt{2026-02-15T19:40:51.819956+00:00}
\item
  Duration: \texttt{0:11:31.001911}
\end{itemize}

\subsection{Effective Configuration}\label{effective-configuration}

\begin{itemize}
\tightlist
\item
  Chat model: \texttt{gpt-5}
\item
  Embedding model: \texttt{text-embedding-3-large}
\item
  Top K: \texttt{10}
\item
  Chunk words / overlap: \texttt{350} / \texttt{75}
\item
  Batch size: \texttt{64}
\item
  Database URL configured: \texttt{True}
\end{itemize}

\subsection{Step Outcomes}\label{step-outcomes}

\begin{itemize}
\tightlist
\item
  \texttt{prep}: \texttt{completed}
\item
  \texttt{ingest}: \texttt{num\_pdfs=1,\ num\_papers=1}
\item
  \texttt{enrich}: \texttt{openalex=0,\ citec=0}
\item
  \texttt{econ\_data}: \texttt{fetched}
\item
  \texttt{agentic}: \texttt{completed}
\item
  \texttt{index}: \texttt{skipped\ (reason:\ }db\_unreachable\texttt{)}
\item
  \texttt{report\_store}: \texttt{skipped\ (reason:\ }db\_unreachable\texttt{)}
\end{itemize}

\subsection{Agentic Summary}\label{agentic-summary}

\begin{itemize}
\tightlist
\item
  Status: \texttt{completed}
\item
  Main question: What is the key contribution?
\item
  Report question set: \texttt{both}
\item
  Structured questions generated: \texttt{84}
\item
  Confidence mean/median: \texttt{0.2921069186783523} / \texttt{0.30376113152112244}
\item
  Confidence labels: low=77, medium=6, high=0
\end{itemize}

\subsubsection{Final Answer}\label{final-answer}

\begin{itemize}
\item
  Core contribution: Formalizes and empirically tests how district homophily governs the diffusion of information about elected officials---modeled as a network statistic that determines the equilibrium share of informed voters---and links this mechanism to electoral behavior. Empirically, it identifies homophily's effects using a within-state border-pairs design and operationalizes participation via a precise roll-off measure. (page 28 words 0-349; page 18 words 275-496)
\item
  Theory to measurement: Develops a model of news diffusion within districts, summarizing district homophily as a network statistic shaping the informed share; operationalizes abstention conditional on showing up via roll-off \(=(\text{VotesTop}-\text{VotesHouse})/\text{VotesTop}\). (page 28 words 0-349; page 18 words 275-496)
\item
  Identification: Implements a neighboring-county border-pairs design with county and pair-by-year fixed effects and time-varying controls to isolate district-assignment-induced differences in homophily; documents robustness and the sensitivity of ``Selected Party'' to adding DMA-by-year FE. (page 28 words 0-349)
\item
  Data infrastructure: Assembles multi-source county and county-by-CD vote counts (precinct-level EDA/EDS merged with Dave Leip's and Daily Kos), CES-based knowledge/preference/turnout measures, and geocoded FEC contributions identifying in-district contributors---enabling comprehensive tests of the theory. (page 47 words 0-207; page 48 words 0-333; page 18 words 275-496)
\item
  Empirical grounding: Provides descriptive baselines for roll-off and turnout (mean roll-off 4.42\%; top-of-ticket turnout 51.34\%; House turnout 49.07\%) and for voter knowledge/behavior (e.g., Heard of Rep 93.2\%; Validated General Turnout 57.5\%), anchoring the analysis. (page 49 words 0-276)
\end{itemize}

\subsubsection{Sub-Answers}\label{sub-answers}

\paragraph{Sub-answer 1}\label{sub-answer-1}

\begin{itemize}
\tightlist
\item
  Question: ResponseTextConfig(format=ResponseFormatText(type=\textquotesingle text\textquotesingle), verbosity=\textquotesingle medium\textquotesingle)
\item
  Question tokens estimate: \texttt{19}
\item
  Answer:
\end{itemize}

No specific question was posed. Below is a concise synthesis of key definitions, data, and identification from the provided context, with citations.

\begin{itemize}
\item
  Roll-off (House turnout relative to top-of-ticket)

  \begin{itemize}
  \tightlist
  \item
    Definition: Share of top-of-ticket voters who abstain from the House race: (VotesTop − VotesHouse) / VotesTop; captures abstention after incurring the cost of voting; related to ``roll-off'' literature (Feddersen \& Pesendorfer 1996; Miller 2022; Snyder \& Strömberg 2010) (page 18 words 275-496).
  \item
    Variable construction and scope: Main analysis at county level from Dave Leip's; robustness at county-by-CD level from Harvard EDA, Daily Kos, Dave Leip's, and MIT EDS Lab; exclude elections without a top-of-ticket race (page 49 words 0-276).
  \item
    Summary stats: Mean 4.42\%, SD 12.22 pp; also report turnout in top-of-ticket (mean 51.34\%) and House (mean 49.07\%) (page 49 words 0-276).
  \end{itemize}
\item
  Campaign contributions data

  \begin{itemize}
  \tightlist
  \item
    Source: Kuziemko et al. 2023 (October), constructed from FEC data in Bonica 2014; geocoding used to identify whether contributor resides in the same congressional district as the candidate (page 18 words 275-496).
  \end{itemize}
\item
  Border pairs identification design

  \begin{itemize}
  \tightlist
  \item
    Setup: Neighboring county pairs that straddle a district boundary; counties are similar except district assignment, yielding different district homophily; restrict to counties fully within one district; collapse outcomes to county level; include one observation per county--pair; compare within-state pairs; precision decreases due to restricted sample (page 28 words 0-349).
  \item
    Specification:

    \begin{itemize}
    \tightlist
    \item
      \(y_{ct} = \alpha_c + \mu_{pt} + \beta\,\bar{\pi}_{c,t} + X'_{ct}\delta + \varepsilon_{ct}\), where \(y_{ct}\) is the outcome, \(\mu_{pt}\) pair-by-year FE, \(X_{ct}\) time-varying county controls; use state-by-year FE (insufficient data for district-by-year FE) (page 28 words 0-349).
    \end{itemize}
  \item
    Results note: Qualitatively similar to redistricting design; estimates on ``Selected Party'' become insignificant after adding DMA-by-year FE (page 28 words 0-349).
  \end{itemize}
\item
  Conceptual framework

  \begin{itemize}
  \tightlist
  \item
    Develops a theoretical model of information diffusion within districts; district homophily summarized as a network statistic; considers news about elected officials diffusing through networks to determine the equilibrium share of informed voters (page 28 words 0-349).
  \end{itemize}
\item
  Vote count data construction

  \begin{itemize}
  \tightlist
  \item
    County-by-CD measures built from precinct-level vote counts (Harvard EDA 2000--2010; MIT EDS Lab 2016--2020) combined with county-by-CD vote counts from Dave Leip's (House) and Daily Kos (President, Senator, Governor) (page 47 words 0-207).
  \end{itemize}
\item
  CES variables and summary statistics (knowledge, preferences, voting)

  \begin{itemize}
  \tightlist
  \item
    Knowledge outcomes (binary): Heard of Representative; Selected Party; Selected Correct Party; defined via pre-survey items on recognizing and assigning party to the House representative (page 47 words 0-207).
  \item
    Voting/preference outcomes (binary): Prefer Incumbent/Opponent/Neither (pre-survey); Voted for Incumbent/Opponent/Neither (post-survey); Validated turnout in general/primary; self-reported general turnout; handling of missing when no incumbent or unmatched rolls (page 48 words 0-333).
  \item
    Summary (sample means): Heard of Rep 93.2\%; Selected Party 68.6\%; Selected Correct Party 61.7\%; Prefer Incumbent 40.1\%; Voted for Incumbent 41.0\%; Validated General Turnout 57.5\%; Self-Reported General Turnout 87.8\% (page 49 words 0-276).
  \end{itemize}
\item
  Additional pointers

  \begin{itemize}
  \tightlist
  \item
    Figures showing effects of homophily on self-reported voting using commuting- and zip-code-based measures (Figures C11, C14); a figure on changes in district homophily with progressively finer bins (Figure C5) (page 61 words 0-15; page 64 words 0-45; page 54 words 0-18).
  \end{itemize}
\end{itemize}

\subsection{Structured Q\&A Appendix}\label{structured-qa-appendix}

This section mirrors \texttt{agentic.report\_questions} for audit traceability.

\subsubsection{A01: What is the main research question of the paper?}\label{a01-what-is-the-main-research-question-of-the-paper}

\begin{itemize}
\tightlist
\item
  Category: \texttt{A)\ Research\ question\ /\ contribution}
\item
  Confidence: \texttt{low\ (0.31323401739665313)}
\item
  Retrieval method: \texttt{local}
\item
  Evidence type: \texttt{direct\ statement\ (introduction)}
\item
  Data source: Introduction
\item
  Assumption flag: \texttt{False}
\item
  Answer:
\end{itemize}

How the alignment between social networks and political boundaries (district homophily) affects voters' political knowledge and behavior, including turnout.

\begin{itemize}
\tightlist
\item
  Quote snippet:
\end{itemize}

\begin{quote}
How does the alignment between social networks and political boundaries impact voters' political knowledge and behavior?
\end{quote}

\begin{itemize}
\tightlist
\item
  Citation anchors:

  \begin{itemize}
  \tightlist
  \item
    page=2, words=0-349, section=Introduction, note=States the research question explicitly.
  \item
    page=5, words=0-349, section=Contribution, note=Reiterates focus on how the match between networks and boundaries affects knowledge and behavior.
  \end{itemize}
\end{itemize}

\subsubsection{A02: What is the paper\textquotesingle s primary contribution relative to the existing literature?}\label{a02-what-is-the-papers-primary-contribution-relative-to-the-existing-literature}

\begin{itemize}
\tightlist
\item
  Category: \texttt{A)\ Research\ question\ /\ contribution}
\item
  Confidence: \texttt{low\ (0.3201369538893216)}
\item
  Retrieval method: \texttt{local}
\item
  Evidence type: \texttt{direct\ statement\ in\ text}
\item
  Assumption flag: \texttt{False}
\item
  Answer:
\end{itemize}

The paper's primary contribution is to provide causal, near-national estimates of how social networks (district homophily) affect voter knowledge---shifting the focus beyond traditional media by leveraging geographic variation in U.S. social ties (and, secondarily, linking social learning to political geography/redistricting).

\begin{itemize}
\tightlist
\item
  Quote snippet:
\end{itemize}

\begin{quote}
This paper makes two primary contributions: First, I contribute to the literature on how voters learn about politics by providing causal estimates of the extent to which social networks impact voter knowledge
\end{quote}

\begin{itemize}
\tightlist
\item
  Citation anchors:

  \begin{itemize}
  \tightlist
  \item
    page=3, words=0-349, section=Introduction, note=States two primary contributions; first is causal, near-national estimates of networks' impact on voter knowledge.
  \item
    page=4, words=275-492, section=Introduction, note=Clarifies leveraging geographic variation in the U.S. social network; relevance for redistricting.
  \item
    page=5, words=275-475, section=Introduction, note=Connects social learning to political geography and implications for gerrymandering measures.
  \end{itemize}
\end{itemize}

\subsubsection{A03: What is the central hypothesis being tested?}\label{a03-what-is-the-central-hypothesis-being-tested}

\begin{itemize}
\tightlist
\item
  Category: \texttt{A)\ Research\ question\ /\ contribution}
\item
  Confidence: \texttt{low\ (0.28653574361407586)}
\item
  Retrieval method: \texttt{local}
\item
  Evidence type: \texttt{textual\ evidence}
\item
  Data source: Paper text in provided context
\item
  Assumption flag: \texttt{False}
\item
  Answer:
\end{itemize}

Greater alignment between social ties and congressional district boundaries (district homophily) facilitates the diffusion of representative-specific political information, making voters more informed and influencing voting behavior.

\begin{itemize}
\tightlist
\item
  Quote snippet:
\end{itemize}

\begin{quote}
Assuming that information about representatives spreads more readily between friends in the same district, the way district boundaries are drawn can shape how political information flows through social networks.
\end{quote}

\begin{itemize}
\tightlist
\item
  Citation anchors:

  \begin{itemize}
  \tightlist
  \item
    page=6, words=0-349, section=main, note=Defines district homophily and posits that information spreads more readily between friends in the same district, implying boundaries shape political information flow.
  \item
    page=28, words=275-405, section=main, note=States empirical link and provides a diffusion framework showing how district homophily determines informed share.
  \item
    page=20, words=0-349, section=main, note=Reports event-study evidence that changes in district homophily change voter knowledge, consistent with the hypothesis.
  \end{itemize}
\end{itemize}

\subsubsection{A04: What are the main outcomes of interest (dependent variables)?}\label{a04-what-are-the-main-outcomes-of-interest-dependent-variables}

\begin{itemize}
\tightlist
\item
  Category: \texttt{A)\ Research\ question\ /\ contribution}
\item
  Confidence: \texttt{low\ (0.34403041836800285)}
\item
  Retrieval method: \texttt{local}
\item
  Evidence type: \texttt{text}
\item
  Data source: CES; Dave Leip's Election Atlas; Harvard Election Data Archive; MIT Election Data and Science Lab; Daily Kos; Census (VAP)
\item
  Table/Figure: Appendix Tables B1, B2, B4--B5
\item
  Assumption flag: \texttt{False}
\item
  Answer:
\end{itemize}

Dependent variables include: (A) Voter information measures from CES: Heard of Representative, Selected Party, Selected Correct Party. (B) Candidate preference (pre-survey): Prefer Incumbent, Prefer Opponent, Prefer Neither. (C) Vote choice (post-survey): Voted for Incumbent, Voted for Opponent, Voted for Neither. (D) Turnout measures from CES: Voted in General Election (Validated), Voted in Primary Election (Validated), Voted in General Election (Self-Report). (E) Vote-count outcomes: House roll-off (share of top-of-ticket voters abstaining in House race), Turnout in Top-of-Ticket Election (as share of VAP), and Turnout in House Election (as share of VAP).

\begin{itemize}
\tightlist
\item
  Quote snippet:
\end{itemize}

\begin{quote}
I study the impact of district homophily on voters' knowledge and political behavior... I then incorporate vote count data to reveal actual voting behavior
\end{quote}

\begin{itemize}
\tightlist
\item
  Citation anchors:

  \begin{itemize}
  \tightlist
  \item
    page=16, words=0-349, note=States study examines voter knowledge, self-reported choices, and actual voting via vote counts.
  \item
    page=47, words=0-207, section=appendix, note=Appendix Table B1 defines voter knowledge outcomes.
  \item
    page=48, words=0-333, section=appendix, note=Appendix Table B2 defines CES voting outcomes (preferences, vote choices, turnout variables).
  \item
    page=49, words=0-276, section=appendix, note=Table B4/B5 define vote-count outcomes: roll-off and turnout measures.
  \item
    page=22, words=0-349, note=Text describes roll-off as share of top-of-ticket voters who abstain in House election.
  \end{itemize}
\end{itemize}

\subsubsection{A05: What are the key treatment/exposure variables (independent variables)?}\label{a05-what-are-the-key-treatmentexposure-variables-independent-variables}

\begin{itemize}
\tightlist
\item
  Category: \texttt{A)\ Research\ question\ /\ contribution}
\item
  Confidence: \texttt{low\ (0.24776601700919515)}
\item
  Retrieval method: \texttt{local}
\item
  Evidence type: \texttt{textual}
\item
  Table/Figure: Table C2
\item
  Assumption flag: \texttt{False}
\item
  Answer:
\end{itemize}

District homophily---both its level (used directly in regressions) and, for the redistricting event study, the change in district homophily a county experiences between 2012 and 2013 (Δπ̄c).

\begin{itemize}
\tightlist
\item
  Quote snippet:
\end{itemize}

\begin{quote}
Δπ̄c is the change in district homophily experienced by county c between 2012 and 2013
\end{quote}

\begin{itemize}
\tightlist
\item
  Citation anchors:

  \begin{itemize}
  \tightlist
  \item
    page=15, words=0-349, section=main, note=Defines Δπ̄c as the change in district homophily between 2012 and 2013 used in event studies.
  \item
    page=65, words=0-288, section=appendix, note=Table C2 shows \textquotesingle District Homophily\textquotesingle{} as the independent variable in regressions.
  \end{itemize}
\end{itemize}

\subsubsection{A06: What setting/context does the paper study (country, market, period)?}\label{a06-what-settingcontext-does-the-paper-study-country-market-period}

\begin{itemize}
\tightlist
\item
  Category: \texttt{A)\ Research\ question\ /\ contribution}
\item
  Confidence: \texttt{low\ (0.31368693464548203)}
\item
  Retrieval method: \texttt{local}
\item
  Evidence type: \texttt{textual}
\item
  Data source: national data on social ties
\item
  Assumption flag: \texttt{False}
\item
  Answer:
\end{itemize}

United States (continental U.S.); U.S. congressional districts/elections and voter information diffusion via social networks; centered on the 2010 Census redistricting with treatment in 2012--2013 and an event-study window spanning roughly 2006--2022.

\begin{itemize}
\tightlist
\item
  Quote snippet:
\end{itemize}

\begin{quote}
The Census was conducted in April 2010, and states needed to draw new congressional district borders in time for the November 2012 elections.
\end{quote}

\begin{itemize}
\tightlist
\item
  Citation anchors:

  \begin{itemize}
  \tightlist
  \item
    page=4, words=275-492, note=Scope across the continental U.S. using national data on social ties
  \item
    page=15, words=0-349, note=Focus on post-2010 Census redistricting; new borders for Nov 2012 elections; treatment timing and event-study years (incl. 2006--2022)
  \end{itemize}
\end{itemize}

\subsubsection{A07: What is the main mechanism proposed by the authors?}\label{a07-what-is-the-main-mechanism-proposed-by-the-authors}

\begin{itemize}
\tightlist
\item
  Category: \texttt{A)\ Research\ question\ /\ contribution}
\item
  Confidence: \texttt{low\ (0.26822049639980433)}
\item
  Retrieval method: \texttt{local}
\item
  Evidence type: \texttt{textual}
\item
  Assumption flag: \texttt{False}
\item
  Answer:
\end{itemize}

Information diffuses through social networks and travels faster among same-district friends (district homophily); thus, alignment between social ties and district boundaries drives voter knowledge about representatives.

\begin{itemize}
\tightlist
\item
  Quote snippet:
\end{itemize}

\begin{quote}
Accordingly, information about representatives is likely to spread more quickly when people are more likely to interact with others from the same district.
\end{quote}

\begin{itemize}
\tightlist
\item
  Citation anchors:

  \begin{itemize}
  \tightlist
  \item
    page=6, words=0-349, note=Defines mechanism: voters learn via networks; same-district ties accelerate spread; introduces district homophily.
  \item
    page=28, words=275-405, note=Conceptual framework of diffusion with homophily; district homophily as summary statistic.
  \item
    page=34, words=0-349, note=Clarifies district homophily arises from intersection of networks and district boundaries.
  \item
    page=4, words=0-349, note=Highlights social networks as key source of political information.
  \end{itemize}
\end{itemize}

\subsubsection{A08: What alternative mechanisms are discussed?}\label{a08-what-alternative-mechanisms-are-discussed}

\begin{itemize}
\tightlist
\item
  Category: \texttt{A)\ Research\ question\ /\ contribution}
\item
  Confidence: \texttt{low\ (0.2426677094907878)}
\item
  Retrieval method: \texttt{local}
\item
  Evidence type: \texttt{textual\ evidence}
\item
  Assumption flag: \texttt{False}
\item
  Answer:
\end{itemize}

Social pressure and recruitment.

\begin{itemize}
\tightlist
\item
  Quote snippet:
\end{itemize}

\begin{quote}
which also explores other mechanisms such as social pressure (Gerber et al. 2008, Sinclair et al. 2012) and recruitment (Klofstad 2007).
\end{quote}

\begin{itemize}
\tightlist
\item
  Citation anchors:

  \begin{itemize}
  \tightlist
  \item
    page=4, words=150-210, note=Peer-effects literature mentions other mechanisms.
  \end{itemize}
\end{itemize}

\subsubsection{A09: What are the main policy implications claimed by the paper?}\label{a09-what-are-the-main-policy-implications-claimed-by-the-paper}

\begin{itemize}
\tightlist
\item
  Category: \texttt{A)\ Research\ question\ /\ contribution}
\item
  Confidence: \texttt{low\ (0.34582459069433863)}
\item
  Retrieval method: \texttt{local}
\item
  Evidence type: \texttt{textual}
\item
  Assumption flag: \texttt{False}
\item
  Answer:
\end{itemize}

The paper argues that how district lines are drawn shapes how political information spreads through social networks, which in turn affects voter knowledge and turnout. Thus, redistricting and assessments of partisan bias should incorporate social learning (rather than assume fixed voter behavior). It further notes that newly available social network data (e.g., the SCI) could help policymakers draw fairer districts, but the same data could be exploited by partisan gerrymanderers.

\begin{itemize}
\tightlist
\item
  Quote snippet:
\end{itemize}

\begin{quote}
enabling its use by policymakers to draw fairer districts, but also by partisan gerrymanderers who may seek to exploit it.
\end{quote}

\begin{itemize}
\tightlist
\item
  Citation anchors:

  \begin{itemize}
  \tightlist
  \item
    page=6, words=0-349, section=Empirical Strategy: District Homophily, note=District boundaries can shape political information flow via network--district alignment.
  \item
    page=5, words=275-475, section=Main text (gerrymandering implications), note=Findings imply turnout depends on how borders group one's social network; literature overlooks social learning.
  \item
    page=36, words=0-309, section=Implications preceding References, note=Social network data (SCI) could be used to draw fairer districts but may also be exploited by partisan gerrymanderers.
  \end{itemize}
\end{itemize}

\subsubsection{A10: What is the welfare interpretation (if any) of the results?}\label{a10-what-is-the-welfare-interpretation-if-any-of-the-results}

\begin{itemize}
\tightlist
\item
  Category: \texttt{A)\ Research\ question\ /\ contribution}
\item
  Confidence: \texttt{low\ (0.31149671482875635)}
\item
  Retrieval method: \texttt{local}
\item
  Evidence type: \texttt{textual\_evidence}
\item
  Data source: Main text (pages 19, 36)
\item
  Assumption flag: \texttt{False}
\item
  Answer:
\end{itemize}

The paper does not provide a formal welfare analysis. Its results imply that higher district homophily increases voter knowledge and reduces abstention, and it notes policy relevance: such network data could be used to draw fairer districts but also exploited by gerrymanderers, making welfare implications ambiguous.

\begin{itemize}
\tightlist
\item
  Quote snippet:
\end{itemize}

\begin{quote}
district homophily increases voters' knowledge about their representatives, and accordingly decreases abstention in House elections.
\end{quote}

\begin{itemize}
\tightlist
\item
  Citation anchors:

  \begin{itemize}
  \tightlist
  \item
    page=19, words=40-140, note=States main results: knowledge increases and abstention decreases.
  \item
    page=36, words=150-250, note=Policy relevance highlighting potential for fairer districts and for exploitation by gerrymanderers.
  \end{itemize}
\end{itemize}

\subsubsection{A11: What are the main limitations acknowledged by the authors?}\label{a11-what-are-the-main-limitations-acknowledged-by-the-authors}

\begin{itemize}
\tightlist
\item
  Category: \texttt{A)\ Research\ question\ /\ contribution}
\item
  Confidence: \texttt{low\ (0.26450904786970586)}
\item
  Retrieval method: \texttt{local}
\item
  Evidence type: \texttt{textual}
\item
  Assumption flag: \texttt{True}
\item
  Assumption notes: Identification requires: (1) limited fragmentation with a dominant district per county; (2) simple boundaries so each county spans ≤2 districts and nearby maps preserve this; (3) leakage to out-of-state nodes (ρ(Πs)\textless1); (4) non-degenerate network (independent rows).
\item
  Answer:
\end{itemize}

They note two broad limitations. Substantively, the paper does not yet estimate the diffusion model's key parameters or run counterfactual map simulations---those are deferred to future work. Methodologically, their identification result relies on restrictive map/network assumptions: limited fragmentation with a dominant district per county, simple boundaries where a county intersects at most two districts (and perturbations preserve this), non-trivial out-of-state network leakage (spectral radius \textless{} 1), and network non-degeneracy (linearly independent rows). They acknowledge the two-district split condition is restrictive, though it holds for most U.S. counties.

\begin{itemize}
\tightlist
\item
  Quote snippet:
\end{itemize}

\begin{quote}
Future work will focus on estimating the parameters of the diffusion process, which would allow for the simulation of counterfactual district maps.
\end{quote}

\begin{itemize}
\tightlist
\item
  Citation anchors:

  \begin{itemize}
  \tightlist
  \item
    page=34, words=200-320, note=Future work to estimate diffusion parameters and simulate counterfactual maps
  \item
    page=44, words=300-491, section=appendix, note=Assumptions 1--2: Limited fragmentation and simple boundaries; note on two-district split prevalence
  \item
    page=45, words=0-180, section=appendix, note=Assumptions 3--4: Non-trivial out-of-state connections and network non-degeneracy
  \item
    page=46, words=175-320, section=appendix, note=Identification depends on leakage and non-degeneracy to ensure invertibility
  \end{itemize}
\end{itemize}

\subsubsection{A12: What does the paper claim is novel about its data or identification?}\label{a12-what-does-the-paper-claim-is-novel-about-its-data-or-identification}

\begin{itemize}
\tightlist
\item
  Category: \texttt{A)\ Research\ question\ /\ contribution}
\item
  Confidence: \texttt{low\ (0.31282513555240943)}
\item
  Retrieval method: \texttt{local}
\item
  Evidence type: \texttt{textual\ evidence\ from\ main\ text}
\item
  Data source: Facebook friendship graph (national social ties)
\item
  Assumption flag: \texttt{False}
\item
  Answer:
\end{itemize}

The paper claims novelty in using nationwide social-network data (the Facebook friendship graph) to construct a district homophily measure that aligns social ties with congressional districts, and in identifying causal effects by exploiting geographic variation and redistricting-induced changes in homophily over time.

\begin{itemize}
\tightlist
\item
  Quote snippet:
\end{itemize}

\begin{quote}
By employing national data on social ties, my analysis comprehensively captures social networks
\end{quote}

\begin{itemize}
\tightlist
\item
  Citation anchors:

  \begin{itemize}
  \tightlist
  \item
    page=4, words=275-492, note=National data on social ties; leveraging geographic variation across the U.S.
  \item
    page=6, words=0-349, note=Defines district homophily and states use of Facebook friendship graph
  \item
    page=6, words=275-415, note=Identification leverages redistricting-driven changes over time
  \end{itemize}
\end{itemize}

\subsubsection{B01: What is the identification strategy (in one sentence)?}\label{b01-what-is-the-identification-strategy-in-one-sentence}

\begin{itemize}
\tightlist
\item
  Category: \texttt{B)\ Identification\ strategy\ /\ causal\ design}
\item
  Confidence: \texttt{low\ (0.3420695296675894)}
\item
  Retrieval method: \texttt{local}
\item
  Evidence type: \texttt{textual\ description\ of\ methods}
\item
  Data source: 2010 U.S. congressional redistricting; SCI-based district homophily
\item
  Answer:
\end{itemize}

Exploit changes in county-level district homophily induced by the 2010 redistricting in an event-study design (with fixed effects), with a district-border county-pairs comparison as an alternative robustness strategy.

\begin{itemize}
\tightlist
\item
  Quote snippet:
\end{itemize}

\begin{quote}
Focusing on a single redistricting event allows me to avoid concerns related to staggered treatment events
\end{quote}

\begin{itemize}
\tightlist
\item
  Citation anchors:

  \begin{itemize}
  \tightlist
  \item
    page=15, words=0-45, note=Focus on single 2010 redistricting event to avoid staggered treatment and permit pre-trend tests.
  \item
    page=15, words=120-260, note=Event-study setup around treatment timing (2012 as last pre-treatment year) and specification.
  \item
    page=28, words=0-90, note=Alternative identification via border county pairs across district lines.
  \end{itemize}
\end{itemize}

\subsubsection{B02: Is the design experimental, quasi-experimental, or observational?}\label{b02-is-the-design-experimental-quasi-experimental-or-observational}

\begin{itemize}
\tightlist
\item
  Category: \texttt{B)\ Identification\ strategy\ /\ causal\ design}
\item
  Confidence: \texttt{low\ (0.2960513458866162)}
\item
  Retrieval method: \texttt{local}
\item
  Evidence type: \texttt{textual}
\item
  Assumption flag: \texttt{False}
\item
  Answer:
\end{itemize}

Quasi-experimental

\begin{itemize}
\tightlist
\item
  Quote snippet:
\end{itemize}

\begin{quote}
changes in district homophily due to redistricting largely are not. Consequently, plausibly causal identification
\end{quote}

\begin{itemize}
\tightlist
\item
  Citation anchors:

  \begin{itemize}
  \tightlist
  \item
    page=6, words=0-349, section=2 Empirical Strategy and Networks Data, note=Uses redistricting-induced changes in district homophily for plausibly causal identification (natural experiment).
  \item
    page=15, words=0-349, section=Event study setup around 2010 redistricting, note=Event-study design leveraging a single redistricting event to estimate causal effects.
  \item
    page=20, words=0-349, section=Voters' Choices event studies, note=Event studies treat 2010 as base year with new district boundaries affecting behavior from 2012.
  \item
    page=28, words=0-349, section=Border pairs design, note=Border-pairs comparison across district lines is a quasi-experimental spatial design.
  \end{itemize}
\end{itemize}

\subsubsection{B03: What is the source of exogenous variation used for identification?}\label{b03-what-is-the-source-of-exogenous-variation-used-for-identification}

\begin{itemize}
\tightlist
\item
  Category: \texttt{B)\ Identification\ strategy\ /\ causal\ design}
\item
  Confidence: \texttt{low\ (0.3452658951371074)}
\item
  Retrieval method: \texttt{local}
\item
  Evidence type: \texttt{textual}
\item
  Data source: 2010 Census redistricting (new congressional district borders for 2012 elections)
\item
  Table/Figure: Appendix Figure C4
\item
  Assumption flag: \texttt{False}
\item
  Answer:
\end{itemize}

The 2010 Census--driven congressional redistricting (implemented for the 2012 elections), which induced changes in district homophily between 2012 and 2013; additionally, a border-pairs design comparing adjacent counties across district lines serves as an alternative source of variation.

\begin{itemize}
\tightlist
\item
  Quote snippet:
\end{itemize}

\begin{quote}
∆π̄c is the change in district homophily experienced by county c between 2012 and 2013
\end{quote}

\begin{itemize}
\tightlist
\item
  Citation anchors:

  \begin{itemize}
  \tightlist
  \item
    page=15, words=0-60, note=Focus on the single redistricting event following the 2010 Census
  \item
    page=15, words=220-270, note=Defines the treatment as the change in district homophily between 2012 and 2013
  \item
    page=28, words=0-120, note=Border-pairs identification comparing counties across district borders
  \end{itemize}
\end{itemize}

\subsubsection{B04: What is the treatment definition and timing?}\label{b04-what-is-the-treatment-definition-and-timing}

\begin{itemize}
\tightlist
\item
  Category: \texttt{B)\ Identification\ strategy\ /\ causal\ design}
\item
  Confidence: \texttt{low\ (0.11053936640375273)}
\item
  Retrieval method: \texttt{local}
\item
  Evidence type: \texttt{text}
\item
  Data source: 2010 Census redistricting
\item
  Table/Figure: Appendix Figure C4
\item
  Assumption flag: \texttt{False}
\item
  Answer:
\end{itemize}

Treatment is the change in district homophily caused by the 2010 redistricting, measured as Δπ̄c---the change a county experiences between 2012 and 2013. Timing: Census in Apr 2010; new borders used for Nov 2012 elections; representatives under new borders seated Jan 2013. The last pre-treatment year is 2012 for outcomes tied to the current representative, and 2011 (or 2010 for even-year outcomes) for upcoming-election outcomes; event studies assume 2012 as the last pre-treatment year.

\begin{itemize}
\tightlist
\item
  Quote snippet:
\end{itemize}

\begin{quote}
``∆π̄c is the change in district homophily experienced by county c between 2012 and 2013.''
\end{quote}

\begin{itemize}
\tightlist
\item
  Citation anchors:

  \begin{itemize}
  \tightlist
  \item
    page=15, words=0-349, note=Defines Δπ̄c as the change in district homophily between 2012 and 2013; details timing of Census, election, seating, and last pre-treatment year(s).
  \end{itemize}
\end{itemize}

\subsubsection{B05: What is the control/comparison group definition?}\label{b05-what-is-the-controlcomparison-group-definition}

\begin{itemize}
\tightlist
\item
  Category: \texttt{B)\ Identification\ strategy\ /\ causal\ design}
\item
  Confidence: \texttt{low\ (0.21389700792283337)}
\item
  Retrieval method: \texttt{local}
\item
  Evidence type: \texttt{Study\ design\ description\ (event\ study\ and\ border-pairs)}
\item
  Table/Figure: Appendix Table C2 (Border pairs results)
\item
  Assumption flag: \texttt{False}
\item
  Answer:
\end{itemize}

In the event‑study, the comparison is counties with little or no change in district homophily around the 2012 redistricting, using the pre‑treatment year as baseline (2012 for current‑representative outcomes; 2011/2010 for election‑related outcomes). In the border‑pairs design, the comparison group is the neighboring county across the district border in the same pair‑year (via pair‑by‑year fixed effects).

\begin{itemize}
\tightlist
\item
  Quote snippet:
\end{itemize}

\begin{quote}
we can identify the impact of district homophily by comparing deviations from the county-pair's mean in one county to deviations from the county-pair's mean in the neighboring county.
\end{quote}

\begin{itemize}
\tightlist
\item
  Citation anchors:

  \begin{itemize}
  \tightlist
  \item
    page=15, words=0-349, note=Defines last pre-treatment year(s) and event-study setup
  \item
    page=28, words=0-349, note=Border-pairs comparison defined (neighboring county within pair-year)
  \end{itemize}
\end{itemize}

\subsubsection{B06: What is the estimating equation / baseline regression specification?}\label{b06-what-is-the-estimating-equation--baseline-regression-specification}

\begin{itemize}
\tightlist
\item
  Category: \texttt{B)\ Identification\ strategy\ /\ causal\ design}
\item
  Confidence: \texttt{low\ (0.3288873548233163)}
\item
  Retrieval method: \texttt{local}
\item
  Evidence type: \texttt{direct\ text\ (equations)}
\item
  Data source: Equations (8) and (11) in text
\item
  Assumption flag: \texttt{False}
\item
  Answer:
\end{itemize}

Baseline specifications given in the text are:

\begin{enumerate}
\def\labelenumi{\arabic{enumi})}
\tightlist
\item
  Event-study (redistricting) specification: yict = λt + Στ βτ Δπ̄c I(τ = t) + Xct δ + Zict γ + εict, with year fixed effects, errors clustered at the county level, and optional additions of district-by-year fixed effects, DMA-by-year fixed effects, and partisan exposure controls.
\item
  Border-pairs specification: yct = αc + µpt + β π̄c,t + Xct δ + εct, which includes county fixed effects and pair-by-year fixed effects (and, in this restricted sample, state-by-year fixed effects rather than district-by-year fixed effects).
\end{enumerate}

\begin{itemize}
\tightlist
\item
  Quote snippet:
\end{itemize}

\begin{quote}
The specification for this design is ′ yct = αc + µpt + β π̄c,t + Xct δ + εct (11)
\end{quote}

\begin{itemize}
\tightlist
\item
  Citation anchors:

  \begin{itemize}
  \tightlist
  \item
    page=15, words=0-349, note=Event-study estimating equation (Eq. 8) and variable definitions.
  \item
    page=15, words=275-450, note=Optional district-by-year and DMA-by-year fixed effects; partisan exposure control.
  \item
    page=28, words=0-349, note=Border-pairs design estimating equation (Eq. 11).
  \end{itemize}
\end{itemize}

\subsubsection{B07: What fixed effects are included (unit, time, two-way, higher dimensional)?}\label{b07-what-fixed-effects-are-included-unit-time-two-way-higher-dimensional}

\begin{itemize}
\tightlist
\item
  Category: \texttt{B)\ Identification\ strategy\ /\ causal\ design}
\item
  Confidence: \texttt{low\ (0.3400683494503729)}
\item
  Retrieval method: \texttt{local}
\item
  Evidence type: \texttt{textual}
\item
  Data source: Appendix Table C2
\item
  Table/Figure: Appendix Table C2
\item
  Assumption flag: \texttt{False}
\item
  Answer:
\end{itemize}

Included fixed effects: (1) Time: year fixed effects. (2) Unit: county fixed effects; congressional district fixed effects; media market (DMA) fixed effects. (3) Two-way/higher-dimensional: district-by-year fixed effects; DMA-by-year fixed effects; pair-by-year fixed effects; state-by-year fixed effects. Note: in the border-pairs design, district-by-year FEs are not used; state-by-year and pair-by-year (and, in some specs, DMA-by-year) FEs are included.

\begin{itemize}
\tightlist
\item
  Quote snippet:
\end{itemize}

\begin{quote}
µpt is the pair-by-year fixed effect
\end{quote}

\begin{itemize}
\tightlist
\item
  Citation anchors:

  \begin{itemize}
  \tightlist
  \item
    page=page 15, words=0-349, note=Event study includes year fixed effects (λt).
  \item
    page=page 15, words=275-450, note=Mentions adding district-by-year and DMA-by-year fixed effects.
  \item
    page=page 28, words=0-349, note=Border pairs spec: county FE (αc), pair-by-year FE (μpt); uses state-by-year FEs; omits district-by-year FEs.
  \item
    page=page 14, words=275-429, note=Regression includes media market and congressional district fixed effects.
  \item
    page=page 19, words=275-406, note=Notes results similar when county fixed effects are included.
  \item
    page=page 65, words=0-288, section=appendix, note=Appendix Table C2 shows FEs: County, Pair x Year, State x Year, DMA x Year.
  \end{itemize}
\end{itemize}

\subsubsection{B08: What standard errors are used (robust, clustered; at what level)?}\label{b08-what-standard-errors-are-used-robust-clustered-at-what-level}

\begin{itemize}
\tightlist
\item
  Category: \texttt{B)\ Identification\ strategy\ /\ causal\ design}
\item
  Confidence: \texttt{low\ (0.3215341803039984)}
\item
  Retrieval method: \texttt{local}
\item
  Evidence type: \texttt{textual}
\item
  Table/Figure: Table C2
\item
  Assumption flag: \texttt{False}
\item
  Answer:
\end{itemize}

Clustered standard errors at the county level.

\begin{itemize}
\tightlist
\item
  Quote snippet:
\end{itemize}

\begin{quote}
Standard errors clustered at the county level in parentheses.
\end{quote}

\begin{itemize}
\tightlist
\item
  Citation anchors:

  \begin{itemize}
  \tightlist
  \item
    page=15, words=0-349, note=Model specification states errors are clustered at the county level.
  \item
    page=65, words=0-288, section=appendix, note=Table C2 note specifies standard errors clustered at the county level.
  \end{itemize}
\end{itemize}

\subsubsection{B09: What is the key identifying assumption (parallel trends, exclusion restriction, ignorability)?}\label{b09-what-is-the-key-identifying-assumption-parallel-trends-exclusion-restriction-ignorability}

\begin{itemize}
\tightlist
\item
  Category: \texttt{B)\ Identification\ strategy\ /\ causal\ design}
\item
  Confidence: \texttt{low\ (0.33852113349453294)}
\item
  Retrieval method: \texttt{local}
\item
  Evidence type: \texttt{textual\ (methods:\ event-study\ design\ and\ pre-trend\ checks)}
\item
  Assumption flag: \texttt{False}
\item
  Answer:
\end{itemize}

Parallel trends.

\begin{itemize}
\tightlist
\item
  Quote snippet:
\end{itemize}

\begin{quote}
allows for a visual test of pre-trends in changes in district homophily.
\end{quote}

\begin{itemize}
\tightlist
\item
  Citation anchors:

  \begin{itemize}
  \tightlist
  \item
    page=15, words=24-46, note=Mentions a visual test of pre-trends around the 2012 redistricting
  \item
    page=15, words=120-200, note=Introduces event-study specification (Eq. 8), which relies on parallel trends
  \end{itemize}
\end{itemize}

\subsubsection{B10: What evidence is provided to support the identifying assumption?}\label{b10-what-evidence-is-provided-to-support-the-identifying-assumption}

\begin{itemize}
\tightlist
\item
  Category: \texttt{B)\ Identification\ strategy\ /\ causal\ design}
\item
  Confidence: \texttt{low\ (0.26440040229632683)}
\item
  Retrieval method: \texttt{local}
\item
  Evidence type: \texttt{robustness\ checks,\ placebo\ tests,\ and\ alternative\ identification\ design}
\item
  Data source: Appendix Section C.4 (CES + SCI zip-code networks) and Appendix Table C2
\item
  Table/Figure: Appendix Table C2
\item
  Assumption flag: \texttt{False}
\item
  Answer:
\end{itemize}

The paper supports the identifying assumption with (i) robustness to constructing district homophily using zip-code--level networks---showing similar patterns, slightly larger effects, and no effects on placebo outcomes---and (ii) an alternative border-pairs specification that yields qualitatively similar results to the redistricting design (reported in Appendix Table C2). It also notes institutional/empirical constraints (e.g., most counties lie wholly in one district) that make district homophily informative about the assignment.

\begin{itemize}
\tightlist
\item
  Quote snippet:
\end{itemize}

\begin{quote}
With the border pairs design, I find qualitatively similar results as in the redistricting design
\end{quote}

\begin{itemize}
\tightlist
\item
  Citation anchors:

  \begin{itemize}
  \tightlist
  \item
    page=27, words=275-439, note=Zip-code--level network robustness and placebo outcomes show similar patterns and no placebo effects.
  \item
    page=28, words=0-349, note=Border-pairs specification yields qualitatively similar results; results in Appendix Table C2.
  \item
    page=44, words=0-349, section=appendix, note=Institutional constraints (many counties entirely in one district; \textasciitilde90\%) supporting informativeness of homophily.
  \end{itemize}
\end{itemize}

\subsubsection{B11: Are there event-study or pre-trend tests? What do they show?}\label{b11-are-there-event-study-or-pre-trend-tests-what-do-they-show}

\begin{itemize}
\tightlist
\item
  Category: \texttt{B)\ Identification\ strategy\ /\ causal\ design}
\item
  Confidence: \texttt{low\ (0.3265706672763601)}
\item
  Retrieval method: \texttt{local}
\item
  Evidence type: \texttt{textual\_evidence}
\item
  Data source: CES (Cooperative Election Study)
\item
  Answer:
\end{itemize}

Yes. The study uses event‑study designs (which also enable a visual pre‑trend test). The event studies show that effects on voter knowledge emerge right after redistricting---most strongly in the 2014 survey---and remain relatively stable over time.

\begin{itemize}
\tightlist
\item
  Quote snippet:
\end{itemize}

\begin{quote}
most strongly takes effect in the first survey after redistricting (2014). Impacts are relatively stable over time.
\end{quote}

\begin{itemize}
\tightlist
\item
  Citation anchors:

  \begin{itemize}
  \tightlist
  \item
    page=15, words=0-349, note=States that focusing on a single redistricting allows a visual pre-trend test and lays out the event-study framework.
  \item
    page=20, words=0-349, note=Reports event-study findings: effects appear in 2014 and are stable over time.
  \end{itemize}
\end{itemize}

\subsubsection{B12: What instruments are used (if IV)? Define instrument and first stage.}\label{b12-what-instruments-are-used-if-iv-define-instrument-and-first-stage}

\begin{itemize}
\tightlist
\item
  Category: \texttt{B)\ Identification\ strategy\ /\ causal\ design}
\item
  Confidence: \texttt{low\ (0.13210708535614082)}
\item
  Retrieval method: \texttt{local}
\item
  Evidence type: \texttt{inference\_from\_text}
\item
  Data source: CES surveys and Dave Leip's Election Atlas data (as described)
\item
  Answer:
\end{itemize}

No instrumental variables are described in the provided text. Identification relies on model structure (invertibility of the Jacobian via the Inverse Function Theorem), not an IV strategy; thus no instrument or first stage is defined here.

\begin{itemize}
\tightlist
\item
  Quote snippet:
\end{itemize}

\begin{quote}
By the Inverse Function Theorem, local injectivity is determined by the invertibility of the Jacobian DF (Q0s ).
\end{quote}

\begin{itemize}
\tightlist
\item
  Citation anchors:

  \begin{itemize}
  \tightlist
  \item
    page=45, words=275-370, section=appendix, note=Identification via Inverse Function Theorem and Jacobian invertibility, not IV.
  \item
    page=46, words=275-372, section=appendix, note=States the map is locally identified; no mention of instruments or first-stage regressions.
  \item
    page=32, words=0-349, note=Presents structural dynamics and steady-state expressions; no IV setup.
  \end{itemize}
\end{itemize}

\subsubsection{B13: What is the first-stage strength (F-stat, partial R\^{}2, relevance evidence)?}\label{b13-what-is-the-first-stage-strength-f-stat-partial-r2-relevance-evidence}

\begin{itemize}
\tightlist
\item
  Category: \texttt{B)\ Identification\ strategy\ /\ causal\ design}
\item
  Confidence: \texttt{low\ (0.30081851757688044)}
\item
  Retrieval method: \texttt{local}
\item
  Evidence type: \texttt{regression\_table\_and\_placebo\_text}
\item
  Data source: CES within-border-pairs regressions (appendix)
\item
  Table/Figure: Table C2
\item
  Answer:
\end{itemize}

Using the within--border-pairs regressions (Table C2), approximate single-instrument first-stage F-stats (t\^{}2 from coef/SE) in the most saturated specification (col 6) are: Heard of Incumbent F ≈ 7.9 (0.256/0.091); Selected Correct Party F ≈ 4.6 (0.343/0.160); Selected Party F ≈ 0.8 (0.136/0.151). Across specifications, F ranges are ≈ 7.4--10.6 (Heard), 4.4--14.3 (Selected Correct Party), and 0.8--6.8 (Selected Party). Partial R\^{}2 is not reported in the provided context. Relevance evidence: statistically significant positive effects for key familiarity measures (p=0.005 for Heard; p=0.032 for Selected Correct Party in col 6), and placebo tests show no significant relationships for governor/senators, supporting relevance specific to House-representative information.

\begin{itemize}
\tightlist
\item
  Quote snippet:
\end{itemize}

\begin{quote}
District Homophily 0.256*** (0.091) ... 0.136 (0.151) ... 0.343** (0.160)
\end{quote}

\begin{itemize}
\tightlist
\item
  Citation anchors:

  \begin{itemize}
  \tightlist
  \item
    page=65, words=0-288, section=appendix, note=Table C2 coefficients, standard errors, and p-values for effects of District Homophily on familiarity outcomes within border pairs.
  \item
    page=55, words=0-152, section=appendix, note=Placebo summary: district homophily does not significantly predict statewide-office outcomes.
  \end{itemize}
\end{itemize}

\subsubsection{B14: If RDD: what is the running variable and cutoff? bandwidth choice?}\label{b14-if-rdd-what-is-the-running-variable-and-cutoff-bandwidth-choice}

\begin{itemize}
\tightlist
\item
  Category: \texttt{B)\ Identification\ strategy\ /\ causal\ design}
\item
  Confidence: \texttt{low\ (0.2982535532267215)}
\item
  Retrieval method: \texttt{local}
\item
  Evidence type: \texttt{textual}
\item
  Assumption flag: \texttt{False}
\item
  Answer:
\end{itemize}

Not applicable---the study does not employ a regression discontinuity design. It uses event-study analyses (with 2010 as the base year) and a border-pairs specification; thus no running variable, cutoff, or bandwidth is defined.

\begin{itemize}
\tightlist
\item
  Quote snippet:
\end{itemize}

\begin{quote}
I run event studies analogous to equation 8 to examine the impact of district homophily on voting-related outcomes
\end{quote}

\begin{itemize}
\tightlist
\item
  Citation anchors:

  \begin{itemize}
  \tightlist
  \item
    page=20, words=275-461, note=States use of event-study design (2010 base year) for voting-related outcomes.
  \item
    page=19, words=275-406, note=Describes event studies and focus on even years for interpretation.
  \item
    page=28, words=0-349, note=Describes border-pairs specification as an alternative identification strategy.
  \end{itemize}
\end{itemize}

\subsubsection{B15: If DiD: what is the timing variation (staggered adoption)? estimator used?}\label{b15-if-did-what-is-the-timing-variation-staggered-adoption-estimator-used}

\begin{itemize}
\tightlist
\item
  Category: \texttt{B)\ Identification\ strategy\ /\ causal\ design}
\item
  Confidence: \texttt{low\ (0.3241949630430817)}
\item
  Retrieval method: \texttt{local}
\item
  Evidence type: \texttt{textual}
\item
  Data source: 2010 Census redistricting (Nov 2012 elections)
\item
  Assumption flag: \texttt{False}
\item
  Answer:
\end{itemize}

No staggered adoption. The design centers on a single redistricting shock from the 2010 Census, with effects realized for the Nov 2012 elections and new representatives seated Jan 2013. The estimator is an event-study difference-in-differences that interacts the county's change in district homophily (Δπ̄c, 2012--2013) with event-time indicators, includes year fixed effects, county-by-year and individual controls, errors clustered at the county level, and can include district-by-year fixed effects.

\begin{itemize}
\tightlist
\item
  Quote snippet:
\end{itemize}

\begin{quote}
Focusing on a single redistricting event allows me to avoid concerns related to staggered treatment events\ldots{} the congressional representatives first elected under the new borders assumed office in January 2013.
\end{quote}

\begin{itemize}
\tightlist
\item
  Citation anchors:

  \begin{itemize}
  \tightlist
  \item
    page=15, words=0-349, note=Single event avoids staggered adoption; timing (2012 elections, Jan 2013 seating) and event-study specification with FE and clustering.
  \item
    page=19, words=275-406, note=Event-study interpretation context (even-year focus; outcome scaling).
  \end{itemize}
\end{itemize}

\subsubsection{C01: What dataset(s) are used? (name sources explicitly)}\label{c01-what-datasets-are-used-name-sources-explicitly}

\begin{itemize}
\tightlist
\item
  Category: \texttt{C)\ Data,\ sample,\ and\ measurement}
\item
  Confidence: \texttt{low\ (0.3256958101891945)}
\item
  Retrieval method: \texttt{local}
\item
  Evidence type: \texttt{textual\ evidence\ from\ methods\ and\ appendix}
\item
  Data source: Appendix B.1.1; Sections 3.1--3.2; Section 5.3; Page 8 (SCI details)
\item
  Assumption flag: \texttt{False}
\item
  Answer:
\end{itemize}

Cooperative Election Study (CES, formerly CCES); Meta's Social Connectedness Index (SCI); Dave Leip's Atlas of U.S. Presidential Elections (Election Atlas); precinct-level vote counts from the Harvard Election Data Archive (2000--2010) and the MIT Election Data and Science Lab (2016--2020); county-by-congressional-district vote counts from Daily Kos.

\begin{itemize}
\tightlist
\item
  Quote snippet:
\end{itemize}

\begin{quote}
precinct-level vote count data from the Harvard Election Data Archive (for 2000-2010) and the MIT Election Data and Science Lab (for 2016-2020), combined with county-by-congressional-district vote count data from Dave Leip's Election Atlas (for House elections) and Daily Kos
\end{quote}

\begin{itemize}
\tightlist
\item
  Citation anchors:

  \begin{itemize}
  \tightlist
  \item
    page=16, words=0-349, note=Use of Cooperative Election Study (CES) for voter information and behavior
  \item
    page=8, words=275-624, note=Description of the Social Connectedness Index (SCI) from Meta apps
  \item
    page=47, words=0-207, section=appendix, note=Vote count data sources: Harvard EDA, MIT Election Data and Science Lab, Dave Leip's Election Atlas, and Daily Kos
  \item
    page=17, words=275-542, note=Use of Dave Leip's Election Atlas in turnout/vote share analysis
  \item
    page=27, words=0-349, note=Zip-code-level construction using SCI
  \end{itemize}
\end{itemize}

\subsubsection{C02: What is the unit of observation (individual, household, firm, county, transaction, product)?}\label{c02-what-is-the-unit-of-observation-individual-household-firm-county-transaction-product}

\begin{itemize}
\tightlist
\item
  Category: \texttt{C)\ Data,\ sample,\ and\ measurement}
\item
  Confidence: \texttt{low\ (0.33652792509374285)}
\item
  Retrieval method: \texttt{local}
\item
  Evidence type: \texttt{text}
\item
  Data source: Cooperative Election Study (CES)
\item
  Table/Figure: Table B3: CES Data: Summary Statistics
\item
  Assumption flag: \texttt{False}
\item
  Answer:
\end{itemize}

individual

\begin{itemize}
\tightlist
\item
  Quote snippet:
\end{itemize}

\begin{quote}
yict is the outcome for a given individual i in county c in year t
\end{quote}

\begin{itemize}
\tightlist
\item
  Citation anchors:

  \begin{itemize}
  \tightlist
  \item
    page=15, words=180-260, note=Defines the outcome at the individual level: yict is for a given individual i in county c in year t.
  \item
    page=17, words=300-370, note=States the analysis uses CES survey responses with hundreds of thousands of respondents (individuals).
  \item
    page=49, words=0-60, section=appendix, note=Summary statistics list large numbers of observations for respondent-level variables.
  \end{itemize}
\end{itemize}

\subsubsection{C03: What is the sample period and geographic coverage?}\label{c03-what-is-the-sample-period-and-geographic-coverage}

\begin{itemize}
\tightlist
\item
  Category: \texttt{C)\ Data,\ sample,\ and\ measurement}
\item
  Confidence: \texttt{low\ (0.2497455184946808)}
\item
  Retrieval method: \texttt{local}
\item
  Evidence type: \texttt{text}
\item
  Assumption flag: \texttt{False}
\item
  Answer:
\end{itemize}

2006--2022; the 48 contiguous U.S. states (continental U.S.), analyzed at the county level (3,136 counties).

\begin{itemize}
\tightlist
\item
  Quote snippet:
\end{itemize}

\begin{quote}
I only include respondents in the 48 contiguous states... across all 17 years of the CES
\end{quote}

\begin{itemize}
\tightlist
\item
  Citation anchors:

  \begin{itemize}
  \tightlist
  \item
    page=17, words=0-349, note=Geographic restriction to 48 contiguous states.
  \item
    page=17, words=275-542, note=Coverage across 17 years of CES.
  \item
    page=15, words=0-349, note=Event-study window includes 2006--2022.
  \item
    page=8, words=275-624, note=County coverage: 3,136 counties.
  \end{itemize}
\end{itemize}

\subsubsection{C04: What are the sample restrictions / inclusion criteria?}\label{c04-what-are-the-sample-restrictions--inclusion-criteria}

\begin{itemize}
\tightlist
\item
  Category: \texttt{C)\ Data,\ sample,\ and\ measurement}
\item
  Confidence: \texttt{low\ (0.25791566656218423)}
\item
  Retrieval method: \texttt{local}
\item
  Evidence type: \texttt{text}
\item
  Data source: CES (Cooperative Election Study); Dave Leip's Election Atlas; Meta Social Connectedness Index (SCI)
\item
  Assumption flag: \texttt{False}
\item
  Answer:
\end{itemize}

CES analyses include only respondents in the 48 contiguous states. Exclusions: (1) respondents with missing answers to the candidate party recognition question; (2) respondents missing required demographic answers when demographic controls are used; (3) a small number in 2006--2007 assigned to counties outside their state; (4) 925 North Carolina respondents in 2020 assigned to incorrect congressional districts (shown wrong candidate names). For voting/event‑study analyses, the sample focuses on even years (odd years are smaller and noisier; most voting outcomes are only available in even years). Coverage note: in even years at least one respondent is present in 80--90\% of counties, about two‑thirds in odd years. For SCI-based measures, the SCI includes only users who interacted with a Meta app in the prior 30 days, and areas with very small populations are removed.

\begin{itemize}
\tightlist
\item
  Quote snippet:
\end{itemize}

\begin{quote}
I only include respondents in the 48 contiguous states.
\end{quote}

\begin{itemize}
\tightlist
\item
  Citation anchors:

  \begin{itemize}
  \tightlist
  \item
    page=17, words=0-349, note=Includes 48 contiguous states and county coverage notes.
  \item
    page=17, words=275-542, note=Footnote detailing exclusions (missing responses, demographic nonresponse, misassigned counties, NC 2020 district error).
  \item
    page=19, words=275-406, note=Restriction to even years for consistency and data availability.
  \item
    page=8, words=275-624, note=SCI inclusion criterion (recent Meta app use) and small-population area removal.
  \end{itemize}
\end{itemize}

\subsubsection{C05: What is the sample size (N) in the main analysis?}\label{c05-what-is-the-sample-size-n-in-the-main-analysis}

\begin{itemize}
\tightlist
\item
  Category: \texttt{C)\ Data,\ sample,\ and\ measurement}
\item
  Confidence: \texttt{low\ (0.2534699615932913)}
\item
  Retrieval method: \texttt{local}
\item
  Evidence type: \texttt{text}
\item
  Data source: CES (Cooperative Election Study)
\item
  Assumption flag: \texttt{True}
\item
  Assumption notes: Interprets \textquotesingle main analysis\textquotesingle{} as the CES-based analysis sample; the paper also cites SCI coverage (3,136 counties; 9,834,496 county pairs) for network data.
\item
  Answer:
\end{itemize}

612,085 respondents

\begin{itemize}
\tightlist
\item
  Quote snippet:
\end{itemize}

\begin{quote}
In these 48 states and across all 17 years of the CES, there are 612,085 respondents
\end{quote}

\begin{itemize}
\tightlist
\item
  Citation anchors:

  \begin{itemize}
  \tightlist
  \item
    page=17, words=275-542, note=States total CES respondents used across 17 years in 48 states
  \end{itemize}
\end{itemize}

\subsubsection{C06: How is the key outcome measured? Any transformations (logs, z-scores, indices)?}\label{c06-how-is-the-key-outcome-measured-any-transformations-logs-z-scores-indices}

\begin{itemize}
\tightlist
\item
  Category: \texttt{C)\ Data,\ sample,\ and\ measurement}
\item
  Confidence: \texttt{low\ (0.3102898319611432)}
\item
  Retrieval method: \texttt{local}
\item
  Evidence type: \texttt{Appendix\ variable\ descriptions\ and\ methods\ text}
\item
  Data source: CES (Cooperative Election Study); Dave Leip's Election Atlas; Harvard Election Data Archive; MIT Election Data and Science Lab; Daily Kos
\item
  Table/Figure: Table B1; Table B2; Table B4
\item
  Assumption flag: \texttt{True}
\item
  Assumption notes: Interprets the paper's primary outcomes as the CES voter knowledge/voting measures and county-level turnout/roll-off; no evidence of additional transformations provided.
\item
  Answer:
\end{itemize}

The key outcomes are measured as binary survey indicators (e.g., Heard of Representative, Selected Party, Selected Correct Party) from the CES, and as voting outcomes that are either binary choices or turnout shares of the Voting Age Population (VAP) and a roll-off difference. No log, z-score, or index transformations are applied; estimates are interpreted as changes in probability since outcomes are binary and district homophily is scaled 0--1.

\begin{itemize}
\tightlist
\item
  Quote snippet:
\end{itemize}

\begin{quote}
District homophily is measured on a scale from 0 to 1, and outcome variables are binary.
\end{quote}

\begin{itemize}
\tightlist
\item
  Citation anchors:

  \begin{itemize}
  \tightlist
  \item
    page=19, words=275-406, note=States outcomes are binary and district homophily is 0--1, so estimates are probability changes.
  \item
    page=47, words=0-207, section=appendix, note=Defines voter knowledge outcomes as Binary from pre-survey.
  \item
    page=48, words=0-333, section=appendix, note=Defines voting preference/choice and validated turnout variables as Binary from pre/post surveys.
  \item
    page=49, words=0-276, section=appendix, note=Defines roll-off and turnout measures; turnout is share of VAP.
  \end{itemize}
\end{itemize}

\subsubsection{C07: How is treatment/exposure measured? Any constructed variables?}\label{c07-how-is-treatmentexposure-measured-any-constructed-variables}

\begin{itemize}
\tightlist
\item
  Category: \texttt{C)\ Data,\ sample,\ and\ measurement}
\item
  Confidence: \texttt{low\ (0.3034837296648951)}
\item
  Retrieval method: \texttt{local}
\item
  Evidence type: \texttt{textual}
\item
  Data source: CES surveys; Dave Leip's Election Atlas; Harvard Election Data Archive; MIT Election Data and Science Lab; Daily Kos
\item
  Assumption flag: \texttt{False}
\item
  Answer:
\end{itemize}

Treatment/exposure is the change in district homophily at the county level induced by the 2010 redistricting, operationalized as Δπ̄c---the difference in a county's district homophily between 2012 (pre) and 2013 (post). Constructed variables include: (1) county exposure to Democrats (friendship-share--weighted Democratic vote shares across connected counties); (2) aggregated network measures connecting the model to data---the county--county friendship matrix Π (πc,c′), district population shares q(c,d), and aggregated county steady-state informed share ρoc(t); (3) CES voter knowledge binaries (Heard of Representative, Selected Party, Selected Correct Party); and (4) voting outcomes such as House roll-off (top-of-ticket votes minus House votes) and turnout rates as shares of VAP. Models also use district-by-year and DMA-by-year fixed effects to net out election- and media-market shocks.

\begin{itemize}
\tightlist
\item
  Quote snippet:
\end{itemize}

\begin{quote}
∆π̄c is the change in district homophily experienced by county c between 2012 and 2013
\end{quote}

\begin{itemize}
\tightlist
\item
  Citation anchors:

  \begin{itemize}
  \tightlist
  \item
    page=15, words=0-349, note=Defines Δπ̄c as the change in district homophily between 2012 and 2013 and frames treatment timing.
  \item
    page=15, words=275-450, note=Constructed exposure to Democrats and DMA-by-year fixed effects.
  \item
    page=33, words=0-349, note=Constructed network aggregation: Π (πc,c′), q(c,d), and aggregated ρoc(t).
  \item
    page=47, words=0-207, section=appendix, note=Constructed CES knowledge variables: definitions for Heard/Selected/Selected Correct Party.
  \item
    page=49, words=0-276, section=appendix, note=Constructed voting outcomes: Roll-Off and turnout measures; data sources.
  \end{itemize}
\end{itemize}

\subsubsection{C08: Are there key covariates/controls? Which ones are always included?}\label{c08-are-there-key-covariatescontrols-which-ones-are-always-included}

\begin{itemize}
\tightlist
\item
  Category: \texttt{C)\ Data,\ sample,\ and\ measurement}
\item
  Confidence: \texttt{low\ (0.34956408360977403)}
\item
  Retrieval method: \texttt{local}
\item
  Evidence type: \texttt{direct\ text}
\item
  Data source: Main text (p. 15) and Appendix Table C2 (p. 65)
\item
  Table/Figure: Table C2
\item
  Assumption flag: \texttt{False}
\item
  Answer:
\end{itemize}

Yes. Key covariates/controls include: (a) year fixed effects (always in the event-study specs); (b) county-by-year demographic controls; and (c) individual-level controls. Additional controls used in some specifications are district-by-year fixed effects, DMA-by-year fixed effects, and a county-level Democratic exposure measure. In the within-border-pair regressions (Table C2), County and Pair×Year fixed effects are included in every column, with State×Year and DMA×Year added in some columns.

\begin{itemize}
\tightlist
\item
  Quote snippet:
\end{itemize}

\begin{quote}
λt are year fixed effects, Xct is a vector of county-by-year controls, and Zict is a vector of individual controls. I can additionally include district-by-year fixed effects\ldots{} include DMA-by-year fixed effects.
\end{quote}

\begin{itemize}
\tightlist
\item
  Citation anchors:

  \begin{itemize}
  \tightlist
  \item
    page=15, words=0-349, note=Event-study spec: year FEs, county-by-year controls (Xct), individual controls (Zict); clustering.
  \item
    page=15, words=275-450, note=Optional additions: district-by-year FEs, DMA-by-year FEs, Democratic exposure control.
  \item
    page=65, words=0-288, section=appendix, note=Within border-pair regressions: FE structure across columns; definitions of individual and county-by-year controls.
  \end{itemize}
\end{itemize}

\subsubsection{C09: How are missing data handled (dropping, imputation, weighting)?}\label{c09-how-are-missing-data-handled-dropping-imputation-weighting}

\begin{itemize}
\tightlist
\item
  Category: \texttt{C)\ Data,\ sample,\ and\ measurement}
\item
  Confidence: \texttt{low\ (0.31647121514709664)}
\item
  Retrieval method: \texttt{local}
\item
  Evidence type: \texttt{text\_from\_document}
\item
  Data source: CES; Dave Leip's Election Atlas; Harvard Election Data Archive; MIT Election Data and Science Lab; Daily Kos
\item
  Table/Figure: Table B2
\item
  Assumption flag: \texttt{False}
\item
  Answer:
\end{itemize}

Missing/invalid observations are excluded rather than imputed. Specifically: (1) survey cases with missing candidate-party recognition responses are dropped; respondents missing demographic answers are also excluded when those controls are used; misassigned county/district cases (including 925 NC respondents in 2020) are removed. (2) Several voting variables are coded missing when there is no incumbent running, or when both validated voting indicators are missing. (3) In vote-count data, elections with no top-of-ticket race and cases where turnout exceeds VAP are excluded. Weighting: CES cumulative weights are used to re-weight observations across years; weights do not adjust for pre/post attrition. No imputation is described.

\begin{itemize}
\tightlist
\item
  Quote snippet:
\end{itemize}

\begin{quote}
I exclude missing responses to the candidate party recognition question (\textless2\% of respondents in each year...). When including individual demographic controls, I similarly exclude respondents who did not answer...
\end{quote}

\begin{itemize}
\tightlist
\item
  Citation anchors:

  \begin{itemize}
  \tightlist
  \item
    page=17, words=275-542, note=Drops missing candidate-party recognition responses; excludes respondents missing demographics; removes misassigned county/district cases; weights do not account for attrition.
  \item
    page=16, words=0-349, note=Uses CES cumulative weights to re-weight observations across years.
  \item
    page=48, words=0-333, section=appendix, note=Voting variables set to missing if no incumbent; post-survey voting variables missing if both validated indicators are missing.
  \item
    page=49, words=0-276, section=appendix, note=Excludes elections with no top-of-ticket race and cases where turnout exceeds VAP.
  \end{itemize}
\end{itemize}

\subsubsection{C10: Are weights used (survey weights, propensity weights)? How?}\label{c10-are-weights-used-survey-weights-propensity-weights-how}

\begin{itemize}
\tightlist
\item
  Category: \texttt{C)\ Data,\ sample,\ and\ measurement}
\item
  Confidence: \texttt{medium\ (0.3628265093281664)}
\item
  Retrieval method: \texttt{local}
\item
  Evidence type: \texttt{textual}
\item
  Data source: CES (Cooperative Election Study) survey weights
\item
  Assumption flag: \texttt{False}
\item
  Answer:
\end{itemize}

Yes. CES survey weights are used so the respondent sample represents people living in the U.S.; however, the weights do not adjust for attrition between the pre- and post-election surveys.

\begin{itemize}
\tightlist
\item
  Quote snippet:
\end{itemize}

\begin{quote}
Because the weighted sample is representative of people living in the U.S. (rather than of U.S. counties)
\end{quote}

\begin{itemize}
\tightlist
\item
  Citation anchors:

  \begin{itemize}
  \tightlist
  \item
    page=17, words=270-349, note=States that the weighted sample is representative of U.S. residents.
  \item
    page=20, words=430-461, note=Notes that weights do not account for attrition between survey waves.
  \end{itemize}
\end{itemize}

\subsubsection{C11: Are data linked/merged across sources? How is linkage performed?}\label{c11-are-data-linkedmerged-across-sources-how-is-linkage-performed}

\begin{itemize}
\tightlist
\item
  Category: \texttt{C)\ Data,\ sample,\ and\ measurement}
\item
  Confidence: \texttt{low\ (0.30299321978219823)}
\item
  Retrieval method: \texttt{local}
\item
  Evidence type: \texttt{textual}
\item
  Data source: Facebook Social Connectedness Index (SCI); county--district population shares q(c,d); commuting flows; CES respondent zip codes; county-level vote/contribution data.
\item
  Assumption flag: \texttt{False}
\item
  Answer:
\end{itemize}

Yes. The paper links Facebook SCI network data to congressional districts by aggregating county--county friendship shares and weighting by county--district population shares. Specifically, it builds πc,k from SCI, then computes district homophily as Σd∈D(c) Σk πc,k × q(c,d) × q(k,d), where q(c,d) is the share of county c's population in district d. For robustness, the same aggregation is done using commuting flows. At the zip level, SCI is linked to CES respondents via reported zip codes. County-level SCI is used to match with county-level vote counts and campaign contributions.

\begin{itemize}
\tightlist
\item
  Quote snippet:
\end{itemize}

\begin{quote}
I do this by using the SCI to construct the Π matrix of county-county friendship shares, and then for each county summing friendship shares across same-district counties (adjusting for counties that intersect multiple districts).
\end{quote}

\begin{itemize}
\tightlist
\item
  Citation anchors:

  \begin{itemize}
  \tightlist
  \item
    page=7, words=0-349, section=2.1.1 Definition of District Homophily, note=Defines district homophily and introduces population-weighted linkage via q(c,d).
  \item
    page=9, words=0-257, section=2.1.3 Construction of District Homophily from SCI, note=Explains aggregating SCI to Π and summing within-district shares; constructs πc,k from SCI.
  \item
    page=8, words=0-349, section=2.1.2 Proxy for Social Networks: Facebook Social Connectedness Index, note=Defines SCI and notes county focus to facilitate matching to county-level outcomes.
  \item
    page=27, words=0-349, section=5.3 Zip-Code-Level Social Network Data, note=Links zip-level SCI to CES via respondents' reported zip codes.
  \item
    page=33, words=0-349, section=6.7 Aggregating to County-Level, note=Details aggregation using county--district population shares and county-county friendship probabilities.
  \end{itemize}
\end{itemize}

\subsubsection{C12: What summary statistics are reported for main variables?}\label{c12-what-summary-statistics-are-reported-for-main-variables}

\begin{itemize}
\tightlist
\item
  Category: \texttt{C)\ Data,\ sample,\ and\ measurement}
\item
  Confidence: \texttt{medium\ (0.3757044005327591)}
\item
  Retrieval method: \texttt{local}
\item
  Evidence type: \texttt{text\ and\ tables}
\item
  Table/Figure: Table B3: CES Data: Summary Statistics; Table B5: Voting Outcomes: Summary Statistics; Table C1: Placebo Outcomes Summary Statistics
\item
  Assumption flag: \texttt{False}
\item
  Answer:
\end{itemize}

Summary statistics reported include: (1) for district homophily---mean, standard deviation, minimum, maximum, 1st and 99th percentiles, and ranges for the middle 50\% and 80\% of counties; (2) for CES knowledge/voting and county voting outcomes---number of observations, mean (percent), and standard deviation (percentage points).

\begin{itemize}
\tightlist
\item
  Quote snippet:
\end{itemize}

\begin{quote}
mean district homophily is 41\% with a standard deviation of 14pp; minimum district homophily is 2\% and maximum is 87\%, while the 1st percentile is 8\% and the 99th percentile is 67\%
\end{quote}

\begin{itemize}
\tightlist
\item
  Citation anchors:

  \begin{itemize}
  \tightlist
  \item
    page=13, words=0-349, note=District homophily summary statistics: mean, SD, min, max, percentiles, middle ranges.
  \item
    page=49, words=0-276, section=appendix, note=Tables B3 and B5 show Observations, Mean (\%), SD (pp) for CES and voting outcomes.
  \item
    page=55, words=0-152, section=appendix, note=Table C1 (placebo outcomes) also uses Observations, Mean (\%), SD (pp).
  \end{itemize}
\end{itemize}

\subsubsection{C13: Are there descriptive figures/maps that establish baseline patterns?}\label{c13-are-there-descriptive-figuresmaps-that-establish-baseline-patterns}

\begin{itemize}
\tightlist
\item
  Category: \texttt{C)\ Data,\ sample,\ and\ measurement}
\item
  Confidence: \texttt{low\ (0.33953190147297013)}
\item
  Retrieval method: \texttt{local}
\item
  Evidence type: \texttt{text}
\item
  Data source: Paper appendix (figures cited)
\item
  Table/Figure: Appendix Figures C1--C2; Figures C4--C5
\item
  Assumption flag: \texttt{False}
\item
  Answer:
\end{itemize}

Yes. Appendix Figures C1--C2 provide descriptive figures on how geographic and demographic features correlate with district homophily (2012 and 2020), and Appendix Figures C4--C5 map the nationwide distribution of changes.

\begin{itemize}
\tightlist
\item
  Quote snippet:
\end{itemize}

\begin{quote}
Appendix Figures C1--C2 summarize how various geographic and demographic features correlate with district homophily, separately in 2012 and 2020.
\end{quote}

\begin{itemize}
\tightlist
\item
  Citation anchors:

  \begin{itemize}
  \tightlist
  \item
    page=13, words=0-349, note=Mentions Appendix Figures C1--C2 summarizing correlations in 2012 and 2020.
  \item
    page=15, words=0-349, note=States Appendix Figure C4 provides a map of changes.
  \item
    page=53, words=0-40, section=appendix, note=Notes that Figures C4--C5 illustrate how changes are distributed around the country.
  \end{itemize}
\end{itemize}

\subsubsection{D01: What is the headline main effect estimate (sign and magnitude)?}\label{d01-what-is-the-headline-main-effect-estimate-sign-and-magnitude}

\begin{itemize}
\tightlist
\item
  Category: \texttt{D)\ Results,\ magnitudes,\ heterogeneity,\ robustness}
\item
  Confidence: \texttt{medium\ (0.3668964869451069)}
\item
  Retrieval method: \texttt{local}
\item
  Evidence type: \texttt{text\ (main\ text)}
\item
  Table/Figure: Figure 8
\item
  Answer:
\end{itemize}

Positive: a 10pp increase in district homophily raises the share of contributions to in-district candidates by 7.4pp.

\begin{itemize}
\tightlist
\item
  Quote snippet:
\end{itemize}

\begin{quote}
In particular, a 10pp increase in district homophily is associated with a 7.4pp increase in the share of contributions to in- district candidates, from a mean of 51\%.
\end{quote}

\begin{itemize}
\tightlist
\item
  Citation anchors:

  \begin{itemize}
  \tightlist
  \item
    page=24, words=0-229, note=Campaign contributions main effect
  \end{itemize}
\end{itemize}

\subsubsection{D02: What is the preferred specification and why is it preferred?}\label{d02-what-is-the-preferred-specification-and-why-is-it-preferred}

\begin{itemize}
\tightlist
\item
  Category: \texttt{D)\ Results,\ magnitudes,\ heterogeneity,\ robustness}
\item
  Confidence: \texttt{low\ (0.19272777411442193)}
\item
  Retrieval method: \texttt{local}
\item
  Evidence type: \texttt{textual}
\item
  Data source: Facebook Social Connectedness Index (SCI)
\item
  Table/Figure: Figure 1 (page 9)
\item
  Assumption flag: \texttt{False}
\item
  Answer:
\end{itemize}

The preferred specification constructs district homophily using the Facebook Social Connectedness Index (SCI), because SCI is a strong proxy for real-world social networks, whereas commuting flows are more geographically concentrated and tend to overstate homophily.

\begin{itemize}
\tightlist
\item
  Quote snippet:
\end{itemize}

\begin{quote}
I use the Facebook Social Connectedness Index (SCI)---one of the best existing proxies for real-world social networks.
\end{quote}

\begin{itemize}
\tightlist
\item
  Citation anchors:

  \begin{itemize}
  \tightlist
  \item
    page=7, words=275-379, note=States the use of SCI as the data source for social networks and justifies it as a top proxy.
  \item
    page=9, words=0-257, note=Explains that commuting flows are more concentrated, leading to higher commuting-based homophily than SCI.
  \end{itemize}
\end{itemize}

\subsubsection{D03: How economically meaningful is the effect (percent change, elasticity, dollars)?}\label{d03-how-economically-meaningful-is-the-effect-percent-change-elasticity-dollars}

\begin{itemize}
\tightlist
\item
  Category: \texttt{D)\ Results,\ magnitudes,\ heterogeneity,\ robustness}
\item
  Confidence: \texttt{low\ (0.33059979194916095)}
\item
  Retrieval method: \texttt{local}
\item
  Evidence type: \texttt{Reported\ quantitative\ estimates\ from\ event-study\ analyses\ in\ the\ paper}
\item
  Data source: CES survey; Dave Leip's Election Atlas (county-level votes)
\item
  Table/Figure: Figure 6 (Roll-Off); Figure 8 (Contributions); Table B3 (CES summary stats)
\item
  Assumption flag: \texttt{True}
\item
  Assumption notes: Interprets effects per 10pp assuming linear scaling and computes relative percent changes using reported means; adopts paper's stated 10\% roll-off reduction and 2pp VAP-based roll-off change.
\item
  Answer:
\end{itemize}

A 10pp rise in district homophily produces modest but nontrivial shifts: voter knowledge rises by 0.7pp (≈0.75\% relative) for name recognition, 3.2pp (≈4.7\%) for picking a party, and 3.3pp (≈5.3\%) for picking the correct party. It reduces ballot roll-off by about 10\% (and by about 2pp when roll-off is constructed using votes/VAP). For money, it reallocates giving toward local races: the in-district share of House contributions rises by 7.4pp from a 51\% mean (\textasciitilde14.5\% relative), with no change in total donations.

\begin{itemize}
\tightlist
\item
  Quote snippet:
\end{itemize}

\begin{quote}
a 10pp increase in district homophily is associated with a 7.4pp increase in the share of contributions to in-district candidates, from a mean of 51\%.
\end{quote}

\begin{itemize}
\tightlist
\item
  Citation anchors:

  \begin{itemize}
  \tightlist
  \item
    page=20, words=0-349, note=Knowledge effects per 10pp increase and baseline means
  \item
    page=24, words=0-229, note=Roll-off reduction (\textasciitilde10\%) and contributions shift (+7.4pp from 51\%)
  \item
    page=67, words=0-258, section=appendix, note=Alternative roll-off measure shows \textasciitilde2pp decrease; no change in turnout levels
  \item
    page=49, words=0-276, section=appendix, note=Baseline means for knowledge outcomes (Table B3)
  \end{itemize}
\end{itemize}

\subsubsection{D04: What are the key robustness checks and do results survive them?}\label{d04-what-are-the-key-robustness-checks-and-do-results-survive-them}

\begin{itemize}
\tightlist
\item
  Category: \texttt{D)\ Results,\ magnitudes,\ heterogeneity,\ robustness}
\item
  Confidence: \texttt{low\ (0.3283513759686765)}
\item
  Retrieval method: \texttt{local}
\item
  Evidence type: \texttt{mixed}
\item
  Data source: CES (Cooperative Election Study)
\item
  Table/Figure: Table C2: Effect of District Homophily on Voter Familiarity with Representative, within Border Pairs
\item
  Assumption flag: \texttt{False}
\item
  Answer:
\end{itemize}

Key checks include: (1) adding district-by-year fixed effects to absorb election-specific shocks; (2) adding DMA-by-year fixed effects to rule out media-market confounds; (3) controlling for partisan bias in network ties via each county's exposure to Democrats; (4) using state-by-year fixed effects; (5) exploiting a within--border-pair design with county and pair-by-year fixed effects; (6) adding individual demographic controls (gender, race, education, age, co-partisanship) and (7) adding county-by-year demographics (population, race/age/gender shares, urban share); and (8) placebo tests on governors and senators. Results largely survive: the effects on `Heard of Incumbent' and `Selected Correct Party' remain positive and statistically significant across specifications with these controls, while the weaker `Selected Party' measure loses significance. Placebo outcomes show no significant relationship, supporting robustness.

\begin{itemize}
\tightlist
\item
  Quote snippet:
\end{itemize}

\begin{quote}
In general, district homophily does not significantly predict the placebo outcomes.
\end{quote}

\begin{itemize}
\tightlist
\item
  Citation anchors:

  \begin{itemize}
  \tightlist
  \item
    page=15, words=275-450, note=Describes adding district-by-year and DMA-by-year fixed effects; controls for partisan exposure to Democrats.
  \item
    page=65, words=0-288, section=appendix, note=Table C2 shows robustness across County \& Pair×Year, State×Year, DMA×Year FEs; adds Democratic exposure, individual and county-year controls; significance for Heard of Incumbent and Selected Correct Party.
  \item
    page=55, words=0-152, section=appendix, note=Placebo tests summary: district homophily does not significantly predict governor/senator outcomes.
  \end{itemize}
\end{itemize}

\subsubsection{D05: What placebo tests are run and what do they show?}\label{d05-what-placebo-tests-are-run-and-what-do-they-show}

\begin{itemize}
\tightlist
\item
  Category: \texttt{D)\ Results,\ magnitudes,\ heterogeneity,\ robustness}
\item
  Confidence: \texttt{low\ (0.18274002630391975)}
\item
  Retrieval method: \texttt{local}
\item
  Evidence type: \texttt{textual}
\item
  Data source: Cooperative Election Study (CES)
\item
  Table/Figure: Appendix Table C1
\item
  Assumption flag: \texttt{False}
\item
  Answer:
\end{itemize}

Placebo tests replicate the knowledge measures for statewide offices unaffected by congressional district borders: for the governor and each U.S. senator, indicators for heard of, selected party, and selected correct party (nine outcomes). These tests show no significant effect of district homophily on any of the placebo outcomes.

\begin{itemize}
\tightlist
\item
  Quote snippet:
\end{itemize}

\begin{quote}
I do not find evidence that district homophily increases voters' knowledge on placebo outcomes (i.e., the same three outcomes but for the respondent's governor and senators).
\end{quote}

\begin{itemize}
\tightlist
\item
  Citation anchors:

  \begin{itemize}
  \tightlist
  \item
    page=26, words=0-170, note=Describes placebo outcomes for governor and senators and states no significant impact.
  \item
    page=55, words=0-152, section=appendix, note=Appendix summary reiterating that district homophily does not significantly predict placebo outcomes.
  \item
    page=20, words=0-120, note=Mentions lack of evidence for impacts on placebo outcomes (same three outcomes for governor and senators).
  \end{itemize}
\end{itemize}

\subsubsection{D06: What falsification outcomes are tested (unaffected outcomes)?}\label{d06-what-falsification-outcomes-are-tested-unaffected-outcomes}

\begin{itemize}
\tightlist
\item
  Category: \texttt{D)\ Results,\ magnitudes,\ heterogeneity,\ robustness}
\item
  Confidence: \texttt{low\ (0.275338042494289)}
\item
  Retrieval method: \texttt{local}
\item
  Evidence type: \texttt{textual\ evidence}
\item
  Data source: Cooperative Election Study (CES)
\item
  Table/Figure: Table C1: CES Data: Summary Statistics for Placebo Outcomes
\item
  Answer:
\end{itemize}

Placebo (falsification) outcomes are voters' knowledge measures for statewide offices: for the governor and for each of the two U.S. senators---whether the respondent has heard of them, selects a party, and selects the correct party (nine outcomes total).

\begin{itemize}
\tightlist
\item
  Quote snippet:
\end{itemize}

\begin{quote}
I test whether district homophily impacts voters' knowledge of their governor and senators... I find no significant impact of district homophily on these nine outcomes.
\end{quote}

\begin{itemize}
\tightlist
\item
  Citation anchors:

  \begin{itemize}
  \tightlist
  \item
    page=26, words=0-220, note=Defines placebo tests: governor and senators; outcomes are heard of, select party, select correct party; reports no significant impact.
  \item
    page=55, words=0-120, section=appendix, note=Appendix C.2 summary and Table C1 list the nine placebo outcomes and note no significant prediction by district homophily.
  \item
    page=20, words=120-260, note=States they do not find evidence that district homophily increases knowledge on placebo outcomes for governor and senators.
  \item
    page=16, words=170-320, note=CES asks about House representative, both senators, and governor---basis for constructing analogous placebo outcomes.
  \end{itemize}
\end{itemize}

\subsubsection{D07: What heterogeneity results are reported (by income, size, baseline exposure, region)?}\label{d07-what-heterogeneity-results-are-reported-by-income-size-baseline-exposure-region}

\begin{itemize}
\tightlist
\item
  Category: \texttt{D)\ Results,\ magnitudes,\ heterogeneity,\ robustness}
\item
  Confidence: \texttt{medium\ (0.37521798394600775)}
\item
  Retrieval method: \texttt{local}
\item
  Evidence type: \texttt{textual}
\item
  Assumption flag: \texttt{False}
\item
  Answer:
\end{itemize}

Reported heterogeneity pertains to district homophily levels rather than treatment-effect splits: (1) Income: With media-market and district fixed effects, higher white non-Hispanic share correlates positively with changes in homophily, while higher poverty share correlates negatively. (2) Size: Larger counties (by population) have lower homophily---each 1\% population increase predicts a 0.05pp decrease; urban areas are more often split. (3) Baseline exposure: Homophily is higher where residents are farther from district borders/areas with geographically larger districts; authors also control for social-network partisanship exposure in outcome models. (4) Region: Counties in single-district states have higher average homophily (53\%), consistent with networks following state boundaries; rural areas show higher homophily than urban areas.

\begin{itemize}
\tightlist
\item
  Quote snippet:
\end{itemize}

\begin{quote}
Counties in single district states have higher district homophily on average (53\%)\ldots{} a one percent increase in county population is associated with a 0.05pp decrease in district homophily.
\end{quote}

\begin{itemize}
\tightlist
\item
  Citation anchors:

  \begin{itemize}
  \tightlist
  \item
    page=13, words=0-349, note=Size, urban/rural patterns; single-district states; population-population effect on homophily; distance/border logic.
  \item
    page=14, words=275-429, note=Income and demographic correlations with changes in district homophily; controls used.
  \item
    page=19, words=0-200, note=Models include control for partisanship of the social network (baseline exposure control).
  \end{itemize}
\end{itemize}

\subsubsection{D08: What mechanism tests are performed and what do they imply?}\label{d08-what-mechanism-tests-are-performed-and-what-do-they-imply}

\begin{itemize}
\tightlist
\item
  Category: \texttt{D)\ Results,\ magnitudes,\ heterogeneity,\ robustness}
\item
  Confidence: \texttt{low\ (0.1494383940132082)}
\item
  Retrieval method: \texttt{local}
\item
  Evidence type: \texttt{Textual\ summary\ of\ figure/caption\ results\ and\ appendix\ placebo\ tests}
\item
  Data source: CES (roll-off); county donations to House candidates (contribution shares)
\item
  Table/Figure: Figure 6 and Figure 8
\item
  Assumption flag: \texttt{False}
\item
  Answer:
\end{itemize}

Two mechanism tests are reported: (1) Ballot roll-off: higher district homophily reduces House roll-off (e.g., a 10pp increase cuts roll-off by about 0.04pp, ≈10\%), indicating greater down-ballot participation. (2) Campaign contributions: a 10pp increase in district homophily raises the share of county donations going to in-district House candidates by 7.4pp (from a 51\% mean) with no change in total House donations, implying a reallocation toward local candidates rather than more overall giving. Placebo tests for governors and senators show no significant effects, consistent with a district-specific information/salience mechanism.

\begin{itemize}
\tightlist
\item
  Quote snippet:
\end{itemize}

\begin{quote}
district homophily reduces roll-off; a 10pp increase raises the in-district contribution share by 7.4pp, with no impact on total donations.
\end{quote}

\begin{itemize}
\tightlist
\item
  Citation anchors:

  \begin{itemize}
  \tightlist
  \item
    page=24, words=0-229, note=Roll-off reduction and in-district contribution share increase; no change in total donations
  \item
    page=55, words=0-152, section=appendix, note=Placebo outcomes show no significant effects for governor/senators
  \end{itemize}
\end{itemize}

\subsubsection{D09: How sensitive are results to alternative samples/bandwidths/controls?}\label{d09-how-sensitive-are-results-to-alternative-samplesbandwidthscontrols}

\begin{itemize}
\tightlist
\item
  Category: \texttt{D)\ Results,\ magnitudes,\ heterogeneity,\ robustness}
\item
  Confidence: \texttt{low\ (0.3485816104887248)}
\item
  Retrieval method: \texttt{local}
\item
  Evidence type: \texttt{Appendix\ table\ and\ narrative\ robustness\ text}
\item
  Data source: CES; Nielsen DMA boundaries; 2016 5-Year ACS County-County Commuting Flows; Placebo tests (Appendix C.2)
\item
  Table/Figure: Appendix Table C2
\item
  Assumption flag: \texttt{True}
\item
  Assumption notes: Bandwidth sensitivity (if any) is not discussed in the provided excerpts.
\item
  Answer:
\end{itemize}

The main findings are robust across alternative samples and control sets. Estimates are similar with county fixed effects and when including odd years (though odd-year coefficients are noisier). Results remain after adding district-by-year and DMA-by-year fixed effects and controlling for Democratic exposure, demographics, and county-year covariates; the `heard of incumbent' and `correct party' effects stay positive and significant, while the `selected party' (any party) effect attenuates and loses precision in the within-border-pairs design. Placebo tests for governors/senators show no effects, and using commuting flows as an alternative network measure yields qualitatively similar effects at about half the magnitude. Effects appear quickly post-redistricting and are relatively stable over time.

\begin{itemize}
\tightlist
\item
  Quote snippet:
\end{itemize}

\begin{quote}
Impacts are relatively stable over time.
\end{quote}

\begin{itemize}
\tightlist
\item
  Citation anchors:

  \begin{itemize}
  \tightlist
  \item
    page=19, words=275-406, note=Similar results with county fixed effects; odd years yield similar results but noisier estimates.
  \item
    page=15, words=275-450, note=Controls: district-by-year FEs, DMA-by-year FEs, and Democratic exposure.
  \item
    page=65, words=0-288, section=appendix, note=Table C2 shows robustness across added controls; `Heard of Incumbent' and `Selected Correct Party' remain significant; `Selected Party' attenuates.
  \item
    page=26, words=0-349, note=Placebo outcomes show no significant impacts; commuting flows robustness with similar direction, \textasciitilde half the magnitude.
  \item
    page=55, words=0-152, section=appendix, note=Placebo tests for governor/senators: generally no significant prediction by district homophily.
  \item
    page=20, words=0-349, note=Effects emerge in first post-redistricting survey and are stable over time.
  \end{itemize}
\end{itemize}

\subsubsection{D10: What are the main takeaways in the conclusion (bullet summary)?}\label{d10-what-are-the-main-takeaways-in-the-conclusion-bullet-summary}

\begin{itemize}
\item
  Category: \texttt{D)\ Results,\ magnitudes,\ heterogeneity,\ robustness}
\item
  Confidence: \texttt{low\ (0.2241474231421388)}
\item
  Retrieval method: \texttt{local}
\item
  Evidence type: \texttt{textual}
\item
  Data source: Social Connectedness Index (SCI)
\item
  Assumption flag: \texttt{False}
\item
  Answer:
\item
  Evidence points toward effects concentrating support towards in-district House candidates and away from out-of-district candidates.
\item
  Calls for future research on how candidates adjust campaign strategies to district homophily, aiming for a general-equilibrium view of social-learning effects from district boundaries.
\item
  Policy implication: newly available social network data (e.g., SCI) can help draw fairer districts but could also be exploited for partisan gerrymandering.
\item
  Quote snippet:
\end{itemize}

\begin{quote}
towards House candidates in their own districts, away from House candidates running in other districts... enabling its use by policymakers to draw fairer districts, but also by partisan gerrymanderers
\end{quote}

\begin{itemize}
\tightlist
\item
  Citation anchors:

  \begin{itemize}
  \tightlist
  \item
    page=36, words=0-309, section=Conclusion/Implications, note=Concluding implications about in-district tilt, research agenda, and policy risks/uses of SCI.
  \end{itemize}
\end{itemize}

\subsubsection{E01: What are the most important prior papers cited and why are they central here?}\label{e01-what-are-the-most-important-prior-papers-cited-and-why-are-they-central-here}

\begin{itemize}
\tightlist
\item
  Category: \texttt{E)\ Citations\ and\ related\ literature}
\item
  Confidence: \texttt{low\ (0.33105038096722883)}
\item
  Retrieval method: \texttt{local}
\item
  Evidence type: \texttt{textual\ synthesis\ from\ introduction,\ literature\ review,\ methods,\ and\ references}
\item
  Assumption flag: \texttt{False}
\item
  Answer:
\end{itemize}

Key prior work falls into three pillars that the paper builds on: (1) Redistricting fairness measures that largely ignore turnout/network effects---Stephanopoulos \& McGhee (2015) and McCartan \& Imai (2023)---which the paper critiques by incorporating social-network-driven information and turnout responses; (2) Social learning and political networks showing information spreads through ties and calling for causal estimates---foundationally Lazarsfeld et al. (1944); evidence on diffusion in networks (Conley \& Udry 2010; Banerjee et al. 2013; Beaman et al. 2021); and explicit calls for causal work (Fowler et al. 2011)---motivating the paper's causal designs and its focus on district homophily; (3) Empirical and experimental studies of peer influence and information transmission---lab/field evidence (Klar \& Shmargad 2017; Druckman et al. 2018; Fafchamps et al. 2019; Arias et al. 2019) and turnout contagion/peer effects (Gerber et al. 2008; Sinclair et al. 2012; Klofstad 2007; Nickerson 2008; Pons 2018), plus place effects (Cantoni \& Pons 2022; Brown et al. 2023)---which ground the mechanisms linking networks to knowledge and participation. Methodologically, the paper's border-pairs identification follows Spenkuch \& Toniatti (2018). It also leverages national social-tie data in the spirit of Alt et al. (2022) and demonstrates large-scale social influence akin to Bond et al. (2012). Conceptually, its diffusion model aligns with network diffusion under homophily (Jackson \& López-Pintado 2013) and segregation measures (Echenique \& Fryer 2007). Finally, the knowledge--turnout link (Snyder \& Strömberg 2010) underpins why increased information from higher district homophily can shift participation.

\begin{itemize}
\tightlist
\item
  Quote snippet:
\end{itemize}

\begin{quote}
I build on this work by highlighting the role of social networks as a key source of information for voters.
\end{quote}

\begin{itemize}
\tightlist
\item
  Citation anchors:

  \begin{itemize}
  \tightlist
  \item
    page=2, words=0-349, note=Frames critique of redistricting fairness measures; positions social networks and knowledge-turnout link.
  \item
    page=4, words=0-349, note=Literature on social networks, calls for causal estimates, media avoidance; experimental/peer-effects studies.
  \item
    page=4, words=275-492, note=Use of national social-tie data; links to Alt et al. (2022) and Bond et al. (2012).
  \item
    page=28, words=0-349, note=Border-pairs design referencing Spenkuch \& Toniatti (2018) and conceptual diffusion framework.
  \item
    page=37, words=0-331, note=References for Bond (2012), Cantoni \& Pons (2022), Brown et al. (2023), and other peer-effects works.
  \item
    page=38, words=0-326, note=References including Echenique \& Fryer (2007) and several diffusion/political info studies.
  \item
    page=39, words=0-327, note=References including Jackson \& López-Pintado (2013) and McCartan \& Imai (2023).
  \item
    page=40, words=0-303, note=References including Snyder \& Strömberg (2010) and additional turnout/peer influence works.
  \end{itemize}
\end{itemize}

\subsubsection{E02: Which papers does this work most directly build on or extend?}\label{e02-which-papers-does-this-work-most-directly-build-on-or-extend}

\begin{itemize}
\tightlist
\item
  Category: \texttt{E)\ Citations\ and\ related\ literature}
\item
  Confidence: \texttt{low\ (0.30810576278123053)}
\item
  Retrieval method: \texttt{local}
\item
  Evidence type: \texttt{textual\ citations\ from\ the\ paper’s\ framing}
\item
  Assumption flag: \texttt{False}
\item
  Answer:
\end{itemize}

Most directly, it builds on work using large-scale social network data to study political behavior and diffusion (Alt et al. 2022; Bond et al. 2012) and on place/peer-effects studies (Cantoni \& Pons 2022; Brown et al. 2023). It also extends strategic gerrymandering models by incorporating social learning effects (Owen \& Grofman 1988; Friedman \& Holden 2008, 2020; Gul \& Pesendorfer 2010; Kolotilin \& Wolitzky 2020; Bouton et al. 2023).

\begin{itemize}
\tightlist
\item
  Quote snippet:
\end{itemize}

\begin{quote}
I build on these studies by estimating how social network structure impacts voter knowledge\ldots{} Second, I bridge the literature on social learning with the literature on models of political geography
\end{quote}

\begin{itemize}
\tightlist
\item
  Citation anchors:

  \begin{itemize}
  \tightlist
  \item
    page=4, words=275-492, section=Main text, note=Author states they build on prior studies, citing Alt et al. (2022), Bond et al. (2012), and positioning relative to Cantoni \& Pons (2022) and Brown et al. (2023).
  \item
    page=5, words=0-349, section=Main text, note=Author bridges social learning with political geography/redistricting models and lists canonical gerrymandering papers.
  \item
    page=5, words=275-475, section=Main text, note=Notes implications once social learning is considered and references broader gerrymandering measurement work.
  \end{itemize}
\end{itemize}

\subsubsection{E03: Which papers are used as benchmarks or comparisons in the results?}\label{e03-which-papers-are-used-as-benchmarks-or-comparisons-in-the-results}

\begin{itemize}
\tightlist
\item
  Category: \texttt{E)\ Citations\ and\ related\ literature}
\item
  Confidence: \texttt{low\ (0.2745280798867526)}
\item
  Retrieval method: \texttt{local}
\item
  Evidence type: \texttt{textual\ citation}
\item
  Data source: McCartan et al. (2021) 50-State Redistricting Simulations
\item
  Assumption flag: \texttt{False}
\item
  Answer:
\end{itemize}

The results benchmark against the 50-State Redistricting Simulations by McCartan, Kenny, Simko, Kuriwaki, et al. (2021), comparing features of actual maps to the distribution across 5,000 simulated maps per state.

\begin{itemize}
\tightlist
\item
  Quote snippet:
\end{itemize}

\begin{quote}
McCartan, Kenny, Simko, Kuriwaki, et al. 2021 simulate 5,000 congressional district maps for each of the 50 states... enabling... comparing a feature of interest of a given map against the distribution
\end{quote}

\begin{itemize}
\tightlist
\item
  Citation anchors:

  \begin{itemize}
  \tightlist
  \item
    page=34, words=275-379, section=main text, note=Describes using McCartan et al. (2021) simulated maps to compare features against a distribution.
  \item
    page=39, words=0-327, section=references, note=Bibliographic entry for McCartan et al. (2021) 50-State Redistricting Simulations.
  \end{itemize}
\end{itemize}

\subsubsection{E04: What data sources or datasets are cited and how are they used?}\label{e04-what-data-sources-or-datasets-are-cited-and-how-are-they-used}

\begin{itemize}
\tightlist
\item
  Category: \texttt{E)\ Citations\ and\ related\ literature}
\item
  Confidence: \texttt{low\ (0.3308023899105018)}
\item
  Retrieval method: \texttt{local}
\item
  Evidence type: \texttt{direct\ textual\ evidence\ from\ methods/data\ descriptions}
\item
  Data source: Facebook Social Connectedness Index (Oct 2021, county--county and zip--zip); CES (2006--2022); Commuting flows; Dave Leip's Atlas of U.S. Presidential Elections; Pew Research Center usage surveys
\item
  Assumption flag: \texttt{False}
\item
  Answer:
\end{itemize}

Cited datasets and uses: (1) Facebook Social Connectedness Index (SCI): used as the main proxy for real-world social networks to construct district homophily, primarily with U.S. county--county pairs from the Oct 2021 snapshot; also constructed at the zip code (ZCTA) level for robustness. (2) Commuting flows: used as an alternative proxy to construct district homophily; strongly correlated with SCI-based measures and yields qualitatively similar results. (3) Cooperative Election Study (CES): nationally representative surveys (2006--2022) used to measure voters' information about representatives and self-reported voting/preferences; CES county and district identifiers enable linkage to homophily; CES zip codes allow zip-level robustness checks. (4) Dave Leip's Atlas of U.S. Presidential Elections: used for vote count data to study actual voting behavior. (5) Pew Research Center surveys on Facebook usage (Auxier \& Anderson 2021; Vogels et al. 2021): cited to document that Facebook usage rates are relatively even across demographic groups and parties, supporting SCI's suitability.

\begin{itemize}
\tightlist
\item
  Quote snippet:
\end{itemize}

\begin{quote}
I use the SCI for U.S. county-county pairs from the October 2021 snapshot.
\end{quote}

\begin{itemize}
\tightlist
\item
  Citation anchors:

  \begin{itemize}
  \tightlist
  \item
    page=7, words=275-379, section=2.1.2 Proxy for Social Networks: Facebook Social Connectedness Index, note=Introduces SCI as proxy for social networks aggregated from Facebook friendships.
  \item
    page=8, words=0-349, section=2.1.2 Proxy for Social Networks: Facebook Social Connectedness Index, note=Defines SCI, states use of Oct 2021 county--county pairs; mentions commuting flows as alternative and correlation.
  \item
    page=8, words=275-624, section=2.1.2 Proxy for Social Networks: Facebook Social Connectedness Index, note=Details SCI snapshot construction and cites Pew evidence on even Facebook usage across demographics and parties.
  \item
    page=16, words=0-349, section=3 Outcomes Data; 3.1 Voters' Information, note=Describes CES data, years, content, and linkage to county/district for voter knowledge outcomes.
  \item
    page=17, words=275-542, section=Outcomes and vote data linkage, note=Cites Dave Leip's Atlas; explains use of CES pre/post election surveys for voting outcomes.
  \item
    page=27, words=0-349, section=5.3 Zip-Code-Level Social Network Data, note=Explains construction of district homophily using SCI at the zip (ZCTA) level and use of CES zip codes.
  \end{itemize}
\end{itemize}

\subsubsection{E05: What methodological or econometric references are cited (e.g., DiD, IV, RDD methods)?}\label{e05-what-methodological-or-econometric-references-are-cited-eg-did-iv-rdd-methods}

\begin{itemize}
\tightlist
\item
  Category: \texttt{E)\ Citations\ and\ related\ literature}
\item
  Confidence: \texttt{medium\ (0.3997880642088801)}
\item
  Retrieval method: \texttt{local}
\item
  Evidence type: \texttt{references/citations}
\item
  Assumption flag: \texttt{False}
\item
  Answer:
\end{itemize}

Cited methodological/econometric references include: Gourieroux, Monfort, \& Renault (1993) on indirect inference; Sinclair, McConnell, \& Green (2012) on detecting spillover effects in multilevel experiments; McCartan \& Imai (2023) on sequential Monte Carlo sampling for redistricting plans; McCartan et al. (2021) on 50-state redistricting simulations; McCormick, Salganik, \& Zheng (2010) on estimating personal network size; Echenique \& Fryer (2007) on a segregation measure based on social interactions; Snyder \& Strömberg (2010) on press coverage and political accountability (used for identification strategies); and Spenkuch \& Toniatti (2018) on political advertising and election results (informing a border-pairs/border-discontinuity design).

\begin{itemize}
\tightlist
\item
  Quote snippet:
\end{itemize}

\begin{quote}
Gourieroux, C., Monfort, A., \& Renault, E. (1993). Indirect inference. Journal of Applied Econometrics, 8 (S1), S85--S118.
\end{quote}

\begin{itemize}
\tightlist
\item
  Citation anchors:

  \begin{itemize}
  \tightlist
  \item
    page=38, words=0-326, section=References, note=Gourieroux, Monfort, \& Renault (1993). Indirect inference (econometric method).
  \item
    page=40, words=0-303, section=References, note=Sinclair, McConnell, \& Green (2012). Detecting Spillover Effects: design and analysis of multilevel experiments.
  \item
    page=39, words=0-327, section=References, note=McCartan \& Imai (2023). Sequential Monte Carlo for sampling balanced and compact redistricting plans.
  \item
    page=39, words=0-327, section=References, note=McCartan et al. (2021). 50-State Redistricting Simulations (simulation methodology).
  \item
    page=40, words=0-303, section=References, note=McCormick, Salganik, \& Zheng (2010). Estimating personal network size (network measurement method).
  \item
    page=38, words=0-326, section=References, note=Echenique \& Fryer (2007). A Measure of Segregation Based on Social Interactions (measurement/statistic).
  \item
    page=40, words=0-303, section=References, note=Snyder \& Strömberg (2010). Press Coverage and Political Accountability (identification using media-market congruence).
  \item
    page=28, words=0-349, section=Border pairs specification, note=Explicit use of Spenkuch \& Toniatti (2018) to justify border-pairs design.
  \item
    page=40, words=0-303, section=References, note=Spenkuch \& Toniatti (2018). Political Advertising and Election Results* (empirical design at media-market borders).
  \end{itemize}
\end{itemize}

\subsubsection{E06: Are there any seminal or classic references the paper positions itself against?}\label{e06-are-there-any-seminal-or-classic-references-the-paper-positions-itself-against}

\begin{itemize}
\tightlist
\item
  Category: \texttt{E)\ Citations\ and\ related\ literature}
\item
  Confidence: \texttt{low\ (0.27354932674976956)}
\item
  Retrieval method: \texttt{local}
\item
  Evidence type: \texttt{text}
\item
  Assumption flag: \texttt{False}
\item
  Answer:
\end{itemize}

Yes. The paper challenges classic strategic gerrymandering models and measures that assume voter distributions and decisions are independent of district maps, citing Owen \& Grofman (1988), Friedman \& Holden (2008, 2020), Gul \& Pesendorfer (2010), Kolotilin \& Wolitzky (2020), and Efficiency Gap work (Stephanopoulos \& McGhee, 2015).

\begin{itemize}
\tightlist
\item
  Quote snippet:
\end{itemize}

\begin{quote}
Existing models of strategic partisan redistricting \ldots{} assume that changes to district boundaries do not affect the distribution of partisans\ldots{} Voters' decisions are independent of the district map.
\end{quote}

\begin{itemize}
\tightlist
\item
  Citation anchors:

  \begin{itemize}
  \tightlist
  \item
    page=5, words=0-349, section=main text, note=States existing redistricting models assume voter distributions and decisions are independent of the map; lists classic references and contrasts with findings.
  \end{itemize}
\end{itemize}

\subsubsection{E07: Are there citations to code, data repositories, or appendices that are essential to the claims?}\label{e07-are-there-citations-to-code-data-repositories-or-appendices-that-are-essential-to-the-claims}

\begin{itemize}
\tightlist
\item
  Category: \texttt{E)\ Citations\ and\ related\ literature}
\item
  Confidence: \texttt{low\ (0.3061721700291295)}
\item
  Retrieval method: \texttt{local}
\item
  Evidence type: \texttt{appendix\ and\ data\ repository\ citations}
\item
  Data source: Appendix B; Dave Leip's Election Atlas; Harvard Election Data Archive; MIT Election Data and Science Lab; Daily Kos; U.S. Census; CES datasets (DOIs in references).
\item
  Table/Figure: Table B4: Descriptions for Voting Outcome Variables
\item
  Assumption flag: \texttt{False}
\item
  Answer:
\end{itemize}

Yes. The paper cites essential appendices (Appendix B) detailing data construction and variables, and references multiple data repositories used in the analysis (Dave Leip's Election Atlas, Harvard Election Data Archive, MIT Election Data and Science Lab, Daily Kos, Census, and CES datasets with DOIs). No explicit code repository is cited in the provided text.

\begin{itemize}
\tightlist
\item
  Quote snippet:
\end{itemize}

\begin{quote}
combined with county-by-congressional-district vote count data from Dave Leip's Election Atlas (for House elections) and Daily Kos (for President, Senator, and Governor elections).
\end{quote}

\begin{itemize}
\tightlist
\item
  Citation anchors:

  \begin{itemize}
  \tightlist
  \item
    page=47, words=0-207, section=appendix, note=Appendix B describes construction of vote count measures using Harvard Election Data Archive, MIT Election Data and Science Lab, Dave Leip's Atlas, and Daily Kos.
  \item
    page=49, words=0-276, section=appendix, note=Appendix B tables and variable descriptions cite data sources for roll-off and turnout (Dave Leip's Atlas, Harvard Election Data Archive, Daily Kos, MIT Election Data and Science Lab, Census).
  \item
    page=17, words=275-542, note=Main text cites Dave Leip's Election Atlas and CES survey data; references CES methodology and exclusions.
  \item
    page=40, words=0-303, section=references, note=References include CES 2020 and 2022 datasets with DOIs (data repository citations).
  \item
    page=39, words=0-327, section=references, note=References include Kuriwaki (2018) Cumulative CES Common Content (Dataverse DOI) and 50-State Redistricting Simulations (data resource DOI).
  \end{itemize}
\end{itemize}

\subsubsection{E08: What gaps in the literature do the authors say these citations leave open?}\label{e08-what-gaps-in-the-literature-do-the-authors-say-these-citations-leave-open}

\begin{itemize}
\tightlist
\item
  Category: \texttt{E)\ Citations\ and\ related\ literature}
\item
  Confidence: \texttt{low\ (0.30376113152112244)}
\item
  Retrieval method: \texttt{local}
\item
  Evidence type: \texttt{Direct\ textual\ evidence\ from\ introduction\ and\ literature\ review}
\item
  Data source: Provided context excerpts (pages 2, 4--5)
\item
  Assumption flag: \texttt{False}
\item
  Answer:
\end{itemize}

They highlight three main gaps: (1) a lack of causal estimates on how social networks affect political knowledge, especially the causal role of district boundaries aligning (or not) with social ties; (2) existing peer-effects and mobilization studies often focus on turnout or persuasion and many are lab-based or outside the U.S., leaving information transmission within networks and nationwide U.S. networks underexplored; and (3) gerrymandering models and fairness metrics assume voter distributions and turnout are invariant to district maps, overlooking social learning and the possibility that borders change participation by grouping social ties.

\begin{itemize}
\tightlist
\item
  Quote snippet:
\end{itemize}

\begin{quote}
the role of political boundaries---which group some friends together while separating others---in social learning has not been causally explored.
\end{quote}

\begin{itemize}
\tightlist
\item
  Citation anchors:

  \begin{itemize}
  \tightlist
  \item
    page=2, words=0-349, section=Introduction, note=States causal estimates of network effects on political knowledge are limited and the role of political boundaries in social learning has not been causally explored.
  \item
    page=4, words=0-349, section=Literature review, note=Notes calls for more causal estimates in social-network political information literature; discusses lab and field studies.
  \item
    page=4, words=275-492, section=Literature review, note=Says related studies focus on turnout/peer effects and not directly on information transmission; emphasizes leveraging national U.S. social ties.
  \item
    page=5, words=0-349, section=Contribution/Models of gerrymandering, note=Argues existing gerrymandering models and measures assume fixed partisan distributions and ignore social networks' effects on turnout.
  \end{itemize}
\end{itemize}

\subsubsection{F01: Are replication files or code provided? If so, where?}\label{f01-are-replication-files-or-code-provided-if-so-where}

\begin{itemize}
\tightlist
\item
  Category: \texttt{F)\ Replication\ and\ transparency}
\item
  Confidence: \texttt{low\ (0.21170377402826074)}
\item
  Retrieval method: \texttt{local}
\item
  Evidence type: \texttt{textual}
\item
  Assumption flag: \texttt{False}
\item
  Answer:
\end{itemize}

No. The provided context does not mention any replication files or code or where to access them.

\begin{itemize}
\tightlist
\item
  Quote snippet:
\end{itemize}

\begin{quote}
B Data Descriptions
\end{quote}

\begin{itemize}
\tightlist
\item
  Citation anchors:

  \begin{itemize}
  \tightlist
  \item
    page=47, words=0-25, section=appendix, note=Appendix lists data and variable descriptions; no replication files or code referenced.
  \item
    page=1, words=372-388, note=Acknowledgments section ends without mentioning replication files or code.
  \end{itemize}
\end{itemize}

\subsubsection{F02: Is there a pre-analysis plan or registered trial? Provide details if mentioned.}\label{f02-is-there-a-pre-analysis-plan-or-registered-trial-provide-details-if-mentioned}

\begin{itemize}
\tightlist
\item
  Category: \texttt{F)\ Replication\ and\ transparency}
\item
  Confidence: \texttt{low\ (0.25252882149704253)}
\item
  Retrieval method: \texttt{local}
\item
  Evidence type: \texttt{textual\ (no\ mention\ in\ provided\ excerpts)}
\item
  Answer:
\end{itemize}

No. The provided excerpts do not mention a pre-analysis plan or any trial/registry; no registration details are given.

\begin{itemize}
\tightlist
\item
  Quote snippet:
\end{itemize}

\begin{quote}
Focusing on a single redistricting event allows me to avoid concerns related to staggered treatment events, and also allows for a visual test of pre-trends in changes in district homophily.
\end{quote}

\begin{itemize}
\tightlist
\item
  Citation anchors:

  \begin{itemize}
  \tightlist
  \item
    page=15, words=0-349, note=Methods and event-study setup discussed; no pre-analysis plan or registration mentioned.
  \item
    page=48, words=0-333, section=appendix, note=Appendix defines variables and outcomes; no pre-registration details provided.
  \end{itemize}
\end{itemize}

\subsubsection{F03: Are data access constraints disclosed (restricted access, proprietary data, NDAs)?}\label{f03-are-data-access-constraints-disclosed-restricted-access-proprietary-data-ndas}

\begin{itemize}
\tightlist
\item
  Category: \texttt{F)\ Replication\ and\ transparency}
\item
  Confidence: \texttt{low\ (0.2639892199729601)}
\item
  Retrieval method: \texttt{local}
\item
  Evidence type: \texttt{textual\ evidence\ (privacy-related\ data\ limitations)}
\item
  Data source: Meta Social Connectedness Index (SCI)
\item
  Assumption flag: \texttt{True}
\item
  Assumption notes: Conclusion based only on provided excerpts; other parts of the document may discuss access terms not shown here.
\item
  Answer:
\end{itemize}

No explicit restricted-access, proprietary, or NDA requirements are disclosed. The text only notes privacy-related limitations to the SCI (small-population areas removed, noise added; some zip codes omitted for privacy).

\begin{itemize}
\tightlist
\item
  Quote snippet:
\end{itemize}

\begin{quote}
due to privacy concerns, zip codes with small populations do not appear in the SCI data.
\end{quote}

\begin{itemize}
\tightlist
\item
  Citation anchors:

  \begin{itemize}
  \tightlist
  \item
    page=8, words=275-624, note=SCI removes small-population areas and adds noise to preserve privacy.
  \item
    page=27, words=0-349, note=SCI omits some zip codes due to privacy concerns.
  \end{itemize}
\end{itemize}

\subsubsection{F04: Are key steps in data cleaning and construction documented?}\label{f04-are-key-steps-in-data-cleaning-and-construction-documented}

\begin{itemize}
\tightlist
\item
  Category: \texttt{F)\ Replication\ and\ transparency}
\item
  Confidence: \texttt{low\ (0.2328718636910073)}
\item
  Retrieval method: \texttt{local}
\item
  Evidence type: \texttt{methods\ and\ appendix\ text}
\item
  Data source: Harvard Election Data Archive; MIT Election Data and Science Lab; Dave Leip's Election Atlas; Daily Kos; Meta Social Connectedness Index
\item
  Table/Figure: Tables B1--B5; Figure 1 reference (methods text)
\item
  Assumption flag: \texttt{False}
\item
  Answer:
\end{itemize}

Yes. The document explicitly details how key datasets are constructed (e.g., district homophily from SCI; county-by-CD vote measures) and records key cleaning steps and exclusions for survey data and voting outcomes.

\begin{itemize}
\tightlist
\item
  Quote snippet:
\end{itemize}

\begin{quote}
I exclude missing responses to the candidate party recognition question (\textless2\% of respondents in each year; ...). Lastly, in the 2020 survey, 925 respondents in North Carolina were assigned to incorrect congressional districts...
\end{quote}

\begin{itemize}
\tightlist
\item
  Citation anchors:

  \begin{itemize}
  \tightlist
  \item
    page=9, words=0-257, note=Construction of district homophily from SCI with aggregation procedure and formulas.
  \item
    page=47, words=0-207, section=appendix, note=Construction of vote count measures and CES variable descriptions.
  \item
    page=48, words=0-333, section=appendix, note=Definitions of voting preference and vote-choice variables.
  \item
    page=49, words=0-276, section=appendix, note=Summary statistics and precise definitions for turnout and roll-off with data sources.
  \item
    page=17, words=275-542, note=Survey data cleaning: explicit exclusions for missing responses, demographic nonresponse, misassignments.
  \item
    page=8, words=275-624, note=SCI data construction choices (scaling, privacy noise, user inclusion, location assignment).
  \end{itemize}
\end{itemize}

\subsubsection{F05: Are robustness and sensitivity analyses fully reported or partially omitted?}\label{f05-are-robustness-and-sensitivity-analyses-fully-reported-or-partially-omitted}

\begin{itemize}
\tightlist
\item
  Category: \texttt{F)\ Replication\ and\ transparency}
\item
  Confidence: \texttt{low\ (0.271322823909785)}
\item
  Retrieval method: \texttt{local}
\item
  Evidence type: \texttt{narrative\ description\ in\ main\ text\ and\ appendix}
\item
  Table/Figure: Table C1 (Appendix C.2)
\item
  Assumption flag: \texttt{False}
\item
  Answer:
\end{itemize}

Fully reported.

\begin{itemize}
\tightlist
\item
  Quote snippet:
\end{itemize}

\begin{quote}
I explore the robustness of this finding by testing whether district homophily impacts placebo outcomes, by constructing an alternative measure of district homophily using commuting flows,
\end{quote}

\begin{itemize}
\tightlist
\item
  Citation anchors:

  \begin{itemize}
  \tightlist
  \item
    page=24, words=0-229, section=5 Robustness, note=Dedicated Robustness section outlines multiple checks (placebo outcomes, alternative homophily measure).
  \item
    page=55, words=0-152, section=appendix, note=Appendix C.2 reports placebo outcome tests and notes generally null effects, indicating reported robustness results.
  \end{itemize}
\end{itemize}

\subsubsection{G01: What populations or settings are most likely to generalize from this study?}\label{g01-what-populations-or-settings-are-most-likely-to-generalize-from-this-study}

\begin{itemize}
\tightlist
\item
  Category: \texttt{G)\ External\ validity\ and\ generalization}
\item
  Confidence: \texttt{low\ (0.2854722064950393)}
\item
  Retrieval method: \texttt{local}
\item
  Evidence type: \texttt{Study\ design\ and\ data\ coverage\ details}
\item
  Data source: Cooperative Election Study (CES); Meta Social Connectedness Index (SCI)
\item
  Assumption flag: \texttt{False}
\item
  Answer:
\end{itemize}

Adults in the continental United States, across counties and congressional districts, especially in U.S. House election and post‑redistricting settings. Generalizability is strongest to U.S. voters because the study links a nationally representative CES sample to counties/districts and uses nationwide social-network measures (SCI) that cover all 3,136 counties, with Facebook usage relatively even across most demographic and geographic groups. Findings also likely extend to settings where non--social-media proxies for ties (e.g., commuting flows) capture similar network structure.

\begin{itemize}
\tightlist
\item
  Quote snippet:
\end{itemize}

\begin{quote}
The CES is a nationally representative survey that has run annually from 2006 to 2022
\end{quote}

\begin{itemize}
\tightlist
\item
  Citation anchors:

  \begin{itemize}
  \tightlist
  \item
    page=4, words=275-492, section=main, note=Uses data from across the continental U.S.; leverages geographic variation in U.S. social networks; relevance to redistricting.
  \item
    page=8, words=275-624, section=main, note=SCI coverage (3,136 counties; nationwide pairs) and relatively even Facebook usage across demographic and geographic groups.
  \item
    page=16, words=0-349, section=main, note=CES is nationally representative; large samples; linked to county and congressional district.
  \item
    page=20, words=0-349, section=main, note=Similar results when using commuting flows as an alternative proxy for social networks.
  \end{itemize}
\end{itemize}

\subsubsection{G02: What populations or settings are least likely to generalize?}\label{g02-what-populations-or-settings-are-least-likely-to-generalize}

\begin{itemize}
\tightlist
\item
  Category: \texttt{G)\ External\ validity\ and\ generalization}
\item
  Confidence: \texttt{low\ (0.26154170789138104)}
\item
  Retrieval method: \texttt{local}
\item
  Evidence type: \texttt{Textual\ evidence\ from\ data/methods\ and\ appendix\ assumptions}
\item
  Data source: Meta Social Connectedness Index (SCI) and U.S. Facebook usage rates cited (Pew, 2021)
\item
  Assumption flag: \texttt{False}
\item
  Answer:
\end{itemize}

Least likely to generalize are: (1) geographies excluded from results (Alaska, Hawaii, Washington, D.C., and territories); (2) very small-population areas removed from the SCI; (3) groups with low Meta/Facebook use---especially older adults (65+); and (4) counties with highly fragmented or complex district splits that violate the limited-fragmentation/simple-boundaries assumptions.

\begin{itemize}
\tightlist
\item
  Quote snippet:
\end{itemize}

\begin{quote}
areas with particularly small populations are removed
\end{quote}

\begin{itemize}
\tightlist
\item
  Citation anchors:

  \begin{itemize}
  \tightlist
  \item
    page=13, words=275-555, note=Results exclude Alaska, Hawaii, Washington, D.C., and territories
  \item
    page=8, words=275-624, note=SCI removes very small-population areas; includes only users active in last 30 days
  \item
    page=8, words=275-624, note=Facebook usage lowest among 65+ (50\%)
  \item
    page=44, words=275-491, section=appendix, note=Local mapping assumes limited fragmentation/simple boundaries; some counties are split across many districts (rare)
  \end{itemize}
\end{itemize}

\subsubsection{G03: Do the authors discuss boundary conditions or scope limits?}\label{g03-do-the-authors-discuss-boundary-conditions-or-scope-limits}

\begin{itemize}
\tightlist
\item
  Category: \texttt{G)\ External\ validity\ and\ generalization}
\item
  Confidence: \texttt{low\ (0.2842624617857855)}
\item
  Retrieval method: \texttt{local}
\item
  Evidence type: \texttt{text}
\item
  Assumption flag: \texttt{True}
\item
  Assumption notes: Appendix imposes: (1) Limited fragmentation with a dominant district per county; (2) Simple boundaries (each county intersects ≤2 districts); (3) Non-trivial out-of-state connections (network leakage, spectral radius \textless{} 1); (4) Network non-degeneracy (linearly independent rows), ensuring local identification.
\item
  Answer:
\end{itemize}

Yes. They explicitly note scope limits (e.g., findings specific to the U.S.; analysis over the continental 48 states) and impose boundary/identification conditions in the appendix (limited fragmentation, simple boundaries, out-of-state leakage, and network non-degeneracy). They also restrict the border-pairs design to counties fully within one district and within-state pairs due to data/precision limits.

\begin{itemize}
\tightlist
\item
  Quote snippet:
\end{itemize}

\begin{quote}
Though my findings are specific to the U.S., this approach may hold relevance in any context where political boundaries are drawn.
\end{quote}

\begin{itemize}
\tightlist
\item
  Citation anchors:

  \begin{itemize}
  \tightlist
  \item
    page=page 5, words=275-475, note=Scope statement that findings are specific to the U.S.
  \item
    page=page 13, words=0-349, note=Scope over continental 48 states and determinants of district homophily
  \item
    page=page 28, words=0-349, note=Border-pairs design restrictions and precision limits
  \item
    page=page 44, words=275-491, section=appendix, note=Assumptions: Limited fragmentation and Simple boundaries
  \item
    page=page 45, words=0-349, section=appendix, note=Assumptions: Non-trivial out-of-state connections; Network non-degeneracy
  \item
    page=page 46, words=0-349, section=appendix, note=Identification result relying on assumptions (invertibility argument)
  \end{itemize}
\end{itemize}

\subsubsection{G04: How might the results change in different time periods or markets?}\label{g04-how-might-the-results-change-in-different-time-periods-or-markets}

\begin{itemize}
\tightlist
\item
  Category: \texttt{G)\ External\ validity\ and\ generalization}
\item
  Confidence: \texttt{low\ (0.31940451009186777)}
\item
  Retrieval method: \texttt{local}
\item
  Evidence type: \texttt{Event-study\ estimates\ over\ time;\ fixed-effects\ controls\ (DMA-by-year,\ district-by-year);\ border-pairs\ robustness}
\item
  Assumption flag: \texttt{False}
\item
  Answer:
\end{itemize}

Over time, effects appear quickly and persist: voter-knowledge impacts take hold in the first post-redistricting survey (2014) and remain relatively stable. Using odd-year data yields similar patterns but with noisier estimates due to much smaller samples. Across media markets, controlling for DMA-by-year fixed effects generally leaves results qualitatively similar---suggesting they are not driven by market-level media---but in the stricter border-pairs design the `Selected Party' effect becomes insignificant after adding DMA-by-year fixed effects. Moreover, correlations between county traits and changes in homophily largely vanish once media market and district fixed effects are included.

\begin{itemize}
\tightlist
\item
  Quote snippet:
\end{itemize}

\begin{quote}
most strongly takes effect in the first survey after redistricting (2014). Impacts are relatively stable over time.
\end{quote}

\begin{itemize}
\tightlist
\item
  Citation anchors:

  \begin{itemize}
  \tightlist
  \item
    page=20, words=0-349, section=Event studies timing, note=Impacts strongest in 2014 and relatively stable over time
  \item
    page=19, words=275-406, section=Sample timing, note=Odd years have smaller samples; similar results but noisier estimates
  \item
    page=15, words=275-450, section=Media market controls, note=Includes DMA-by-year fixed effects to address media market confounding
  \item
    page=28, words=0-349, section=Border-pairs robustness, note=`Selected Party' becomes insignificant after adding DMA-by-year fixed effects
  \item
    page=14, words=0-349, section=Predictors and fixed effects, note=County-level correlations disappear once media market and district FEs are included
  \end{itemize}
\end{itemize}

\subsubsection{H01: Are key variables measured directly or via proxies?}\label{h01-are-key-variables-measured-directly-or-via-proxies}

\begin{itemize}
\tightlist
\item
  Category: \texttt{H)\ Measurement\ validity}
\item
  Confidence: \texttt{low\ (0.3328462292312862)}
\item
  Retrieval method: \texttt{local}
\item
  Evidence type: \texttt{text}
\item
  Data source: Facebook Social Connectedness Index; CES surveys; Dave Leip's Election Atlas; Harvard Election Data Archive; MIT Election Data and Science Lab; Daily Kos
\item
  Assumption flag: \texttt{False}
\item
  Answer:
\end{itemize}

Both. Social networks are proxied using Facebook's Social Connectedness Index and a constructed partisan-exposure measure, while knowledge and voting outcomes are measured directly from CES surveys and administrative vote counts/validated rolls.

\begin{itemize}
\tightlist
\item
  Quote snippet:
\end{itemize}

\begin{quote}
I use the Facebook Social Connectedness Index (SCI)---one of the best existing proxies for real-world social networks.
\end{quote}

\begin{itemize}
\tightlist
\item
  Citation anchors:

  \begin{itemize}
  \tightlist
  \item
    page=7, words=275-379, note=SCI used as proxy for real-world social networks
  \item
    page=15, words=275-450, note=Constructed proxy for partisan exposure from friendships and vote shares
  \item
    page=48, words=0-333, section=appendix, note=Voting outcomes measured from pre/post CES surveys and validated rolls
  \item
    page=47, words=0-207, section=appendix, note=Vote counts built from Harvard EDA, MIT/MedSL, Dave Leip, Daily Kos
  \end{itemize}
\end{itemize}

\subsubsection{H02: What measurement error risks are acknowledged or likely?}\label{h02-what-measurement-error-risks-are-acknowledged-or-likely}

\begin{itemize}
\tightlist
\item
  Category: \texttt{H)\ Measurement\ validity}
\item
  Confidence: \texttt{low\ (0.27088332269027793)}
\item
  Retrieval method: \texttt{local}
\item
  Evidence type: \texttt{quoted\_and\_paraphrased}
\item
  Data source: Social Connectedness Index (SCI); CES; Dave Leip's Election Atlas; Harvard Election Data Archive; MIT Election Data and Science Lab; Daily Kos
\item
  Assumption flag: \texttt{True}
\item
  Assumption notes: Coverage heterogeneity in SCI by age and potential inconsistencies from merging multiple vote data sources are inferred risks based on the described data construction and usage-rate differences.
\item
  Answer:
\end{itemize}

Acknowledged and likely measurement error risks include: (1) SCI measurement noise and suppression: small-population areas are removed and noise is added; (2) SCI coverage/assignment issues: only active Meta users in last 30 days are included, usage varies by age, and locations are assigned via self-reported and device data; (3) CES response error: respondents may guess parties and lucky guesses cannot be ruled out; (4) Survey assignment errors: some respondents were assigned to incorrect districts (e.g., 2020 NC), though excluded; (5) Voting data quality/construct issues: turnout metrics exclude cases where turnout exceeds VAP; (6) Potential harmonization inconsistencies from combining multiple vote data sources.

\begin{itemize}
\tightlist
\item
  Quote snippet:
\end{itemize}

\begin{quote}
Lastly, the third dummy variable, ``Selected Correct Party'', is coded as 1 if the respondent selected the correct party for the incumbent and 0 otherwise. While lucky guesses cannot be ruled out
\end{quote}

\begin{itemize}
\tightlist
\item
  Citation anchors:

  \begin{itemize}
  \tightlist
  \item
    page=8, words=275-315, note=SCI adds privacy noise and removes small-population areas
  \item
    page=8, words=315-380, note=SCI only includes users active in last 30 days
  \item
    page=8, words=380-430, note=SCI location assignment based on self-reported info and device data
  \item
    page=8, words=520-624, note=Facebook usage rates vary by age (coverage heterogeneity)
  \item
    page=17, words=60-140, note=Selected Party may involve guessing
  \item
    page=17, words=140-220, note=Lucky guesses cannot be ruled out for Selected Correct Party
  \item
    page=17, words=420-520, note=2020 NC respondents assigned to incorrect districts; excluded
  \item
    page=49, words=180-230, section=appendix, note=Elections where turnout exceeds VAP are excluded
  \item
    page=47, words=0-120, section=appendix, note=Voting outcomes constructed by combining multiple data sources
  \end{itemize}
\end{itemize}

\subsubsection{H03: Are there validation checks for key measures?}\label{h03-are-there-validation-checks-for-key-measures}

\begin{itemize}
\tightlist
\item
  Category: \texttt{H)\ Measurement\ validity}
\item
  Confidence: \texttt{low\ (0.26064539994033026)}
\item
  Retrieval method: \texttt{local}
\item
  Evidence type: \texttt{Placebo\ tests\ and\ external\ stability\ checks}
\item
  Data source: CES; Social Connectedness Index (Meta apps); Dave Leip's Election Atlas
\item
  Table/Figure: Table C1
\item
  Assumption flag: \texttt{False}
\item
  Answer:
\end{itemize}

Yes. The study validates key measures via placebo tests for voter information (district homophily does not predict placebo outcomes), external stability/representativeness checks for the SCI (e.g., \textgreater0.99 year-to-year correlation and broadly even usage rates), and robustness noting similar results with county fixed effects.

\begin{itemize}
\tightlist
\item
  Quote snippet:
\end{itemize}

\begin{quote}
In general, district homophily does not significantly predict the placebo outcomes.
\end{quote}

\begin{itemize}
\tightlist
\item
  Citation anchors:

  \begin{itemize}
  \tightlist
  \item
    page=55, words=0-152, section=appendix, note=Placebo outcomes show no significant prediction by district homophily.
  \item
    page=8, words=275-624, note=SCI validation: \textgreater0.99 cross-year correlation; usage rates relatively even across demographics.
  \item
    page=19, words=275-406, note=Robustness: results similar with county fixed effects.
  \end{itemize}
\end{itemize}

\subsubsection{H04: Do the authors discuss construct validity for core outcomes?}\label{h04-do-the-authors-discuss-construct-validity-for-core-outcomes}

\begin{itemize}
\tightlist
\item
  Category: \texttt{H)\ Measurement\ validity}
\item
  Confidence: \texttt{low\ (0.22703125944891173)}
\item
  Retrieval method: \texttt{local}
\item
  Evidence type: \texttt{placebo\ tests\ and\ robustness\ checks}
\item
  Data source: Cooperative Election Study (CES)
\item
  Table/Figure: Table C1
\item
  Assumption flag: \texttt{False}
\item
  Answer:
\end{itemize}

Yes. They assess construct validity via placebo outcome tests---showing district homophily does not affect knowledge of statewide officials (governor, senators) where no effect is expected---and corroborate this with robustness using alternative network measures.

\begin{itemize}
\tightlist
\item
  Quote snippet:
\end{itemize}

\begin{quote}
I find no significant impact of district homophily on these nine outcomes.
\end{quote}

\begin{itemize}
\tightlist
\item
  Citation anchors:

  \begin{itemize}
  \tightlist
  \item
    page=26, words=0-349, note=Placebo outcomes for governor and senators; no significant impact reported.
  \item
    page=27, words=0-349, note=No significant impact on placebo outcomes when using commuting-flow-based homophily.
  \item
    page=55, words=0-152, section=appendix, note=Appendix C.2 summary: district homophily does not significantly predict placebo outcomes; Table C1.
  \end{itemize}
\end{itemize}

\subsubsection{I01: What policy counterfactuals are considered or implied?}\label{i01-what-policy-counterfactuals-are-considered-or-implied}

\begin{itemize}
\tightlist
\item
  Category: \texttt{I)\ Policy\ counterfactuals\ and\ welfare}
\item
  Confidence: \texttt{low\ (0.34996576379917405)}
\item
  Retrieval method: \texttt{local}
\item
  Evidence type: \texttt{textual}
\item
  Data source: McCartan et al. (2021) 50-State Redistricting Simulations
\item
  Assumption flag: \texttt{False}
\item
  Answer:
\end{itemize}

The implied policy counterfactuals are alternative congressional district maps: (1) simulating counterfactual district maps via the information-diffusion model to forecast voter information under different boundaries, and (2) comparing outcomes across many legally compliant simulated redistricting plans (5,000 per state).

\begin{itemize}
\tightlist
\item
  Quote snippet:
\end{itemize}

\begin{quote}
allow for the simulation of counterfactual district maps. \ldots{} simulate 5,000 congressional district maps for each of the 50 states\ldots{} constrained to follow the given state's redistricting laws.
\end{quote}

\begin{itemize}
\tightlist
\item
  Citation anchors:

  \begin{itemize}
  \tightlist
  \item
    page=34, words=150-215, section=Conceptual Framework: Information Diffusion within Districts, note=Model enables simulation of counterfactual district maps.
  \item
    page=34, words=250-349, section=Comparisons Across Many Simulated Maps, note=5,000 simulated congressional district maps per state, constrained by state laws.
  \end{itemize}
\end{itemize}

\subsubsection{I02: What are the main welfare tradeoffs or distributional impacts discussed?}\label{i02-what-are-the-main-welfare-tradeoffs-or-distributional-impacts-discussed}

\begin{itemize}
\tightlist
\item
  Category: \texttt{I)\ Policy\ counterfactuals\ and\ welfare}
\item
  Confidence: \texttt{low\ (0.30339278782441154)}
\item
  Retrieval method: \texttt{local}
\item
  Evidence type: \texttt{Textual\ evidence\ from\ results\ and\ discussion}
\item
  Data source: CES survey; campaign contributions data; Facebook friendship graph/SCI; commuting flows
\item
  Table/Figure: Figure 6 and Figure 8
\item
  Answer:
\end{itemize}

The paper highlights distributional impacts from how social networks align with district lines: (1) Higher district homophily raises voters' information about their House representative and reduces roll-off, concentrating these gains in places where networks and districts align; (2) It reallocates campaign money toward in-district candidates and away from out-of-district ones without changing total donations; (3) Using social-network data to draw districts could improve fairness but also risks partisan exploitation.

\begin{itemize}
\tightlist
\item
  Quote snippet:
\end{itemize}

\begin{quote}
enabling its use by policymakers to draw fairer districts, but also by partisan gerrymanderers who may seek to exploit it.
\end{quote}

\begin{itemize}
\tightlist
\item
  Citation anchors:

  \begin{itemize}
  \tightlist
  \item
    page=20, words=0-349, section=Section 4.1.1--4.1.2, note=Increases in district homophily raise voter knowledge; timing and magnitudes reported.
  \item
    page=24, words=0-229, section=Section 4.2, note=Roll-off reduced; donations shift toward in-district candidates with no change in totals.
  \item
    page=36, words=0-309, section=Conclusion/Implications, note=Policy tradeoff: social network data could aid fair maps but also be exploited by gerrymanderers.
  \end{itemize}
\end{itemize}

\subsubsection{I03: Are cost-benefit or incidence analyses provided?}\label{i03-are-cost-benefit-or-incidence-analyses-provided}

\begin{itemize}
\tightlist
\item
  Category: \texttt{I)\ Policy\ counterfactuals\ and\ welfare}
\item
  Confidence: \texttt{low\ (0.24479389793772252)}
\item
  Retrieval method: \texttt{local}
\item
  Evidence type: \texttt{textual\ review\ of\ provided\ excerpts\ (absence\ of\ mention)}
\item
  Table/Figure: Table B4; Table C1
\item
  Assumption flag: \texttt{True}
\item
  Assumption notes: Conclusion is limited to the provided excerpts; other parts of the full document may exist but are not included here.
\item
  Answer:
\end{itemize}

No. The provided excerpts contain event studies, placebo tests, and variable descriptions but do not present cost-benefit or incidence analyses.

\begin{itemize}
\tightlist
\item
  Quote snippet:
\end{itemize}

\begin{quote}
In general, district homophily does not significantly predict the placebo outcomes.
\end{quote}

\begin{itemize}
\tightlist
\item
  Citation anchors:

  \begin{itemize}
  \tightlist
  \item
    page=55, words=0-152, section=appendix, note=Describes placebo outcomes; no cost-benefit or incidence analysis indicated.
  \item
    page=49, words=0-276, section=appendix, note=Defines voting outcome variables and summary stats; not cost-benefit or incidence.
  \item
    page=15, words=0-349, section=main, note=Event study specification and design; no cost-benefit or incidence framing.
  \end{itemize}
\end{itemize}

\subsubsection{I04: What policy recommendations are stated or implied?}\label{i04-what-policy-recommendations-are-stated-or-implied}

\begin{itemize}
\tightlist
\item
  Category: \texttt{I)\ Policy\ counterfactuals\ and\ welfare}
\item
  Confidence: \texttt{low\ (0.3071000189687213)}
\item
  Retrieval method: \texttt{local}
\item
  Evidence type: \texttt{textual\ inference\ from\ main\ text}
\item
  Data source: Social Connectedness Index (SCI)
\item
  Assumption flag: \texttt{False}
\item
  Answer:
\end{itemize}

Use social-network data when drawing districts to make them fairer and improve voter information, while instituting safeguards because the same data could be exploited for partisan gerrymandering. The findings are especially relevant for redistricting, and the framework implies policymakers can evaluate or simulate alternative district maps with network-based measures (e.g., district homophily).

\begin{itemize}
\tightlist
\item
  Quote snippet:
\end{itemize}

\begin{quote}
\ldots enabling its use by policymakers to draw fairer districts, but also by partisan gerrymanderers who may seek to exploit it.
\end{quote}

\begin{itemize}
\tightlist
\item
  Citation anchors:

  \begin{itemize}
  \tightlist
  \item
    page=36, words=0-309, section=Main text, note=Policymakers can use SCI to draw fairer districts; caution about partisan exploitation.
  \item
    page=4, words=275-492, section=Main text, note=Findings especially relevant for redistricting.
  \item
    page=34, words=0-349, section=Main text, note=Framework enables simulating counterfactual district maps based on information diffusion.
  \end{itemize}
\end{itemize}

\subsubsection{J01: What missingness or attrition patterns are reported?}\label{j01-what-missingness-or-attrition-patterns-are-reported}

\begin{itemize}
\tightlist
\item
  Category: \texttt{J)\ Data\ quality\ and\ integrity}
\item
  Confidence: \texttt{low\ (0.31242748280810845)}
\item
  Retrieval method: \texttt{local}
\item
  Evidence type: \texttt{document\ text}
\item
  Data source: CES (Cooperative Election Study)
\item
  Assumption flag: \texttt{False}
\item
  Answer:
\end{itemize}

Reported patterns include: (1) attrition between the CES pre- and post-election surveys; (2) incomplete county coverage each year (even years have at least one respondent from 80--90\% of counties; odd years about two-thirds); (3) much smaller samples in odd years (about one-fifth the size of even years); and (4) exclusions for missing/incorrect data: \textless2\% missing party-recognition responses, respondents missing demographic answers (when such controls are used), a small number in 2006--2007 misassigned to counties outside their state, and 925 North Carolina respondents in 2020 misassigned to the wrong district.

\begin{itemize}
\tightlist
\item
  Quote snippet:
\end{itemize}

\begin{quote}
\ldots utilize the same sample as the information outcomes (because there is some attrition between surveys).
\end{quote}

\begin{itemize}
\tightlist
\item
  Citation anchors:

  \begin{itemize}
  \tightlist
  \item
    page=17, words=275-542, section=main, note=Attrition between pre- and post-election surveys; exclusions for missing party-recognition (\textless2\%), missing demographics, 2006--2007 county/state misassignment, and 2020 NC wrong-district misassignment.
  \item
    page=17, words=0-349, section=main, note=County coverage by year: even years 80--90\% of counties have a respondent; odd years about two-thirds.
  \item
    page=19, words=275-406, section=main, note=Odd-year samples are about one-fifth the size of even years.
  \end{itemize}
\end{itemize}

\subsubsection{J02: How are outliers handled (winsorization, trimming, exclusions)?}\label{j02-how-are-outliers-handled-winsorization-trimming-exclusions}

\begin{itemize}
\tightlist
\item
  Category: \texttt{J)\ Data\ quality\ and\ integrity}
\item
  Confidence: \texttt{low\ (0.3092116149025004)}
\item
  Retrieval method: \texttt{local}
\item
  Evidence type: \texttt{document\_text}
\item
  Data source: CES; Dave Leip's Election Atlas; Census (VAP)
\item
  Table/Figure: Table B4
\item
  Assumption flag: \texttt{False}
\item
  Answer:
\end{itemize}

No winsorization or trimming is described. Outliers are not specially transformed; instead, observations are excluded in specific cases: (1) CES respondents with missing candidate-party recognition, missing demographic answers, or erroneous county/district assignments (including 925 NC respondents in 2020); (2) elections with no top-of-ticket race; and (3) cases where turnout exceeds the voting-age population. Some variables are set missing when no incumbent runs.

\begin{itemize}
\tightlist
\item
  Quote snippet:
\end{itemize}

\begin{quote}
I exclude missing responses to the candidate party recognition question (\textless2\% of respondents in each year; for most of these cases, the House candidate name is missing in the survey).
\end{quote}

\begin{itemize}
\tightlist
\item
  Citation anchors:

  \begin{itemize}
  \tightlist
  \item
    page=17, words=275-542, note=CES exclusions: missing responses, missing demographics, misassigned counties/districts; 2020 NC misassignment excluded.
  \item
    page=49, words=0-276, section=appendix, note=Voting outcomes data exclusions: no top-of-ticket race; turnout exceeds VAP.
  \item
    page=48, words=0-333, section=appendix, note=Variables marked missing when no incumbent runs or when validation variables are missing.
  \end{itemize}
\end{itemize}

\subsubsection{J03: Are there data audits or validation steps described?}\label{j03-are-there-data-audits-or-validation-steps-described}

\begin{itemize}
\tightlist
\item
  Category: \texttt{J)\ Data\ quality\ and\ integrity}
\item
  Confidence: \texttt{low\ (0.24185108042339895)}
\item
  Retrieval method: \texttt{local}
\item
  Evidence type: \texttt{Data\ cleaning\ rules,\ validated\ measures,\ and\ placebo\ tests\ documented\ in\ appendix}
\item
  Data source: CES linked to state voter rolls; Dave Leip's Election Atlas
\item
  Table/Figure: Table B2
\item
  Assumption flag: \texttt{False}
\item
  Answer:
\end{itemize}

Yes. The study describes several validation and data-cleaning steps: use of validated voting records by linking CES respondents to state voter rolls, exclusion of misassigned or missing-response CES records, dropping elections where turnout exceeds the VAP (and those without a top-of-ticket race), and placebo outcome tests to validate findings.

\begin{itemize}
\tightlist
\item
  Quote snippet:
\end{itemize}

\begin{quote}
Voted in General Election (Validated) Respondent can be linked to state voter rolls, and there is a record of the (Validated) respondent voting in the general election.
\end{quote}

\begin{itemize}
\tightlist
\item
  Citation anchors:

  \begin{itemize}
  \tightlist
  \item
    page=48, words=0-333, section=appendix, note=Validated voting variables via linkage to state voter rolls (CES)
  \item
    page=17, words=275-542, note=Exclusions for missing responses, demographic nonresponse, county/state mismatches, and misassigned districts
  \item
    page=49, words=0-276, section=appendix, note=Excluding elections where turnout exceeds VAP and without top-of-ticket race
  \item
    page=55, words=0-152, section=appendix, note=Placebo outcome tests show no significant effects for non-House offices
  \end{itemize}
\end{itemize}

\subsubsection{J04: Is there evidence of reporting bias or selective sample inclusion?}\label{j04-is-there-evidence-of-reporting-bias-or-selective-sample-inclusion}

\begin{itemize}
\tightlist
\item
  Category: \texttt{J)\ Data\ quality\ and\ integrity}
\item
  Confidence: \texttt{low\ (0.30692777890252254)}
\item
  Retrieval method: \texttt{local}
\item
  Evidence type: \texttt{Direct\ textual\ evidence\ from\ methods\ and\ results\ descriptions}
\item
  Data source: CES (Cooperative Election Study); Dave Leip's Atlas of U.S. Presidential Elections
\item
  Table/Figure: Table C1 (Appendix)
\item
  Assumption flag: \texttt{False}
\item
  Answer:
\end{itemize}

No. The study transparently reports sample restrictions and weighting, justifies focusing on even years, includes placebo tests and null findings (e.g., no turnout effects), and notes robustness checks---showing no clear signs of reporting bias or selective inclusion.

\begin{itemize}
\tightlist
\item
  Quote snippet:
\end{itemize}

\begin{quote}
In general, district homophily does not significantly predict the placebo outcomes.
\end{quote}

\begin{itemize}
\tightlist
\item
  Citation anchors:

  \begin{itemize}
  \tightlist
  \item
    page=16, words=0-349, note=Uses CES cumulative weights; describes large, nationally representative samples and linkage to counties/districts.
  \item
    page=17, words=0-349, note=States inclusion of 48 contiguous states; coverage variation across years; notes representativeness differences and transparency on exclusions.
  \item
    page=19, words=275-406, note=Explains focusing on even years due to smaller odd-year samples; similar results with more noise---justifies, not cherry-picks.
  \item
    page=55, words=0-152, section=appendix, note=Reports placebo outcomes with generally no significant effects---transparent reporting of null results.
  \item
    page=67, words=0-258, section=appendix, note=Vote count analysis: reports no turnout effect; shows roll-off reduction and discusses precision/pre-trend---balanced reporting.
  \item
    page=20, words=0-349, note=Notes robustness (commuting flows proxy) and that placebo outcomes show no effect---supports lack of selective reporting.
  \end{itemize}
\end{itemize}

\subsubsection{K01: What goodness-of-fit or diagnostic metrics are reported?}\label{k01-what-goodness-of-fit-or-diagnostic-metrics-are-reported}

\begin{itemize}
\tightlist
\item
  Category: \texttt{K)\ Model\ fit\ and\ diagnostics}
\item
  Confidence: \texttt{low\ (0.28805989117751185)}
\item
  Retrieval method: \texttt{local}
\item
  Evidence type: \texttt{appendix\ regression\ table}
\item
  Table/Figure: Table C2
\item
  Assumption flag: \texttt{False}
\item
  Answer:
\end{itemize}

Reported diagnostics include R-squared (R2), the number of observations (Obs), clustered standard errors (clustered at the county level) in parentheses, p-values in square brackets, and significance stars.

\begin{itemize}
\tightlist
\item
  Quote snippet:
\end{itemize}

\begin{quote}
Standard errors clustered at the county level in parentheses. P-values in square brackets.
\end{quote}

\begin{itemize}
\tightlist
\item
  Citation anchors:

  \begin{itemize}
  \tightlist
  \item
    page=65, words=0-288, section=appendix, note=Table C2 lists Obs and R2 for each specification; notes state SEs are clustered at county level and p-values in brackets.
  \end{itemize}
\end{itemize}

\subsubsection{K02: Are functional form choices tested (logs, levels, nonlinearities)?}\label{k02-are-functional-form-choices-tested-logs-levels-nonlinearities}

\begin{itemize}
\tightlist
\item
  Category: \texttt{K)\ Model\ fit\ and\ diagnostics}
\item
  Confidence: \texttt{low\ (0.2771974868223541)}
\item
  Retrieval method: \texttt{local}
\item
  Evidence type: \texttt{textual\ (methods)\ and\ table\ (specification\ robustness)}
\item
  Table/Figure: Table C2
\item
  Answer:
\end{itemize}

Not specified in the provided context. The excerpts indicate linear probability interpretations and robustness via added fixed effects and controls, but do not report tests of alternative functional forms (logs or nonlinearities).

\begin{itemize}
\tightlist
\item
  Quote snippet:
\end{itemize}

\begin{quote}
District homophily is measured on a scale from 0 to 1, and outcome variables are binary. As such, reported estimates give the change in probability of the outcome
\end{quote}

\begin{itemize}
\tightlist
\item
  Citation anchors:

  \begin{itemize}
  \tightlist
  \item
    page=19, words=275-406, note=Outcomes are binary and estimates interpreted as changes in probability, implying a linear specification; no mention of functional form tests.
  \item
    page=65, words=0-288, section=appendix, note=Robustness shown by adding fixed effects and controls (Table C2), not by changing functional form.
  \item
    page=15, words=275-450, note=Discussion focuses on additional fixed effects and controls (DMA-by-year, partisan exposure), not functional form variations.
  \end{itemize}
\end{itemize}

\subsubsection{K03: Are residual checks or specification tests reported?}\label{k03-are-residual-checks-or-specification-tests-reported}

\begin{itemize}
\tightlist
\item
  Category: \texttt{K)\ Model\ fit\ and\ diagnostics}
\item
  Confidence: \texttt{low\ (0.22376402325218922)}
\item
  Retrieval method: \texttt{local}
\item
  Evidence type: \texttt{appendix\ text}
\item
  Data source: Appendix C.2
\item
  Assumption flag: \texttt{False}
\item
  Answer:
\end{itemize}

Yes. Specification tests are reported via placebo outcomes; district homophily does not significantly predict these placebo outcomes. No residual diagnostics are described in the provided text.

\begin{itemize}
\tightlist
\item
  Quote snippet:
\end{itemize}

\begin{quote}
The nine figures that follow show the results of the placebo tests. In general, district homophily does not significantly predict the placebo outcomes.
\end{quote}

\begin{itemize}
\tightlist
\item
  Citation anchors:

  \begin{itemize}
  \tightlist
  \item
    page=55, words=0-152, section=appendix, note=Appendix C.2 summarizes placebo tests and their null results.
  \item
    page=27, words=275-439, note=Notes additional placebo tests with zip-code networks also showing no significant impact.
  \end{itemize}
\end{itemize}

\subsubsection{K04: How sensitive are results to alternative specifications or estimators?}\label{k04-how-sensitive-are-results-to-alternative-specifications-or-estimators}

\begin{itemize}
\tightlist
\item
  Category: \texttt{K)\ Model\ fit\ and\ diagnostics}
\item
  Confidence: \texttt{medium\ (0.3655497023341587)}
\item
  Retrieval method: \texttt{local}
\item
  Evidence type: \texttt{text\_and\_table}
\item
  Data source: CES survey; 2016 5-Year ACS County-County Commuting Flows
\item
  Table/Figure: Appendix Table C2
\item
  Assumption flag: \texttt{False}
\item
  Answer:
\end{itemize}

Results are generally robust across alternative specifications and estimators. They remain similar when adding county fixed effects and when including odd-year surveys (though odd years are noisier). Using commuting flows as an alternative network measure yields qualitatively similar positive effects on voter familiarity and incumbent support, with magnitudes about half as large and shifts coming from reduced no preference/not voting rather than reduced opponent support. Placebo tests for governors and senators show no significant effects. An alternative border-pairs design delivers qualitatively similar findings; however, the `Selected Party' outcome becomes insignificant after adding DMA-by-year fixed effects, while `Heard of Incumbent' and `Selected Correct Party' remain significant in most specifications (as shown in Appendix Table C2), with reduced precision due to the restricted sample.

\begin{itemize}
\tightlist
\item
  Quote snippet:
\end{itemize}

\begin{quote}
With the border pairs design, I find qualitatively similar results as in the redistricting design, except estimates on ``Selected Party'' become insignificant after adding DMA-by-year fixed effects.
\end{quote}

\begin{itemize}
\tightlist
\item
  Citation anchors:

  \begin{itemize}
  \tightlist
  \item
    page=19, words=275-406, note=Similar results with county fixed effects; odd years similar but noisier.
  \item
    page=20, words=0-349, note=Stability over time; similar results with commuting flows; placebo outcomes null.
  \item
    page=26, words=0-349, note=Placebo outcomes show no significant impact; describes commuting flows measure.
  \item
    page=26, words=275-375, note=Commuting homophily effects about half as large; reductions from no preference/not voting.
  \item
    page=28, words=0-349, note=Border-pairs design: qualitatively similar; `Selected Party' becomes insignificant with DMA×year FEs; precision decreases.
  \item
    page=65, words=0-288, section=appendix, note=Appendix Table C2 shows robustness across FE sets; `Selected Party' loses significance with richer FEs; other outcomes remain significant.
  \item
    page=55, words=0-152, section=appendix, note=Placebo outcomes summary: generally no significant prediction by district homophily.
  \end{itemize}
\end{itemize}

\subsubsection{P01: ResponseTextConfig(format=ResponseFormatText(type=\textquotesingle text\textquotesingle), verbosity=\textquotesingle medium\textquotesingle)}\label{p01-responsetextconfigformatresponseformattexttypetext-verbositymedium}

\begin{itemize}
\tightlist
\item
  Category: \texttt{P)\ Previous\ questions}
\item
  Confidence: \texttt{n/a}
\item
  Retrieval method: \texttt{n/a}
\item
  Evidence type: \texttt{n/a}
\item
  Answer:
\end{itemize}

No specific question was posed. Below is a concise synthesis of key definitions, data, and identification from the provided context, with citations.

\begin{itemize}
\item
  Roll-off (House turnout relative to top-of-ticket)

  \begin{itemize}
  \tightlist
  \item
    Definition: Share of top-of-ticket voters who abstain from the House race: (VotesTop − VotesHouse) / VotesTop; captures abstention after incurring the cost of voting; related to ``roll-off'' literature (Feddersen \& Pesendorfer 1996; Miller 2022; Snyder \& Strömberg 2010) (page 18 words 275-496).
  \item
    Variable construction and scope: Main analysis at county level from Dave Leip's; robustness at county-by-CD level from Harvard EDA, Daily Kos, Dave Leip's, and MIT EDS Lab; exclude elections without a top-of-ticket race (page 49 words 0-276).
  \item
    Summary stats: Mean 4.42\%, SD 12.22 pp; also report turnout in top-of-ticket (mean 51.34\%) and House (mean 49.07\%) (page 49 words 0-276).
  \end{itemize}
\item
  Campaign contributions data

  \begin{itemize}
  \tightlist
  \item
    Source: Kuziemko et al. 2023 (October), constructed from FEC data in Bonica 2014; geocoding used to identify whether contributor resides in the same congressional district as the candidate (page 18 words 275-496).
  \end{itemize}
\item
  Border pairs identification design

  \begin{itemize}
  \tightlist
  \item
    Setup: Neighboring county pairs that straddle a district boundary; counties are similar except district assignment, yielding different district homophily; restrict to counties fully within one district; collapse outcomes to county level; include one observation per county--pair; compare within-state pairs; precision decreases due to restricted sample (page 28 words 0-349).
  \item
    Specification:

    \begin{itemize}
    \tightlist
    \item
      \(y_{ct} = \alpha_c + \mu_{pt} + \beta\,\bar{\pi}_{c,t} + X'_{ct}\delta + \varepsilon_{ct}\), where \(y_{ct}\) is the outcome, \(\mu_{pt}\) pair-by-year FE, \(X_{ct}\) time-varying county controls; use state-by-year FE (insufficient data for district-by-year FE) (page 28 words 0-349).
    \end{itemize}
  \item
    Results note: Qualitatively similar to redistricting design; estimates on ``Selected Party'' become insignificant after adding DMA-by-year FE (page 28 words 0-349).
  \end{itemize}
\item
  Conceptual framework

  \begin{itemize}
  \tightlist
  \item
    Develops a theoretical model of information diffusion within districts; district homophily summarized as a network statistic; considers news about elected officials diffusing through networks to determine the equilibrium share of informed voters (page 28 words 0-349).
  \end{itemize}
\item
  Vote count data construction

  \begin{itemize}
  \tightlist
  \item
    County-by-CD measures built from precinct-level vote counts (Harvard EDA 2000--2010; MIT EDS Lab 2016--2020) combined with county-by-CD vote counts from Dave Leip's (House) and Daily Kos (President, Senator, Governor) (page 47 words 0-207).
  \end{itemize}
\item
  CES variables and summary statistics (knowledge, preferences, voting)

  \begin{itemize}
  \tightlist
  \item
    Knowledge outcomes (binary): Heard of Representative; Selected Party; Selected Correct Party; defined via pre-survey items on recognizing and assigning party to the House representative (page 47 words 0-207).
  \item
    Voting/preference outcomes (binary): Prefer Incumbent/Opponent/Neither (pre-survey); Voted for Incumbent/Opponent/Neither (post-survey); Validated turnout in general/primary; self-reported general turnout; handling of missing when no incumbent or unmatched rolls (page 48 words 0-333).
  \item
    Summary (sample means): Heard of Rep 93.2\%; Selected Party 68.6\%; Selected Correct Party 61.7\%; Prefer Incumbent 40.1\%; Voted for Incumbent 41.0\%; Validated General Turnout 57.5\%; Self-Reported General Turnout 87.8\% (page 49 words 0-276).
  \end{itemize}
\item
  Additional pointers

  \begin{itemize}
  \tightlist
  \item
    Figures showing effects of homophily on self-reported voting using commuting- and zip-code-based measures (Figures C11, C14); a figure on changes in district homophily with progressively finer bins (Figure C5) (page 61 words 0-15; page 64 words 0-45; page 54 words 0-18).
  \end{itemize}
\end{itemize}

\end{document}
