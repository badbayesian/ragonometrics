% Options for packages loaded elsewhere

\PassOptionsToPackage{unicode}{hyperref}

\PassOptionsToPackage{hyphens}{url}

\documentclass[

]{article}

\usepackage{xcolor}

\usepackage{amsmath,amssymb}

\setcounter{secnumdepth}{-\maxdimen} % remove section numbering

\usepackage{iftex}

\ifPDFTeX

  \usepackage[T1]{fontenc}

  \usepackage[utf8]{inputenc}

  \usepackage{textcomp} % provide euro and other symbols

\else % if luatex or xetex

  \usepackage{unicode-math} % this also loads fontspec

  \defaultfontfeatures{Scale=MatchLowercase}

  \defaultfontfeatures[\rmfamily]{Ligatures=TeX,Scale=1}

\fi

\usepackage{lmodern}

\ifPDFTeX\else

  % xetex/luatex font selection

\fi

% Use upquote if available, for straight quotes in verbatim environments

\IfFileExists{upquote.sty}{\usepackage{upquote}}{}

\IfFileExists{microtype.sty}{% use microtype if available

  \usepackage[]{microtype}

  \UseMicrotypeSet[protrusion]{basicmath} % disable protrusion for tt fonts

}{}

\makeatletter

\@ifundefined{KOMAClassName}{% if non-KOMA class

  \IfFileExists{parskip.sty}{%

    \usepackage{parskip}

  }{% else

    \setlength{\parindent}{0pt}

    \setlength{\parskip}{6pt plus 2pt minus 1pt}}

}{% if KOMA class

  \KOMAoptions{parskip=half}}

\makeatother

\setlength{\emergencystretch}{3em} % prevent overfull lines

\providecommand{\tightlist}{%

  \setlength{\itemsep}{0pt}\setlength{\parskip}{0pt}}

\usepackage{bookmark}

\IfFileExists{xurl.sty}{\usepackage{xurl}}{} % add URL line breaks if available

\urlstyle{same}

\hypersetup{

  hidelinks,

  pdfcreator={LaTeX via pandoc}}



\author{}

\date{}



\begin{document}



\section{\texorpdfstring{Audit Report: Workflow

\texttt{1308532de7a9446d813e57129826aa71}}{Audit Report: Workflow 1308532de7a9446d813e57129826aa71}}\label{audit-report-workflow-1308532de7a9446d813e57129826aa71}



\subsection{Overview}\label{overview}



\begin{itemize}

\tightlist

\item

  Source JSON:

  \texttt{reports\textbackslash{}workflow-report-1308532de7a9446d813e57129826aa71.json}

\item

  Run ID: \texttt{1308532de7a9446d813e57129826aa71}

\item

  Papers input:

  \texttt{papers\textbackslash{}Use\_of\_Cumulative\_Sums\_of\_Squares\_for\_Re.pdf}

\item

  Started at: \texttt{2026-02-15T17:40:01.994514+00:00}

\item

  Finished at: \texttt{2026-02-15T17:53:00.314525+00:00}

\item

  Duration: \texttt{0:12:58.320011}

\end{itemize}



\subsection{Effective Configuration}\label{effective-configuration}



\begin{itemize}

\tightlist

\item

  Chat model: \texttt{gpt-5}

\item

  Embedding model: \texttt{text-embedding-3-large}

\item

  Top K: \texttt{10}

\item

  Chunk words / overlap: \texttt{350} / \texttt{75}

\item

  Batch size: \texttt{64}

\item

  Database URL configured: \texttt{True}

\end{itemize}



\subsection{Step Outcomes}\label{step-outcomes}



\begin{itemize}

\tightlist

\item

  \texttt{prep}: \texttt{completed}

\item

  \texttt{ingest}: \texttt{n/a}

\item

  \texttt{enrich}: \texttt{n/a}

\item

  \texttt{econ\_data}: \texttt{fetched}

\item

  \texttt{agentic}: \texttt{completed}

\item

  \texttt{index}: \texttt{skipped} (reason: \texttt{db\_unreachable})

\item

  \texttt{evaluate}: \texttt{n/a}

\item

  \texttt{report\_store}: \texttt{skipped} (reason:

  \texttt{db\_unreachable})

\end{itemize}



\subsection{Agentic Summary}\label{agentic-summary}



\begin{itemize}

\tightlist

\item

  Status: \texttt{completed}

\item

  Main question: What is the key contribution?

\item

  Report question set: \texttt{structured}

\item

  Structured questions generated: \texttt{83}

\item

  Confidence mean/median: \texttt{0.27249828745446014} /

  \texttt{0.268446744281151}

\item

  Confidence labels: low=77, medium=6, high=0

\end{itemize}



\subsubsection{Final Answer}\label{final-answer}



\begin{itemize}

\item

  Primary: Introduces the ICSS procedure for variance change-point

  detection---built on a statistic \(D_k\) that is a monotone transform

  of the two-sample variance F statistic

  \(F_{T-k,k}=\frac{(C_T-C_k)/(T-k)}{C_k/k}\) with

  \(C_k=\sum_{t=1}^k a_t^2\)---and locates breaks by maximizing

  \(\max_k |D_k|\); under homogeneity, the \(D_k\) process converges to

  a Brownian bridge, enabling use of fixed critical values (e.g., 5\% ≈

  1.358) for inference {[}page 3 words 0--349; page 12 words 0--349{]}.

\item

  Supporting:



  \begin{itemize}

  \tightlist

  \item

    Provides asymptotic moments under no breaks (e.g.,

    \(E[C_k]=k\sigma^2\), \(E[C_k C_T]=k(T+2)\sigma^4\)) and shows

    \(E[D_k]\to 0\) as \(T\) grows, justifying the null behavior used

    for thresholding {[}page 11 words 275--507{]}.

  \item

    Benchmarks ICSS against likelihood-ratio and Bayesian posterior-odds

    approaches via extensive simulations (various \(T\), change

    locations, and variance ratios), showing strong detection except

    when changes are very early and small; performance improves with

    larger \(T\) and variance ratios {[}page 7 words 0--349; page 7

    words 275--556; page 9 words 550--899{]}.

  \item

    Establishes ICSS as a computationally light alternative with stable,

    lower CPU cost versus LR (roughly \(O(T^3)\)), especially beneficial

    for long series and multiple changes {[}page 9 words 825--904; page

    11 words 275--507{]}.

  \item

    Extends application to AR(1) settings via prewhitened residuals,

    yielding results close to the i.i.d. case {[}page 9 words 550--899;

    page 10 words 0--84{]}.

  \item

    Outlines complementary testing frameworks: sequential LR tests for

    the number of change points with extreme-value (Gumbel)

    standardization, and Bayesian posterior odds with per-time change

    probability, to compare modeling strategies {[}page 8 words

    0--329{]}.

  \end{itemize}

\end{itemize}



\subsubsection{Sub-Answers}\label{sub-answers}



\paragraph{Sub-answer 1}\label{sub-answer-1}



\begin{itemize}

\tightlist

\item

  Question:

  ResponseTextConfig(format=ResponseFormatText(type=\textquotesingle text\textquotesingle),

  verbosity=\textquotesingle medium\textquotesingle)

\item

  Answer:

\end{itemize}



Could you clarify what you'd like from these excerpts (e.g., statistic

definitions, testing procedure, simulation setup/results, or

computational complexity)? Key points available here include:



\begin{itemize}

\item

  ICSS statistic and F-test link



  \begin{itemize}

  \tightlist

  \item

    For a fixed k, the usual two-sample variance F statistic is

    \(F_{T-k,k} = \frac{(C_T - C_k)/(T-k)}{C_k/k}\), where

    \(C_k=\sum_{t=1}^k a_t^2\); \(D_k\) is a monotone function of

    \(F_{T-k,k}\) and depends on k only via \(k/T\). Unlike the F-test

    (known k), ICSS searches \(\max_k |D_k|\) to locate a change point

    {[}page 3 words 0--349{]}.

  \item

    Asymptotically (Normal noise), the \(D_k\) process is a Brownian

    bridge; an example uses the 5\% critical value 1.358 with detection

    at \(k^*=342\) when \(\max_k|D_k|\) exceeds this boundary {[}page 3

    words 0--349{]}.

  \end{itemize}

\item

  Asymptotics and moments under homogeneity



  \begin{itemize}

  \tightlist

  \item

    Under \(N_T=0\): \(E[C_k]=k\sigma^2\),

    \(E[C_k C_T]=k(T+2)\sigma^4\), and \(E[D_k]\to 0\) as \(T\) grows

    {[}page 11 words 275--507{]}.

  \item

    Appendix sketch: with \(t_i=a_i^2-\sigma^2\), a Donsker-type

    argument yields convergence to a Brownian bridge \(W^0\) {[}page 12

    words 0--349{]}.

  \end{itemize}

\item

  Likelihood ratio (LR) testing for number of change points



  \begin{itemize}

  \tightlist

  \item

    Strategy: sequentially test \(H_0:N_T=n-1\) vs \(H_a:N_T=n\) using

    \(LR_{n-1,n}\) for \(n=1,2,\dots\); critical values require

    simulation because the problem is nonregular {[}page 8 words

    0--329{]}.

  \item

    Standardize \(LR_{n-1,n}\) to reference the extreme value limit with

    cdf \(F(x)=\exp(-e^{-x})\); formulas for standardization are given

    in Table 3 (referenced) {[}page 8 words 0--329{]}.

  \end{itemize}

\item

  Bayesian posterior odds (multiple change points)



  \begin{itemize}

  \tightlist

  \item

    Posterior odds compare \(p(N_T=n\mid a)\) across n, using

    \(p(a,K\mid N_T)\) and a per-time prior change probability \(X\);

    details and Table 5 summarize two-change cases {[}page 8 words

    0--329{]}.

  \end{itemize}

\item

  Simulation design (comparisons: ICSS, LR, Bayesian)



  \begin{itemize}

  \tightlist

  \item

    1,000 replicates per design; \(T\in\{100,200,500\}\); one-change

    locations \(K/T\in\{.25,.5,.75\}\); variance ratios \(A\in\{2,3\}\).

    Two-change locations

    \((K_1/T,K_2/T)\in\{(.33,.66),(.20,.80),(.58,.80)\}\) with

    permutations of \((\sigma_1^2,\sigma_2^2,\sigma_3^2)\) constrained

    by \(A_1=\sigma_2^2/\sigma_1^2\), \(A_2=\sigma_3^2/\sigma_2^2\) and

    symmetry reductions. Data are iid \(N(0,1)\); Fortran

    implementation; RNG via Bratley--Fox--Schrage; Sparcstation 1+

    {[}page 7 words 0--349; page 7 words 275--556{]}.

  \end{itemize}

\item

  Main empirical findings (one change)



  \begin{itemize}

  \tightlist

  \item

    With \(T=100\) and small variance ratio (\(A=2\)), detection is

    hard---especially if the change is near the beginning---with ICSS.

    With \(T=200\) or \(A=3\), correct identification exceeds 80\% when

    the change is near the middle. ICSS improves with larger T and A.

    The Bayesian approach performs best for \(A=2\) and is comparable to

    others for \(A=3\) {[}page 9 words 550--899{]}.

  \end{itemize}

\item

  AR(1) residual application



  \begin{itemize}

  \tightlist

  \item

    Generate \(Y_t=\phi Y_{t-1}+a_t\); apply ICSS to residuals

    \(\varepsilon_t=Y_t-\hat\phi Y_{t-1}\). The percentage with

    \(N_T=1\) is close to the iid results in Table 4 (see Table 8)

    {[}page 9 words 550--899; page 10 words 0--84{]}.

  \end{itemize}

\item

  Computational burden



  \begin{itemize}

  \tightlist

  \item

    ICSS was designed to avoid the heavy computation of posterior odds

    and the need to evaluate functions at all possible change locations.

    LR uses Baufays--Rasson's algorithm for maximizing the likelihood

    but remains heavy for long series and multiple changes {[}page 9

    words 825--904{]}.

  \item

    CPU-time scaling: LR is roughly \(O(T^3)\) and of similar order for

    \(N_T=2\) and \(N_T=3\); ICSS's dependence on T is the same across

    values of \(N_T\) (lower and more stable cost) {[}page 11 words

    275--507{]}.

  \end{itemize}

\end{itemize}



\subsection{Structured Q\&A Appendix}\label{structured-qa-appendix}



This section mirrors \texttt{agentic.report\_questions} for audit

traceability.



\subsubsection{A01: What is the main research question of the

paper?}\label{a01-what-is-the-main-research-question-of-the-paper}



\begin{itemize}

\tightlist

\item

  Category: \texttt{A)\ Research\ question\ /\ contribution}

\item

  Confidence: \texttt{low} (0.25738287842443736)

\item

  Retrieval method: \texttt{local}

\item

  Evidence type: \texttt{text}

\item

  Assumption flag: \texttt{False}

\item

  Answer:

\end{itemize}



To develop and assess a method---based on cumulative sums of squares

(ICSS)---for retrospectively detecting changes in variance in time

series, including determining the number and locations of variance

changepoints, and to compare its performance with likelihood ratio and

Bayesian approaches.



\begin{itemize}

\tightlist

\item

  Quote snippet:

\end{itemize}



\begin{quote}

The performance of different procedures used to determine the number of

variance changes in a series

\end{quote}



\begin{itemize}

\tightlist

\item

  Citation anchors:



  \begin{itemize}

  \tightlist

  \item

    page=7, words=0-120, section=4. SIMULATION EXPERIMENT, note=States

    the study evaluates procedures to determine the number of variance

    changes and compares ICSS with likelihood ratio and Bayesian

    methods.

  \item

    page=4, words=0-40, section=Inclán and Tiao: Retrospective Detection

    of Changes in Variance, note=Section header emphasizes focus on

    retrospective detection of variance changes.

  \item

    page=6, words=550-640, section=ICSS algorithm application,

    note=Describes the ICSS procedure as a systematic search for points

    of change.

  \end{itemize}

\end{itemize}



\subsubsection{A02: What is the paper\textquotesingle s primary

contribution relative to the existing

literature?}\label{a02-what-is-the-papers-primary-contribution-relative-to-the-existing-literature}



\begin{itemize}

\tightlist

\item

  Category: \texttt{A)\ Research\ question\ /\ contribution}

\item

  Confidence: \texttt{low} (0.27419868133310066)

\item

  Retrieval method: \texttt{local}

\item

  Evidence type: \texttt{direct\ textual\ evidence}

\item

  Assumption flag: \texttt{False}

\item

  Answer:

\end{itemize}



It introduces the ICSS algorithm---a centered cumulative-sum-of-squares,

iterative method---to retrospectively detect multiple variance change

points efficiently; it develops properties of the centered CSS statistic

and shows via simulations that ICSS matches Bayesian/likelihood-ratio

performance for moderate samples without their heavy computational

burden.



\begin{itemize}

\tightlist

\item

  Quote snippet:

\end{itemize}



\begin{quote}

We propose a procedure to detect variance changes based on an iterated

cumulative sums of squares (ICSS) algorithm.

\end{quote}



\begin{itemize}

\tightlist

\item

  Citation anchors:



  \begin{itemize}

  \tightlist

  \item

    page=2, words=0-349, section=Introduction, note=States proposal of

    ICSS, study of centered CSS properties, and comparable performance

    without heavy computation.

  \item

    page=2, words=275-624, section=Introduction, note=Describes using

    centered cumulative sums of squares and an iterative algorithm to

    find multiple change points.

  \end{itemize}

\end{itemize}



\subsubsection{A03: What is the central hypothesis being

tested?}\label{a03-what-is-the-central-hypothesis-being-tested}



\begin{itemize}

\tightlist

\item

  Category: \texttt{A)\ Research\ question\ /\ contribution}

\item

  Confidence: \texttt{low} (0.25484008361822535)

\item

  Retrieval method: \texttt{local}

\item

  Evidence type: \texttt{textual}

\item

  Answer:

\end{itemize}



H0: no change in variance (NT = 0) versus H1: one change in variance (NT

= 1).



\begin{itemize}

\tightlist

\item

  Quote snippet:

\end{itemize}



\begin{quote}

testing one change against no change in variance

\end{quote}



\begin{itemize}

\tightlist

\item

  Citation anchors:



  \begin{itemize}

  \tightlist

  \item

    page=3, words=300-430, section=Section 2.1, note=Defines NT=0 (no

    change) and NT=1 (one change) and states testing one change against

    no change in variance.

  \item

    page=7, words=310-400, section=Section 4.1, note=Defines LR\_mn for

    testing H0: NT = m against Ha: NT = n (number of variance changes).

  \end{itemize}

\end{itemize}



\subsubsection{A04: What are the main outcomes of interest (dependent

variables)?}\label{a04-what-are-the-main-outcomes-of-interest-dependent-variables}



\begin{itemize}

\tightlist

\item

  Category: \texttt{A)\ Research\ question\ /\ contribution}

\item

  Confidence: \texttt{low} (0.26037546805984413)

\item

  Retrieval method: \texttt{local}

\item

  Evidence type: \texttt{explicit\ statement}

\item

  Data source: Section 4. SIMULATION EXPERIMENT description

\item

  Assumption flag: \texttt{False}

\item

  Answer:

\end{itemize}



The main outcomes are the number of correct identifications of variance

changes and the computational time (CPU time).



\begin{itemize}

\tightlist

\item

  Quote snippet:

\end{itemize}



\begin{quote}

the main way is by the number of \textquotesingle correct

identifications.\textquotesingle{} Another aspect to consider is the

computational requirements of each approach, which can be measured in

terms of the CPU time.

\end{quote}



\begin{itemize}

\tightlist

\item

  Citation anchors:



  \begin{itemize}

  \tightlist

  \item

    page=7, words=0-349, section=4. SIMULATION EXPERIMENT, note=Defines

    primary performance measures: correct identifications and CPU time.

  \end{itemize}

\end{itemize}



\subsubsection{A05: What are the key treatment/exposure variables

(independent

variables)?}\label{a05-what-are-the-key-treatmentexposure-variables-independent-variables}



\begin{itemize}

\tightlist

\item

  Category: \texttt{A)\ Research\ question\ /\ contribution}

\item

  Confidence: \texttt{low} (0.2180384234455463)

\item

  Retrieval method: \texttt{local}

\item

  Evidence type: \texttt{methods/simulation\ design\ description}

\item

  Data source: Inclán and Tiao (1994), Journal of the American

  Statistical Association

\item

  Assumption flag: \texttt{False}

\item

  Answer:

\end{itemize}



Series length (T); location(s) of changepoint(s) (K or K1, K2); variance

ratio(s) (A for one change; A1 = τ1/τ0 and A2 = τ2/τ1 for two changes);

configuration/permutation of segment variances; and the number of

changepoints (NT).



\begin{itemize}

\tightlist

\item

  Quote snippet:

\end{itemize}



\begin{quote}

For Part One, we used three series lengths ( T = 100,200, and 500),three

locations of the changepoint ( K , = .25T, .50T, and . 7 5 T ) ,and two

values for the variance ratio ( A = 2 and 3 ) . For Part Two, we used

the same three series lengths ... three different sets oflocations ofthe

change points ... and six possible configurations of variances ... keep

the corresponding ratios A, = T : / T \textasciitilde{} and A2 = T ; / T

; .

\end{quote}



\begin{itemize}

\tightlist

\item

  Citation anchors:



  \begin{itemize}

  \tightlist

  \item

    page=page 7, words=0-349, section=Simulation Experiment (Part One,

    Part Two), note=Enumerates manipulated factors: T, K (or K1,K2), A

    (or A1,A2), and variance configurations.

  \end{itemize}

\end{itemize}



\subsubsection{A06: What setting/context does the paper study (country,

market,

period)?}\label{a06-what-settingcontext-does-the-paper-study-country-market-period}



\begin{itemize}

\tightlist

\item

  Category: \texttt{A)\ Research\ question\ /\ contribution}

\item

  Confidence: \texttt{low} (0.29809896205083153)

\item

  Retrieval method: \texttt{local}

\item

  Evidence type: \texttt{direct\_text}

\item

  Data source: Box and Jenkins (1976)

\item

  Assumption flag: \texttt{False}

\item

  Answer:

\end{itemize}



Applied example: IBM stock closing prices (first differences of log

prices) over May 17, 1961 to November 2, 1962; country not specified in

the provided text.



\begin{itemize}

\tightlist

\item

  Quote snippet:

\end{itemize}



\begin{quote}

The series analyzed is the first difference of the logarithm of the IBM

stock closing prices from May 17, 1961, to November 2, 1962, as reported

by Box and Jenkins (1976).

\end{quote}



\begin{itemize}

\tightlist

\item

  Citation anchors:



  \begin{itemize}

  \tightlist

  \item

    page=6, words=360-430, section=3.1 Example: IBM Stock Prices,

    note=States the data series and its time span.

  \end{itemize}

\end{itemize}



\subsubsection{A07: What is the main mechanism proposed by the

authors?}\label{a07-what-is-the-main-mechanism-proposed-by-the-authors}



\begin{itemize}

\tightlist

\item

  Category: \texttt{A)\ Research\ question\ /\ contribution}

\item

  Confidence: \texttt{low} (0.23553237245887174)

\item

  Retrieval method: \texttt{local}

\item

  Evidence type: \texttt{direct\ textual\ evidence}

\item

  Data source: Inclán \& Tiao (1994), Journal of the American

  Statistical Association

\item

  Assumption flag: \texttt{False}

\item

  Answer:

\end{itemize}



The Iterated Cumulative Sums of Squares (ICSS) algorithm---an iterative

use of the centered cumulative sum of squares (Dk) to systematically

locate multiple variance-change points.



\begin{itemize}

\tightlist

\item

  Quote snippet:

\end{itemize}



\begin{quote}

We propose a procedure to detect variance changes based on an iterated

cumulative sums of squares (ICSS) algorithm.

\end{quote}



\begin{itemize}

\tightlist

\item

  Citation anchors:



  \begin{itemize}

  \tightlist

  \item

    page=2, words=0-349, section=Introduction/Abstract, note=States

    proposal of ICSS to detect variance changes.

  \item

    page=2, words=275-624, section=Introduction, note=Describes using

    cumulative sums of squares and iterative algorithm.

  \item

    page=5, words=825-1174, section=ICSS Algorithm description,

    note=Algorithm labeled and illustrated.

  \end{itemize}

\end{itemize}



\subsubsection{A08: What alternative mechanisms are

discussed?}\label{a08-what-alternative-mechanisms-are-discussed}



\begin{itemize}

\tightlist

\item

  Category: \texttt{A)\ Research\ question\ /\ contribution}

\item

  Confidence: \texttt{low} (0.242333650293353)

\item

  Retrieval method: \texttt{local}

\item

  Evidence type: \texttt{textual}

\item

  Assumption flag: \texttt{True}

\item

  Assumption notes: Interpreted

  \textquotesingle mechanisms\textquotesingle{} as alternative

  explanations for observed features/regime shifts: heavy-tailed vs

  variance-shift modeling, and variance vs AR-parameter changes.

\item

  Answer:

\end{itemize}



Two alternative mechanisms are discussed: (1) modeling heavy tails in

returns (e.g., Pareto/heavy‑tailed distributions) versus explaining them

by variance shifts (nonstationary variance with step changes), and (2)

changes in variance versus changes in the autoregressive parameter as

drivers of regime shifts.



\begin{itemize}

\tightlist

\item

  Quote snippet:

\end{itemize}



\begin{quote}

A Bayesian analysis that allows both the variance and the autoregressive

parameter to change at the same points ... indicates that there are both

variance changes and parameter changes.

\end{quote}



\begin{itemize}

\tightlist

\item

  Citation anchors:



  \begin{itemize}

  \tightlist

  \item

    page=2, words=275-624, section=Introduction, note=Contrasts

    heavy-tailed distributions with a normal model having variance step

    changes.

  \item

    page=6, words=550-899, section=Section 3.1 Example, note=States that

    both variance and autoregressive parameter changes can occur.

  \end{itemize}

\end{itemize}



\subsubsection{A09: What are the main policy implications claimed by the

paper?}\label{a09-what-are-the-main-policy-implications-claimed-by-the-paper}



\begin{itemize}

\tightlist

\item

  Category: \texttt{A)\ Research\ question\ /\ contribution}

\item

  Confidence: \texttt{low} (0.2661350049764556)

\item

  Retrieval method: \texttt{local}

\item

  Evidence type: \texttt{textual}

\item

  Assumption flag: \texttt{False}

\item

  Answer:

\end{itemize}



The paper does not present policy implications; it focuses on developing

and evaluating the ICSS statistical method for retrospectively detecting

variance changes and comparing computational costs.



\begin{itemize}

\tightlist

\item

  Quote snippet:

\end{itemize}



\begin{quote}

The main interest is to study the variance of a given sequence of

observations retrospectively, so we can use all the information on the

series to indicate the points of variance change.

\end{quote}



\begin{itemize}

\tightlist

\item

  Citation anchors:



  \begin{itemize}

  \tightlist

  \item

    page=2, words=560-650, section=2. CENTERED CUMULATIVE SUMS OF

    SQUARES, note=States the paper's main methodological objective

    (variance change detection).

  \item

    page=6, words=550-640, section=ICSS algorithm application,

    note=Describes the procedure's systematic search for change points.

  \item

    page=7, words=0-120, section=4. SIMULATION EXPERIMENT, note=Frames

    performance comparison of procedures rather than policy discussion.

  \item

    page=9, words=720-870, section=4.6 CPU Time Requirements,

    note=Emphasizes computational burden and efficiency, not policy.

  \end{itemize}

\end{itemize}



\subsubsection{A10: What is the welfare interpretation (if any) of the

results?}\label{a10-what-is-the-welfare-interpretation-if-any-of-the-results}



\begin{itemize}

\tightlist

\item

  Category: \texttt{A)\ Research\ question\ /\ contribution}

\item

  Confidence: \texttt{low} (0.2910150337102933)

\item

  Retrieval method: \texttt{local}

\item

  Evidence type: \texttt{text}

\item

  Assumption flag: \texttt{False}

\item

  Assumption notes: The provided text discusses statistical detection

  accuracy and computation; it does not introduce any welfare framework.

\item

  Answer:

\end{itemize}



None. The results are statistical---focused on detecting variance

changes, their locations, and computational performance---not on welfare

or economic surplus.



\begin{itemize}

\tightlist

\item

  Quote snippet:

\end{itemize}



\begin{quote}

The performance of different procedures used to determine the number of

variance changes in a series can be measured in several ways; the main

way is by the number of "correct identifications."

\end{quote}



\begin{itemize}

\tightlist

\item

  Citation anchors:



  \begin{itemize}

  \tightlist

  \item

    page=7, words=0-349, section=4. SIMULATION EXPERIMENT, note=Defines

    performance as correct identifications and CPU time; no welfare

    concepts are introduced.

  \item

    page=9, words=550-899, section=4.2--4.6 Results, note=Discusses

    detection success rates and computational burden; still no welfare

    interpretation.

  \end{itemize}

\end{itemize}



\subsubsection{A11: What are the main limitations acknowledged by the

authors?}\label{a11-what-are-the-main-limitations-acknowledged-by-the-authors}



\begin{itemize}

\tightlist

\item

  Category: \texttt{A)\ Research\ question\ /\ contribution}

\item

  Confidence: \texttt{low} (0.2484759032229551)

\item

  Retrieval method: \texttt{local}

\item

  Evidence type:

  \texttt{Direct\ textual\ evidence\ from\ results\ and\ method\ discussion}

\item

  Assumption flag: \texttt{False}

\item

  Answer:

\end{itemize}



They note several limitations: (1) heavy computational burden for

likelihood/Bayesian methods when searching multiple change points

(posterior-odds runs can be infeasible for long series); (2) a masking

effect with multiple changepoints, where Dk tends to highlight only one;

(3) ICSS struggles with small variance shifts or early changes, and

detecting two monotone-increasing changes often requires large samples;

and (4) the location estimate k* is biased toward the middle of the

series, improving only with larger T or variance ratios.



\begin{itemize}

\tightlist

\item

  Quote snippet:

\end{itemize}



\begin{quote}

it is hard to detect a small variance change (variance ratio A =

2)---particularly when it appears at the beginning of the series and we

are using the ICSS algorithm.

\end{quote}



\begin{itemize}

\tightlist

\item

  Citation anchors:



  \begin{itemize}

  \tightlist

  \item

    page=2, words=550-899, section=Background/Section 2, note=Heavy

    computational burden when looking for several points of change

    simultaneously

  \item

    page=4, words=0-349, section=Discussion of E{[}Dk{]} with two

    changepoints, note=Masking effect when more than one changepoint

  \item

    page=9, words=550-899, section=Results (4.2, 4.3) and CPU time

    (4.6), note=Hard to detect small/early variance change with ICSS;

    posterior odds infeasible for T=500 (1,358 CPU hours)

  \item

    page=9, words=275-624, section=4.4 Sampling Distribution of k*,

    note=k* distribution skewed and biased toward the middle;

    concentrates with larger T or variance ratio

  \end{itemize}

\end{itemize}



\subsubsection{A12: What does the paper claim is novel about its data or

identification?}\label{a12-what-does-the-paper-claim-is-novel-about-its-data-or-identification}



\begin{itemize}

\tightlist

\item

  Category: \texttt{A)\ Research\ question\ /\ contribution}

\item

  Confidence: \texttt{low} (0.2866564534027284)

\item

  Retrieval method: \texttt{local}

\item

  Evidence type: \texttt{paraphrase\_of\_context}

\item

  Data source: Inclán \& Tiao (1994), Journal of the American

  Statistical Association

\item

  Assumption flag: \texttt{False}

\item

  Answer:

\end{itemize}



The paper's novelty is an iterated cumulative sums of squares (ICSS)

algorithm based on a centered cumulative sum of squares statistic to

systematically and retrospectively detect multiple variance

changepoints, achieving performance comparable to Bayesian and

likelihood-ratio methods for moderate samples while avoiding their heavy

computational burden and mitigating masking when multiple changes exist.



\begin{itemize}

\tightlist

\item

  Quote snippet:

\end{itemize}



\begin{quote}

the ICSS algorithm offers results comparable to those obtained by a

Bayesian approach or by likelihood ratio tests, without the heavy

computational burden required by these approaches.

\end{quote}



\begin{itemize}

\tightlist

\item

  Citation anchors:



  \begin{itemize}

  \tightlist

  \item

    page=2, words=0-200, section=Introduction, note=Introduces ICSS and

    claims comparable performance without heavy computational burden.

  \item

    page=2, words=275-410, section=Introduction, note=States use of

    cumulative sums of squares and iterative search for multiple

    changepoints.

  \item

    page=2, words=550-750, section=2. Centered Cumulative Sums of

    Squares, note=Defines the centered and normalized cumulative sum of

    squares statistic Dk used by ICSS.

  \item

    page=6, words=0-140, section=3.1 Example: IBM Stock Prices / ICSS

    illustration, note=Shows ICSS eliminates masking when more than one

    variance change exists.

  \item

    page=9, words=825-900, section=4.6 CPU Time Requirements, note=Notes

    ICSS avoids evaluating functions at all possible changepoint

    locations, reducing computational burden.

  \end{itemize}

\end{itemize}



\subsubsection{B01: What is the identification strategy (in one

sentence)?}\label{b01-what-is-the-identification-strategy-in-one-sentence}



\begin{itemize}

\tightlist

\item

  Category: \texttt{B)\ Identification\ strategy\ /\ causal\ design}

\item

  Confidence: \texttt{low} (0.27741585874335584)

\item

  Retrieval method: \texttt{local}

\item

  Evidence type: \texttt{direct\ description\ from\ text}

\item

  Assumption flag: \texttt{False}

\item

  Answer:

\end{itemize}



Detect variance change points by computing the centered cumulative sum

of squares Dk, flagging k* where \textbar Dk\textbar{} crosses

asymptotic boundaries, then iteratively splitting the series at k* and

reapplying on subsegments with adjacent-point checks to fine-tune until

convergence (ICSS).



\begin{itemize}

\tightlist

\item

  Quote snippet:

\end{itemize}



\begin{quote}

A solution is an iterative scheme based on successive application of Dk

to pieces of the series, dividing consecutively after a possible

changepoint is found.

\end{quote}



\begin{itemize}

\tightlist

\item

  Citation anchors:



  \begin{itemize}

  \tightlist

  \item

    page=5, words=825-1174, section=ICSS algorithm steps, note=Iterative

    scheme applying Dk to pieces, dividing after a possible changepoint

    is found.

  \item

    page=6, words=550-899, section=Fine-tuning and convergence,

    note=Systematic search from extremes toward middle; fine-tune by

    checking each point given adjacent ones until convergence.

  \item

    page=3, words=275-624, section=Dk boundaries, note=Dk exhibits

    excursions beyond specified boundaries derived from asymptotic

    distribution under constant variance.

  \item

    page=2, words=0-349, section=Introduction, note=Procedure based on

    an iterated cumulative sums of squares (ICSS) algorithm.

  \end{itemize}

\end{itemize}



\subsubsection{B02: Is the design experimental, quasi-experimental, or

observational?}\label{b02-is-the-design-experimental-quasi-experimental-or-observational}



\begin{itemize}

\tightlist

\item

  Category: \texttt{B)\ Identification\ strategy\ /\ causal\ design}

\item

  Confidence: \texttt{low} (0.26840387625640166)

\item

  Retrieval method: \texttt{local}

\item

  Evidence type: \texttt{text}

\item

  Data source: Inclan \& Tiao (1994), Journal of the American

  Statistical Association

\item

  Assumption flag: \texttt{False}

\item

  Answer:

\end{itemize}



Experimental (simulation experiment)



\begin{itemize}

\tightlist

\item

  Quote snippet:

\end{itemize}



\begin{quote}

The simulation experiment has two separate parts, the first for one

changepoint and the second for two changepoints... For each part, we

used 1,000 replicates per design point

\end{quote}



\begin{itemize}

\tightlist

\item

  Citation anchors:



  \begin{itemize}

  \tightlist

  \item

    page=7, words=0-349, section=4. SIMULATION EXPERIMENT,

    note=Introduces and describes the simulation experiment design.

  \item

    page=7, words=275-556, section=4.1 and setup details, note=Details

    controlled simulation with replicates and generated data.

  \end{itemize}

\end{itemize}



\subsubsection{B03: What is the source of exogenous variation used for

identification?}\label{b03-what-is-the-source-of-exogenous-variation-used-for-identification}



\begin{itemize}

\tightlist

\item

  Category: \texttt{B)\ Identification\ strategy\ /\ causal\ design}

\item

  Confidence: \texttt{low} (0.3252492401531617)

\item

  Retrieval method: \texttt{local}

\item

  Evidence type: \texttt{textual}

\item

  Data source: Inclán \& Tiao (1994), Journal of the American

  Statistical Association

\item

  Assumption flag: \texttt{False}

\item

  Answer:

\end{itemize}



Not applicable. The paper is methodological and does not use exogenous

variation for identification; it detects variance changes via the ICSS

cumulative sums of squares approach.



\begin{itemize}

\tightlist

\item

  Quote snippet:

\end{itemize}



\begin{quote}

We study the detection of multiple changes of variance in a sequence of

independent observations. We propose a procedure to detect variance

changes based on an ICSS algorithm.

\end{quote}



\begin{itemize}

\tightlist

\item

  Citation anchors:



  \begin{itemize}

  \tightlist

  \item

    page=2, words=0-349, section=Introduction, note=States the study

    focuses on detecting multiple variance changes using ICSS, not on

    causal identification.

  \item

    page=2, words=550-899, section=Centered cumulative sums of squares,

    note=Defines the centered cumulative sum of squares used for

    detection.

  \item

    page=3, words=275-624, section=Methodological details,

    note=Discusses testing for a change in variance and likelihood

    ratio---again methodological, not exogenous variation.

  \end{itemize}

\end{itemize}



\subsubsection{B04: What is the treatment definition and

timing?}\label{b04-what-is-the-treatment-definition-and-timing}



\begin{itemize}

\tightlist

\item

  Category: \texttt{B)\ Identification\ strategy\ /\ causal\ design}

\item

  Confidence: \texttt{low} (0.12010050431450066)

\item

  Retrieval method: \texttt{local}

\item

  Evidence type: \texttt{inferred\_from\_context}

\item

  Data source: Inclán \& Tiao (1994), Use of Cumulative Sums of Squares

  for Retrospective Detection of Changes of Variance, JASA 89(427):

  913-923.

\item

  Assumption flag: \texttt{False}

\item

  Answer:

\end{itemize}



Not applicable---the article is a methodological study (ICSS for

detecting variance change points) and does not define a treatment or its

timing.



\begin{itemize}

\tightlist

\item

  Quote snippet:

\end{itemize}



\begin{quote}

This article studies the problem of multiple change points in the

variance of a sequence of independent observations.

\end{quote}



\begin{itemize}

\tightlist

\item

  Citation anchors:



  \begin{itemize}

  \tightlist

  \item

    page=2, words=0-349, section=Introduction, note=States study focus

    on detecting variance change points; no treatment is described.

  \item

    page=11, words=0-349, section=Summary/Appendix, note=Recommends ICSS

    for analyzing series with multiple change points; still no treatment

    definition.

  \end{itemize}

\end{itemize}



\subsubsection{B05: What is the control/comparison group

definition?}\label{b05-what-is-the-controlcomparison-group-definition}



\begin{itemize}

\tightlist

\item

  Category: \texttt{B)\ Identification\ strategy\ /\ causal\ design}

\item

  Confidence: \texttt{low} (0.1971543586339832)

\item

  Retrieval method: \texttt{local}

\item

  Evidence type: \texttt{text}

\item

  Data source:

  \url{http://links.jstor.org/sici?sici=0162-1459\%28199409\%2989\%3A427\%3C913\%3AUOCSOS\%3E2.0.CO\%3B2-H}

\item

  Answer:

\end{itemize}



There is no traditional control group; performance is compared by

applying alternative procedures (likelihood ratio tests and a Bayesian

approach) alongside ICSS to the same simulated series for paired

comparisons.



\begin{itemize}

\tightlist

\item

  Quote snippet:

\end{itemize}



\begin{quote}

The same series is subject to each of the procedures considered to

facilitate paired comparisons between procedures.

\end{quote}



\begin{itemize}

\tightlist

\item

  Citation anchors:



  \begin{itemize}

  \tightlist

  \item

    page=7, words=40-140, section=4. SIMULATION EXPERIMENT, note=States

    the same series is subjected to each procedure for paired

    comparisons.

  \item

    page=7, words=0-60, section=4. SIMULATION EXPERIMENT,

    note=Introduces simulation comparing ICSS with likelihood ratio and

    a Bayesian approach.

  \item

    page=2, words=0-120, section=Abstract/Intro, note=Mentions

    simulation results comparing ICSS to other approaches.

  \end{itemize}

\end{itemize}



\subsubsection{B06: What is the estimating equation / baseline

regression

specification?}\label{b06-what-is-the-estimating-equation--baseline-regression-specification}



\begin{itemize}

\tightlist

\item

  Category: \texttt{B)\ Identification\ strategy\ /\ causal\ design}

\item

  Confidence: \texttt{low} (0.2852559821489955)

\item

  Retrieval method: \texttt{local}

\item

  Evidence type: \texttt{text\_and\_formula}

\item

  Data source: Inclán and Tiao (1994), Journal of the American

  Statistical Association

\item

  Assumption flag: \texttt{True}

\item

  Assumption notes: No regression is estimated; the method assumes a

  zero-mean Normal series with piecewise-constant variance and uses the

  centered cumulative sum of squares Dk for detection.

\item

  Answer:

\end{itemize}



Baseline model: a\_t are independent Normal(0, σ\_t\^{}2) with

piecewise-constant variance. Estimating equation/statistic: D\_k =

C\_k/C\_T − k/T, where C\_k = ∑\_\{i=1\}\^{}k a\_i\^{}2; inference is

based on max\_k \textbar D\_k\textbar.



\begin{itemize}

\tightlist

\item

  Quote snippet:

\end{itemize}



\begin{quote}

Let Dk = Ck/Cn - kin, where Ck = Cfi=la:.

\end{quote}



\begin{itemize}

\tightlist

\item

  Citation anchors:



  \begin{itemize}

  \tightlist

  \item

    page=5, words=60-130, section=Theorem 1, note=Definition of Dk and

    Ck

  \item

    page=3, words=120-200, section=Main text, note=Assumption: \{a\_t\}

    Normally distributed with mean 0 and variances σ\_t\^{}2

  \item

    page=11, words=430-500, section=Main text, note=Under variance

    homogeneity, E{[}Dk{]} → 0

  \item

    page=2, words=860-910, section=Intro/overview, note=Dk oscillates

    around 0 for homogeneous variance

  \end{itemize}

\end{itemize}



\subsubsection{B07: What fixed effects are included (unit, time,

two-way, higher

dimensional)?}\label{b07-what-fixed-effects-are-included-unit-time-two-way-higher-dimensional}



\begin{itemize}

\tightlist

\item

  Category: \texttt{B)\ Identification\ strategy\ /\ causal\ design}

\item

  Confidence: \texttt{low} (0.2763758758257676)

\item

  Retrieval method: \texttt{local}

\item

  Evidence type: \texttt{textual\ absence\ within\ provided\ context}

\item

  Data source: Inclán and Tiao (1994), Journal of the American

  Statistical Association

\item

  Assumption flag: \texttt{False}

\item

  Answer:

\end{itemize}



None (not applicable). The context discusses variance-change detection

in time series and does not include unit, time, two-way, or

higher-dimensional fixed effects.



\begin{itemize}

\tightlist

\item

  Quote snippet:

\end{itemize}



\begin{quote}

This behavior leads to the search for a variance change point via maxk I

Dk I.

\end{quote}



\begin{itemize}

\tightlist

\item

  Citation anchors:



  \begin{itemize}

  \tightlist

  \item

    page=3, words=550-882, section=Discussion of Dk for variance change,

    note=Focuses on Dk and detecting variance change points; no mention

    of fixed effects.

  \item

    page=6, words=550-899, section=IBM example and ICSS algorithm,

    note=Applies ICSS to time series variance changes; no fixed-effects

    framework referenced.

  \end{itemize}

\end{itemize}



\subsubsection{B08: What standard errors are used (robust, clustered; at

what

level)?}\label{b08-what-standard-errors-are-used-robust-clustered-at-what-level}



\begin{itemize}

\tightlist

\item

  Category: \texttt{B)\ Identification\ strategy\ /\ causal\ design}

\item

  Confidence: \texttt{low} (0.3315274087668142)

\item

  Retrieval method: \texttt{local}

\item

  Evidence type: \texttt{textual}

\item

  Table/Figure:

  \texttt{Table\ 1\ and\ Table\ 2\ (SE\ formulas\ described\ in\ text)}

\item

  Assumption flag: \texttt{False}

\item

  Answer:

\end{itemize}



Standard errors are Monte Carlo (simulation-based), not robust or

clustered. For quantile estimates they use SE ≈

sqrt{[}p(1−p){]}/(sqrt(n)·f), with n=10,000 and f the estimated density

at the pth quantile; for proportions they use SE = sqrt(p̂(1−p̂)/10,000).

No clustering level applies.



\begin{itemize}

\tightlist

\item

  Quote snippet:

\end{itemize}



\begin{quote}

The standard errors were obtained as dp(1 - p)/nf 2, where n = 10,000 is

the number of replicates and f is the density at the pth quantile... The

standard errors in this table are obtained as SE = Vb(l - \$)/ 10,000

\end{quote}



\begin{itemize}

\tightlist

\item

  Citation anchors:



  \begin{itemize}

  \tightlist

  \item

    page=5, words=275-450, section=Discussion of Table 1 quantiles,

    note=SE formula for quantile estimates from simulations

  \item

    page=5, words=450-624, section=Discussion of Table 2 proportions,

    note=SE formula for proportion of series meeting criterion

  \end{itemize}

\end{itemize}



\subsubsection{B09: What is the key identifying assumption (parallel

trends, exclusion restriction,

ignorability)?}\label{b09-what-is-the-key-identifying-assumption-parallel-trends-exclusion-restriction-ignorability}



\begin{itemize}

\tightlist

\item

  Category: \texttt{B)\ Identification\ strategy\ /\ causal\ design}

\item

  Confidence: \texttt{low} (0.2725651917028813)

\item

  Retrieval method: \texttt{local}

\item

  Evidence type: \texttt{text}

\item

  Assumption flag: \texttt{False}

\item

  Answer:

\end{itemize}



Not specified in the provided context; the document concerns

variance-change detection (ICSS, likelihood ratio) rather than causal

identification, so none of parallel trends, exclusion restriction, or

ignorability is stated.



\begin{itemize}

\tightlist

\item

  Quote snippet:

\end{itemize}



\begin{quote}

The performance of different procedures used to determine the number of

variance changes in a series can be measured

\end{quote}



\begin{itemize}

\tightlist

\item

  Citation anchors:



  \begin{itemize}

  \tightlist

  \item

    page=7, words=0-60, section=4. SIMULATION EXPERIMENT, note=Context

    frames the paper around detecting variance changes, not causal

    inference assumptions.

  \item

    page=4, words=275-350, section=2.3 Asymptotic Behavior of Dk Under

    Homogeneous Variance, note=Asymptotic discussion of Dk (Brownian

    bridge) indicates focus on statistical testing for variance changes.

  \end{itemize}

\end{itemize}



\subsubsection{B10: What evidence is provided to support the identifying

assumption?}\label{b10-what-evidence-is-provided-to-support-the-identifying-assumption}



\begin{itemize}

\tightlist

\item

  Category: \texttt{B)\ Identification\ strategy\ /\ causal\ design}

\item

  Confidence: \texttt{low} (0.2664735476513488)

\item

  Retrieval method: \texttt{local}

\item

  Evidence type: \texttt{theoretical\ +\ simulation}

\item

  Table/Figure: \texttt{Table\ 2}

\item

  Data source: 10,000 Monte Carlo replicates of independent N(0,1)

  series; asymptotic theory (Theorem 1)

\item

  Assumption flag: \texttt{True}

\item

  Assumption notes: Interprets the \textquotesingle identifying

  assumption\textquotesingle{} as (i) the null distribution of Dk being

  Brownian-bridge and (ii) max\textbar Dk\textbar{} locating the true

  variance-change point.

\item

  Answer:

\end{itemize}



They provide both theory and simulations: Theorem 1 shows the centered

cumulative sum of squares converges to a Brownian bridge under

homoscedastic Normal noise, supporting use of Brownian-bridge critical

values; and Monte Carlo evidence (10,000 N(0,1) series) shows the

empirical probability that max\textbar Dk\textbar{} is below the 1.358

critical value is ≈95\%. They also report that the mode of k* (where

max\textbar Dk\textbar{} occurs) is at the true variance-change point

and concentrates there as T or the variance ratio increases.



\begin{itemize}

\tightlist

\item

  Quote snippet:

\end{itemize}



\begin{quote}

What makes the ICSS algorithm work well is that the mode of k* is

exactly at the point where the change in variance occurs.

\end{quote}



\begin{itemize}

\tightlist

\item

  Citation anchors:



  \begin{itemize}

  \tightlist

  \item

    page=5, words=140-310, section=Theorem 1, note=Asymptotic null: Dk

    converges to a Brownian bridge (W0).

  \item

    page=5, words=0-120, section=Table 2, note=Monte Carlo: percentage

    with max\textbar Dk\textbar{} \textless{} 1.358 near 95\% (10,000

    N(0,1) replicates).

  \item

    page=9, words=360-500, section=4.4, note=Distribution of k*: mode

    exactly at the true change point; concentration increases with T and

    variance ratio.

  \end{itemize}

\end{itemize}



\subsubsection{B11: Are there event-study or pre-trend tests? What do

they

show?}\label{b11-are-there-event-study-or-pre-trend-tests-what-do-they-show}



\begin{itemize}

\tightlist

\item

  Category: \texttt{B)\ Identification\ strategy\ /\ causal\ design}

\item

  Confidence: \texttt{medium} (0.38895839421501494)

\item

  Retrieval method: \texttt{local}

\item

  Evidence type: \texttt{textual\ description}

\item

  Data source: Inclán and Tiao (1994), JASA 89(427): 913--923

\item

  Assumption flag: \texttt{True}

\item

  Assumption notes: Within the provided excerpts, there is no mention of

  event-study or pre-trend tests; content centers on ICSS variance

  change detection, simulations, and an IBM stock example.

\item

  Answer:

\end{itemize}



No. The article does not report event-study or pre-trend tests; it

focuses on detecting variance changes using the ICSS algorithm and

compares it with likelihood ratio and Bayesian methods.



\begin{itemize}

\tightlist

\item

  Quote snippet:

\end{itemize}



\begin{quote}

We propose a procedure to detect variance changes based on an iterated

cumulative sums of squares (ICSS) algorithm.

\end{quote}



\begin{itemize}

\tightlist

\item

  Citation anchors:



  \begin{itemize}

  \tightlist

  \item

    page=2, words=0-349, section=Introduction/scope, note=States the

    paper's focus on ICSS for variance change detection.

  \item

    page=2, words=275-624, section=Method overview, note=Describes

    cumulative sums of squares approach and paper organization; no

    mention of event-study or pre-trends.

  \item

    page=7, words=0-349, section=Simulation experiment, note=Details

    simulations comparing ICSS to likelihood ratio and Bayesian

    approaches; no event-study or pre-trend tests.

  \item

    page=6, words=275-624, section=Empirical example, note=IBM stock

    price example analyzed via ICSS; no event-study framework discussed.

  \end{itemize}

\end{itemize}



\subsubsection{B12: What instruments are used (if IV)? Define instrument

and first

stage.}\label{b12-what-instruments-are-used-if-iv-define-instrument-and-first-stage}



\begin{itemize}

\tightlist

\item

  Category: \texttt{B)\ Identification\ strategy\ /\ causal\ design}

\item

  Confidence: \texttt{low} (0.1772272201535017)

\item

  Retrieval method: \texttt{local}

\item

  Evidence type: \texttt{text}

\item

  Data source: Provided context excerpts from Inclán \& Tiao on ICSS and

  variance change detection

\item

  Assumption flag: \texttt{True}

\item

  Assumption notes: No IV terminology or regression framework appears in

  the provided excerpts; inferred that instruments and first-stage

  definitions are not applicable to this method-focused variance-change

  study.

\item

  Answer:

\end{itemize}



Not applicable. The provided context describes the ICSS algorithm for

detecting variance changes in time series and does not use instrumental

variables (IV). Therefore, there are no instruments or first-stage

regressions to define in this study.



\begin{itemize}

\tightlist

\item

  Quote snippet:

\end{itemize}



\begin{quote}

Iterated Cumulative Sums of Squares (ICSS) Algorithm

\end{quote}



\begin{itemize}

\tightlist

\item

  Citation anchors:



  \begin{itemize}

  \tightlist

  \item

    page=5, words=900-1120, section=3. Detections of Multiple Changes:

    The Iterated Cumulative Sums of Squares Algorithm, note=Introduces

    and details the ICSS algorithm; no mention of IV or instruments.

  \item

    page=6, words=0-120, section=ICSS discussion and convergence,

    note=Describes algorithmic convergence and usage; context is

    variance-change diagnostics, not IV.

  \item

    page=6, words=550-880, section=Procedure refinement and example,

    note=Focuses on systematic search and diagnostics for variance

    changes; no IV framework.

  \end{itemize}

\end{itemize}



\subsubsection{B13: What is the first-stage strength (F-stat, partial

R\^{}2, relevance

evidence)?}\label{b13-what-is-the-first-stage-strength-f-stat-partial-r2-relevance-evidence}



\begin{itemize}

\tightlist

\item

  Category: \texttt{B)\ Identification\ strategy\ /\ causal\ design}

\item

  Confidence: \texttt{low} (0.3120605828645241)

\item

  Retrieval method: \texttt{local}

\item

  Evidence type: \texttt{textual}

\item

  Data source: Inclán and Tiao (1994), Journal of the American

  Statistical Association

\item

  Assumption flag: \texttt{False}

\item

  Answer:

\end{itemize}



Not available in the provided context. The text does not report any

first-stage regression results (e.g., IV first-stage F-statistic,

partial R\^{}2, or other relevance evidence). The only F-statistic

discussed pertains to testing equality of variances in variance-change

detection, not instrument relevance.



\begin{itemize}

\tightlist

\item

  Quote snippet:

\end{itemize}



\begin{quote}

For a fixed k, the value of Dk can be written as a function of the usual

F statistic for testing equality of variances... F\_\{T−k,k\} = ((CT −

Ck)/(T − k))/(Ck/k).

\end{quote}



\begin{itemize}

\tightlist

\item

  Citation anchors:



  \begin{itemize}

  \tightlist

  \item

    page=3, words=0-349, section=2.1 Relationship of Dk to the F

    Statistic and the Likelihood Ratio, note=F-stat here is for equality

    of variances, not IV first-stage strength.

  \end{itemize}

\end{itemize}



\subsubsection{B14: If RDD: what is the running variable and cutoff?

bandwidth

choice?}\label{b14-if-rdd-what-is-the-running-variable-and-cutoff-bandwidth-choice}



\begin{itemize}

\tightlist

\item

  Category: \texttt{B)\ Identification\ strategy\ /\ causal\ design}

\item

  Confidence: \texttt{low} (0.3424343411773066)

\item

  Retrieval method: \texttt{local}

\item

  Evidence type: \texttt{text}

\item

  Table/Figure: \texttt{Figure\ 4}

\item

  Data source: Inclán and Tiao (1994), Journal of the American

  Statistical Association

\item

  Assumption flag: \texttt{False}

\item

  Answer:

\end{itemize}



Not applicable. The study uses ICSS/change-point detection for variance

shifts, not a regression discontinuity design; thus there is no running

variable, no cutoff, and no bandwidth choice.



\begin{itemize}

\tightlist

\item

  Quote snippet:

\end{itemize}



\begin{quote}

we will be looking for maxk \textbar Dk\textbar{} to determine the

location of the change point.

\end{quote}



\begin{itemize}

\tightlist

\item

  Citation anchors:



  \begin{itemize}

  \tightlist

  \item

    page=3, words=0-349, section=Section 2.1, note=Method looks for

    max\textbar Dk\textbar{} to locate a change point, not an RDD setup.

  \item

    page=3, words=275-624, section=Sections 2.1--2.2, note=Focus on

    testing variance changes and using Dk/CUSUM of squares.

  \item

    page=6, words=275-624, section=Section 3.1, note=ICSS algorithm

    performs a systematic search for points of change.

  \item

    page=9, words=275-624, section=Section 4.4, note=Discussion of k*

    where max\textbar Dk\textbar{} is attained; confirms change-point

    framework.

  \end{itemize}

\end{itemize}



\subsubsection{B15: If DiD: what is the timing variation (staggered

adoption)? estimator

used?}\label{b15-if-did-what-is-the-timing-variation-staggered-adoption-estimator-used}



\begin{itemize}

\tightlist

\item

  Category: \texttt{B)\ Identification\ strategy\ /\ causal\ design}

\item

  Confidence: \texttt{medium} (0.41074100604643277)

\item

  Retrieval method: \texttt{local}

\item

  Evidence type: \texttt{text}

\item

  Table/Figure: \texttt{Table\ 1}

\item

  Data source: Journal of the American Statistical Association,

  September 1994 (Inclán \& Tiao)

\item

  Assumption flag: \texttt{False}

\item

  Assumption notes: The provided context covers variance change-point

  detection (ICSS), not difference-in-differences.

\item

  Answer:

\end{itemize}



Not applicable. The study is not a DiD design; it detects variance

change points using the ICSS (Iterative Cumulative Sum of Squares)

algorithm, estimating changepoints via max\_k \textbar Dk\textbar{} with

Brownian-bridge critical values (Table 1). There is no staggered

adoption or DiD estimator.



\begin{itemize}

\tightlist

\item

  Quote snippet:

\end{itemize}



\begin{quote}

This behavior leads to the search for a variance change point via maxk I

Dk I . Let k* be the value of k at which maxk I Dk I is attained.

\end{quote}



\begin{itemize}

\tightlist

\item

  Citation anchors:



  \begin{itemize}

  \tightlist

  \item

    page=3, words=550-882, section=Method: change-point search,

    note=Defines search via max\_k \textbar Dk\textbar{} to locate

    variance change points.

  \item

    page=5, words=0-349, section=Asymptotics and boundaries, note=Dk

    -\textgreater{} Brownian bridge; Table 1 gives quantiles for

    decision thresholds.

  \item

    page=5, words=1100-1198, section=Algorithm steps, note=Uses boundary

    D* from Table 1 to declare changepoints.

  \item

    page=6, words=0-349, section=ICSS algorithm usage, note=Describes

    the ICSS algorithm and its purpose in detecting variance changes.

  \end{itemize}

\end{itemize}



\subsubsection{C01: What dataset(s) are used? (name sources

explicitly)}\label{c01-what-datasets-are-used-name-sources-explicitly}



\begin{itemize}

\tightlist

\item

  Category: \texttt{C)\ Data,\ sample,\ and\ measurement}

\item

  Confidence: \texttt{low} (0.26690114886370353)

\item

  Retrieval method: \texttt{local}

\item

  Evidence type: \texttt{direct\_quote\_and\_summary}

\item

  Data source: Box and Jenkins (1976); author-generated simulated series

  and AR(1) residuals

\item

  Assumption flag: \texttt{False}

\item

  Answer:

\end{itemize}



The paper uses: (1) a real dataset of IBM stock closing prices (analyzed

as first differences of log prices) from May 17, 1961 to November 2,

1962, as reported by Box and Jenkins (1976); and (2) simulated time

series (and AR(1) residuals) generated by the authors for the simulation

experiments.



\begin{itemize}

\tightlist

\item

  Quote snippet:

\end{itemize}



\begin{quote}

The series analyzed is the first difference of the logarithm of the IBM

stock closing prices ... as reported by Box and Jenkins (1976).

\end{quote}



\begin{itemize}

\tightlist

\item

  Citation anchors:



  \begin{itemize}

  \tightlist

  \item

    page=6, words=275-624, section=3.1 Example: IBM Stock Prices,

    note=Describes the IBM stock prices dataset and cites Box and

    Jenkins (1976).

  \item

    page=7, words=0-349, section=4. SIMULATION EXPERIMENT, note=States

    that simulation experiments were conducted with generated series.

  \item

    page=9, words=550-899, section=4.5 The ICSS Algorithm Applied to

    Residuals, note=Details application to AR(1) residuals from

    simulated series.

  \end{itemize}

\end{itemize}



\subsubsection{C02: What is the unit of observation (individual,

household, firm, county, transaction,

product)?}\label{c02-what-is-the-unit-of-observation-individual-household-firm-county-transaction-product}



\begin{itemize}

\tightlist

\item

  Category: \texttt{C)\ Data,\ sample,\ and\ measurement}

\item

  Confidence: \texttt{low} (0.24640099422268022)

\item

  Retrieval method: \texttt{local}

\item

  Evidence type: \texttt{text}

\item

  Data source: IBM stock closing prices (Box and Jenkins, 1976)

\item

  Assumption flag: \texttt{True}

\item

  Assumption notes: Unit inferred from the empirical example focused on

  IBM, a single firm; other parts use simulated series.

\item

  Answer:

\end{itemize}



firm



\begin{itemize}

\tightlist

\item

  Quote snippet:

\end{itemize}



\begin{quote}

The series analyzed is the first difference of the logarithm of the IBM

stock closing prices from May 17, 196 1, to No- vember 2, 1962, as

reported by Box and Jenkins (1976).

\end{quote}



\begin{itemize}

\tightlist

\item

  Citation anchors:



  \begin{itemize}

  \tightlist

  \item

    page=6, words=275-624, section=3.1 Example: IBM Stock Prices,

    note=Empirical example analyzes IBM stock closing prices (a single

    company\textquotesingle s series).

  \end{itemize}

\end{itemize}



\subsubsection{C03: What is the sample period and geographic

coverage?}\label{c03-what-is-the-sample-period-and-geographic-coverage}



\begin{itemize}

\tightlist

\item

  Category: \texttt{C)\ Data,\ sample,\ and\ measurement}

\item

  Confidence: \texttt{low} (0.268446744281151)

\item

  Retrieval method: \texttt{local}

\item

  Evidence type: \texttt{article\_text}

\item

  Data source: Box and Jenkins (1976) IBM stock closing prices

\item

  Assumption flag: \texttt{False}

\item

  Assumption notes: Geographic coverage is not specified in the provided

  context.

\item

  Answer:

\end{itemize}



Sample period: May 17, 1961 to November 2, 1962. Geographic coverage:

Not specified in the provided context (IBM stock closing prices

dataset).



\begin{itemize}

\tightlist

\item

  Quote snippet:

\end{itemize}



\begin{quote}

The series analyzed is the first difference of the logarithm of the IBM

stock closing prices from May 17, 1961, to November 2, 1962, as reported

by Box and Jenkins (1976).

\end{quote}



\begin{itemize}

\tightlist

\item

  Citation anchors:



  \begin{itemize}

  \tightlist

  \item

    page=6, words=275-624, section=3.1 Example: IBM Stock Prices,

    note=States the IBM stock series dates: May 17, 1961 to November 2,

    1962.

  \end{itemize}

\end{itemize}



\subsubsection{C04: What are the sample restrictions / inclusion

criteria?}\label{c04-what-are-the-sample-restrictions--inclusion-criteria}



\begin{itemize}

\tightlist

\item

  Category: \texttt{C)\ Data,\ sample,\ and\ measurement}

\item

  Confidence: \texttt{low} (0.23034433773356772)

\item

  Retrieval method: \texttt{local}

\item

  Evidence type: \texttt{textual\ description\ (methods)}

\item

  Assumption flag: \texttt{True}

\item

  Assumption notes: Interprets \textquotesingle sample\textquotesingle{}

  as the simulated series used in the study (not an observational

  dataset).

\item

  Answer:

\end{itemize}



Simulated time-series only. Inclusion criteria: (1) innovations assumed

Normal with mean 0 and piecewise-constant variance; (2) series lengths T

in \{100, 200, 500\}; (3) one-change cases use K1 at \{0.25T, 0.50T,

0.75T\} with variance ratio A in \{2, 3\}; (4) two-change cases use (K1,

K2) in \{(.33T, .66T), (.20T, .80T), (.58T, .80T)\} with base variance

set to 1 and varying ratios A1 = σ1\^{}2/σ0\^{}2 and A2 =

σ2\^{}2/σ1\^{}2; symmetric variance permutations excluded; (5) the same

generated series are subjected to ICSS, likelihood ratio, and Bayesian

procedures for paired comparisons; (6) additional runs apply ICSS to

residuals from AR(1) models fitted to the same simulated series.



\begin{itemize}

\tightlist

\item

  Quote snippet:

\end{itemize}



\begin{quote}

The same series is subject to each of the procedures considered to

facilitate paired comparisons between procedures.

\end{quote}



\begin{itemize}

\tightlist

\item

  Citation anchors:



  \begin{itemize}

  \tightlist

  \item

    page=7, words=0-349, section=4. SIMULATION EXPERIMENT, note=Defines

    T, change-point locations, variance ratios; same series used across

    procedures; variance permutations and base variance constraint.

  \item

    page=3, words=0-349, section=2.1 Relationship of Dk to the F

    Statistic and the Likelihood Ratio, note=States the Normal, mean-0

    assumption for \{a\_t\}.

  \item

    page=9, words=550-899, section=4.5 The ICSS Algorithm Applied to

    Residuals, note=ICSS applied to residuals from AR(1) fitted to the

    same series.

  \end{itemize}

\end{itemize}



\subsubsection{C05: What is the sample size (N) in the main

analysis?}\label{c05-what-is-the-sample-size-n-in-the-main-analysis}



\begin{itemize}

\tightlist

\item

  Category: \texttt{C)\ Data,\ sample,\ and\ measurement}

\item

  Confidence: \texttt{low} (0.29216650868207844)

\item

  Retrieval method: \texttt{local}

\item

  Evidence type: \texttt{text}

\item

  Table/Figure: \texttt{Figure\ 1}

\item

  Assumption flag: \texttt{True}

\item

  Assumption notes: Interpreted \textquotesingle main

  analysis\textquotesingle{} as the primary empirical series analyzed

  around Figure 1/Section 3, where indices run k=1,...,700.

\item

  Answer:

\end{itemize}



700



\begin{itemize}

\tightlist

\item

  Quote snippet:

\end{itemize}



\begin{quote}

Dk plot k=1,....700

\end{quote}



\begin{itemize}

\tightlist

\item

  Citation anchors:



  \begin{itemize}

  \tightlist

  \item

    page=3, words=200-230, section=Figure 1 caption/context, note=Shows

    Dk plot indexed k=1,...,700, indicating series length 700.

  \item

    page=6, words=820-870, section=ICSS algorithm discussion,

    note=Refers to Dk(a{[}398: 700{]}), confirming T ends at 700.

  \end{itemize}

\end{itemize}



\subsubsection{C06: How is the key outcome measured? Any transformations

(logs, z-scores,

indices)?}\label{c06-how-is-the-key-outcome-measured-any-transformations-logs-z-scores-indices}



\begin{itemize}

\tightlist

\item

  Category: \texttt{C)\ Data,\ sample,\ and\ measurement}

\item

  Confidence: \texttt{low} (0.2937680977760846)

\item

  Retrieval method: \texttt{local}

\item

  Evidence type:

  \texttt{Direct\ quotes\ and\ contextual\ summary\ from\ article\ text}

\item

  Table/Figure:

  \texttt{Table\ 1\ (Empirical\ and\ Asymptotic\ Quantiles\ of\ max\ \textbar{}Dk\textbar{})}

\item

  Data source: Inclán and Tiao (1994), Journal of the American

  Statistical Association

\item

  Answer:

\end{itemize}



The key outcome is the variance-change statistic Dk, defined as a

centered and normalized cumulative sum of squares of the series.

Evidence for a change is assessed by maxk \textbar Dk\textbar{}

exceeding asymptotic Brownian-bridge boundaries (e.g., Table 1), with

the change-point estimated at k* where the maximum occurs.

Transformations used include: (1) the intrinsic centering/normalization

in Dk; (2) applying ICSS to transformed data such as the first

difference of log prices (IBM example); and (3) applying ICSS to

residuals from an AR(1) fit to remove autocorrelation.



\begin{itemize}

\tightlist

\item

  Quote snippet:

\end{itemize}



\begin{quote}

the centered (and normalized) cumulative sum of squares.

\end{quote}



\begin{itemize}

\tightlist

\item

  Citation anchors:



  \begin{itemize}

  \tightlist

  \item

    page=2, words=825-916, section=Definition/overview, note=Defines Dk

    as the centered and normalized cumulative sum of squares; behavior

    around 0 for homogenous variance.

  \item

    page=3, words=550-882, section=Detection rule and asymptotics,

    note=Decision via max \textbar Dk\textbar{} vs boundary;

    Brownian-bridge behavior; Table 1 quantiles.

  \item

    page=6, words=275-624, section=Example (IBM), note=Uses first

    difference of the logarithm of IBM stock prices.

  \item

    page=9, words=550-899, section=Residuals application, note=Applies

    ICSS to residuals from an AR(1) model.

  \end{itemize}

\end{itemize}



\subsubsection{C07: How is treatment/exposure measured? Any constructed

variables?}\label{c07-how-is-treatmentexposure-measured-any-constructed-variables}



\begin{itemize}

\tightlist

\item

  Category: \texttt{C)\ Data,\ sample,\ and\ measurement}

\item

  Confidence: \texttt{low} (0.2466897861561292)

\item

  Retrieval method: \texttt{local}

\item

  Evidence type: \texttt{textual\_evidence}

\item

  Data source: Methodological simulations with independent N(0,1)

  series; no treatment/exposure variables

\item

  Assumption flag: \texttt{False}

\item

  Answer:

\end{itemize}



Not applicable. This is a methodological/statistical paper; no treatment

or exposure is measured. The study analyzes sequences of independent

observations (often simulated N(0,1) series) to detect variance changes.

Constructed variables/statistics include the centered cumulative sum of

squares Dk = Ck/Cn − k/n (with Ck the cumulative sum of squares), its

linkage to the F statistic, and likelihood ratio statistics standardized

to an extreme value distribution within the ICSS framework.



\begin{itemize}

\tightlist

\item

  Quote snippet:

\end{itemize}



\begin{quote}

Let Dk = Ck/Cn - k/n, where Ck = Cfi=la:.

\end{quote}



\begin{itemize}

\tightlist

\item

  Citation anchors:



  \begin{itemize}

  \tightlist

  \item

    page=2, words=0-349, section=Introduction, note=States focus on

    detecting multiple changes of variance in a sequence of independent

    observations; no treatment/exposure construct.

  \item

    page=5, words=0-349, section=Theorem 1 and definitions, note=Defines

    Dk = Ck/Cn − k/n with Ck as cumulative sum of squares; also notes

    simulations with independent N(0,1) observations.

  \item

    page=3, words=0-349, section=Relationship of Dk to the F Statistic,

    note=Explains Dk as a function of the usual F statistic, reinforcing

    that constructed statistics are used.

  \item

    page=8, words=0-329, section=Likelihood ratio standardization,

    note=Describes standardizing LR statistics to an extreme value

    distribution; another constructed diagnostic.

  \end{itemize}

\end{itemize}



\subsubsection{C08: Are there key covariates/controls? Which ones are

always

included?}\label{c08-are-there-key-covariatescontrols-which-ones-are-always-included}



\begin{itemize}

\tightlist

\item

  Category: \texttt{C)\ Data,\ sample,\ and\ measurement}

\item

  Confidence: \texttt{low} (0.2842639867015846)

\item

  Retrieval method: \texttt{local}

\item

  Evidence type: \texttt{textual}

\item

  Assumption flag: \texttt{False}

\item

  Answer:

\end{itemize}



No. The method analyzes sequences of observations (or residuals) to

detect variance shifts and does not include covariates or control

variables. None are always included.



\begin{itemize}

\tightlist

\item

  Quote snippet:

\end{itemize}



\begin{quote}

This article studies the problem of multiple change points in the

variance of a sequence of independent observations.

\end{quote}



\begin{itemize}

\tightlist

\item

  Citation anchors:



  \begin{itemize}

  \tightlist

  \item

    page=2, words=0-349, section=Introduction, note=States the focus on

    variance changes in a sequence of independent observations, not on

    covariates.

  \item

    page=2, words=550-899, section=2. Centered Cumulative Sums of

    Squares, note=Defines the setup using uncorrelated mean-zero

    variables and cumulative sums of squares; no covariates.

  \item

    page=3, words=275-624, section=Methods/Likelihood formulation,

    note=Assumes \{a\_t\} Normally distributed with mean 0 and

    time-varying variances; focuses on variance change testing.

  \item

    page=10, words=0-84, section=Results (tables), note=Mentions

    application to AR(1) residuals, reinforcing that inputs are residual

    series, not covariates.

  \end{itemize}

\end{itemize}



\subsubsection{C09: How are missing data handled (dropping, imputation,

weighting)?}\label{c09-how-are-missing-data-handled-dropping-imputation-weighting}



\begin{itemize}

\tightlist

\item

  Category: \texttt{C)\ Data,\ sample,\ and\ measurement}

\item

  Confidence: \texttt{low} (0.24359517343422313)

\item

  Retrieval method: \texttt{local}

\item

  Evidence type: \texttt{not\ specified\ in\ context}

\item

  Data source: Inclán and Tiao (1994), Journal of the American

  Statistical Association (provided context)

\item

  Assumption flag: \texttt{False}

\item

  Answer:

\end{itemize}



Not discussed in the provided context; there is no description of

handling missing data (e.g., dropping, imputation, or weighting).



\begin{itemize}

\tightlist

\item

  Quote snippet:

\end{itemize}



\begin{quote}

It is important to look at the plots, because a big outlier would create

a significant peak in the Dk plot... advisable to complement the search

with a procedure for outlier detection.

\end{quote}



\begin{itemize}

\tightlist

\item

  Citation anchors:



  \begin{itemize}

  \tightlist

  \item

    page=6, words=0-349, section=ICSS algorithm discussion / 3.1

    Example: IBM Stock Prices, note=Context discusses outliers and ICSS

    procedure but gives no guidance on missing data handling.

  \end{itemize}

\end{itemize}



\subsubsection{C10: Are weights used (survey weights, propensity

weights)?

How?}\label{c10-are-weights-used-survey-weights-propensity-weights-how}



\begin{itemize}

\tightlist

\item

  Category: \texttt{C)\ Data,\ sample,\ and\ measurement}

\item

  Confidence: \texttt{low} (0.28048630291205834)

\item

  Retrieval method: \texttt{local}

\item

  Evidence type:

  \texttt{Inference\ from\ absence\ in\ provided\ methodological\ description}

\item

  Data source: Inclán and Tiao (1994), Journal of the American

  Statistical Association: Use of Cumulative Sums of Squares for

  Retrospective Detection of Changes of Variance

\item

  Assumption flag: \texttt{True}

\item

  Assumption notes: Weights are not mentioned in the provided excerpts;

  conclusion is based on the absence of any reference to

  survey/propensity weighting and the focus on unweighted cumulative

  sums of squares.

\item

  Answer:

\end{itemize}



No. The methods and simulations described rely on unweighted cumulative

sums of squares (the ICSS algorithm) and related likelihood/Bayesian

comparisons; survey or propensity weights are not mentioned or used.



\begin{itemize}

\tightlist

\item

  Quote snippet:

\end{itemize}



\begin{quote}

We propose a procedure to detect variance changes based on an iterated

cumulative sums of squares (ICSS) algorithm.

\end{quote}



\begin{itemize}

\tightlist

\item

  Citation anchors:



  \begin{itemize}

  \tightlist

  \item

    page=2, words=0-349, section=Introduction, note=Describes ICSS

    approach; no mention of any weighting scheme.

  \item

    page=2, words=825-916, section=Introduction/Method, note=Defines

    centered/normalized cumulative sum of squares Dk; no weights

    referenced.

  \item

    page=7, words=0-349, section=Simulation experiment setup,

    note=Details simulation design and procedures compared; no use of

    survey or propensity weights.

  \end{itemize}

\end{itemize}



\subsubsection{C11: Are data linked/merged across sources? How is

linkage

performed?}\label{c11-are-data-linkedmerged-across-sources-how-is-linkage-performed}



\begin{itemize}

\tightlist

\item

  Category: \texttt{C)\ Data,\ sample,\ and\ measurement}

\item

  Confidence: \texttt{low} (0.20254398734717888)

\item

  Retrieval method: \texttt{local}

\item

  Evidence type: \texttt{direct\_text}

\item

  Table/Figure: \texttt{Table\ 1\ (note)}

\item

  Data source: IBM stock prices (Box and Jenkins, 1976); simulated

  N(0,1) series

\item

  Assumption flag: \texttt{False}

\item

  Answer:

\end{itemize}



No. The analyses use either a single time series (IBM stock prices from

Box and Jenkins, 1976) or simulated independent N(0,1) series. There is

no cross-source linkage or merging performed.



\begin{itemize}

\tightlist

\item

  Quote snippet:

\end{itemize}



\begin{quote}

The series analyzed is the first difference of the logarithm of the IBM

stock closing prices

\end{quote}



\begin{itemize}

\tightlist

\item

  Citation anchors:



  \begin{itemize}

  \tightlist

  \item

    page=6, words=275-624, section=3.1 Example: IBM Stock Prices,

    note=Describes use of a single real-world series (IBM stock prices)

    from a cited source.

  \item

    page=3, words=550-882, section=Table 1 note, note=States results are

    from simulations of independent N(0,1) series.

  \end{itemize}

\end{itemize}



\subsubsection{C12: What summary statistics are reported for main

variables?}\label{c12-what-summary-statistics-are-reported-for-main-variables}



\begin{itemize}

\tightlist

\item

  Category: \texttt{C)\ Data,\ sample,\ and\ measurement}

\item

  Confidence: \texttt{medium} (0.3529576886460369)

\item

  Retrieval method: \texttt{local}

\item

  Evidence type:

  \texttt{Tables\ and\ accompanying\ text\ in\ the\ provided\ excerpts}

\item

  Table/Figure: \texttt{Table\ 1;\ Table\ 2;\ Table\ 7;\ Table\ 8}

\item

  Data source: Inclán and Tiao (1994), Journal of the American

  Statistical Association

\item

  Assumption flag: \texttt{True}

\item

  Assumption notes: Interpreted the \textquotesingle main

  variables\textquotesingle{} as k*, max\textbar Dk\textbar, and ICSS

  detection proportions (NT=1). If a different set of variables was

  intended, the reported summaries may differ.

\item

  Answer:

\end{itemize}



They report: (1) for k*: mode, mean, and standard deviation; (2) for

max\textbar Dk\textbar: empirical and asymptotic quantiles with standard

errors; and (3) detection rates as percentages of series meeting

criteria (e.g., max\textbar Dk\textbar{} \textless{} 1.358) with

standard errors, plus percentages of NT = 1 from ICSS runs.



\begin{itemize}

\tightlist

\item

  Quote snippet:

\end{itemize}



\begin{quote}

Mode of k* Mean of k* SD of k*

\end{quote}



\begin{itemize}

\tightlist

\item

  Citation anchors:



  \begin{itemize}

  \tightlist

  \item

    page=3, words=550-882, section=Table 1, note=Empirical and

    asymptotic quantiles of max\textbar Dk\textbar{} with SEs are

    presented.

  \item

    page=5, words=0-349, section=Table 2, note=Percentages of series

    with max\textbar Dk\textbar{} \textless{} 1.358 and their standard

    errors.

  \item

    page=10, words=0-84, section=Table 7 and Table 8, note=Summaries for

    k* (Mode, Mean, SD) and percentage of NT = 1.

  \end{itemize}

\end{itemize}



\subsubsection{C13: Are there descriptive figures/maps that establish

baseline

patterns?}\label{c13-are-there-descriptive-figuresmaps-that-establish-baseline-patterns}



\begin{itemize}

\tightlist

\item

  Category: \texttt{C)\ Data,\ sample,\ and\ measurement}

\item

  Confidence: \texttt{low} (0.289297076290918)

\item

  Retrieval method: \texttt{local}

\item

  Evidence type: \texttt{textual\_figure\_reference}

\item

  Table/Figure: \texttt{Figure\ 1;\ Figure\ 2;\ Figure\ 3;\ Figure\ 4}

\item

  Assumption flag: \texttt{False}

\item

  Answer:

\end{itemize}



Yes. The paper includes descriptive figures (e.g., Figure 1) that show

baseline patterns under homogeneous variance and illustrate expected Ck

and Dk behaviors, with additional figures for change-point scenarios.



\begin{itemize}

\tightlist

\item

  Quote snippet:

\end{itemize}



\begin{quote}

For the series with homogeneous variance, Ck presents roughly a straight

line with slope a2 = 1.

\end{quote}



\begin{itemize}

\tightlist

\item

  Citation anchors:



  \begin{itemize}

  \tightlist

  \item

    page=3, words=275-624, section=Figure 1 discussion, note=Mentions

    Figure 1 and example series used to illustrate baseline and

    changed-variance patterns.

  \item

    page=3, words=550-882, section=Main text on baseline behavior,

    note=Describes baseline pattern for homogeneous variance (straight

    line for Ck, characteristic Dk behavior).

  \item

    page=4, words=550-869, section=Figure 3 caption/discussion,

    note=References Figure 3 showing expected Dk patterns with specified

    change points.

  \item

    page=6, words=275-624, section=Figure 4 caption/discussion,

    note=References Figure 4 sequence of Dk functions, further

    illustrating patterns.

  \end{itemize}

\end{itemize}



\subsubsection{D01: What is the headline main effect estimate (sign and

magnitude)?}\label{d01-what-is-the-headline-main-effect-estimate-sign-and-magnitude}



\begin{itemize}

\tightlist

\item

  Category:

  \texttt{D)\ Results,\ magnitudes,\ heterogeneity,\ robustness}

\item

  Confidence: \texttt{low} (0.276356738767391)

\item

  Retrieval method: \texttt{local}

\item

  Evidence type: \texttt{text}

\item

  Answer:

\end{itemize}



Not reported in the provided context; no main-effect estimate (sign or

magnitude) is given.



\begin{itemize}

\tightlist

\item

  Quote snippet:

\end{itemize}



\begin{quote}

Note that although LRo,, and max, \textbar Dk\textbar{} are related,

they are not equivalent for finite T.

\end{quote}



\begin{itemize}

\tightlist

\item

  Citation anchors:



  \begin{itemize}

  \tightlist

  \item

    page=3, words=0-349, section=2.1, note=Context discusses Dk,

    F-statistic relation, and critical value, but no main-effect

    coefficient is presented.

  \item

    page=4, words=275-624, section=2.3, note=Explains properties of

    max\textbar Dk\textbar{} and likelihood ratio but does not report

    any main-effect estimate.

  \end{itemize}

\end{itemize}



\subsubsection{D02: What is the preferred specification and why is it

preferred?}\label{d02-what-is-the-preferred-specification-and-why-is-it-preferred}



\begin{itemize}

\tightlist

\item

  Category:

  \texttt{D)\ Results,\ magnitudes,\ heterogeneity,\ robustness}

\item

  Confidence: \texttt{low} (0.22501087583053234)

\item

  Retrieval method: \texttt{local}

\item

  Evidence type: \texttt{text}

\item

  Assumption flag: \texttt{True}

\item

  Assumption notes: Interpreted `preferred specification' as the

  preferred method/approach for variance-change detection in this

  context; the article recommends ICSS for long series with multiple

  changepoints.

\item

  Answer:

\end{itemize}



Use the ICSS algorithm for detecting variance changepoints---especially

in long series with multiple changes---because it has O(T) computational

cost and, in simulations, performs as well or better than likelihood

ratio tests as T grows, while Bayesian posterior odds are

computationally burdensome.



\begin{itemize}

\tightlist

\item

  Quote snippet:

\end{itemize}



\begin{quote}

\ldots lend support to the recommendation to use the ICSS algorithm when

we need to analyze long series with multiple change points.

\end{quote}



\begin{itemize}

\tightlist

\item

  Citation anchors:



  \begin{itemize}

  \tightlist

  \item

    page=11, words=0-349, section=4.6 CPU Time Requirements,

    note=Explicit recommendation to use ICSS for long series with

    multiple change points

  \item

    page=11, words=275-507, section=4.6 CPU Time Requirements, note=CPU

    time order: LR roughly O(T\^{}2) for NT≥2; ICSS dependence

    consistent across NT

  \item

    page=9, words=0-349, section=Results of Part Two (Two Changepoints),

    note=ICSS often better than likelihood ratio tests with ≥200

    observations

  \item

    page=9, words=550-899, section=4.6 CPU Time Requirements (lead-in),

    note=Heavy computation for LR/Bayesian; ICSS avoids intensive

    function evaluations

  \end{itemize}

\end{itemize}



\subsubsection{D03: How economically meaningful is the effect (percent

change, elasticity,

dollars)?}\label{d03-how-economically-meaningful-is-the-effect-percent-change-elasticity-dollars}



\begin{itemize}

\tightlist

\item

  Category:

  \texttt{D)\ Results,\ magnitudes,\ heterogeneity,\ robustness}

\item

  Confidence: \texttt{low} (0.30583567056319455)

\item

  Retrieval method: \texttt{local}

\item

  Evidence type: \texttt{textual}

\item

  Assumption flag: \texttt{False}

\item

  Answer:

\end{itemize}



Not specified. The context describes a statistical method (ICSS) for

detecting variance change points and reports thresholds/quantiles, but

it does not quantify an economic effect in percent change, elasticity,

or dollars.



\begin{itemize}

\tightlist

\item

  Quote snippet:

\end{itemize}



\begin{quote}

This behavior leads to the search for a variance change point via maxk I

Dk I.

\end{quote}



\begin{itemize}

\tightlist

\item

  Citation anchors:



  \begin{itemize}

  \tightlist

  \item

    page=3, words=560-700, section=2. CENTERED CUMULATIVE SUMS OF

    SQUARES, note=Describes searching for variance change points via

    max\textbar Dk\textbar; no economic magnitude reported.

  \item

    page=6, words=275-420, section=3.1 Example: IBM Stock Prices,

    note=Example uses stock returns and reports change points only, not

    economic effect sizes.

  \item

    page=5, words=0-120, section=Asymptotic behavior/quantiles,

    note=Presents asymptotic results and percentages for

    max\textbar Dk\textbar; still methodological, not economic

    magnitudes.

  \end{itemize}

\end{itemize}



\subsubsection{D04: What are the key robustness checks and do results

survive

them?}\label{d04-what-are-the-key-robustness-checks-and-do-results-survive-them}



\begin{itemize}

\tightlist

\item

  Category:

  \texttt{D)\ Results,\ magnitudes,\ heterogeneity,\ robustness}

\item

  Confidence: \texttt{low} (0.33810902789115976)

\item

  Retrieval method: \texttt{local}

\item

  Evidence type: \texttt{Simulation\ and\ empirical\ corroboration}

\item

  Table/Figure: \texttt{Table\ 2;\ Table\ 4;\ Figure\ 4}

\item

  Data source: Simulated N(0,1) series; IBM stock prices (1961--1962)

\item

  Assumption flag: \texttt{False}

\item

  Answer:

\end{itemize}



Key checks: (1) Monte Carlo simulations varying series length,

changepoint location, and variance ratios for both one and two

changepoints; results show correct identification improves with longer T

and larger variance ratios, exceeding 80\% for T≥200 or A=3 when the

change is central, while A=2 with T=100 and early change is harder. (2)

Applying ICSS to AR(1) residuals yields detection rates very close to

the baseline simulation, indicating robustness to modeling residuals.

(3) Real-data corroboration: on IBM returns, ICSS changepoints match

maximum-likelihood estimates and Bayesian posterior modes at the same

points. (4) Type I error check: empirical 95th quantile of

max\textbar Dk\textbar{} is below the asymptotic 1.358, implying

conservative size under the null. (5) Localization property: the mode of

k* is at the true changepoint and concentrates with larger T or variance

ratio. Overall, the main results survive these robustness checks.



\begin{itemize}

\tightlist

\item

  Quote snippet:

\end{itemize}



\begin{quote}

the percentage of series for which NT = 1 is very close to the figures

presented in Table 4.

\end{quote}



\begin{itemize}

\tightlist

\item

  Citation anchors:



  \begin{itemize}

  \tightlist

  \item

    page=7, words=0-349, section=Simulation design (T, K/T, A; one and

    two changepoints), note=Design spans T=100,200,500; K/T=.25,.50,.75;

    A=2,3; and multiple two-changepoint configurations.

  \item

    page=9, words=550-899, section=Simulation results summary,

    note=Performance patterns: harder at T=100, A=2 (early change);

    \textgreater80\% when T≥200 or A=3 in middle; improvement with T and

    A.

  \item

    page=9, words=550-899, section=ICSS applied to AR(1) residuals,

    note=Residuals check shows NT=1 rates very close to Table 4.

  \item

    page=6, words=550-899, section=IBM example; cross-method

    corroboration, note=ICSS points agree with ML; Bayesian posterior

    modes at same points.

  \item

    page=5, words=275-624, section=Type I error/quantiles,

    note=Empirical 95th quantile \textless{} asymptotic 1.358, implying

    smaller empirical type I error.

  \item

    page=9, words=275-624, section=Distribution of k*, note=Mode at true

    change; concentration increases with T and variance ratio.

  \end{itemize}

\end{itemize}



\subsubsection{D05: What placebo tests are run and what do they

show?}\label{d05-what-placebo-tests-are-run-and-what-do-they-show}



\begin{itemize}

\tightlist

\item

  Category:

  \texttt{D)\ Results,\ magnitudes,\ heterogeneity,\ robustness}

\item

  Confidence: \texttt{low} (0.20050309178168896)

\item

  Retrieval method: \texttt{local}

\item

  Evidence type:

  \texttt{simulation\ (Table\ 2)\ and\ asymptotic\ theory}

\item

  Table/Figure: \texttt{Table\ 2}

\item

  Data source: 10,000 simulated IID N(0,1) series per T under no change

\item

  Assumption flag: \texttt{False}

\item

  Answer:

\end{itemize}



Placebo (null) tests simulate IID N(0,1) series with no variance changes

and compute max\_k \textbar Dk\textbar{} against the asymptotic boundary

1.358. Across T=100,200,300,400,500 (10,000 reps each), 97.13\%,

96.51\%, 96.31\%, 96.07\%, and 95.53\% of series stay below 1.358,

showing false positives are near the intended \textasciitilde5\% and

consistent with Brownian-bridge asymptotics.



\begin{itemize}

\tightlist

\item

  Quote snippet:

\end{itemize}



\begin{quote}

NOTE: 10,000 replicates, series of T independent N ( 0 , 1 )

observations.

\end{quote}



\begin{itemize}

\tightlist

\item

  Citation anchors:



  \begin{itemize}

  \tightlist

  \item

    page=5, words=0-349, section=Table 2: Null simulation results,

    note=Percent of series with max\textbar Dk\textbar{} \textless{}

    1.358 for T=100--500; 10,000 replicates of IID N(0,1)

  \item

    page=5, words=0-349, section=Theorem 1, note=Asymptotic result: Dk

    converges to a Brownian bridge, basis for boundary 1.358

  \item

    page=3, words=0-349, section=Figure discussion / critical value,

    note=Asymptotic critical value D.35 = 1.358 and boundary

    interpretation for \textbar Dk\textbar{}

  \end{itemize}

\end{itemize}



\subsubsection{D06: What falsification outcomes are tested (unaffected

outcomes)?}\label{d06-what-falsification-outcomes-are-tested-unaffected-outcomes}



\begin{itemize}

\tightlist

\item

  Category:

  \texttt{D)\ Results,\ magnitudes,\ heterogeneity,\ robustness}

\item

  Confidence: \texttt{low} (0.28755518983303757)

\item

  Retrieval method: \texttt{local}

\item

  Evidence type: \texttt{text}

\item

  Data source: Inclán and Tiao (1994), JASA, provided excerpts

\item

  Assumption flag: \texttt{False}

\item

  Answer:

\end{itemize}



Not specified in the provided context; no falsification (unaffected)

outcomes are discussed or tested.



\begin{itemize}

\tightlist

\item

  Quote snippet:

\end{itemize}



\begin{quote}

The simulation experiment has two separate parts, the first for one

changepoint and the second for two changepoints

\end{quote}



\begin{itemize}

\tightlist

\item

  Citation anchors:



  \begin{itemize}

  \tightlist

  \item

    page=7, words=300-460, section=4. SIMULATION EXPERIMENT,

    note=Context focuses on simulation to detect variance changes; no

    mention of falsification or unaffected outcomes.

  \end{itemize}

\end{itemize}



\subsubsection{D07: What heterogeneity results are reported (by income,

size, baseline exposure,

region)?}\label{d07-what-heterogeneity-results-are-reported-by-income-size-baseline-exposure-region}



\begin{itemize}

\tightlist

\item

  Category:

  \texttt{D)\ Results,\ magnitudes,\ heterogeneity,\ robustness}

\item

  Confidence: \texttt{low} (0.32051980795754204)

\item

  Retrieval method: \texttt{local}

\item

  Evidence type: \texttt{text}

\item

  Assumption flag: \texttt{False}

\item

  Answer:

\end{itemize}



None. The provided context reports no heterogeneity results by income,

size, baseline exposure, or region.



\begin{itemize}

\tightlist

\item

  Quote snippet:

\end{itemize}



\begin{quote}

The simulation experiment has two separate parts, the first for one

changepoint and the second for two changepoints

\end{quote}



\begin{itemize}

\tightlist

\item

  Citation anchors:



  \begin{itemize}

  \tightlist

  \item

    page=7, words=275-556, section=4. Simulation experiment setup and

    focus, note=Describes simulations focused on detecting variance

    changepoints, not subgroup heterogeneity.

  \item

    page=3, words=550-882, section=Method description and Table 1

    context, note=Focus on variance change detection via Dk and

    empirical/asymptotic quantiles.

  \item

    page=9, words=275-624, section=Posterior odds and sampling

    distribution of k*, note=Discusses number of variance changes and

    distribution of the estimated changepoint.

  \end{itemize}

\end{itemize}



\subsubsection{D08: What mechanism tests are performed and what do they

imply?}\label{d08-what-mechanism-tests-are-performed-and-what-do-they-imply}



\begin{itemize}

\tightlist

\item

  Category:

  \texttt{D)\ Results,\ magnitudes,\ heterogeneity,\ robustness}

\item

  Confidence: \texttt{low} (0.2055530935249787)

\item

  Retrieval method: \texttt{local}

\item

  Evidence type: \texttt{textual\_evidence\_and\_simulation\_results}

\item

  Data source: Inclán and Tiao (1994), JASA -- provided excerpts

\item

  Assumption flag: \texttt{False}

\item

  Answer:

\end{itemize}



They perform three mechanisms to test for variance changes: (1) the ICSS

max\textbar Dk\textbar{} threshold test applied iteratively to segments

(if max\textbar Dk\textbar{} ≤ D*, no change; if \textgreater{} D*,

declare a changepoint and continue), (2) sequential likelihood-ratio

tests LR\_\{n−1,n\} to assess the number of changepoints, standardized

to an extreme-value reference distribution, and (3) Bayesian

posterior-odds K\_\{n,n−1\} to select the number of changepoints up to a

specified maximum. Implications: max\textbar Dk\textbar{} is biased

toward mid-series but its mode occurs at the true changepoint and

concentrates there as T or the variance ratio increases; small changes

(A=2) in short series (T=100), especially early in the series, are hard

to detect with ICSS, while for T≥200 or A=3, correct identification

exceeds \textasciitilde80\% when the change is near the middle; the

Bayesian approach performs best for A=2, and ICSS improves with larger T

and A. They also note F-test power features for variance differences.



\begin{itemize}

\tightlist

\item

  Quote snippet:

\end{itemize}



\begin{quote}

If M(t1:T) \textless{} D*, there is no evidence of variance changes...

If M(t1:t2) \textgreater{} D*, then we have a new point of change

\end{quote}



\begin{itemize}

\tightlist

\item

  Citation anchors:



  \begin{itemize}

  \tightlist

  \item

    page=5, words=0-349, section=ICSS algorithm steps, note=Decision

    rule for ICSS: compare max\textbar Dk\textbar{} to D*; if below,

    stop; if above, declare a change and iterate.

  \item

    page=8, words=0-329, section=4.1 Likelihood ratios and posterior

    odds, note=LR\_\{n−1,n\} tests, standardization to extreme value

    distribution; posterior odds framework.

  \item

    page=7, words=275-556, section=4.1 Likelihood ratios and posterior

    odds, note=Definition/use of LR\_\{n−1,n\} for testing numbers of

    changepoints.

  \item

    page=9, words=275-624, section=Distribution of k* and implications,

    note=k* is biased toward middle; mode at true changepoint;

    concentration increases with T or variance ratio.

  \item

    page=9, words=550-899, section=4.2 Results (one changepoint),

    note=Performance implications: hard to detect A=2 with T=100; ≥80\%

    correct for T≥200 or A=3 when change is in the middle; Bayesian best

    for A=2; ICSS improves with T and A.

  \item

    page=4, words=0-349, section=Discussion of tests, note=F-test power

    characteristics for variance differences and interaction with

    degrees of freedom.

  \end{itemize}

\end{itemize}



\subsubsection{D09: How sensitive are results to alternative

samples/bandwidths/controls?}\label{d09-how-sensitive-are-results-to-alternative-samplesbandwidthscontrols}



\begin{itemize}

\tightlist

\item

  Category:

  \texttt{D)\ Results,\ magnitudes,\ heterogeneity,\ robustness}

\item

  Confidence: \texttt{medium} (0.4155334148582157)

\item

  Retrieval method: \texttt{local}

\item

  Evidence type:

  \texttt{simulation\ results\ and\ asymptotic\ theory\ from\ the\ provided\ text}

\item

  Table/Figure: \texttt{Tables\ 2,\ 4,\ 6–7\ (as\ referenced\ in\ text)}

\item

  Assumption flag: \texttt{True}

\item

  Assumption notes: No bandwidths or control-variable choices are

  discussed; sensitivity is interpreted in terms of sample size,

  variance ratio, change-point location, and configuration as reported.

\item

  Answer:

\end{itemize}



Sensitivity is driven by sample length (T), variance ratio (A), and

where/how variance changes occur---bandwidths/controls are not

discussed. Specifically: (1) With T=100 and small A=2, detection is

hard, especially for early changes; with T≥200 or A=3 and a mid-series

change, correct identifications exceed \textasciitilde80\%, and

performance improves with larger T and A. (2) With two changepoints,

monotone variance increases are most difficult; T≈500 is often needed to

get NT=2 more than half the time; best results occur when changes are

equally spaced and the largest variance is in the middle. (3)

Finite-sample critical values are conservative: the 95th quantile of

max\textbar Dk\textbar{} is below the asymptotic 1.358, lowering type I

error. (4) The location estimator k* is skewed toward the middle but has

its mode at the true change and concentrates there as T or A increase.



\begin{itemize}

\tightlist

\item

  Quote snippet:

\end{itemize}



\begin{quote}

Once we have 200 observations or the variance ratio is larger (A = 3),

the correct identifications occur more than 80\% of the time

\end{quote}



\begin{itemize}

\tightlist

\item

  Citation anchors:



  \begin{itemize}

  \tightlist

  \item

    page=9, words=550-899, section=Simulation results (one changepoint),

    note=Detection depends on T, A, and position; \textgreater80\% with

    T≥200 or A=3 when change is in the middle; harder with T=100 and

    A=2, especially early changes.

  \item

    page=9, words=0-349, section=Simulation results (two changepoints),

    note=Monotone increases are hardest; need large T (e.g., 500) to get

    NT=2 more than half the time; best when changes are equally spaced

    and large variance is in the middle.

  \item

    page=9, words=275-624, section=Sampling distribution of k*, note=k*

    is biased toward the center but its mode is at the true change;

    concentration improves with larger T or A.

  \item

    page=5, words=275-624, section=Finite-sample quantiles, note=95th

    quantile of max\textbar Dk\textbar{} is below asymptotic 1.358,

    implying smaller empirical type I error (conservative critical

    value).

  \end{itemize}

\end{itemize}



\subsubsection{D10: What are the main takeaways in the conclusion

(bullet

summary)?}\label{d10-what-are-the-main-takeaways-in-the-conclusion-bullet-summary}



\begin{itemize}

\item

  Category:

  \texttt{D)\ Results,\ magnitudes,\ heterogeneity,\ robustness}

\item

  Confidence: \texttt{low} (0.204945056307743)

\item

  Retrieval method: \texttt{local}

\item

  Evidence type: \texttt{Paraphrase\ with\ direct\ citations}

\item

  Table/Figure: \texttt{Tables\ 4–8}

\item

  Assumption flag: \texttt{True}

\item

  Assumption notes: The explicit conclusion bullets were not provided;

  key takeaways are summarized from the provided conclusion and results

  excerpts.

\item

  Answer:

\item

  ICSS is recommended for analyzing long series with multiple variance

  change points, based on results and CPU-time evidence.

\item

  Computationally efficient: iterative search averages O(T) operations

  and avoids exhaustive evaluation; convergence typically in few

  iterations.

\item

  One change point: with T=100 and small change (A=2), detection is

  hard---especially early in the series; with T≥200 or larger change

  (A=3) near the middle, correct identification exceeds

  \textasciitilde80\%. Bayesian is best for A=2 and comparable for A=3.

\item

  Two change points: with T≥200, ICSS generally outperforms likelihood

  ratio tests; best when change points are evenly spaced (.33T, .67T)

  and when the largest variance is in the middle (A1,A2)=(4,2).

  Monotone-increasing variances require larger T (e.g., 500) to identify

  both changes \textgreater50\% of the time.

\item

  Posterior-odds approach is computationally prohibitive (e.g.,

  \textasciitilde1,358 CPU hours for T=500), limiting practicality.

\item

  Location statistic k*: distribution is skewed and biased toward the

  middle, but the mode aligns with the true change; concentration around

  the true point improves with larger T and variance ratios.

\item

  Applying ICSS to AR(1) residuals yields similar identification rates

  to the i.i.d. case.

\item

  Quote snippet:

\end{itemize}



\begin{quote}

\ldots lend support to the recommendation to use the ICSS algorithm when

we need to analyze long series with multiple change points.

\end{quote}



\begin{itemize}

\tightlist

\item

  Citation anchors:



  \begin{itemize}

  \tightlist

  \item

    page=11, words=0-349, section=CPU Time Requirements / Conclusion,

    note=Recommends ICSS for long series with multiple change points;

    O(T) operations.

  \item

    page=5, words=825-1174, section=ICSS Algorithm and Convergence,

    note=Iterative scheme; convergence in few iterations.

  \item

    page=9, words=550-899, section=4.2 Results Part One; 4.6 CPU Time

    (setup), note=One change point detection difficulty at T=100, A=2;

    improves with T≥200 or A=3; Bayesian best at A=2.

  \item

    page=9, words=0-349, section=4.3 Results Part Two, note=ICSS better

    than LR with T≥200; best at (.33,.67) and when large variance in the

    middle; posterior odds computational burden (1,358 CPU hours).

  \item

    page=9, words=275-624, section=4.4 Sampling Distribution of k*,

    note=k* mode at true change; distribution biased to middle;

    concentration improves with T and variance ratio.

  \item

    page=9, words=550-899, section=4.5 ICSS on Residuals, note=Similar

    identification rates on AR(1) residuals.

  \end{itemize}

\end{itemize}



\subsubsection{E01: What are the most important prior papers cited and

why are they central

here?}\label{e01-what-are-the-most-important-prior-papers-cited-and-why-are-they-central-here}



\begin{itemize}

\tightlist

\item

  Category: \texttt{E)\ Citations\ and\ related\ literature}

\item

  Confidence: \texttt{low} (0.27717837898716885)

\item

  Retrieval method: \texttt{local}

\item

  Evidence type: \texttt{textual\ citations\ from\ the\ article}

\item

  Data source: Inclán \& Tiao (1994), JASA 89(427):913--923 (provided

  context)

\item

  Assumption flag: \texttt{False}

\item

  Answer:

\end{itemize}



Key prior works are: (1) Brown, Durbin, and Evans (1975) because the

ICSS procedure is built on their centered cumulative sum of squares; (2)

Hsu, Miller, and Wichern (1974) for initiating the variance-change

literature with a step-change variance model for stock returns; (3) Hsu

(1977, 1979, 1982) and Booth \& Smith (1982) for single-change detection

methods (frequentist and Bayesian) that highlight the computational

burden of multiple changes and serve as baselines; (4) Baufays \& Rasson

(1985) for MLE methods handling several change points and for providing

the algorithm used to compute LR benchmarks; (5) Wichern, Miller, \& Hsu

(1976) and Abraham \& Wei (1984) for variance-change models in

autocorrelated (AR) series, relevant to applying ICSS to residuals; (6)

Worsley (1986) as a related MLE framework for change-point estimation

(mean changes) that informs single-change inference.



\begin{itemize}

\tightlist

\item

  Quote snippet:

\end{itemize}



\begin{quote}

It is based on a centered version of the cumulative sum of squares

presented by Brown, Durbin, and Evans (1975).

\end{quote}



\begin{itemize}

\tightlist

\item

  Citation anchors:



  \begin{itemize}

  \tightlist

  \item

    page=2, words=275-624, section=Introduction, note=ICSS is based on

    the centered CUSUM of squares of Brown, Durbin, and Evans (1975);

    foundational change-of-variance literature and single-change

    Bayesian/frequentist works cited.

  \item

    page=2, words=0-349, section=Introduction, note=Hsu, Miller, and

    Wichern (1974) noted as starting the variance-change literature with

    step changes in variance for stock returns.

  \item

    page=2, words=550-899, section=2. Centered Cumulative Sums of

    Squares, note=Worsley (1986) MLE for change in mean; Wichern,

    Miller, \& Hsu (1976) and Abraham \& Wei (1984) on AR(1) with

    variance change; Baufays \& Rasson (1985) MLE for multiple change

    points.

  \item

    page=9, words=550-899, section=4.6 CPU Time Requirements,

    note=Baufays \& Rasson (1985) algorithm used to obtain the ML in LR

    tests; computational motivation for ICSS.

  \end{itemize}

\end{itemize}



\subsubsection{E02: Which papers does this work most directly build on

or

extend?}\label{e02-which-papers-does-this-work-most-directly-build-on-or-extend}



\begin{itemize}

\tightlist

\item

  Category: \texttt{E)\ Citations\ and\ related\ literature}

\item

  Confidence: \texttt{low} (0.25288654187197895)

\item

  Retrieval method: \texttt{local}

\item

  Evidence type: \texttt{explicit\ statements\ in\ the\ text}

\item

  Data source: Inclán \& Tiao (1994) article text (JSTOR excerpt)

\item

  Assumption flag: \texttt{False}

\item

  Answer:

\end{itemize}



The paper builds most directly on Brown, Durbin, and Evans (1975) for

the centered cumulative sum of squares, and extends the variance

change-point literature initiated by Hsu, Miller, and Wichern (1974) and

developed by Hsu (1977, 1979, 1982) and Booth \& Smith (1982).



\begin{itemize}

\tightlist

\item

  Quote snippet:

\end{itemize}



\begin{quote}

It is based on a centered version of the cumulative sum of squares

presented by Brown, Durbin, and Evans (1975).

\end{quote}



\begin{itemize}

\tightlist

\item

  Citation anchors:



  \begin{itemize}

  \tightlist

  \item

    page=2, words=275-624, section=Introduction/Overview, note=States

    the method is based on a centered version of CUSUM of squares by

    Brown, Durbin, and Evans (1975); lists prior variance change-point

    works including Hsu (1977, 1979, 1982) and Booth \& Smith (1982).

  \item

    page=2, words=0-349, section=Introduction, note=Notes the

    statistical literature on changes of variance started with Hsu,

    Miller, and Wichern (1974).

  \end{itemize}

\end{itemize}



\subsubsection{E03: Which papers are used as benchmarks or comparisons

in the

results?}\label{e03-which-papers-are-used-as-benchmarks-or-comparisons-in-the-results}



\begin{itemize}

\tightlist

\item

  Category: \texttt{E)\ Citations\ and\ related\ literature}

\item

  Confidence: \texttt{low} (0.30012011333351307)

\item

  Retrieval method: \texttt{local}

\item

  Evidence type:

  \texttt{textual\ evidence\ from\ methods/results\ description\ and\ cited\ references}

\item

  Data source: JASA 1994 article text (provided context)

\item

  Assumption flag: \texttt{False}

\item

  Answer:

\end{itemize}



The results compare ICSS against (1) likelihood ratio tests implemented

using the Baufays and Rasson (1985) algorithm, and (2) a Bayesian

posterior-odds approach specified with an inverted-gamma prior as in

Inclán (1991).



\begin{itemize}

\tightlist

\item

  Quote snippet:

\end{itemize}



\begin{quote}

compare the performance of the ICSS algorithm with the likelihood ratio

and a Bayesian approach.

\end{quote}



\begin{itemize}

\tightlist

\item

  Citation anchors:



  \begin{itemize}

  \tightlist

  \item

    page=7, words=0-349, section=4. SIMULATION EXPERIMENT, note=States

    comparison of ICSS with likelihood ratio and a Bayesian approach.

  \item

    page=9, words=550-899, section=4.6 CPU Time Requirements, note=LR

    tests computed using the Baufays and Rasson (1985) algorithm.

  \item

    page=9, words=275-624, section=4.4 Sampling Distribution of k* / 4.2

    Results context, note=Bayesian posterior odds uses inverted-gamma

    prior (see Inclán 1991).

  \end{itemize}

\end{itemize}



\subsubsection{E04: What data sources or datasets are cited and how are

they

used?}\label{e04-what-data-sources-or-datasets-are-cited-and-how-are-they-used}



\begin{itemize}

\tightlist

\item

  Category: \texttt{E)\ Citations\ and\ related\ literature}

\item

  Confidence: \texttt{low} (0.2478824092067886)

\item

  Retrieval method: \texttt{local}

\item

  Evidence type: \texttt{textual}

\item

  Table/Figure:

  \texttt{Tables\ 6–8\ (simulation\ results\ and\ AR(1)\ residuals)}

\item

  Data source: Simulated time series (T=100,200,500) with specified

  changepoints and variance ratios; AR(1) residuals derived from these

  simulations

\item

  Assumption flag: \texttt{False}

\item

  Answer:

\end{itemize}



Only simulated series are cited. The study uses artificially generated

time series with specified lengths (T = 100, 200, 500), changepoint

locations, and variance ratios (A = 2, 3) for simulation experiments; it

also applies the ICSS algorithm to AR(1) residuals derived from these

simulated series. No external empirical datasets are cited in the

provided excerpt; an illustrative series is mentioned but not

identified.



\begin{itemize}

\tightlist

\item

  Quote snippet:

\end{itemize}



\begin{quote}

For Part One, we used three series lengths (T = 100, 200, and

500)\ldots{} and two values for the variance ratio (A = 2 and 3)\ldots{}

Table 8\ldots{} Residuals from AR(1).

\end{quote}



\begin{itemize}

\tightlist

\item

  Citation anchors:



  \begin{itemize}

  \tightlist

  \item

    page=7, words=0-349, section=4. SIMULATION EXPERIMENT, note=Details

    of simulated series: lengths, changepoint locations, variance

    ratios; paired comparisons.

  \item

    page=9, words=0-349, section=4.4 Sampling Distribution of k*,

    note=States there are 1,000 replicates per configuration, confirming

    simulation-based data.

  \item

    page=10, words=0-84, section=Tables 7--8, note=Mentions applying

    ICSS to AR(1) residuals from the simulated series.

  \item

    page=6, words=550-899, section=Application example, note=Refers to

    an analyzed series illustrating ICSS, but the dataset is not

    identified in the excerpt.

  \end{itemize}

\end{itemize}



\subsubsection{E05: What methodological or econometric references are

cited (e.g., DiD, IV, RDD

methods)?}\label{e05-what-methodological-or-econometric-references-are-cited-eg-did-iv-rdd-methods}



\begin{itemize}

\tightlist

\item

  Category: \texttt{E)\ Citations\ and\ related\ literature}

\item

  Confidence: \texttt{low} (0.3465047268726521)

\item

  Retrieval method: \texttt{local}

\item

  Evidence type: \texttt{explicit\ references\ list}

\item

  Data source: Inclán \& Tiao (1994), Journal of the American

  Statistical Association (provided context)

\item

  Assumption flag: \texttt{False}

\item

  Answer:

\end{itemize}



The article cites methodological/econometric works on change-point

detection, CUSUM/CUSUM-of-squares, likelihood-ratio and Bayesian

methods, time-series models with changing variance/parameters, and

related asymptotics/simulation, including: Hinkley (1971); Brown,

Durbin, and Evans (1975); Smith (1975, 1980); Booth and Smith (1982,

Journal of Econometrics); Menzefricke (1981); Hsu (1977, 1979, 1982);

Haccou and Meelis (1988); Haccou, Meelis, and van de Geer (1988);

Worsley (1986); Wichern, Miller, and Hsu (1976); Abraham and Wei (1984);

Baufays and Rasson (1985); Tyssedal and Tjostheim (1988); Tsay (1988);

Chang, Tiao, and Chen (1988); Cobb (1978); Commenges, Seal, and Pinatel

(1986); Billingsley (1968); Box and Jenkins (1976); Bratley, Fox, and

Schrage (1987); Broemeling and Tsurumi (1987). No DiD, IV, or RDD

methods are cited.



\begin{itemize}

\tightlist

\item

  Quote snippet:

\end{itemize}



\begin{quote}

Booth \& Smith (1982), \textquotesingle A Bayesian Approach to

Retrospective Identification of Change-Points\textquotesingle; Brown,

Durbin \& Evans (1975), \textquotesingle Techniques for Testing the

Constancy of Regression Relationships Over Time\textquotesingle{}

\end{quote}



\begin{itemize}

\tightlist

\item

  Citation anchors:



  \begin{itemize}

  \tightlist

  \item

    page=12, words=0-349, section=References and Appendix, note=Multiple

    methodological references listed (Booth \& Smith 1982; Brown, Durbin

    \& Evans 1975; Billingsley 1968; Box \& Jenkins 1976; etc.)

  \item

    page=12, words=275-624, section=References and Appendix,

    note=Additional change-point and likelihood-ratio references (Haccou

    \& Meelis 1988; Worsley 1986; Hsu 1977, 1979, 1982).

  \item

    page=12, words=550-858, section=References and Appendix,

    note=Further references on change-point tests and asymptotics

    (Menzefricke 1981; Smith 1975, 1980; Worsley 1986).

  \end{itemize}

\end{itemize}



\subsubsection{E06: Are there any seminal or classic references the

paper positions itself

against?}\label{e06-are-there-any-seminal-or-classic-references-the-paper-positions-itself-against}



\begin{itemize}

\tightlist

\item

  Category: \texttt{E)\ Citations\ and\ related\ literature}

\item

  Confidence: \texttt{low} (0.23712234626341297)

\item

  Retrieval method: \texttt{local}

\item

  Evidence type: \texttt{textual}

\item

  Data source: Inclan \& Tiao (1994), JASA (JSTOR excerpt)

\item

  Assumption flag: \texttt{True}

\item

  Assumption notes: Interpreted \textquotesingle positions itself

  against\textquotesingle{} as contrasting with prior benchmark methods

  and frameworks rather than refuting them.

\item

  Answer:

\end{itemize}



Yes. The paper contrasts its ICSS method with established Bayesian and

likelihood‑ratio change‑point procedures and related maximum‑likelihood

approaches, citing classic works such as Smith (1975, 1980), Booth \&

Smith (1982), Hsu (1977, 1979, 1982), Worsley (1986), and Baufays \&

Rasson (1985). It also builds on the classic CUSUM-of-squares work of

Brown, Durbin \& Evans (1975) and references the foundational

variance-change formulation of Hsu, Miller \& Wichern (1974).



\begin{itemize}

\tightlist

\item

  Quote snippet:

\end{itemize}



\begin{quote}

the ICSS algorithm offers results comparable to those obtained by a

Bayesian approach or by likelihood ratio tests, without the heavy

computational burden

\end{quote}



\begin{itemize}

\tightlist

\item

  Citation anchors:



  \begin{itemize}

  \tightlist

  \item

    page=2, words=0-349, section=Introduction, note=Positions ICSS

    against Bayesian and likelihood ratio tests due to computational

    burden.

  \item

    page=2, words=275-624, section=Introduction, note=States approach is

    based on centered CUSUM of squares by Brown, Durbin, and Evans

    (1975); lists classic change-point references (Hinkley 1971; Smith

    1975, 1980; Booth \& Smith 1982; Hsu 1977, 1979, 1982).

  \item

    page=2, words=550-899, section=Section 2 setup, note=Mentions

    Worsley (1986) maximum-likelihood tests and AR variance-change works

    (Wichern et al. 1976; Abraham \& Wei 1984; Baufays \& Rasson 1985).

  \end{itemize}

\end{itemize}



\subsubsection{E07: Are there citations to code, data repositories, or

appendices that are essential to the

claims?}\label{e07-are-there-citations-to-code-data-repositories-or-appendices-that-are-essential-to-the-claims}



\begin{itemize}

\tightlist

\item

  Category: \texttt{E)\ Citations\ and\ related\ literature}

\item

  Confidence: \texttt{low} (0.24801479150047653)

\item

  Retrieval method: \texttt{local}

\item

  Evidence type: \texttt{explicit\ textual\ mention}

\item

  Assumption flag: \texttt{False}

\item

  Answer:

\end{itemize}



Yes. The paper references appendices (Appendix A with derivations and

Appendix B with a proof) that underpin core claims, and it notes Fortran

code for the ICSS algorithm is available upon request. No data

repositories are cited.



\begin{itemize}

\tightlist

\item

  Quote snippet:

\end{itemize}



\begin{quote}

The Fortran program of the ICSS algorithm is available from us upon

request; send an electronic mail message to

\href{mailto:inclan@guvax.georgetown.edu}{\nolinkurl{inclan@guvax.georgetown.edu}}.

\end{quote}



\begin{itemize}

\tightlist

\item

  Citation anchors:



  \begin{itemize}

  \tightlist

  \item

    page=6, words=825-968, section=Main text, note=Mentions availability

    of Fortran program for the ICSS algorithm.

  \item

    page=11, words=0-349, section=Appendix A, note=Appendix A provides

    derivations (Approximate Expected Value of Dk).

  \item

    page=12, words=0-349, section=Appendix B, note=Appendix B provides

    the proof of Theorem 1.

  \end{itemize}

\end{itemize}



\subsubsection{E08: What gaps in the literature do the authors say these

citations leave

open?}\label{e08-what-gaps-in-the-literature-do-the-authors-say-these-citations-leave-open}



\begin{itemize}

\tightlist

\item

  Category: \texttt{E)\ Citations\ and\ related\ literature}

\item

  Confidence: \texttt{low} (0.23077763319438027)

\item

  Retrieval method: \texttt{local}

\item

  Evidence type: \texttt{textual}

\item

  Assumption flag: \texttt{False}

\item

  Answer:

\end{itemize}



The cited literature mainly treats a single, unknown variance change and

tests changepoints one at a time due to heavy computational burden,

leaving a gap for an efficient, systematic method to detect multiple

variance changes in moderate-length series.



\begin{itemize}

\tightlist

\item

  Quote snippet:

\end{itemize}



\begin{quote}

single change of variance at an unknown point... focusing on the

detection of points of change one at a time because of the heavy

computational burden involved in looking for several points of change

simultaneously.

\end{quote}



\begin{itemize}

\tightlist

\item

  Citation anchors:



  \begin{itemize}

  \tightlist

  \item

    page=2, words=275-624, section=Introduction, note=Prior work focuses

    on single changepoint; multiple points are computationally

    burdensome.

  \item

    page=2, words=0-349, section=Introduction, note=Authors position

    ICSS as comparable to Bayesian/LR without heavy computational

    burden.

  \item

    page=9, words=550-899, section=Simulation: CPU Time Requirements,

    note=Motivation cites heavy computational burden of posterior odds

    and likelihood-based searches.

  \end{itemize}

\end{itemize}



\subsubsection{F01: Are replication files or code provided? If so,

where?}\label{f01-are-replication-files-or-code-provided-if-so-where}



\begin{itemize}

\tightlist

\item

  Category: \texttt{F)\ Replication\ and\ transparency}

\item

  Confidence: \texttt{low} (0.19380720232206966)

\item

  Retrieval method: \texttt{local}

\item

  Evidence type: \texttt{textual}

\item

  Data source: Article text (Inclán \& Tiao, 1994) in provided context

\item

  Assumption flag: \texttt{False}

\item

  Answer:

\end{itemize}



Yes. The Fortran program (ICSS algorithm) is available from the authors

upon request via email to

\href{mailto:inclan@guvax.georgetown.edu}{\nolinkurl{inclan@guvax.georgetown.edu}}.



\begin{itemize}

\tightlist

\item

  Quote snippet:

\end{itemize}



\begin{quote}

The Fortran program of the ICSS algorithm is available from us upon

request; send an electronic mail message to

\href{mailto:inclan@guvax.georgetown.edu}{\nolinkurl{inclan@guvax.georgetown.edu}}.

\end{quote}



\begin{itemize}

\tightlist

\item

  Citation anchors:



  \begin{itemize}

  \tightlist

  \item

    page=6, words=825-968, note=States the Fortran program is available

    upon request via email to

    \href{mailto:inclan@guvax.georgetown.edu}{\nolinkurl{inclan@guvax.georgetown.edu}}.

  \end{itemize}

\end{itemize}



\subsubsection{F02: Is there a pre-analysis plan or registered trial?

Provide details if

mentioned.}\label{f02-is-there-a-pre-analysis-plan-or-registered-trial-provide-details-if-mentioned}



\begin{itemize}

\tightlist

\item

  Category: \texttt{F)\ Replication\ and\ transparency}

\item

  Confidence: \texttt{low} (0.22306541673272023)

\item

  Retrieval method: \texttt{local}

\item

  Evidence type: \texttt{textual}

\item

  Data source: Inclán \& Tiao (1994), Journal of the American

  Statistical Association

\item

  Assumption flag: \texttt{False}

\item

  Answer:

\end{itemize}



No. The provided excerpts do not mention any pre-analysis plan or

registered trial; they describe a methodological study with simulations,

not a registered study.



\begin{itemize}

\tightlist

\item

  Quote snippet:

\end{itemize}



\begin{quote}

The simulation experiment has two separate parts, the first for one

changepoint and the second for two changepoints

\end{quote}



\begin{itemize}

\tightlist

\item

  Citation anchors:



  \begin{itemize}

  \tightlist

  \item

    page=7, words=275-556, section=4. SIMULATION EXPERIMENT,

    note=Describes simulation setup; no registration or pre-analysis

    plan referenced.

  \end{itemize}

\end{itemize}



\subsubsection{F03: Are data access constraints disclosed (restricted

access, proprietary data,

NDAs)?}\label{f03-are-data-access-constraints-disclosed-restricted-access-proprietary-data-ndas}



\begin{itemize}

\tightlist

\item

  Category: \texttt{F)\ Replication\ and\ transparency}

\item

  Confidence: \texttt{low} (0.18228175986289236)

\item

  Retrieval method: \texttt{local}

\item

  Evidence type:

  \texttt{Terms\ of\ Use\ notice\ in\ article\ front\ matter}

\item

  Data source: JSTOR Terms and Conditions of Use notice (page 1)

\item

  Assumption flag: \texttt{False}

\item

  Answer:

\end{itemize}



Yes. The article's front matter discloses restricted-use terms via

JSTOR's Terms and Conditions (personal, non-commercial use only; no

downloading entire issues or multiple copies without prior permission).

No NDAs are mentioned.



\begin{itemize}

\tightlist

\item

  Quote snippet:

\end{itemize}



\begin{quote}

you may use content in the JSTOR archive only for your personal,

non-commercial use.

\end{quote}



\begin{itemize}

\tightlist

\item

  Citation anchors:



  \begin{itemize}

  \tightlist

  \item

    page=1, words=0-230, section=JSTOR Terms and Conditions notice,

    note=States personal, non-commercial use only and prohibits

    downloading entire issues or multiple copies without prior

    permission.

  \end{itemize}

\end{itemize}



\subsubsection{F04: Are key steps in data cleaning and construction

documented?}\label{f04-are-key-steps-in-data-cleaning-and-construction-documented}



\begin{itemize}

\tightlist

\item

  Category: \texttt{F)\ Replication\ and\ transparency}

\item

  Confidence: \texttt{low} (0.2456742970693667)

\item

  Retrieval method: \texttt{local}

\item

  Evidence type: \texttt{textual\ evidence}

\item

  Data source: IBM stock prices as reported by Box and Jenkins (1976)

\item

  Assumption flag: \texttt{False}

\item

  Answer:

\end{itemize}



Yes---key construction steps are documented (e.g., transforming IBM

prices to first differences of logs and detailed simulation setups).

Explicit data-cleaning is limited to advising outlier checks.



\begin{itemize}

\tightlist

\item

  Quote snippet:

\end{itemize}



\begin{quote}

The series analyzed is the first difference of the logarithm of the IBM

stock closing prices...

\end{quote}



\begin{itemize}

\tightlist

\item

  Citation anchors:



  \begin{itemize}

  \tightlist

  \item

    page=6, words=275-624, section=3.1 Example: IBM Stock Prices,

    note=Documents construction of the empirical series as first

    differences of log IBM prices.

  \item

    page=7, words=0-349, section=4. SIMULATION EXPERIMENT, note=Details

    parameters for constructing simulated data (lengths, changepoint

    locations, variance ratios).

  \item

    page=6, words=0-349, section=ICSS Algorithm discussion, note=Advises

    complementing variance-change search with outlier detection, a

    cleaning consideration.

  \end{itemize}

\end{itemize}



\subsubsection{F05: Are robustness and sensitivity analyses fully

reported or partially

omitted?}\label{f05-are-robustness-and-sensitivity-analyses-fully-reported-or-partially-omitted}



\begin{itemize}

\tightlist

\item

  Category: \texttt{F)\ Replication\ and\ transparency}

\item

  Confidence: \texttt{low} (0.309728123445472)

\item

  Retrieval method: \texttt{local}

\item

  Evidence type: \texttt{text}

\item

  Assumption flag: \texttt{False}

\item

  Answer:

\end{itemize}



Partially omitted



\begin{itemize}

\tightlist

\item

  Quote snippet:

\end{itemize}



\begin{quote}

The posterior odds ratio was not obtained for series with T = 500,

because it would have required a total of 1,358 CPU hours.

\end{quote}



\begin{itemize}

\tightlist

\item

  Citation anchors:



  \begin{itemize}

  \tightlist

  \item

    page=9, words=200-340, section=4.3 Results of Part Two,

    note=Posterior odds not obtained for T=500; reduced replicates for

    T=200.

  \item

    page=9, words=700-780, section=4.5 The ICSS Algorithm Applied to

    Residuals, note=Indicates further work pending, suggesting

    incomplete reporting.

  \end{itemize}

\end{itemize}



\subsubsection{G01: What populations or settings are most likely to

generalize from this

study?}\label{g01-what-populations-or-settings-are-most-likely-to-generalize-from-this-study}



\begin{itemize}

\tightlist

\item

  Category: \texttt{G)\ External\ validity\ and\ generalization}

\item

  Confidence: \texttt{low} (0.21142405836654996)

\item

  Retrieval method: \texttt{local}

\item

  Evidence type:

  \texttt{Direct\ text\ statements\ on\ scope/assumptions\ and\ simulation\ design}

\item

  Assumption flag: \texttt{False}

\item

  Answer:

\end{itemize}



Sequences of independent (or prewhitened/uncorrelated) observations with

piecewise-constant variance---especially financial time series---with

moderate-to-large sample sizes (about 200+ observations). Theory and

simulations assume i.i.d. Normal innovations; autocorrelated series can

be handled by applying ICSS to AR-model residuals.



\begin{itemize}

\tightlist

\item

  Quote snippet:

\end{itemize}



\begin{quote}

We study the detection of multiple changes of variance in a sequence of

independent observations. There are series, particularly in the area of

finance,

\end{quote}



\begin{itemize}

\tightlist

\item

  Citation anchors:



  \begin{itemize}

  \tightlist

  \item

    page=2, words=0-349, section=Introduction, note=Scope: independent

    observations; finance context; moderate size (200+) applicability.

  \item

    page=2, words=550-899, section=Centered Cumulative Sums of Squares,

    note=Defines method for uncorrelated mean-zero series with

    time-varying variances.

  \item

    page=5, words=0-349, section=Asymptotics (Brownian bridge),

    note=Assumes i.i.d. Normal innovations for asymptotic distribution.

  \item

    page=9, words=550-899, section=ICSS Applied to Residuals,

    note=Application to AR(1) residuals indicates use with

    autocorrelated series after prewhitening.

  \end{itemize}

\end{itemize}



\subsubsection{G02: What populations or settings are least likely to

generalize?}\label{g02-what-populations-or-settings-are-least-likely-to-generalize}



\begin{itemize}

\tightlist

\item

  Category: \texttt{G)\ External\ validity\ and\ generalization}

\item

  Confidence: \texttt{low} (0.20618870169803688)

\item

  Retrieval method: \texttt{local}

\item

  Evidence type:

  \texttt{Simulation\ results\ and\ theoretical\ properties}

\item

  Data source: Inclán and Tiao (1994), Journal of the American

  Statistical Association

\item

  Assumption flag: \texttt{False}

\item

  Answer:

\end{itemize}



Settings least likely to generalize are short series with small variance

changes occurring near the ends, and series with two changepoints where

the variance increases monotonically. The max\textbar Dk\textbar{}

criterion is biased toward the middle, further reducing reliability for

edge changes.



\begin{itemize}

\tightlist

\item

  Quote snippet:

\end{itemize}



\begin{quote}

it is hard to detect a small variance change (variance ratio A =

2)---particularly when it appears at the beginning of the series

\end{quote}



\begin{itemize}

\tightlist

\item

  Citation anchors:



  \begin{itemize}

  \tightlist

  \item

    page=9, words=550-899, section=4.2 Results of Part One: Series With

    One Changepoint, note=Hard to detect small change (A=2) in short

    series, especially at the beginning.

  \item

    page=9, words=0-349, section=4.3 Results of Part Two: Series With

    Two Changepoints, note=Most difficult when variances increase

    monotonically; needs large T (\textasciitilde500).

  \item

    page=9, words=275-624, section=4.4 Sampling Distribution of k*,

    note=k* biased toward the middle of the series.

  \item

    page=4, words=275-624, section=2.3 Asymptotic Behavior of Dk Under

    Homogeneous Variance, note=max\textbar Dk\textbar{} puts more weight

    near the middle, inducing bias toward T/2.

  \end{itemize}

\end{itemize}



\subsubsection{G03: Do the authors discuss boundary conditions or scope

limits?}\label{g03-do-the-authors-discuss-boundary-conditions-or-scope-limits}



\begin{itemize}

\tightlist

\item

  Category: \texttt{G)\ External\ validity\ and\ generalization}

\item

  Confidence: \texttt{low} (0.2617001163362451)

\item

  Retrieval method: \texttt{local}

\item

  Evidence type: \texttt{explicit\ text}

\item

  Table/Figure: \texttt{Table\ 1;\ Figure\ 2}

\item

  Assumption flag: \texttt{False}

\item

  Answer:

\end{itemize}



Yes. They define statistical boundaries for Dk from its asymptotic

distribution and use a critical value D* to decide changes; they also

limit iterations and set a convergence rule (within two observations).



\begin{itemize}

\tightlist

\item

  Quote snippet:

\end{itemize}



\begin{quote}

These boundaries can be obtained from the asymptotic distribution of Dk

assuming constant variance.

\end{quote}



\begin{itemize}

\tightlist

\item

  Citation anchors:



  \begin{itemize}

  \tightlist

  \item

    page=3, words=275-624, section=Main text on Dk plots and boundaries,

    note=Boundaries from asymptotic distribution of Dk assuming constant

    variance

  \item

    page=5, words=0-349, section=ICSS Algorithm steps, note=Decision

    rule using critical value D*; stopping condition

  \item

    page=5, words=825-1174, section=ICSS implementation details,

    note=Controls to avoid indefinite cycling; iteration limit mentioned

  \item

    page=6, words=550-899, section=Convergence criteria,

    note=Convergence when change points move by no more than two

    observations

  \item

    page=4, words=550-869, section=Asymptotic boundaries illustration,

    note=Reference to asymptotic boundaries (e.g., 1.358) shown in plots

  \end{itemize}

\end{itemize}



\subsubsection{G04: How might the results change in different time

periods or

markets?}\label{g04-how-might-the-results-change-in-different-time-periods-or-markets}



\begin{itemize}

\tightlist

\item

  Category: \texttt{G)\ External\ validity\ and\ generalization}

\item

  Confidence: \texttt{medium} (0.37517108644342595)

\item

  Retrieval method: \texttt{local}

\item

  Evidence type: \texttt{text}

\item

  Assumption flag: \texttt{True}

\item

  Assumption notes: Interprets different time periods or markets as time

  series that differ in sample length, spacing and configuration of

  variance shifts, variance ratios, outliers, and autocorrelation;

  conclusions follow the stated simulation and algorithm behavior.

\item

  Answer:

\end{itemize}



Results would vary with the series' characteristics. Detection

performance depends on sample size, spacing and configuration of

variance changes, variance ratios, and data idiosyncrasies. Longer

samples improve identification and localization; equal spacing between

change points tends to perform best; monotone variance increases are

hardest (often requiring many more observations). Performance improves

as variance ratios increase. Multiple change points can mask each other,

but the iterative ICSS procedure mitigates this. Outliers can create

false peaks and should be checked. Applying ICSS to autoregressive

residuals yields results similar to independent data. Hence, different

time periods or markets---if they differ in length, volatility regime

patterns, outlier prevalence, or autocorrelation---can produce different

numbers and locations of detected variance changes and different

accuracy levels.



\begin{itemize}

\tightlist

\item

  Quote snippet:

\end{itemize}



\begin{quote}

The most difficult situation is when the variances change in a monotone

way; that is, the variance increases at the first change point and

increases again at the second change point.

\end{quote}



\begin{itemize}

\tightlist

\item

  Citation anchors:



  \begin{itemize}

  \tightlist

  \item

    page=9, words=0-349, section=Simulation results (two changepoints),

    note=Hardest when variances change monotonically; need larger T

    (e.g., 500); best when spacing is (.33,.67) and large variance in

    the middle; ICSS often better with T ≥ 200.

  \item

    page=9, words=275-624, section=Sampling distribution of k*, note=k*

    biased toward middle; concentration around true changepoint

    increases with sample size and variance ratio.

  \item

    page=4, words=0-349, section=Behavior of E{[}Dk{]} with multiple

    changepoints, note=Piecewise linear E{[}Dk{]}; masking effect when

    \textgreater1 changepoint; dependence on configuration of variances.

  \item

    page=6, words=0-349, section=Practical considerations, note=Outliers

    can create significant Dk peaks; recommend outlier detection; ICSS

    on AR residuals performs similarly to independent data.

  \item

    page=6, words=275-624, section=Example and convergence, note=Applied

    example on IBM returns; illustrates practical variance-change

    detection in a market series.

  \end{itemize}

\end{itemize}



\subsubsection{H01: Are key variables measured directly or via

proxies?}\label{h01-are-key-variables-measured-directly-or-via-proxies}



\begin{itemize}

\tightlist

\item

  Category: \texttt{H)\ Measurement\ validity}

\item

  Confidence: \texttt{low} (0.2784010202574155)

\item

  Retrieval method: \texttt{local}

\item

  Evidence type: \texttt{textual}

\item

  Assumption flag: \texttt{True}

\item

  Assumption notes: Interpreted \textquotesingle key

  variables\textquotesingle{} as variance changes/changepoints;

  underlying series values (e.g., IBM price log-differences) are

  directly observed, but variance changes are inferred.

\item

  Answer:

\end{itemize}



Proxies/inferred. The key quantities---variance changes and

changepoints---are not observed directly; they are inferred from the

data using statistics like max\textbar Dk\textbar{} (and LR), while the

raw series (e.g., IBM price log-differences) are directly observed.



\begin{itemize}

\tightlist

\item

  Quote snippet:

\end{itemize}



\begin{quote}

This behavior leads to the search for a variance change point via maxk I

Dk I ... If this maximum absolute value exceeds a predetermined

boundary, then we may conclude that there is a change point

\end{quote}



\begin{itemize}

\tightlist

\item

  Citation anchors:



  \begin{itemize}

  \tightlist

  \item

    page=3, words=550-882, section=Detection via Dk, note=Variance

    change points concluded from the statistic max\textbar Dk\textbar{}

    crossing a boundary.

  \item

    page=6, words=275-624, section=IBM example data, note=Raw data

    described as first differences of log IBM closing prices (directly

    observed then transformed).

  \end{itemize}

\end{itemize}



\subsubsection{H02: What measurement error risks are acknowledged or

likely?}\label{h02-what-measurement-error-risks-are-acknowledged-or-likely}



\begin{itemize}

\tightlist

\item

  Category: \texttt{H)\ Measurement\ validity}

\item

  Confidence: \texttt{low} (0.2677151175224959)

\item

  Retrieval method: \texttt{local}

\item

  Evidence type: \texttt{textual\_evidence}

\item

  Table/Figure: \texttt{Table\ 2}

\item

  Assumption flag: \texttt{True}

\item

  Assumption notes: Inference that autocorrelation in raw series may

  distort detection is based on authors' application of ICSS to AR(1)

  residuals rather than an explicit statement of risk.

\item

  Answer:

\end{itemize}



Acknowledged/likely risks include: (1) miscalibrated thresholds from

asymptotic approximations, yielding conservative type I error and

potential under-detection; (2) biased location estimates of the change

point (k* skewed toward the series middle); (3) low power for small

samples or small variance ratios, especially when the change occurs

early, and in monotone-increasing variance scenarios, which may require

large T to detect multiple changes; (4) dependence in data can affect

detection, prompting use on AR(1) residuals rather than raw series.



\begin{itemize}

\tightlist

\item

  Quote snippet:

\end{itemize}



\begin{quote}

This distribution is skewed; k* is biased towards the middle of the

series.

\end{quote}



\begin{itemize}

\tightlist

\item

  Citation anchors:



  \begin{itemize}

  \tightlist

  \item

    page=5, words=500-610, section=Asymptotics and empirical quantiles,

    note=Empirical 95th quantile below asymptotic 1.358 implying

    conservative type I error

  \item

    page=9, words=210-270, section=4.4 Sampling Distribution of k*,

    note=k* biased toward the middle

  \item

    page=9, words=550-760, section=Results and detection difficulty,

    note=Hard to detect small variance change, especially early in

    series

  \item

    page=9, words=0-120, section=Two changepoints results, note=Most

    difficult when variances increase monotonically; need large T

  \item

    page=8, words=120-200, section=Likelihood ratio testing,

    note=Critical values must be obtained by simulation; non-regular

    problem

  \item

    page=9, words=760-899, section=AR(1) residuals application,

    note=Applied ICSS to AR(1) residuals, indicating concern about

    dependence

  \end{itemize}

\end{itemize}



\subsubsection{H03: Are there validation checks for key

measures?}\label{h03-are-there-validation-checks-for-key-measures}



\begin{itemize}

\tightlist

\item

  Category: \texttt{H)\ Measurement\ validity}

\item

  Confidence: \texttt{low} (0.2598324384323857)

\item

  Retrieval method: \texttt{local}

\item

  Evidence type:

  \texttt{asymptotic\ theory;\ simulation\ benchmarks;\ algorithmic\ convergence\ checks;\ LR\ standardization}

\item

  Table/Figure:

  \texttt{Table\ 1\ (Empirical\ vs\ asymptotic\ quantiles\ of\ max\textbar{}Dk\textbar{});\ Table\ 2\ (percent\ below\ 1.358);\ Table\ 3\ (LR\ standardization)}

\item

  Data source: 10,000 replicates of T independent N(0,1) series (noted

  in Tables 1--2)

\item

  Answer:

\end{itemize}



Yes. The paper validates key measures by (1) deriving asymptotic

reference values for max\textbar Dk\textbar{} from a Brownian bridge and

(2) benchmarking them with large-scale simulations (Tables 1--2, with

SEs). It also standardizes likelihood-ratio tests to an extreme-value

distribution (Table 3) and uses algorithmic stopping and convergence

checks for detected changepoints.



\begin{itemize}

\tightlist

\item

  Quote snippet:

\end{itemize}



\begin{quote}

Table 1 compares selected asymptotic quantiles of max, I D, I with

corresponding em- pirical quantiles for various series lengths obtained

by sim- ulation.

\end{quote}



\begin{itemize}

\tightlist

\item

  Citation anchors:



  \begin{itemize}

  \tightlist

  \item

    page=5, words=0-349, section=Brownian bridge asymptotics; empirical

    vs asymptotic quantiles (Table 1), note=Asymptotic distribution of

    Dk and comparison to simulations

  \item

    page=3, words=550-882, section=Table 1 details and simulation note,

    note=Empirical and asymptotic quantiles of max\textbar Dk\textbar;

    10,000 replicates; Brownian-bridge reference

  \item

    page=5, words=275-624, section=Table 2 percentages and SE formula,

    note=Percentage of series with max\textbar Dk\textbar{} \textless{}

    1.358; standard errors from simulations

  \item

    page=6, words=550-899, section=Algorithm convergence checks (Step

    3), note=Convergence when number of points stabilizes and locations

    move by ≤2

  \item

    page=8, words=0-329, section=LR test standardization (Table 3) and

    extreme value reference, note=LR statistics standardized for

    reference to extreme value distribution; simulation for critical

    values

  \end{itemize}

\end{itemize}



\subsubsection{H04: Do the authors discuss construct validity for core

outcomes?}\label{h04-do-the-authors-discuss-construct-validity-for-core-outcomes}



\begin{itemize}

\tightlist

\item

  Category: \texttt{H)\ Measurement\ validity}

\item

  Confidence: \texttt{low} (0.21229758667580595)

\item

  Retrieval method: \texttt{local}

\item

  Evidence type: \texttt{no\ direct\ mention\ in\ provided\ context}

\item

  Data source: Inclán \& Tiao (1994) Use of Cumulative Sums of Squares

  for Retrospective Detection of Changes of Variance, JASA

\item

  Assumption flag: \texttt{True}

\item

  Assumption notes: Conclusion is limited to the provided excerpts;

  other parts of the full article might address different topics.

\item

  Answer:

\end{itemize}



No. In the provided excerpts, the authors focus on detecting variance

change points using the ICSS algorithm and related simulations; there is

no discussion of construct validity for core outcomes.



\begin{itemize}

\tightlist

\item

  Quote snippet:

\end{itemize}



\begin{quote}

This article studies the problem of multiple change points in the

variance of a sequence of independent observations. We propose a

procedure to detect variance changes based on an ICSS algorithm.

\end{quote}



\begin{itemize}

\tightlist

\item

  Citation anchors:



  \begin{itemize}

  \tightlist

  \item

    page=2, words=0-349, section=Introduction/Abstract, note=States

    study scope on variance change points and ICSS; no mention of

    construct validity or core outcomes.

  \item

    page=2, words=275-624, section=Introduction/Organization,

    note=Outlines Dk function, relation to F statistic, ICSS algorithm,

    and simulations; no construct validity discussion.

  \item

    page=7, words=0-349, section=Simulation experiment setup,

    note=Describes simulation design and performance comparisons; no

    core outcomes or validity constructs.

  \end{itemize}

\end{itemize}



\subsubsection{I01: What policy counterfactuals are considered or

implied?}\label{i01-what-policy-counterfactuals-are-considered-or-implied}



\begin{itemize}

\tightlist

\item

  Category: \texttt{I)\ Policy\ counterfactuals\ and\ welfare}

\item

  Confidence: \texttt{low} (0.2852302375069697)

\item

  Retrieval method: \texttt{local}

\item

  Evidence type: \texttt{textual\ (scope/absence)}

\item

  Assumption flag: \texttt{False}

\item

  Answer:

\end{itemize}



None. The provided text focuses on statistical methods (ICSS, likelihood

ratio, Bayesian) for detecting variance changes, simulations, and an IBM

stock price example; it does not consider or imply any policy

counterfactuals.



\begin{itemize}

\tightlist

\item

  Quote snippet:

\end{itemize}



\begin{quote}

The main interest is to study the variance of a given sequence of

observations retrospectively, so we can use all the information on the

series to indicate the points of variance change.

\end{quote}



\begin{itemize}

\tightlist

\item

  Citation anchors:



  \begin{itemize}

  \tightlist

  \item

    page=2, words=560-620, section=Centered Cumulative Sums of Squares,

    note=States the methodological focus on detecting variance changes

    retrospectively.

  \item

    page=7, words=20-120, section=4. SIMULATION EXPERIMENT,

    note=Describes simulation to compare procedures, not policy

    evaluation.

  \item

    page=6, words=300-380, section=3.1 Example: IBM Stock Prices,

    note=Application example is stock prices, not a policy setting.

  \end{itemize}

\end{itemize}



\subsubsection{I02: What are the main welfare tradeoffs or

distributional impacts

discussed?}\label{i02-what-are-the-main-welfare-tradeoffs-or-distributional-impacts-discussed}



\begin{itemize}

\tightlist

\item

  Category: \texttt{I)\ Policy\ counterfactuals\ and\ welfare}

\item

  Confidence: \texttt{low} (0.24546938630512277)

\item

  Retrieval method: \texttt{local}

\item

  Evidence type: \texttt{absence\_in\_context}

\item

  Data source: Inclán and Tiao (1994), Journal of the American

  Statistical Association excerpt

\item

  Assumption flag: \texttt{False}

\item

  Answer:

\end{itemize}



None. The provided context is a methodological/statistical discussion

(ICSS algorithm, likelihood ratios, sampling distributions, and CPU-time

costs). It does not address welfare tradeoffs or socioeconomic

distributional impacts.



\begin{itemize}

\tightlist

\item

  Quote snippet:

\end{itemize}



\begin{quote}

the main way is by the number of "correct identifications." Another

aspect to consider is the computational requirements of each approach,

which can be measured in terms of the CPU time.

\end{quote}



\begin{itemize}

\tightlist

\item

  Citation anchors:



  \begin{itemize}

  \tightlist

  \item

    page=7, words=0-349, section=4. SIMULATION EXPERIMENT, note=Focuses

    on performance metrics and computational requirements, not welfare

    or distributional impacts.

  \item

    page=2, words=550-899, section=2. CENTERED CUMULATIVE SUMS OF

    SQUARES, note=Mentions measuring cost in CPU time; methodological

    emphasis.

  \item

    page=9, words=550-899, section=4.2 Results of Part One,

    note=Discusses detection performance and computational burden; no

    welfare analysis.

  \end{itemize}

\end{itemize}



\subsubsection{I03: Are cost-benefit or incidence analyses

provided?}\label{i03-are-cost-benefit-or-incidence-analyses-provided}



\begin{itemize}

\tightlist

\item

  Category: \texttt{I)\ Policy\ counterfactuals\ and\ welfare}

\item

  Confidence: \texttt{low} (0.28150735669307536)

\item

  Retrieval method: \texttt{local}

\item

  Evidence type:

  \texttt{textual\ (CPU\ time\ comparison;\ simulation\ summaries)}

\item

  Data source: Provided article text (Sections 4 and 4.6)

\item

  Assumption flag: \texttt{True}

\item

  Assumption notes: Interprets `cost-benefit' as an economic evaluation

  and `incidence analyses' as occurrence rates of outcomes; neither is

  presented---only computational cost (CPU time) and simulation

  frequencies.

\item

  Answer:

\end{itemize}



No. The paper reports computational cost comparisons (CPU time) and

simulation frequencies but does not provide cost-benefit or incidence

analyses.



\begin{itemize}

\tightlist

\item

  Quote snippet:

\end{itemize}



\begin{quote}

Another aspect to consider is the computational requirements of each

approach, which can be measured in terms of the CPU time.

\end{quote}



\begin{itemize}

\tightlist

\item

  Citation anchors:



  \begin{itemize}

  \tightlist

  \item

    page=2, words=550-899, section=Introduction/Background, note=Cost

    framed explicitly as CPU time for comparing approaches.

  \item

    page=7, words=0-349, section=4. SIMULATION EXPERIMENT, note=States

    computational requirements measured in CPU time.

  \item

    page=9, words=550-899, section=4.6 CPU Time Requirements,

    note=Details on CPU time burden and comparisons.

  \item

    page=11, words=0-349, section=CPU time regression summary,

    note=Regression models summarizing CPU time; recommendation based on

    computational cost.

  \end{itemize}

\end{itemize}



\subsubsection{I04: What policy recommendations are stated or

implied?}\label{i04-what-policy-recommendations-are-stated-or-implied}



\begin{itemize}

\tightlist

\item

  Category: \texttt{I)\ Policy\ counterfactuals\ and\ welfare}

\item

  Confidence: \texttt{low} (0.21524732197355106)

\item

  Retrieval method: \texttt{local}

\item

  Evidence type: \texttt{textual}

\item

  Data source: Inclán \& Tiao (1994), Journal of the American

  Statistical Association, 89(427), 913--923.

\item

  Assumption flag: \texttt{True}

\item

  Assumption notes: Interprets ``policy recommendations'' as

  methodological/practical guidance for applying variance-change

  detection procedures; no public policy is discussed.

\item

  Answer:

\end{itemize}



No explicit public policy is given; the article offers methodological

recommendations: (1) Use the ICSS algorithm for detecting variance

changes, especially when computational burden is a concern. (2) Control

ICSS iterations to avoid cycling (cap iterations; declare convergence

when change-point locations move by \textless2 observations). (3) After

detecting change points, estimate segment variances to build a

homogeneous ``corrected series'' and compute residual diagnostics (e.g.,

Q statistic) on that corrected series. (4) For likelihood-ratio testing,

proceed sequentially with LR\_\{n-1,n\}, obtain critical values by

simulation, and standardize to an extreme value distribution. (5) For

Bayesian inference, compute posterior-odds K\_\{n,n-1\} up to a

specified maximum number of changes Nmax and choose NT maximizing the

conditional probability. (6) For time series with dynamics, apply ICSS

to residuals from fitted AR models. (7) Practical power guidance: small

samples (T≈100) and small variance ratios (A=2), especially with early

changes, are hard to detect; performance improves notably with longer

series (T≥200) or larger variance ratios (A=3).



\begin{itemize}

\tightlist

\item

  Quote snippet:

\end{itemize}



\begin{quote}

This "corrected series" should then be used to obtain other residual

diagnostics, like the Q statistic, which would be distorted if

\end{quote}



\begin{itemize}

\tightlist

\item

  Citation anchors:



  \begin{itemize}

  \tightlist

  \item

    page=5, words=825-1174, section=ICSS algorithm and implementation,

    note=Recommends iteration controls to avoid cycling; mentions

    practical cap (\textasciitilde20 iterations).

  \item

    page=6, words=550-899, section=ICSS convergence and post-detection

    use, note=Convergence rule (\textless2 observations movement) and

    recommendation to create a corrected series for subsequent

    diagnostics.

  \item

    page=8, words=0-329, section=Likelihood ratio strategy,

    note=Sequential LR\_\{n-1,n\}; critical values by simulation;

    extreme value standardization.

  \item

    page=9, words=275-624, section=Bayesian posterior odds procedure,

    note=Systematic use of posterior odds with specified Nmax; choose NT

    maximizing conditional probability.

  \item

    page=9, words=550-899, section=Performance and computational

    guidance, note=ICSS for computational efficiency; detection

    difficulty at T=100 and A=2; improved performance with T≥200 or A=3;

    apply ICSS to AR residuals.

  \end{itemize}

\end{itemize}



\subsubsection{J01: What missingness or attrition patterns are

reported?}\label{j01-what-missingness-or-attrition-patterns-are-reported}



\begin{itemize}

\tightlist

\item

  Category: \texttt{J)\ Data\ quality\ and\ integrity}

\item

  Confidence: \texttt{low} (0.304980734681012)

\item

  Retrieval method: \texttt{local}

\item

  Evidence type: \texttt{text}

\item

  Data source: Simulation experiment description and results text

\item

  Assumption flag: \texttt{True}

\item

  Assumption notes: Interprets

  \textquotesingle missingness/attrition\textquotesingle{} as data loss

  or dropouts; none reported. Only computational omissions in

  posterior-odds analyses are noted.

\item

  Answer:

\end{itemize}



No data missingness or attrition is reported. All simulated series were

complete of length T. The only omission was analytical: posterior odds

were not computed for T=500, and for T=200 only 100 (vs 1,000)

replicates were run.



\begin{itemize}

\tightlist

\item

  Quote snippet:

\end{itemize}



\begin{quote}

The posterior odds ratio was not obtained for series with T = 500... the

runs with T = 200 for the log posterior odds have only 100 replicates

instead of the 1,000...

\end{quote}



\begin{itemize}

\tightlist

\item

  Citation anchors:



  \begin{itemize}

  \tightlist

  \item

    page=7, words=275-556, section=Simulation experiment methods,

    note=States each replicate is a full series of length T of

    independent N(0,1) variables.

  \item

    page=9, words=0-349, section=Results/notes on computational

    feasibility, note=Explains posterior odds not obtained for T=500 and

    only 100 replicates for T=200.

  \end{itemize}

\end{itemize}



\subsubsection{J02: How are outliers handled (winsorization, trimming,

exclusions)?}\label{j02-how-are-outliers-handled-winsorization-trimming-exclusions}



\begin{itemize}

\tightlist

\item

  Category: \texttt{J)\ Data\ quality\ and\ integrity}

\item

  Confidence: \texttt{low} (0.3496926880881717)

\item

  Retrieval method: \texttt{local}

\item

  Evidence type: \texttt{textual}

\item

  Assumption flag: \texttt{False}

\item

  Answer:

\end{itemize}



Outliers are not handled by winsorization or trimming. Instead, users

should visually inspect Dk plots---outliers appear as sudden jumps

without a change in slope---and complement the variance-change search

with a dedicated outlier-detection procedure (e.g., Chang, Tiao, and

Chen 1988).



\begin{itemize}

\tightlist

\item

  Quote snippet:

\end{itemize}



\begin{quote}

a big outlier would create a significant peak in the Dk plot... it is

advisable to complement the search for variance changes with a procedure

for outlier detection

\end{quote}



\begin{itemize}

\tightlist

\item

  Citation anchors:



  \begin{itemize}

  \tightlist

  \item

    page=6, words=0-349, section=Section 3 (ICSS algorithm guidance),

    note=Outliers create significant peaks; use outlier detection

    alongside ICSS.

  \end{itemize}

\end{itemize}



\subsubsection{J03: Are there data audits or validation steps

described?}\label{j03-are-there-data-audits-or-validation-steps-described}



\begin{itemize}

\tightlist

\item

  Category: \texttt{J)\ Data\ quality\ and\ integrity}

\item

  Confidence: \texttt{low} (0.2451173391958245)

\item

  Retrieval method: \texttt{local}

\item

  Evidence type:

  \texttt{Algorithm\ description\ and\ simulation\ study\ reported\ in\ the\ text}

\item

  Table/Figure: \texttt{Table\ 1\ and\ Table\ 2}

\item

  Data source: Inclán \& Tiao (1994), Journal of the American

  Statistical Association

\item

  Assumption flag: \texttt{False}

\item

  Answer:

\end{itemize}



Yes. The ICSS procedure includes internal validation checks of detected

changepoints (re-checking each candidate given adjacent ones and

repeating until convergence with a defined tolerance), iteration

controls to prevent cycling, and external validation via simulation

(empirical quantile comparisons, standard errors, and a simulation study

assessing correct identifications and CPU time).



\begin{itemize}

\tightlist

\item

  Quote snippet:

\end{itemize}



\begin{quote}

Check each possible changepoint by calculating

Dk(a{[}cpj−1+1:cpj+1{]})\ldots{} Repeat Step 3 until the number of

changepoints does not change\ldots{} we consider that if each

changepoint is within two observations\ldots{} then the algorithm has

converged.

\end{quote}



\begin{itemize}

\tightlist

\item

  Citation anchors:



  \begin{itemize}

  \tightlist

  \item

    page=5, words=550-899, section=ICSS Algorithm Step 3 (validation and

    convergence), note=Re-check each candidate changepoint; repeat until

    number/locations stabilize within tolerance.

  \item

    page=5, words=825-1174, section=ICSS implementation safeguards,

    note=Controls over number of iterations to avoid cycling; practical

    convergence behavior.

  \item

    page=5, words=0-349, section=Asymptotic vs empirical quantiles,

    note=Table 1 compares asymptotic and empirical quantiles for max

    \textbar Dk\textbar{} via simulation.

  \item

    page=5, words=275-624, section=Empirical validation and Type I

    error, note=Percentages with standard errors; empirical 95th

    quantile lower than asymptotic value.

  \item

    page=7, words=0-349, section=Simulation experiment design,

    note=Simulation to compare ICSS with alternatives; uses correctness

    and CPU time as metrics.

  \end{itemize}

\end{itemize}



\subsubsection{J04: Is there evidence of reporting bias or selective

sample

inclusion?}\label{j04-is-there-evidence-of-reporting-bias-or-selective-sample-inclusion}



\begin{itemize}

\tightlist

\item

  Category: \texttt{J)\ Data\ quality\ and\ integrity}

\item

  Confidence: \texttt{low} (0.2814155547201351)

\item

  Retrieval method: \texttt{local}

\item

  Evidence type:

  \texttt{Textual\ evidence\ from\ methods/results\ and\ table\ notes}

\item

  Table/Figure: \texttt{Table\ 4\ note;\ Section\ 4\ design\ notes}

\item

  Data source: Journal article text (JASA 1994)

\item

  Assumption flag: \texttt{False}

\item

  Answer:

\end{itemize}



No. The paper transparently describes its simulation design, notes when

certain results were not computed due to CPU-time limits, and explains

the elimination of redundant variance permutations by symmetry---none of

which indicate reporting bias or selective sample inclusion.



\begin{itemize}

\tightlist

\item

  Quote snippet:

\end{itemize}



\begin{quote}

The posterior odds ratio was not obtained for series with T = 500,

because it would have required a total of 1,358 CPU hours.

\end{quote}



\begin{itemize}

\tightlist

\item

  Citation anchors:



  \begin{itemize}

  \tightlist

  \item

    page=9, words=0-349, section=Section 4 (Simulation Experiment),

    results discussion, note=Explicitly states omission of

    posterior-odds results for T=500 and reduced replicates for T=200

    due to computational cost.

  \item

    page=7, words=275-556, section=Section 4 (Simulation Experiment),

    design details, note=Explains elimination of some

    variance-permutation cases based on symmetry, not selective

    inclusion.

  \item

    page=10, words=0-84, section=Table note, note=Clarifies differing

    replicate counts across conditions; transparent reporting rather

    than selective.

  \end{itemize}

\end{itemize}



\subsubsection{K01: What goodness-of-fit or diagnostic metrics are

reported?}\label{k01-what-goodness-of-fit-or-diagnostic-metrics-are-reported}



\begin{itemize}

\tightlist

\item

  Category: \texttt{K)\ Model\ fit\ and\ diagnostics}

\item

  Confidence: \texttt{low} (0.31894037148092247)

\item

  Retrieval method: \texttt{local}

\item

  Evidence type: \texttt{text\_and\_tables}

\item

  Table/Figure: \texttt{Table\ 4;\ Table\ 5;\ Table\ 6}

\item

  Data source: Inclán \& Tiao (1994), Journal of the American

  Statistical Association

\item

  Assumption flag: \texttt{False}

\item

  Answer:

\end{itemize}



They report residual diagnostics using the Q statistic (to be applied to

the variance-corrected series), frequency distributions of the number of

detected variance changes (NT) for ICSS, LR tests, and Bayesian

posterior odds, and summaries of the sampling distribution of the

estimated changepoint location k* (mode, mean, and SD), noting its

skewness and bias toward the series middle.



\begin{itemize}

\tightlist

\item

  Quote snippet:

\end{itemize}



\begin{quote}

other residual diagnostics, like the Q statistic, which would be

distorted if calculated

\end{quote}



\begin{itemize}

\tightlist

\item

  Citation anchors:



  \begin{itemize}

  \tightlist

  \item

    page=6, words=550-899, section=Diagnostics (residuals/Q statistic),

    note=Mentions using residual diagnostics like the Q statistic on

    corrected series.

  \item

    page=7, words=275-556, section=4.1 Assessing the Evidence,

    note=Introduces Table 4: frequency distribution of NT for one

    change.

  \item

    page=8, words=0-329, section=Two changepoints results,

    note=Introduces Table 5: frequency distribution of NT for two

    changes.

  \item

    page=9, words=0-349, section=4.4 Sampling Distribution of k*,

    note=Table 6 provides Mode of k*, Mean of k*, SD of k*; discussion

    of skew/bias.

  \item

    page=9, words=275-624, section=4.4 Sampling Distribution of k*,

    note=Explains k* distribution is skewed and biased toward the

    middle.

  \end{itemize}

\end{itemize}



\subsubsection{K02: Are functional form choices tested (logs, levels,

nonlinearities)?}\label{k02-are-functional-form-choices-tested-logs-levels-nonlinearities}



\begin{itemize}

\tightlist

\item

  Category: \texttt{K)\ Model\ fit\ and\ diagnostics}

\item

  Confidence: \texttt{low} (0.3410795504216201)

\item

  Retrieval method: \texttt{local}

\item

  Evidence type: \texttt{text}

\item

  Assumption flag: \texttt{True}

\item

  Assumption notes: Conclusion based on absence of any mention of

  testing logs vs. levels or nonlinearities in the provided excerpts;

  only a log-difference transformation is described for the example.

\item

  Answer:

\end{itemize}



No. In the provided context, there is no evidence that functional-form

choices (logs vs. levels or nonlinearities) were tested. The example

simply uses first-differenced log prices and proceeds with

variance-change detection.



\begin{itemize}

\tightlist

\item

  Quote snippet:

\end{itemize}



\begin{quote}

The series analyzed is the first difference of the logarithm of the IBM

stock closing prices

\end{quote}



\begin{itemize}

\tightlist

\item

  Citation anchors:



  \begin{itemize}

  \tightlist

  \item

    page=6, words=275-624, section=3.1 Example: IBM Stock Prices,

    note=Uses first difference of log prices; no discussion of testing

    alternative functional forms.

  \item

    page=7, words=275-556, section=4. Simulation Experiment,

    note=Simulations use independent N(0,1) series; focus is on variance

    changes, not functional-form tests.

  \end{itemize}

\end{itemize}



\subsubsection{K03: Are residual checks or specification tests

reported?}\label{k03-are-residual-checks-or-specification-tests-reported}



\begin{itemize}

\tightlist

\item

  Category: \texttt{K)\ Model\ fit\ and\ diagnostics}

\item

  Confidence: \texttt{low} (0.32987787498865473)

\item

  Retrieval method: \texttt{local}

\item

  Evidence type: \texttt{textual}

\item

  Table/Figure: \texttt{Table\ 8}

\item

  Data source: Inclán, C., \& Tiao, G. C. (1994). Use of Cumulative Sums

  of Squares for Retrospective Detection of Changes of Variance. JASA,

  89(427), 913--923.

\item

  Assumption flag: \texttt{False}

\item

  Answer:

\end{itemize}



Yes. The paper discusses residual diagnostics (e.g., the Q statistic)

and applies the ICSS algorithm to AR(1) residuals, reporting results in

Table 8.



\begin{itemize}

\tightlist

\item

  Quote snippet:

\end{itemize}



\begin{quote}

This "corrected series" should then be used to obtain other residual

diagnostics, like the Q statistic

\end{quote}



\begin{itemize}

\tightlist

\item

  Citation anchors:



  \begin{itemize}

  \tightlist

  \item

    page=6, words=825-968, note=Mentions using residual diagnostics like

    the Q statistic after variance correction.

  \item

    page=9, words=550-899, note=States the ICSS algorithm was applied to

    AR(1) residuals and summarizes results.

  \item

    page=10, words=0-84, note=Table 8 title confirms residual-based

    results are reported.

  \end{itemize}

\end{itemize}



\subsubsection{K04: How sensitive are results to alternative

specifications or

estimators?}\label{k04-how-sensitive-are-results-to-alternative-specifications-or-estimators}



\begin{itemize}

\tightlist

\item

  Category: \texttt{K)\ Model\ fit\ and\ diagnostics}

\item

  Confidence: \texttt{medium} (0.3882154722563691)

\item

  Retrieval method: \texttt{local}

\item

  Evidence type:

  \texttt{Simulation\ comparisons,\ methodological\ notes,\ and\ algorithmic\ diagnostics}

\item

  Table/Figure: \texttt{Table\ 1;\ Tables\ 4–8}

\item

  Data source: Journal of the American Statistical Association (1994),

  Inclán and Tiao ICSS study

\item

  Assumption flag: \texttt{False}

\item

  Answer:

\end{itemize}



Results are sensitive to the chosen test/estimator, critical values, and

data-generation specifications. Method choice: For small variance

changes (A=2), the Bayesian posterior-odds approach performs best, while

ICSS improves notably with larger samples or larger variance ratios and

often outperforms likelihood-ratio (LR) tests; LR is frequently weaker

in multi-change settings. Threshold choice: Using empirical quantiles

for max\textbar Dk\textbar{} yields lower critical values than the

asymptotic 1.358, reducing type I error relative to the asymptotic

boundary, so conclusions depend on whether asymptotic or

simulation-based cutoffs are used. Model structure and iteration:

Accounting for adjacent changepoints can remove previously `significant'

candidates (e.g., point 342 becomes non-significant after neighboring

points are included), indicating sensitivity to the iterative

specification. Estimation context: Applying ICSS to AR(1) residuals

using OLS produces detection rates very close to the i.i.d. case,

suggesting limited sensitivity to estimating the AR parameter. Estimator

of location k*: its sampling distribution is skewed (biased toward the

center), but the mode is at the true change; concentration improves with

larger T or larger variance ratios. Performance also depends on

changepoint locations and variance configurations (hardest under

monotone increases; easier with equal segment lengths or when the large

variance is in the middle).



\begin{itemize}

\tightlist

\item

  Quote snippet:

\end{itemize}



\begin{quote}

The Bayesian procedure is the best for A = 2 and gives comparable

results to the other approaches with A = 3.

\end{quote}



\begin{itemize}

\tightlist

\item

  Citation anchors:



  \begin{itemize}

  \tightlist

  \item

    page=9, words=640-880, section=4.2, note=Bayesian best for A=2; ICSS

    improves with longer series/larger ratios; LR comparison

  \item

    page=5, words=470-620, section=Quantiles/Type I error,

    note=Empirical 95th quantile lower than asymptotic 1.358; smaller

    empirical type I error

  \item

    page=6, words=560-720, section=Algorithm Step 3, note=Point 342

    becomes non-significant after considering adjacent changepoints

  \item

    page=9, words=820-899, section=Residuals, note=ICSS on AR(1)

    residuals: percentages close to Table 4 (i.i.d. case)

  \item

    page=9, words=275-540, section=4.4, note=k* distribution skewed;

    biased to middle; mode at true change; concentration increases with

    T or A

  \item

    page=9, words=0-220, section=4.3, note=Hardest when variances change

    monotonically; ICSS often better than LR with T≥200; best at equal

    segment lengths or large variance in middle

  \item

    page=3, words=190-349, section=2.1, note=Distinction between F

    (known k) and searching max\textbar Dk\textbar{} (unknown k)

  \end{itemize}

\end{itemize}



\end{document}

