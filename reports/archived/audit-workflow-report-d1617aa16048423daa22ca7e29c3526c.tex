% Options for packages loaded elsewhere
\PassOptionsToPackage{unicode}{hyperref}
\PassOptionsToPackage{hyphens}{url}
\documentclass[
]{article}
\usepackage{xcolor}
\usepackage{amsmath,amssymb}
\setcounter{secnumdepth}{-\maxdimen} % remove section numbering
\usepackage{iftex}
\ifPDFTeX
  \usepackage[T1]{fontenc}
  \usepackage[utf8]{inputenc}
  \usepackage{textcomp} % provide euro and other symbols
\else % if luatex or xetex
  \usepackage{unicode-math} % this also loads fontspec
  \defaultfontfeatures{Scale=MatchLowercase}
  \defaultfontfeatures[\rmfamily]{Ligatures=TeX,Scale=1}
\fi
\usepackage{lmodern}
\ifPDFTeX\else
  % xetex/luatex font selection
\fi
% Use upquote if available, for straight quotes in verbatim environments
\IfFileExists{upquote.sty}{\usepackage{upquote}}{}
\IfFileExists{microtype.sty}{% use microtype if available
  \usepackage[]{microtype}
  \UseMicrotypeSet[protrusion]{basicmath} % disable protrusion for tt fonts
}{}
\makeatletter
\@ifundefined{KOMAClassName}{% if non-KOMA class
  \IfFileExists{parskip.sty}{%
    \usepackage{parskip}
  }{% else
    \setlength{\parindent}{0pt}
    \setlength{\parskip}{6pt plus 2pt minus 1pt}}
}{% if KOMA class
  \KOMAoptions{parskip=half}}
\makeatother
\setlength{\emergencystretch}{3em} % prevent overfull lines
\providecommand{\tightlist}{%
  \setlength{\itemsep}{0pt}\setlength{\parskip}{0pt}}
\usepackage{bookmark}
\IfFileExists{xurl.sty}{\usepackage{xurl}}{} % add URL line breaks if available
\urlstyle{same}
\hypersetup{
  hidelinks,
  pdfcreator={LaTeX via pandoc}}

\author{}
\date{}

\begin{document}

\section{\texorpdfstring{Audit Report: Workflow \texttt{d1617aa16048423daa22ca7e29c3526c}}{Audit Report: Workflow d1617aa16048423daa22ca7e29c3526c}}\label{audit-report-workflow-d1617aa16048423daa22ca7e29c3526c}

\subsection{Overview}\label{overview}

\begin{itemize}
\tightlist
\item
  Source JSON: \texttt{reports\textbackslash{}workflow-report-d1617aa16048423daa22ca7e29c3526c.json}
\item
  Run ID: \texttt{d1617aa16048423daa22ca7e29c3526c}
\item
  Papers input: \texttt{papers\textbackslash{}alcott\ food\ deserts.pdf}
\item
  Started at: \texttt{2026-02-15T19:13:11.001652+00:00}
\item
  Finished at: \texttt{2026-02-15T19:27:16.173575+00:00}
\item
  Duration: \texttt{0:14:05.171923}
\end{itemize}

\subsection{Effective Configuration}\label{effective-configuration}

\begin{itemize}
\tightlist
\item
  Chat model: \texttt{gpt-5}
\item
  Embedding model: \texttt{text-embedding-3-large}
\item
  Top K: \texttt{10}
\item
  Chunk words / overlap: \texttt{350} / \texttt{75}
\item
  Batch size: \texttt{64}
\item
  Database URL configured: \texttt{True}
\end{itemize}

\subsection{Step Outcomes}\label{step-outcomes}

\begin{itemize}
\tightlist
\item
  \texttt{prep}: \texttt{completed}
\item
  \texttt{ingest}: \texttt{num\_pdfs=1,\ num\_papers=1}
\item
  \texttt{enrich}: \texttt{openalex=0,\ citec=0}
\item
  \texttt{econ\_data}: \texttt{fetched}
\item
  \texttt{agentic}: \texttt{completed}
\item
  \texttt{index}: \texttt{skipped\ (reason:\ }db\_unreachable\texttt{)}
\item
  \texttt{report\_store}: \texttt{skipped\ (reason:\ }db\_unreachable\texttt{)}
\end{itemize}

\subsection{Agentic Summary}\label{agentic-summary}

\begin{itemize}
\tightlist
\item
  Status: \texttt{completed}
\item
  Main question: What is the key contribution?
\item
  Report question set: \texttt{both}
\item
  Structured questions generated: \texttt{84}
\item
  Confidence mean/median: \texttt{0.3041450240118245} / \texttt{0.2994701740878031}
\item
  Confidence labels: low=65, medium=18, high=0
\end{itemize}

\subsubsection{Final Answer}\label{final-answer}

\begin{itemize}
\item
  Quantifies the drivers of the nutrition--income gradient, showing that roughly 90\% is due to demand (preferences) and about 10\% to supply (prices/availability), across robust specifications; assigns the unobserved term \(\hat{\xi}\) to demand because it primarily governs demand elasticity with respect to price {[}page 41, words 275--411; page 38, words 275--377{]}.
\item
  Identifies supply-driven price variation using substantial geographic differences in the footprint of top retailers that set distinct relative prices across product groups, with market fixed effects to address endogeneity; validates the approach via a replicated ``migration'' effect for Coke vs. Pepsi (τ̂ ≈ 0.16: a 10 pp higher county Coke share raises a household's Coke share by 1.6 pp) {[}page 32, words 275--431; page 27, words 275--482{]}.
\item
  Estimates nutrient-level preferences by income quartile and introduces a WTP for Health Index, \$ \textbackslash sum\_c \textbackslash hat\{\textbackslash beta\}\_c s\_c r\_c \$, that increases monotonically with income (Q1 ≈ 0.429 → Q4 ≈ 1.141), with higher-income households exhibiting more negative WTP for sodium/added sugar and more positive WTP for whole fruit/whole grains {[}page 36, words 0--349{]}.
\item
  Grounds the analysis in rich linked scanner data (Homescan + RMS), arguing packaged-goods focus reasonably represents produce purchasing in the magnet subsample and documenting broad store/channel coverage with national weighting and CPI-U deflation {[}page 8, words 0--349; page 8, words 275--393{]}.
\end{itemize}

\subsubsection{Sub-Answers}\label{sub-answers}

\paragraph{Sub-answer 1}\label{sub-answer-1}

\begin{itemize}
\item
  Question: ResponseTextConfig(format=ResponseFormatText(type=\textquotesingle text\textquotesingle), verbosity=\textquotesingle medium\textquotesingle)
\item
  Question tokens estimate: \texttt{19}
\item
  Answer:
\item
  Data sources and coverage

  \begin{itemize}
  \tightlist
  \item
    Homescan records packaged groceries; in 2004--2006, an 8,000-household ``magnet'' subsample also recorded nonpackaged items. About 60\% of magnet households' produce calories are from packaged goods, with no statistical income differences, suggesting packaged focus is reasonably representative for produce and that produce is a small share of groceries overall (page 8, words 0--349).
  \item
    Households report income (16 bins), composition, demographics; researchers compute daily calorie needs using Dietary Guidelines and scale children to adult equivalents. PanelViews add-on (2017) provides health importance, nutrition quiz, BMI, diabetes; results are nationally weighted (page 8, words 0--349).
  \item
    RMS: weekly UPC-level prices/volumes at \textasciitilde42,000 stores from 160 chains (≈35,000/year observed) for 2006--2016; excludes liquor and bulk produce. Channel coverage: 53\% grocery, 32\% mass merchandiser, 55\% drug, 2\% convenience. Prices/incomes deflated to 2010 dollars using CPI-U (page 8, words 275--393).
  \end{itemize}
\item
  Identification and market structure facts

  \begin{itemize}
  \tightlist
  \item
    Substantial geographic variation in the presence of the five largest retailers; these chains set different relative prices across product groups, yielding geographic price instruments. Example: produce predicted cheap on West Coast and expensive on East Coast, likely due to California production; market fixed effects address endogeneity from demand elasticity differences across markets (page 32, words 275--431).
  \item
    Within-household regressions: adding demographic controls changes results little and increases R\^{}2 only slightly, implying possible unobserved within-household changes (Oster 2019). A migration-style thought experiment considers moving a household from low- to high-income retail environments (partial equilibrium) (page 27, words 275--482).
  \item
    Validation: replicates immediate brand choice migration effect for Coke vs. Pepsi with τ̂ ≈ 0.16; a 10 pp higher county Coke share raises a household's Coke share by 1.6 pp (page 27, words 275--482).
  \end{itemize}
\item
  Preference estimates (Table IV) and WTP for health

  \begin{itemize}
  \tightlist
  \item
    GMM estimates of nutrient preference parameters by income quartile show:

    \begin{itemize}
    \tightlist
    \item
      Higher-income quartiles have more negative WTP for sodium and added sugar, and more positive WTP for whole fruit and whole grains; patterns are monotonic across quartiles (page 36, words 0--349).
    \item
      ``WTP for Health Index'' rises with income: Q1 ≈ 0.429, Q2 ≈ 0.631, Q3 ≈ 0.820, Q4 ≈ 1.141 (SEs shown in table) (page 36, words 0--349).
    \end{itemize}
  \item
    Units: sodium (g), whole/other fruit and dairy (cups), whole/refined grains and proteins (oz), added sugar (tsp), solid fats (calories). WTP for Health Index defined as \(\sum_c \hat{\beta}_c s_c r_c\) (page 36, words 0--349).
  \end{itemize}
\item
  Decomposition: supply vs. demand in nutrition--income gaps

  \begin{itemize}
  \tightlist
  \item
    Across alternative specifications (including census-region-by--product group fixed effects and interactions with urban/rural and above-median income), supply explains 7\%--12\% and demand explains 88\%--93\% of the nutrition--income relationship; summary: about 90\% demand, 10\% supply (page 41, words 275--411; page 38, words 275--377).
  \item
    The unobserved characteristic \(\hat{\xi}\) is attributed to demand because it primarily determines demand elasticity with respect to price (page 41, words 275--411).
  \end{itemize}
\item
  Demographic controls used in outcomes regressions

  \begin{itemize}
  \tightlist
  \item
    Controls include household size, age indicators, log years of education, race, marital status, employment, weekly work hours, plus add-on variables such as the self-reported importance of staying healthy (page 42, words 275--356; page 8, words 0--349).
  \end{itemize}
\end{itemize}

\subsection{Structured Q\&A Appendix}\label{structured-qa-appendix}

This section mirrors \texttt{agentic.report\_questions} for audit traceability.

\subsubsection{A01: What is the main research question of the paper?}\label{a01-what-is-the-main-research-question-of-the-paper}

\begin{itemize}
\tightlist
\item
  Category: \texttt{A)\ Research\ question\ /\ contribution}
\item
  Confidence: \texttt{low\ (0.33061339623330005)}
\item
  Retrieval method: \texttt{local}
\item
  Evidence type: \texttt{direct\ textual\ evidence}
\item
  Data source: Allcott et al., Quarterly Journal of Economics (2020)
\item
  Assumption flag: \texttt{False}
\item
  Answer:
\end{itemize}

The paper asks what causes nutritional inequality---why higher‑income Americans eat more healthfully than lower‑income Americans---and examines whether neighborhood food environments (supply) versus household demand explain this gap.

\begin{itemize}
\tightlist
\item
  Quote snippet:
\end{itemize}

\begin{quote}
We study the causes of ``nutritional inequality'': why the wealthy eat more healthfully than the poor in the United States.
\end{quote}

\begin{itemize}
\tightlist
\item
  Citation anchors:

  \begin{itemize}
  \tightlist
  \item
    page=1, words=35-70, section=Introduction, note=States the study's focus on causes of nutritional inequality and why the wealthy eat more healthfully than the poor.
  \end{itemize}
\end{itemize}

\subsubsection{A02: What is the paper\textquotesingle s primary contribution relative to the existing literature?}\label{a02-what-is-the-papers-primary-contribution-relative-to-the-existing-literature}

\begin{itemize}
\tightlist
\item
  Category: \texttt{A)\ Research\ question\ /\ contribution}
\item
  Confidence: \texttt{low\ (0.3459851165680258)}
\item
  Retrieval method: \texttt{local}
\item
  Evidence type: \texttt{direct}
\item
  Data source: QJE 2020 article text (Allcott et al.)
\item
  Assumption flag: \texttt{False}
\item
  Answer:
\end{itemize}

The paper's primary contribution is to overturn the food-desert explanation for nutritional inequality by showing---using supermarket entry and migration event studies plus a structural demand model with a novel cross-market price instrument---that equalizing access to products and prices would reduce the gap by only about 10\%, with roughly 90\% driven by demand differences, implying limited impact of supply-side policies.

\begin{itemize}
\tightlist
\item
  Quote snippet:
\end{itemize}

\begin{quote}
reduces nutritional inequality by only about 10\%, while the remaining 90\% is driven by differences in demand.
\end{quote}

\begin{itemize}
\tightlist
\item
  Citation anchors:

  \begin{itemize}
  \tightlist
  \item
    page=2, words=0-349, section=I. INTRODUCTION, note=States 10\% reduction from equalizing supply; 90\% due to demand; counters supply-side policy arguments.
  \item
    page=1, words=0-349, section=Abstract/Overview, note=Rejects meaningful neighborhood (food-desert) effects; introduces structural model and new instrument.
  \item
    page=34, words=0-349, section=VI.C / Instrument validity, note=Describes the price instrument and notes it is novel in the literature.
  \item
    page=42, words=0-349, section=VII.B, note=Reiterates that findings counter food-desert arguments; supply equalization cuts gap by \textasciitilde10\%.
  \end{itemize}
\end{itemize}

\subsubsection{A03: What is the central hypothesis being tested?}\label{a03-what-is-the-central-hypothesis-being-tested}

\begin{itemize}
\tightlist
\item
  Category: \texttt{A)\ Research\ question\ /\ contribution}
\item
  Confidence: \texttt{low\ (0.29062503471444984)}
\item
  Retrieval method: \texttt{local}
\item
  Evidence type: \texttt{Within-household\ event\ study\ of\ movers\ (fixed-effects),\ complemented\ by\ structural\ demand\ model\ counterfactuals}
\item
  Data source: Nielsen Homescan panel and RMS retail data
\item
  Assumption flag: \texttt{False}
\item
  Answer:
\end{itemize}

Whether local place/supply conditions---the neighborhood retail food environment (``food deserts'')---causally change households' healthy grocery purchasing enough to explain the nutrition--income gap.

\begin{itemize}
\tightlist
\item
  Quote snippet:
\end{itemize}

\begin{quote}
To test for place effects, we measure within-household changes in grocery purchases after the 20,031 cross--ZIP code moves and the 11,728 cross-county moves...
\end{quote}

\begin{itemize}
\tightlist
\item
  Citation anchors:

  \begin{itemize}
  \tightlist
  \item
    page=23, words=275-408, section=Event study of place effects, note=Defines the test for place effects using within-household changes after moves.
  \item
    page=4, words=275-423, section=Summary of move-based findings, note=Reports small convergence after moves and bounds place effects at 2\%--3\%.
  \item
    page=42, words=0-349, section=Interpretation relative to food deserts, note=States the findings counter arguments that food deserts drive nutritional inequality.
  \item
    page=48, words=0-323, section=Conclusion on place effects, note=Concludes place effects explain no more than \textasciitilde3\% of differences in healthy eating.
  \end{itemize}
\end{itemize}

\subsubsection{A04: What are the main outcomes of interest (dependent variables)?}\label{a04-what-are-the-main-outcomes-of-interest-dependent-variables}

\begin{itemize}
\tightlist
\item
  Category: \texttt{A)\ Research\ question\ /\ contribution}
\item
  Confidence: \texttt{low\ (0.3047568666778385)}
\item
  Retrieval method: \texttt{local}
\item
  Evidence type: \texttt{Table\ and\ text\ from\ the\ paper}
\item
  Data source: 2004--2016 Nielsen Homescan
\item
  Table/Figure: Table II
\item
  Assumption flag: \texttt{False}
\item
  Answer:
\end{itemize}

Expenditure shares (including the entrant expenditure share and the share at grocery/supercenter/club stores) and the Health Index of household grocery purchases.

\begin{itemize}
\tightlist
\item
  Quote snippet:
\end{itemize}

\begin{quote}
Expenditure shares are the share of total grocery expenditures recorded in Homescan, in units of percentage points.
\end{quote}

\begin{itemize}
\tightlist
\item
  Citation anchors:

  \begin{itemize}
  \tightlist
  \item
    page=21, words=0-349, section=Table II notes, note=Defines expenditure shares and the Health Index and shows these as dependent variables in Panels A and B.
  \item
    page=18, words=0-349, section=Figure IV description, note=Defines the entrant expenditure share outcome.
  \end{itemize}
\end{itemize}

\subsubsection{A05: What are the key treatment/exposure variables (independent variables)?}\label{a05-what-are-the-key-treatmentexposure-variables-independent-variables}

\begin{itemize}
\tightlist
\item
  Category: \texttt{A)\ Research\ question\ /\ contribution}
\item
  Confidence: \texttt{low\ (0.24192713713243177)}
\item
  Retrieval method: \texttt{local}
\item
  Evidence type: \texttt{text}
\item
  Data source: Nielsen Homescan panel and RMS store data
\item
  Assumption flag: \texttt{False}
\item
  Answer:
\end{itemize}

Two main exposures: (1) Ebcqt---an indicator for whether a supermarket entered within distance band b of a tract by q quarters after quarter t (event-study of entry); (2) Hm---the local (ZIP/county) Health Index of packaged groceries, used in the movers design to capture the retail nutrition environment.

\begin{itemize}
\tightlist
\item
  Quote snippet:
\end{itemize}

\begin{quote}
We define Ebcqt as an indicator variable denoting whether one supermarket had entered in distance band b of census tract c by q quarters after quarter t.
\end{quote}

\begin{itemize}
\tightlist
\item
  Citation anchors:

  \begin{itemize}
  \tightlist
  \item
    page=17, words=275-411, section=Event study: supermarket entry, note=Defines Ebcqt as the entry indicator by distance band and event time.
  \item
    page=24, words=0-349, section=Place effects (movers) setup, note=Defines Hm (area Health Index) and its role as the key regressor in movers regression.
  \item
    page=25, words=275-468, section=Place effects identification, note=Variation in local Health Index across moves identifies the effect (supports Hm as exposure).
  \item
    page=21, words=275-503, section=Entry variants, note=Notes analyses considering entry by supercenters as a variant of the entry exposure.
  \end{itemize}
\end{itemize}

\subsubsection{A06: What setting/context does the paper study (country, market, period)?}\label{a06-what-settingcontext-does-the-paper-study-country-market-period}

\begin{itemize}
\tightlist
\item
  Category: \texttt{A)\ Research\ question\ /\ contribution}
\item
  Confidence: \texttt{medium\ (0.3864847718005176)}
\item
  Retrieval method: \texttt{local}
\item
  Evidence type: \texttt{text}
\item
  Data source: Nielsen Homescan and Nielsen Retail Measurement (RMS)
\item
  Assumption flag: \texttt{False}
\item
  Answer:
\end{itemize}

United States; the grocery/supermarket retail market (packaged groceries via Nielsen Homescan and RMS); period 2004--2016 for Homescan and 2006--2016 for RMS.

\begin{itemize}
\tightlist
\item
  Quote snippet:
\end{itemize}

\begin{quote}
The Nielsen RMS data consist of weekly prices and sales vol- umes for each UPC sold at approximately 42,000 unique stores from 160 retail chains for 2006--2016
\end{quote}

\begin{itemize}
\tightlist
\item
  Citation anchors:

  \begin{itemize}
  \tightlist
  \item
    page=8, words=0-349, section=Data description, note=RMS market coverage and period (grocery retail, 2006--2016).
  \item
    page=26, words=0-282, section=Figure V caption, note=Homescan period stated as 2004--2016.
  \item
    page=24, words=0-349, section=Place effects estimation setup, note=Uses U.S. geographic units (ZIP codes, counties), indicating U.S. setting.
  \item
    page=1, words=275-483, section=Acknowledgments, note=References to Nielsen Company (US) and USDA ERS, consistent with U.S. setting.
  \end{itemize}
\end{itemize}

\subsubsection{A07: What is the main mechanism proposed by the authors?}\label{a07-what-is-the-main-mechanism-proposed-by-the-authors}

\begin{itemize}
\tightlist
\item
  Category: \texttt{A)\ Research\ question\ /\ contribution}
\item
  Confidence: \texttt{low\ (0.28020166763200416)}
\item
  Retrieval method: \texttt{local}
\item
  Evidence type: \texttt{Event-study\ and\ structural\ demand\ model\ estimates}
\item
  Data source: Nielsen Homescan household panel and RMS retail scanner data
\item
  Table/Figure: Figure VII (Explaining the Relationship between Income and Healthy Grocery Demand)
\item
  Assumption flag: \texttt{False}
\item
  Answer:
\end{itemize}

Differences in consumer demand---driven by factors like preferences, education, and nutrition knowledge---rather than neighborhood food supply, are the main mechanism behind nutritional inequality (about 90\% demand vs. 10\% supply; place effects ≲3\%).

\begin{itemize}
\tightlist
\item
  Quote snippet:
\end{itemize}

\begin{quote}
Only about 10\% of the nutrition--income relationship is driven by these differences in supply, while 90\% of the relationship is driven by differences in demand.
\end{quote}

\begin{itemize}
\tightlist
\item
  Citation anchors:

  \begin{itemize}
  \tightlist
  \item
    page=6, words=0-349, note=States \textasciitilde10\% of nutrition--income relationship is due to supply, \textasciitilde90\% due to demand.
  \item
    page=42, words=0-349, section=VII.B, note=Emphasizes demand-side factors, not supply, are central.
  \item
    page=48, words=0-323, section=Conclusion, note=Place effects explain no more than 3\%; equalizing supply reduces gap by ≤10\%.
  \item
    page=43, words=0-349, section=VII.B, note=Education and nutrition knowledge help explain demand differences.
  \end{itemize}
\end{itemize}

\subsubsection{A08: What alternative mechanisms are discussed?}\label{a08-what-alternative-mechanisms-are-discussed}

\begin{itemize}
\tightlist
\item
  Category: \texttt{A)\ Research\ question\ /\ contribution}
\item
  Confidence: \texttt{low\ (0.21443295838395354)}
\item
  Retrieval method: \texttt{local}
\item
  Evidence type: \texttt{text\_quote}
\item
  Table/Figure: Figure VII
\item
  Assumption flag: \texttt{False}
\item
  Answer:
\end{itemize}

Demand-side mechanisms---especially education and nutrition knowledge---are discussed as alternatives to supply-side factors (like prices and availability) for explaining differences in healthy eating.

\begin{itemize}
\tightlist
\item
  Quote snippet:
\end{itemize}

\begin{quote}
Education explains the largest share of the relationship between demand for healthy groceries and income, at about 20\%. Nutrition knowledge explains the second-largest share, at about 14\%.
\end{quote}

\begin{itemize}
\tightlist
\item
  Citation anchors:

  \begin{itemize}
  \tightlist
  \item
    page=43, words=0-349, section=VII.B, note=Education and nutrition knowledge cited as key contributors
  \item
    page=38, words=275-377, section=VII.A, note=Decomposition into supply-side vs demand-side factors
  \end{itemize}
\end{itemize}

\subsubsection{A09: What are the main policy implications claimed by the paper?}\label{a09-what-are-the-main-policy-implications-claimed-by-the-paper}

\begin{itemize}
\tightlist
\item
  Category: \texttt{A)\ Research\ question\ /\ contribution}
\item
  Confidence: \texttt{medium\ (0.394069770413192)}
\item
  Retrieval method: \texttt{local}
\item
  Evidence type: \texttt{Empirical\ analyses\ and\ model-based\ counterfactual\ simulations}
\item
  Data source: Provided context excerpt from The Quarterly Journal of Economics (2020) article
\item
  Assumption flag: \texttt{True}
\item
  Assumption notes: Policy conclusions about subsidies rely on model-based simulations with assumptions; authors recommend randomized demonstrations and welfare analysis before adoption.
\item
  Answer:
\end{itemize}

The paper argues that supply-side policies to eliminate ``food deserts'' (e.g., promoting supermarket entry or equalizing local supply conditions) have very limited impact on healthy eating and explain little of the income--nutrition gap. It recommends shifting policy focus toward demand-side tools---specifically, means-tested subsidies for healthy foods (potentially through SNAP)---which model-based simulations suggest could cost-effectively raise low-income households' healthy purchases to high-income levels. The authors also urge testing such subsidies via randomized demonstrations and note that any adoption should consider welfare analysis and practical constraints.

\begin{itemize}
\tightlist
\item
  Quote snippet:
\end{itemize}

\begin{quote}
policy makers interested in reducing nutritional inequality might redirect efforts away from promoting access to healthy groceries and toward means-tested subsidies.
\end{quote}

\begin{itemize}
\tightlist
\item
  Citation anchors:

  \begin{itemize}
  \tightlist
  \item
    page=2, words=0-349, section=Introduction, note=Only \textasciitilde10\% of nutritional inequality from supply; counters policies focused on increasing healthy grocery supply.
  \item
    page=22, words=0-349, section=Empirical results, note=Local supermarket access explains ≈1.5\% of the Health Index gap.
  \item
    page=46, words=0-349, section=VII.C. Using Subsidies to Reduce Nutritional Inequality, note=Cost comparisons: \$84M vs supermarket entry; \$830M to equalize supply; \$11B (\textasciitilde15\% of SNAP) to reach top-quartile health level.
  \item
    page=47, words=0-349, section=VII.C and VIII. Conclusion, note=Recommendation to redirect efforts from access to means-tested subsidies; call for RCTs; note assumptions and considerations.
  \item
    page=48, words=0-323, section=Conclusion, note=Equalizing supply cuts gap by ≤10\%; means-tested subsidy could close gap at \textasciitilde15\% of SNAP; eliminating food deserts yields little progress.
  \end{itemize}
\end{itemize}

\subsubsection{A10: What is the welfare interpretation (if any) of the results?}\label{a10-what-is-the-welfare-interpretation-if-any-of-the-results}

\begin{itemize}
\tightlist
\item
  Category: \texttt{A)\ Research\ question\ /\ contribution}
\item
  Confidence: \texttt{medium\ (0.38966701151833294)}
\item
  Retrieval method: \texttt{local}
\item
  Evidence type: \texttt{textual\_evidence}
\item
  Assumption flag: \texttt{False}
\item
  Answer:
\end{itemize}

The paper does not provide a welfare analysis. Its cost comparisons are illustrative, not full measures of social costs, and the authors state that welfare conclusions would require measuring market failures and conducting an optimal policy analysis in a welfare-maximization framework.

\begin{itemize}
\tightlist
\item
  Quote snippet:
\end{itemize}

\begin{quote}
Before advocating for or against such a subsidy, one would need to measure the relevant market failures and study optimal policy in a principled welfare maximization framework.
\end{quote}

\begin{itemize}
\tightlist
\item
  Citation anchors:

  \begin{itemize}
  \tightlist
  \item
    page=46, words=0-349, section=VII.C. Using Subsidies to Reduce Nutritional Inequality, note=Cautions that government expenditures are not a complete measure of social costs, indicating no full welfare evaluation.
  \item
    page=48, words=0-323, section=VII.C. Using Subsidies to Reduce Nutritional Inequality / Summary, note=Explicitly states the need for a welfare maximization framework before advocating policies.
  \end{itemize}
\end{itemize}

\subsubsection{A11: What are the main limitations acknowledged by the authors?}\label{a11-what-are-the-main-limitations-acknowledged-by-the-authors}

\begin{itemize}
\tightlist
\item
  Category: \texttt{A)\ Research\ question\ /\ contribution}
\item
  Confidence: \texttt{low\ (0.2796642781746991)}
\item
  Retrieval method: \texttt{local}
\item
  Evidence type: \texttt{paraphrase\_with\_quote}
\item
  Assumption flag: \texttt{False}
\item
  Answer:
\end{itemize}

The authors note several limitations: (1) their structural model relies on many assumptions and so the results should be viewed as suggestive rather than definitive, motivating randomized demonstration projects; (2) some key findings (e.g., contributions of education and nutrition knowledge) are correlational and not causal; (3) estimates of place effects are limited to the several-year time horizon allowed by the data and rely on an assumption that any endogeneity would bias effects upward, and reflect only partial-equilibrium impacts; and (4) before advocating subsidies, relevant market failures must be measured and there are substantial economic and political considerations around modifying SNAP.

\begin{itemize}
\tightlist
\item
  Quote snippet:
\end{itemize}

\begin{quote}
Our model requires many assumptions, so we view these results as suggestive and a potential motivation for demonstration projects, ideally using randomized experiments\ldots{}
\end{quote}

\begin{itemize}
\tightlist
\item
  Citation anchors:

  \begin{itemize}
  \tightlist
  \item
    page=47, words=0-349, note=Model requires many assumptions; results suggestive; need randomized experiments; policy considerations around SNAP.
  \item
    page=48, words=0-323, note=Limited several-year time horizon; assumption on endogeneity direction; partial equilibrium place effects; need to measure market failures before advocating subsidies.
  \item
    page=43, words=0-349, note=Decomposition results are correlations, not causal effects.
  \end{itemize}
\end{itemize}

\subsubsection{A12: What does the paper claim is novel about its data or identification?}\label{a12-what-does-the-paper-claim-is-novel-about-its-data-or-identification}

\begin{itemize}
\tightlist
\item
  Category: \texttt{A)\ Research\ question\ /\ contribution}
\item
  Confidence: \texttt{low\ (0.3121536646348006)}
\item
  Retrieval method: \texttt{local}
\item
  Evidence type: \texttt{explicit\ statement\ in\ text}
\item
  Data source: Nielsen Homescan PanelViews add-on survey (2017)
\item
  Answer:
\end{itemize}

The paper introduces a novel identification strategy---a cross‑sectional price instrument based on retailers' chain-specific relative pricing advantages and geographic presence---and leverages new data from a Homescan add-on survey that links scanner purchases to health preferences and nutrition knowledge.

\begin{itemize}
\tightlist
\item
  Quote snippet:
\end{itemize}

\begin{quote}
adds a novel identification strategy and price instrument.
\end{quote}

\begin{itemize}
\tightlist
\item
  Citation anchors:

  \begin{itemize}
  \tightlist
  \item
    page=7, words=0-349, section=Introduction, note=States the study adds a novel identification strategy and price instrument, and highlights the new Homescan add-on survey linking scanner data to health preferences/knowledge.
  \item
    page=34, words=0-349, section=Identification, note=Explicitly says the instrument is novel in the literature and contrasts it with existing instruments.
  \end{itemize}
\end{itemize}

\subsubsection{B01: What is the identification strategy (in one sentence)?}\label{b01-what-is-the-identification-strategy-in-one-sentence}

\begin{itemize}
\tightlist
\item
  Category: \texttt{B)\ Identification\ strategy\ /\ causal\ design}
\item
  Confidence: \texttt{low\ (0.3493418440321413)}
\item
  Retrieval method: \texttt{local}
\item
  Evidence type: \texttt{quasi-experimental\ movers\ design\ +\ instrumental\ variables}
\item
  Data source: Nielsen Homescan panel; retailer price/availability by chain and market (RMS)
\item
  Assumption flag: \texttt{True}
\item
  Assumption notes: IV exclusion: chains' comparative advantages are uncorrelated with unobserved tastes (E{[}ε\_ijt Pjmt{]}=0); movers design relies on within-household changes with controls for observable shocks.
\item
  Answer:
\end{itemize}

Exploit within-household changes after cross--ZIP/county moves to identify place effects, and use an IV based on retail chains' comparative pricing advantages (from national UPC prices interacted with local chain presence), so identification comes from relative price differences across product groups within a market-year.

\begin{itemize}
\tightlist
\item
  Quote snippet:
\end{itemize}

\begin{quote}
identification comes only from variation in the relative prices across product groups within a market in a given year.
\end{quote}

\begin{itemize}
\tightlist
\item
  Citation anchors:

  \begin{itemize}
  \tightlist
  \item
    page=23, words=275-408, section=main text, note=Within-household changes after cross--ZIP/county moves to test place effects
  \item
    page=32, words=0-349, section=main text, note=Identification comes from relative prices across product groups within a market-year
  \item
    page=33, words=275-397, section=main text, note=Identifying assumption: instrument uncorrelated with idiosyncratic tastes (E{[}ε\_ijt Pjmt{]}=0)
  \item
    page=31, words=0-349, section=main text, note=Instrument intuition: chains have heterogeneous comparative advantages across products and markets
  \end{itemize}
\end{itemize}

\subsubsection{B02: Is the design experimental, quasi-experimental, or observational?}\label{b02-is-the-design-experimental-quasi-experimental-or-observational}

\begin{itemize}
\tightlist
\item
  Category: \texttt{B)\ Identification\ strategy\ /\ causal\ design}
\item
  Confidence: \texttt{low\ (0.27884377133987565)}
\item
  Retrieval method: \texttt{local}
\item
  Evidence type: \texttt{textual}
\item
  Data source: Nielsen Homescan; RMS; ZIP Code Business Patterns (ZBP)
\item
  Answer:
\end{itemize}

Quasi-experimental

\begin{itemize}
\tightlist
\item
  Quote snippet:
\end{itemize}

\begin{quote}
Before estimating equation (1), we show graphical results of the event study.
\end{quote}

\begin{itemize}
\tightlist
\item
  Citation anchors:

  \begin{itemize}
  \tightlist
  \item
    page=17, words=275-350, section=Event study around entry, note=Mentions showing graphical results of the event study before estimating eq. (1)
  \item
    page=18, words=0-180, section=Event study details, note=Defines event-study regression around supermarket entry
  \item
    page=24, words=275-349, section=Movers event study, note=Again references graphical results of the event study before eq. (3)
  \item
    page=23, words=275-360, section=Place effects and endogeneity, note=Acknowledges non-random moves and endogeneity concerns
  \end{itemize}
\end{itemize}

\subsubsection{B03: What is the source of exogenous variation used for identification?}\label{b03-what-is-the-source-of-exogenous-variation-used-for-identification}

\begin{itemize}
\tightlist
\item
  Category: \texttt{B)\ Identification\ strategy\ /\ causal\ design}
\item
  Confidence: \texttt{medium\ (0.43571135442327524)}
\item
  Retrieval method: \texttt{local}
\item
  Evidence type: \texttt{textual}
\item
  Data source: Nielsen RMS (Retail Measurement Services) retail scanner data on chains' presence and pricing
\item
  Table/Figure: Online Appendix Table A8; Figures A11, A14, A15
\item
  Answer:
\end{itemize}

Cross-sectional variation in relative prices across product groups within a market-year, generated by the interaction of retail chains' heterogeneous comparative cost advantages and their differing geographic presence (the price instrument Pjmt).

\begin{itemize}
\tightlist
\item
  Quote snippet:
\end{itemize}

\begin{quote}
The variation in our instrument comes from the interaction between retail chains' differing pricing advantages \ldots{} and their differing presence across geographic markets.
\end{quote}

\begin{itemize}
\tightlist
\item
  Citation anchors:

  \begin{itemize}
  \tightlist
  \item
    page=32, words=0-349, section=VI.B. Price Endogeneity, note=Defines Pjmt and states that variation comes from chains' pricing advantages interacted with their market presence; identification is from relative prices across product groups within a market-year.
  \item
    page=31, words=0-349, section=VI.B. Price Endogeneity, note=Introduces the instrument and explains chain-specific comparative advantages across products and geographic presence.
  \end{itemize}
\end{itemize}

\subsubsection{B04: What is the treatment definition and timing?}\label{b04-what-is-the-treatment-definition-and-timing}

\begin{itemize}
\tightlist
\item
  Category: \texttt{B)\ Identification\ strategy\ /\ causal\ design}
\item
  Confidence: \texttt{low\ (0.12914936559263318)}
\item
  Retrieval method: \texttt{local}
\item
  Evidence type: \texttt{text}
\item
  Data source: 2004--2016 Nielsen Homescan household-by-quarter data
\item
  Table/Figure: Table II
\item
  Assumption flag: \texttt{False}
\item
  Answer:
\end{itemize}

Treatment is defined as a supermarket entering within a 0--10 or 10--15 minute drive of a household's census-tract centroid. Timing is at the household-by-quarter level, using event-time indicators/counts relative to the entry quarter.

\begin{itemize}
\tightlist
\item
  Quote snippet:
\end{itemize}

\begin{quote}
We consider the impact of supermarket entries that occur within a 0--10- or 10--15-minute drive of households' census tract centroids.
\end{quote}

\begin{itemize}
\tightlist
\item
  Citation anchors:

  \begin{itemize}
  \tightlist
  \item
    page=17, words=0-349, section=Event study setup and regression specification, note=Defines treatment bands (0--10 and 10--15 minutes) and quarter-level data
  \item
    page=17, words=275-411, section=Event-time indicators, note=Defines Ebcqt as indicator by q quarters after t
  \item
    page=19, words=0-310, section=Figure IV description, note=Confirms use of 2004--2016 household-by-quarter Homescan data
  \item
    page=21, words=0-349, section=Table II notes, note=Independent variables are counts of entries within 0--10 or 10--15 minutes as of quarter t
  \end{itemize}
\end{itemize}

\subsubsection{B05: What is the control/comparison group definition?}\label{b05-what-is-the-controlcomparison-group-definition}

\begin{itemize}
\tightlist
\item
  Category: \texttt{B)\ Identification\ strategy\ /\ causal\ design}
\item
  Confidence: \texttt{low\ (0.1973947657382083)}
\item
  Retrieval method: \texttt{local}
\item
  Evidence type: \texttt{methods\ description}
\item
  Assumption flag: \texttt{True}
\item
  Assumption notes: The text outlines event-study and within-household designs but does not explicitly label the control group; we infer it is the pre-event (pre-entry or pre-move) period for the same unit.
\item
  Answer:
\end{itemize}

A pre-event period for the same unit serves as the comparison group in the event-study designs: for supermarket entry, census tracts before an entry within the specified distance band; for moves, the same households in the balanced pre-move window.

\begin{itemize}
\tightlist
\item
  Quote snippet:
\end{itemize}

\begin{quote}
we measure within-household changes in grocery purchases after the 20,031 cross--ZIP code moves and the 11,728 cross-county moves\ldots{} Bit be an indicator for whether observation it is part of a balanced panel around a move
\end{quote}

\begin{itemize}
\tightlist
\item
  Citation anchors:

  \begin{itemize}
  \tightlist
  \item
    page=17, words=275-411, section=Event study of supermarket entry, note=Defines event-time indicator for entry within distance bands, implying pre-entry periods as comparison.
  \item
    page=23, words=275-408, section=Within-household move analysis, note=States they measure within-household changes after moves, implying pre-move periods as comparison.
  \item
    page=24, words=275-452, section=Event study around moves, note=Defines Bit for a balanced panel around a move, indicating pre- versus post-move comparison.
  \end{itemize}
\end{itemize}

\subsubsection{B06: What is the estimating equation / baseline regression specification?}\label{b06-what-is-the-estimating-equation--baseline-regression-specification}

\begin{itemize}
\tightlist
\item
  Category: \texttt{B)\ Identification\ strategy\ /\ causal\ design}
\item
  Confidence: \texttt{medium\ (0.39087819086762554)}
\item
  Retrieval method: \texttt{local}
\item
  Evidence type: \texttt{direct\ equation\ from\ text}
\item
  Assumption flag: \texttt{False}
\item
  Answer:
\end{itemize}

Baseline estimating equation (annual calorie demand): ln Y\_ijt = − ln p̃\_ijt − Σ\_\{c≥2\} β̃\_c ã\_ijct − ξ + δ\_j + φ\_m + φ\_t + ε\_ijt, where Y\_ijt is total calories purchased by household i in product group j in year t; p̃\_ijt is the average price per calorie; ã\_ijct is the average amount of characteristic c per calorie; β̃\_c are money-metric marginal utilities; ξ is an unobserved product characteristic; and δ\_j, φ\_m, φ\_t are product-group, market, and year fixed effects.

\begin{itemize}
\tightlist
\item
  Quote snippet:
\end{itemize}

\begin{quote}
ln Yi jt = − ln p̃i jt − β̃c ãi jct − ξ + δ j + φm + φt + εi jt
\end{quote}

\begin{itemize}
\tightlist
\item
  Citation anchors:

  \begin{itemize}
  \tightlist
  \item
    page=30, words=0-349, section=VI.A. Empirical Model, note=Presents the annual calorie demand estimating equation (7) with fixed effects and variables defined.
  \item
    page=29, words=275-369, section=VI.A. Empirical Model, note=Leads into aggregating first-order conditions to form the estimating equation.
  \end{itemize}
\end{itemize}

\subsubsection{B07: What fixed effects are included (unit, time, two-way, higher dimensional)?}\label{b07-what-fixed-effects-are-included-unit-time-two-way-higher-dimensional}

\begin{itemize}
\tightlist
\item
  Category: \texttt{B)\ Identification\ strategy\ /\ causal\ design}
\item
  Confidence: \texttt{low\ (0.3346980299233284)}
\item
  Retrieval method: \texttt{local}
\item
  Evidence type: \texttt{textual\ evidence\ from\ provided\ context}
\item
  Assumption flag: \texttt{False}
\item
  Answer:
\end{itemize}

Included fixed effects are: household fixed effects (unit); year indicators (time); census-division-by-quarter fixed effects (time-by-region, two-way); household-by-census-tract fixed effects (interaction, higher dimensional). Additional specifications include county, product group, and year fixed effects in instrument tests, and county-by-quarter fixed effects in robustness checks.

\begin{itemize}
\tightlist
\item
  Quote snippet:
\end{itemize}

\begin{quote}
\ldots census-division-by-quarter of sample indicators, and household-by-census-tract fixed effects.
\end{quote}

\begin{itemize}
\tightlist
\item
  Citation anchors:

  \begin{itemize}
  \tightlist
  \item
    page=21, words=275-503, note=Mentions census-division-by-quarter indicators and household-by-census-tract fixed effects; also notes adding county-by-quarter fixed effects.
  \item
    page=24, words=0-349, note=Defines year indicators (time FE) and household fixed effects (unit FE) for regression (3).
  \item
    page=34, words=0-349, note=States use of county, product group, and year fixed effects in instrument validity tests.
  \end{itemize}
\end{itemize}

\subsubsection{B08: What standard errors are used (robust, clustered; at what level)?}\label{b08-what-standard-errors-are-used-robust-clustered-at-what-level}

\begin{itemize}
\tightlist
\item
  Category: \texttt{B)\ Identification\ strategy\ /\ causal\ design}
\item
  Confidence: \texttt{low\ (0.33384988660866577)}
\item
  Retrieval method: \texttt{local}
\item
  Evidence type: \texttt{textual}
\item
  Assumption flag: \texttt{False}
\item
  Answer:
\end{itemize}

Robust standard errors with two-way clustering by household and census tract.

\begin{itemize}
\tightlist
\item
  Quote snippet:
\end{itemize}

\begin{quote}
we use robust standard errors with two-way clustering by household and census tract.
\end{quote}

\begin{itemize}
\tightlist
\item
  Citation anchors:

  \begin{itemize}
  \tightlist
  \item
    page=17, words=275-411, section=Event study estimation (equations 1 and 2), note=States use of robust SEs with two-way clustering by household and census tract.
  \item
    page=21, words=275-503, section=Results/table notes, note=Confirms robust SEs clustered by household and census tract.
  \end{itemize}
\end{itemize}

\subsubsection{B09: What is the key identifying assumption (parallel trends, exclusion restriction, ignorability)?}\label{b09-what-is-the-key-identifying-assumption-parallel-trends-exclusion-restriction-ignorability}

\begin{itemize}
\tightlist
\item
  Category: \texttt{B)\ Identification\ strategy\ /\ causal\ design}
\item
  Confidence: \texttt{medium\ (0.373532037182493)}
\item
  Retrieval method: \texttt{local}
\item
  Evidence type: \texttt{explicit\ statement\ in\ text}
\item
  Assumption flag: \texttt{True}
\item
  Assumption notes: Instrument exogeneity: chains' comparative pricing advantages are unrelated to unobserved consumer tastes within product groups.
\item
  Answer:
\end{itemize}

Exclusion restriction: the price instrument must be uncorrelated with unobserved tastes (εijt).

\begin{itemize}
\tightlist
\item
  Quote snippet:
\end{itemize}

\begin{quote}
Our identifying assumption is that household i's idiosyncratic preferences for product group j are uncorrelated with the price instrument Pjmt
\end{quote}

\begin{itemize}
\tightlist
\item
  Citation anchors:

  \begin{itemize}
  \tightlist
  \item
    page=33, words=275-397, section=VI.B Price Endogeneity / Instrument, note=States the identifying assumption that εijt is uncorrelated with the instrument Pjmt.
  \item
    page=34, words=0-349, section=VI.B Instrument validity tests, note=Discusses tests consistent with the exclusion restriction (no relation between instrument and predictable tastes).
  \end{itemize}
\end{itemize}

\subsubsection{B10: What evidence is provided to support the identifying assumption?}\label{b10-what-evidence-is-provided-to-support-the-identifying-assumption}

\begin{itemize}
\tightlist
\item
  Category: \texttt{B)\ Identification\ strategy\ /\ causal\ design}
\item
  Confidence: \texttt{low\ (0.26630295512584257)}
\item
  Retrieval method: \texttt{local}
\item
  Evidence type: \texttt{instrument\ validity\ checks\ and\ robustness\ tests}
\item
  Data source: Main text; references to Online Appendix Tables A8 and A9
\item
  Assumption flag: \texttt{False}
\item
  Answer:
\end{itemize}

They support the identifying assumption with three pieces of evidence: (1) most identifying variation in the price instrument comes from retailer-specific pricing advantages that are constant across counties (not from county-specific pricing), implying little scope for endogenous responses to local tastes; (2) two balance tests show no correlation between the instrument and predictable tastes---no association with purchases predicted by demographics and no systematic differences in high--Health Index product groups across low- vs. high-income counties; and (3) estimates are robust to adding/removing controls for preference variation, yielding very similar results.

\begin{itemize}
\tightlist
\item
  Quote snippet:
\end{itemize}

\begin{quote}
We find that the instrument is not associated with predicted purchases conditional on our standard set of county, product group, and year fixed ef- fects.
\end{quote}

\begin{itemize}
\tightlist
\item
  Citation anchors:

  \begin{itemize}
  \tightlist
  \item
    page=32, words=0-349, section=Instrument and identification, note=Variation mainly retailer-specific (constant across counties), limiting endogenous pricing concerns.
  \item
    page=33, words=275-397, section=Identifying assumption, note=States the identifying assumption E{[}ε\_ijt P\_jmt{]}=0 and discusses testing strategy.
  \item
    page=34, words=0-349, section=Instrument validity tests, note=Two tests: no association with predicted purchases; no systematic differences across income in high-Health Index groups.
  \item
    page=38, words=275-377, section=Robustness to preference controls, note=Estimates very similar with added preference controls, reinforcing exclusion restriction plausibility.
  \end{itemize}
\end{itemize}

\subsubsection{B11: Are there event-study or pre-trend tests? What do they show?}\label{b11-are-there-event-study-or-pre-trend-tests-what-do-they-show}

\begin{itemize}
\tightlist
\item
  Category: \texttt{B)\ Identification\ strategy\ /\ causal\ design}
\item
  Confidence: \texttt{low\ (0.32193505733248057)}
\item
  Retrieval method: \texttt{local}
\item
  Evidence type: \texttt{text}
\item
  Data source: 2004--2016 Nielsen Homescan household data
\item
  Table/Figure: Figure IV; Figure V
\item
  Assumption flag: \texttt{False}
\item
  Answer:
\end{itemize}

Yes. The paper runs event-study analyses around supermarket entry (eq. 2; Figure IV) and around household moves across counties (Figure V). After entry, entrant retailers' expenditure share rises by \textasciitilde2 percentage points in a year (≈3\% in food deserts), with Health Index effects small/mostly insignificant and gains largely diverted from other supermarkets. For moves, there is no statistically significant post-move change in the Health Index when moving to healthier counties, though point estimates are positive. The entry event study uses a balanced panel with four pre-entry quarters and sets q = −1 as the omitted pre-period, allowing visual pre-trend assessment.

\begin{itemize}
\tightlist
\item
  Quote snippet:
\end{itemize}

\begin{quote}
The top left panel shows that this entrant expenditure share increases by about 2 percentage points one year after entry.
\end{quote}

\begin{itemize}
\tightlist
\item
  Citation anchors:

  \begin{itemize}
  \tightlist
  \item
    page=18, words=0-349, section=Event study of supermarket entry (eq. 2; Figure IV), note=Describes event-study setup, balanced panel, omitted q = −1, and \textasciitilde2 pp increase post-entry
  \item
    page=19, words=0-310, section=Figure IV description, note=Specifies outcomes including Health Index in event study panels
  \item
    page=21, words=275-503, section=Entry effects interpretation, note=Health Index effects small/mostly insignificant; gains largely diverted from other supermarkets
  \item
    page=26, words=0-282, section=Event study of moves (Figure V), note=Reports no statistically significant post-move Health Index change; point estimates positive
  \item
    page=17, words=275-411, section=Event study setup details, note=Mentions showing graphical event-study results and defines key indicators
  \end{itemize}
\end{itemize}

\subsubsection{B12: What instruments are used (if IV)? Define instrument and first stage.}\label{b12-what-instruments-are-used-if-iv-define-instrument-and-first-stage}

\begin{itemize}
\tightlist
\item
  Category: \texttt{B)\ Identification\ strategy\ /\ causal\ design}
\item
  Confidence: \texttt{low\ (0.18485188819713624)}
\item
  Retrieval method: \texttt{local}
\item
  Evidence type: \texttt{textual\_evidence}
\item
  Table/Figure: Equation (9); Online Appendix Figure A17
\item
  Assumption flag: \texttt{True}
\item
  Assumption notes: Exclusion restriction: E{[}εijt Pjmt{]}=0; chains' comparative pricing advantages are unrelated to unobserved tastes. Market fixed effects address overall demand patterns.
\item
  Answer:
\end{itemize}

Instrument: A chain-level, cross-market pricing instrument Pjmt. For each UPC k and chain r, compute the chain's cost advantage outside market m as Δln(pkrt,−m) = ln(pkrt,−m) − ln(pkt,−m), where prices exclude market m. Then Pjmt is the weighted average of these cost advantages across UPCs in product group j, with weights based on the chain's presence in market m (Nrmt), chain-by-group sales weights (Njrt), and national UPC importance (Nkt) {[}Eq. (9){]}. First stage: Regress log prices on Pjmt with market and product-group fixed effects; there is a robust linear relationship, with first-stage F-statistics around 243--260 across income groups.

\begin{itemize}
\tightlist
\item
  Quote snippet:
\end{itemize}

\begin{quote}
The price instrument Pjmt is the weighted average cost advantage that chains in market m have for UPCs in product group j:
\end{quote}

\begin{itemize}
\tightlist
\item
  Citation anchors:

  \begin{itemize}
  \tightlist
  \item
    page=page 31, words=275-432, note=Defines ln(pkrt,−m), ln(pkt,−m) and exclusion of market m to avoid local demand contamination.
  \item
    page=page 32, words=0-349, note=Defines Pjmt (Eq. 9), weighting scheme, and variation source; mentions fixed effects context.
  \item
    page=page 33, words=0-349, note=States robust linear first-stage and F-statistics for instrument predicting log prices with FE.
  \item
    page=page 33, words=275-397, note=Identifying assumption E{[}εijt Pjmt{]}=0; economic content of exclusion restriction.
  \end{itemize}
\end{itemize}

\subsubsection{B13: What is the first-stage strength (F-stat, partial R\^{}2, relevance evidence)?}\label{b13-what-is-the-first-stage-strength-f-stat-partial-r2-relevance-evidence}

\begin{itemize}
\tightlist
\item
  Category: \texttt{B)\ Identification\ strategy\ /\ causal\ design}
\item
  Confidence: \texttt{low\ (0.30479736690202236)}
\item
  Retrieval method: \texttt{local}
\item
  Evidence type: \texttt{Qualitative\ description\ of\ instrument\ variation\ and\ cited\ appendix\ checks;\ no\ numeric\ first-stage\ stats\ in\ provided\ text}
\item
  Table/Figure: Online Appendix Table A8; Figures A11, A14, A15 (referenced)
\item
  Assumption flag: \texttt{False}
\item
  Answer:
\end{itemize}

Numeric first-stage strength (F-stat or partial R\^{}2) is not reported in the provided excerpts. The relevance evidence given is qualitative: variation comes from retailer-specific pricing advantages and retailers' differing geographic presence, with identification from relative price differences across product groups within a market-year; appendix evidence indicates most within--product-group variation is retailer- (not county-) specific.

\begin{itemize}
\tightlist
\item
  Quote snippet:
\end{itemize}

\begin{quote}
identification comes only from variation in the relative prices across product groups within a market in a given year.
\end{quote}

\begin{itemize}
\tightlist
\item
  Citation anchors:

  \begin{itemize}
  \tightlist
  \item
    page=32, words=0-349, section=Instrument construction and variation, note=Defines instrument; states where variation and identification come from; references appendix showing variation largely retailer-specific.
  \item
    page=34, words=0-349, section=Instrument validity checks, note=Reports tests showing instrument not associated with predicted tastes (exogeneity checks).
  \item
    page=33, words=275-397, section=Identifying assumption, note=States exclusion restriction; context for instrument validity (not first-stage strength).
  \end{itemize}
\end{itemize}

\subsubsection{B14: If RDD: what is the running variable and cutoff? bandwidth choice?}\label{b14-if-rdd-what-is-the-running-variable-and-cutoff-bandwidth-choice}

\begin{itemize}
\tightlist
\item
  Category: \texttt{B)\ Identification\ strategy\ /\ causal\ design}
\item
  Confidence: \texttt{low\ (0.26722121815091543)}
\item
  Retrieval method: \texttt{local}
\item
  Evidence type: \texttt{textual}
\item
  Data source: Nielsen Homescan household panel; Nielsen RMS store data; Google Maps API driving times
\item
  Assumption flag: \texttt{False}
\item
  Answer:
\end{itemize}

Not applicable. The study does not use an RDD; it relies on event‑study regressions around supermarket entries and household moves (and an IV for prices). Therefore, there is no running variable, cutoff, or bandwidth.

\begin{itemize}
\tightlist
\item
  Quote snippet:
\end{itemize}

\begin{quote}
Before estimating equation (1), we show graphical results of the event study.
\end{quote}

\begin{itemize}
\tightlist
\item
  Citation anchors:

  \begin{itemize}
  \tightlist
  \item
    page=17, words=0-349, section=Event study around supermarket entries, note=Describes event-study setup and regression (1), not an RDD.
  \item
    page=24, words=0-349, section=Event study around household moves, note=Describes event-study design and regression (3), not an RDD.
  \item
    page=32, words=0-349, section=Instrumental variables for prices, note=Introduces a cross-sectional IV; again not an RDD.
  \end{itemize}
\end{itemize}

\subsubsection{B15: If DiD: what is the timing variation (staggered adoption)? estimator used?}\label{b15-if-did-what-is-the-timing-variation-staggered-adoption-estimator-used}

\begin{itemize}
\tightlist
\item
  Category: \texttt{B)\ Identification\ strategy\ /\ causal\ design}
\item
  Confidence: \texttt{low\ (0.3211861386487479)}
\item
  Retrieval method: \texttt{local}
\item
  Evidence type: \texttt{textual\_excerpts}
\item
  Data source: Homescan household-by-quarter data
\item
  Table/Figure: Figure IV
\item
  Assumption flag: \texttt{False}
\item
  Answer:
\end{itemize}

Staggered timing comes from supermarket openings occurring in different quarters across census tracts within 0--10 or 10--15 minutes' drive; event time is defined around the quarter of entry (balanced panels: 4 pre, 8 post). The estimator is a two‑way fixed‑effects DiD/event‑study in household‑by‑quarter data: Yict = τb Sbct + γXit + μd(c)t + φic + εict and, for event time, Yict = Bbit Σq τbq Ebcqt + αb + γXit + μd(c)t + φic + εict, with q = −1 omitted and robust SE two‑way clustered by household and census tract.

\begin{itemize}
\tightlist
\item
  Quote snippet:
\end{itemize}

\begin{quote}
We run the following regression in household-by-quarter Homescan data: Yict = τb Sbct + γ Xit + μd(c)t + φic + εict \ldots{} We run the following regression\ldots{} Yict = Bbit τbq Ebcqt + αb \ldots{} The omitted category is q = −1.
\end{quote}

\begin{itemize}
\tightlist
\item
  Citation anchors:

  \begin{itemize}
  \tightlist
  \item
    page=17, words=0-349, section=Equation (1) DiD setup, note=Defines distance bands, Sbct, and TWFE regression with division-by-quarter and household-by-tract FE; mentions two-way clustering.
  \item
    page=18, words=0-349, section=Equation (2) event study, note=Defines balanced panels around entry, event-time regression, and that q = −1 is the omitted category.
  \item
    page=17, words=275-411, section=Event study definitions; clustering, note=Introduces Ebcqt, Bbit and notes two-way clustering by household and census tract.
  \end{itemize}
\end{itemize}

\subsubsection{C01: What dataset(s) are used? (name sources explicitly)}\label{c01-what-datasets-are-used-name-sources-explicitly}

\begin{itemize}
\tightlist
\item
  Category: \texttt{C)\ Data,\ sample,\ and\ measurement}
\item
  Confidence: \texttt{low\ (0.323291429229872)}
\item
  Retrieval method: \texttt{local}
\item
  Evidence type: \texttt{textual\_evidence}
\item
  Data source: Allcott et al. (2020), The Quarterly Journal of Economics article text (provided context)
\item
  Assumption flag: \texttt{False}
\item
  Answer:
\end{itemize}

Nielsen Homescan household panel (including the 2004--2006 "magnet" subsample and the 2017 PanelViews add-on survey); Nielsen Retail Measurement Services (RMS); Nielsen TDLinx (store opening dates/locations); ZIP Code Business Patterns (ZBP); USDA Food and Nutrient Database for Dietary Studies (FNDDS) and National Nutrient Database for Standard Reference (SR); Consumer Price Index for Urban Consumers (CPI-U) for deflation.

\begin{itemize}
\tightlist
\item
  Quote snippet:
\end{itemize}

\begin{quote}
Our nutrition facts are from the Food and Nutrient Database for Dietary Studies and the National Nutrient Database for Standard Reference
\end{quote}

\begin{itemize}
\tightlist
\item
  Citation anchors:

  \begin{itemize}
  \tightlist
  \item
    page=8, words=1-200, section=Data description, note=Nielsen Homescan; 8,000-household \textquotesingle magnet\textquotesingle{} subsample; PanelViews add-on survey
  \item
    page=8, words=275-393, section=Data description, note=Nielsen RMS; CPI-U deflator
  \item
    page=9, words=275-389, section=II.B. Grocery Retail Establishments, note=ZIP Code Business Patterns (ZBP)
  \item
    page=10, words=1-180, section=II.B. Grocery Retail Establishments, note=Nielsen TDLinx store openings data
  \item
    page=10, words=180-349, section=II.C. Nutrition Facts and the Health Index, note=USDA FNDDS and SR nutrition databases; USDA crosswalks
  \end{itemize}
\end{itemize}

\subsubsection{C02: What is the unit of observation (individual, household, firm, county, transaction, product)?}\label{c02-what-is-the-unit-of-observation-individual-household-firm-county-transaction-product}

\begin{itemize}
\tightlist
\item
  Category: \texttt{C)\ Data,\ sample,\ and\ measurement}
\item
  Confidence: \texttt{medium\ (0.40760063179281236)}
\item
  Retrieval method: \texttt{local}
\item
  Evidence type: \texttt{explicit\ statement\ in\ text}
\item
  Data source: Nielsen Homescan
\item
  Assumption flag: \texttt{False}
\item
  Answer:
\end{itemize}

Household-by--product group--by-year

\begin{itemize}
\tightlist
\item
  Quote snippet:
\end{itemize}

\begin{quote}
Data are at the household-by--product group--by-year level.
\end{quote}

\begin{itemize}
\tightlist
\item
  Citation anchors:

  \begin{itemize}
  \tightlist
  \item
    page=34, words=275-385, section=VI.C. Method of Moments Estimation, note=States the estimation dataset unit
  \item
    page=9, words=275-389, note=Notes collapsing to household (or store)-by-year for analyses
  \end{itemize}
\end{itemize}

\subsubsection{C03: What is the sample period and geographic coverage?}\label{c03-what-is-the-sample-period-and-geographic-coverage}

\begin{itemize}
\tightlist
\item
  Category: \texttt{C)\ Data,\ sample,\ and\ measurement}
\item
  Confidence: \texttt{low\ (0.2994701740878031)}
\item
  Retrieval method: \texttt{local}
\item
  Evidence type: \texttt{direct\_author\_text}
\item
  Data source: Nielsen Homescan; Nielsen TDLinx; ZIP Code Business Patterns
\item
  Assumption flag: \texttt{False}
\item
  Answer:
\end{itemize}

2004--2016; United States (nationwide). The Homescan panel is nationally representative, ZIP Code Business Patterns cover every ZIP code, and supermarket entries come from a census of all U.S. food retailers.

\begin{itemize}
\tightlist
\item
  Quote snippet:
\end{itemize}

\begin{quote}
Homescan data include 731,994 household-by-year observations for 2004--2016 and are weighted for national representativeness.
\end{quote}

\begin{itemize}
\tightlist
\item
  Citation anchors:

  \begin{itemize}
  \tightlist
  \item
    page=9, words=0-349, section=Table I notes / Data description, note=Homescan period 2004--2016; nationally representative.
  \item
    page=10, words=0-349, section=II.B. Grocery Retail Establishments, note=TDLinx is a census of all U.S. food retailers; 2004--2016 entries.
  \item
    page=10, words=0-349, section=II.B. Grocery Retail Establishments, note=ZBP provides counts for every ZIP code.
  \item
    page=9, words=275-389, section=II.B. Grocery Retail Establishments, note=ZBP data span 2004--2016.
  \end{itemize}
\end{itemize}

\subsubsection{C04: What are the sample restrictions / inclusion criteria?}\label{c04-what-are-the-sample-restrictions--inclusion-criteria}

\begin{itemize}
\tightlist
\item
  Category: \texttt{C)\ Data,\ sample,\ and\ measurement}
\item
  Confidence: \texttt{low\ (0.2702960096992867)}
\item
  Retrieval method: \texttt{local}
\item
  Evidence type: \texttt{direct\_excerpts}
\item
  Data source: Nielsen Homescan; Nielsen Retail Scanner (RMS); ZIP Code Business Patterns (ZBP)
\item
  Assumption flag: \texttt{False}
\item
  Answer:
\end{itemize}

The analyses use Nielsen Homescan households and Nielsen RMS stores, with the following core sample restrictions/inclusion criteria: (1) RMS excludes liquor stores; (2) both Homescan and RMS exclude bulk produce and other nonpackaged items (focus on packaged groceries); (3) for most empirical analyses (except the supermarket-entry event study), data are collapsed to the household- or store-by-year level; (4) entry and heterogeneity analyses sometimes restrict to subsamples---e.g., households in the bottom income quartile (defined by residualized income) and households in food deserts; (5) the household-move design includes only observations in a balanced panel around a move (Bit = 1) and imposes a 50\% local shopping restriction to ensure exposure to the local retail environment; (6) some estimations stratify the sample into four income quartiles for separate datasets.

\begin{itemize}
\tightlist
\item
  Quote snippet:
\end{itemize}

\begin{quote}
As with Homescan, RMS does not include sales of bulk produce and other nonpackaged items.
\end{quote}

\begin{itemize}
\tightlist
\item
  Citation anchors:

  \begin{itemize}
  \tightlist
  \item
    page=8, words=275-393, section=RMS sample definition, note=Excludes liquor stores; no bulk produce/nonpackaged items
  \item
    page=9, words=275-389, section=Data construction, note=Collapse to household- or store-by-year level
  \item
    page=20, words=275-432, section=Subsample analyses, note=Limits to bottom income quartile and to food deserts; definition of bottom quartile
  \item
    page=24, words=275-452, section=Move design sample, note=Balanced panel around a move (Bit definition)
  \item
    page=25, words=275-468, section=Move design restriction, note=50\% local shopping restriction
  \item
    page=34, words=275-385, section=Estimation stratification, note=Separate datasets for four household income quartiles
  \end{itemize}
\end{itemize}

\subsubsection{C05: What is the sample size (N) in the main analysis?}\label{c05-what-is-the-sample-size-n-in-the-main-analysis}

\begin{itemize}
\tightlist
\item
  Category: \texttt{C)\ Data,\ sample,\ and\ measurement}
\item
  Confidence: \texttt{low\ (0.28304756286944066)}
\item
  Retrieval method: \texttt{local}
\item
  Evidence type: \texttt{Table\ note\ (descriptive\ statistics)}
\item
  Data source: Nielsen Homescan data, 2004--2016
\item
  Table/Figure: Table I (Notes)
\item
  Assumption flag: \texttt{True}
\item
  Assumption notes: Interpreted `main analysis' as the Homescan household-by-year sample used throughout; ZIP code counts (508,951) also reported but pertain to establishment data, not the main household analysis.
\item
  Answer:
\end{itemize}

731,994 household-by-year observations (Homescan, 2004--2016).

\begin{itemize}
\tightlist
\item
  Quote snippet:
\end{itemize}

\begin{quote}
Homescan data include 731,994 household-by-year observations for 2004--2016
\end{quote}

\begin{itemize}
\tightlist
\item
  Citation anchors:

  \begin{itemize}
  \tightlist
  \item
    page=9, words=0-349, section=Table I notes (Panel A), note=States total Homescan household-by-year observations for 2004--2016.
  \end{itemize}
\end{itemize}

\subsubsection{C06: How is the key outcome measured? Any transformations (logs, z-scores, indices)?}\label{c06-how-is-the-key-outcome-measured-any-transformations-logs-z-scores-indices}

\begin{itemize}
\tightlist
\item
  Category: \texttt{C)\ Data,\ sample,\ and\ measurement}
\item
  Confidence: \texttt{low\ (0.3488158062475002)}
\item
  Retrieval method: \texttt{local}
\item
  Evidence type: \texttt{textual\ evidence\ from\ methods/notes}
\item
  Data source: Nielsen Homescan (2004--2016)
\item
  Table/Figure: Table III
\item
  Assumption flag: \texttt{False}
\item
  Answer:
\end{itemize}

The key outcome is the Health Index---an overall, linearized Healthy Eating Index of grocery purchases. It is standardized to mean 0 and SD 1 across households (z-score). In some counterfactual figures it is further renormalized so the base high--low income gap equals 1.

\begin{itemize}
\tightlist
\item
  Quote snippet:
\end{itemize}

\begin{quote}
The Health Index is our overall measure of the healthfulness of grocery purchases and is normalized to have a mean of 0 and a standard deviation of 1 across households.
\end{quote}

\begin{itemize}
\tightlist
\item
  Citation anchors:

  \begin{itemize}
  \tightlist
  \item
    page=27, words=0-349, section=Table III notes, note=Defines Health Index and states it is normalized to mean 0, SD 1 across households.
  \item
    page=11, words=275-399, section=Methods/Measurement, note=Describes linearized HEI and its correlation with the true HEI; notes omission of fatty acid ratio.
  \item
    page=40, words=0-285, section=Figure VI caption, note=States the Health Index is renormalized so the base difference between highest and lowest income groups equals 1.
  \item
    page=36, words=275-508, section=Estimation Results/HEI components, note=Clarifies HEI components and units used in constructing the index.
  \end{itemize}
\end{itemize}

\subsubsection{C07: How is treatment/exposure measured? Any constructed variables?}\label{c07-how-is-treatmentexposure-measured-any-constructed-variables}

\begin{itemize}
\tightlist
\item
  Category: \texttt{C)\ Data,\ sample,\ and\ measurement}
\item
  Confidence: \texttt{low\ (0.2926683796776228)}
\item
  Retrieval method: \texttt{local}
\item
  Evidence type: \texttt{textual\ description\ of\ methods\ and\ variable\ construction}
\item
  Data source: Nielsen Homescan and RMS; ZBP for store-type distribution adjustment
\item
  Assumption flag: \texttt{False}
\item
  Answer:
\end{itemize}

Exposure is measured as: (1) supermarket entry near households via Ebcqt---an indicator for whether one supermarket entered within distance band b of a census tract by q quarters after quarter t (with an additional indicator Bbit); and (2) the local retail environment's healthfulness Hm---the Health Index of packaged groceries in a ZIP code or county, constructed as the calorie-weighted average Health Index of groceries sold in RMS stores and regression-adjusted to match the true distribution of store channel types from ZBP. For mover event studies, Bit flags observations in a balanced panel around a move; they also apply a 50\% local shopping restriction and use the realized change in the local Health Index across moves. Constructed variables include Ebcqt, Bbit/Bit, the area Health Index Hm (calorie-weighted and store-mix adjusted), and a linearized Healthy Eating Index (HEI) used to summarize dietary quality.

\begin{itemize}
\tightlist
\item
  Quote snippet:
\end{itemize}

\begin{quote}
We define Ebcqt as an indicator variable denoting whether one supermarket had entered in distance band b of census tract c by q quarters after quarter t.
\end{quote}

\begin{itemize}
\tightlist
\item
  Citation anchors:

  \begin{itemize}
  \tightlist
  \item
    page=17, words=275-411, section=Event study design (store entry), note=Defines Ebcqt (entry indicator by distance band and time); mentions Bbit indicator.
  \item
    page=24, words=0-349, section=Place effects specification, note=Defines Hm (area Health Index) and Bit (balanced panel around a move).
  \item
    page=24, words=275-452, section=Construction of area Health Index, note=Details calorie-weighted averaging and regression adjustment to match ZBP store-type distribution.
  \item
    page=25, words=275-468, section=Mover exposure measurement, note=Shows change in local Health Index across moves; mentions 50\% local shopping restriction and trips shifting to the new county.
  \item
    page=11, words=275-399, section=Outcome/quality measure construction, note=Describes the linearized HEI and its components.
  \end{itemize}
\end{itemize}

\subsubsection{C08: Are there key covariates/controls? Which ones are always included?}\label{c08-are-there-key-covariatescontrols-which-ones-are-always-included}

\begin{itemize}
\tightlist
\item
  Category: \texttt{C)\ Data,\ sample,\ and\ measurement}
\item
  Confidence: \texttt{medium\ (0.35539971550701116)}
\item
  Retrieval method: \texttt{local}
\item
  Evidence type: \texttt{textual}
\item
  Assumption flag: \texttt{False}
\item
  Answer:
\end{itemize}

Yes. A standard demographic vector Xit is routinely included: log income, log years of education, age indicators, household size, race indicators, marital status, employment status, and weekly work hours. Entry regressions also include census-division-by-quarter indicators and household-by-census-tract fixed effects. Mover/event-study regressions include year indicators and household fixed effects (with the same Xit), and additionally control for changes in income, job responsibilities, household composition, and marriage status.

\begin{itemize}
\tightlist
\item
  Quote snippet:
\end{itemize}

\begin{quote}
Xit includes the natural log of income, natural log of years of education, indicators for each integer age from 23--90, household size, race indicators,
\end{quote}

\begin{itemize}
\tightlist
\item
  Citation anchors:

  \begin{itemize}
  \tightlist
  \item
    page=17, words=275-411, section=Covariates (Xit), note=Lists Xit demographics: log income, log education, age dummies, household size, race, marital status, employment status, weekly hours.
  \item
    page=21, words=275-360, section=Fixed effects in entry regressions, note=States inclusion of census-division-by-quarter indicators and household-by-census-tract fixed effects.
  \item
    page=24, words=0-200, section=Mover/event-study specification, note=Defines regression with year indicators and household fixed effects and reuses Xit.
  \item
    page=23, words=300-390, section=Additional controls in mover analysis, note=Mentions adding controls for changes in income, job responsibilities, household composition, and marriage status.
  \end{itemize}
\end{itemize}

\subsubsection{C09: How are missing data handled (dropping, imputation, weighting)?}\label{c09-how-are-missing-data-handled-dropping-imputation-weighting}

\begin{itemize}
\tightlist
\item
  Category: \texttt{C)\ Data,\ sample,\ and\ measurement}
\item
  Confidence: \texttt{low\ (0.3056674984716571)}
\item
  Retrieval method: \texttt{local}
\item
  Evidence type: \texttt{textual\ evidence}
\item
  Assumption flag: \texttt{False}
\item
  Answer:
\end{itemize}

Missing data are handled by dropping zero-purchase cells and using survey weights. Specifically, they drop 10.6\% of household--product group--year observations with zero purchases, and Homescan results are generally weighted for national representativeness unless noted; some analyses are explicitly unweighted. No imputation is described in the provided context.

\begin{itemize}
\tightlist
\item
  Quote snippet:
\end{itemize}

\begin{quote}
We drop 10.6\% of observations at the household-by--product group--by-year level because they have zero purchases.
\end{quote}

\begin{itemize}
\tightlist
\item
  Citation anchors:

  \begin{itemize}
  \tightlist
  \item
    page=35, words=275-428, section=Main text, note=Dropping zero-purchase observations (10.6\%) and discussion of missing observations
  \item
    page=8, words=275-393, section=Main text, note=Homescan results weighted for national representativeness
  \item
    page=21, words=275-503, section=Main text, note=Example noting observations are not weighted in a particular analysis
  \end{itemize}
\end{itemize}

\subsubsection{C10: Are weights used (survey weights, propensity weights)? How?}\label{c10-are-weights-used-survey-weights-propensity-weights-how}

\begin{itemize}
\tightlist
\item
  Category: \texttt{C)\ Data,\ sample,\ and\ measurement}
\item
  Confidence: \texttt{low\ (0.3415259310295733)}
\item
  Retrieval method: \texttt{local}
\item
  Evidence type: \texttt{text\ from\ provided\ context}
\item
  Data source: Nielsen Homescan (consumer panel) and Nielsen RMS (retail scanner) data
\item
  Assumption flag: \texttt{False}
\item
  Answer:
\end{itemize}

Yes. Homescan survey/sample weights are used to make estimates nationally representative---"unless otherwise stated, all Homescan results are weighted." They explicitly note using Homescan sample weights in an analysis, while some regressions are explicitly not weighted for national representativeness. Additionally, certain constructs (e.g., area Health Index) use calorie-weighted averages of store sales. No propensity weights are mentioned in the provided text.

\begin{itemize}
\tightlist
\item
  Quote snippet:
\end{itemize}

\begin{quote}
Unless otherwise stated, all Homescan results are weighted for national representativeness.
\end{quote}

\begin{itemize}
\tightlist
\item
  Citation anchors:

  \begin{itemize}
  \tightlist
  \item
    page=8, words=275-393, section=main text, note=States that Homescan results are weighted for national representativeness unless otherwise noted.
  \item
    page=17, words=275-411, section=main text, note=Explicitly mentions using Homescan sample weights for the analysis.
  \item
    page=21, words=275-503, section=main text, note=Notes an instance where observations are not weighted for national representativeness.
  \item
    page=24, words=275-452, section=main text, note=Describes using calorie-weighted averages when constructing area Health Index measures.
  \end{itemize}
\end{itemize}

\subsubsection{C11: Are data linked/merged across sources? How is linkage performed?}\label{c11-are-data-linkedmerged-across-sources-how-is-linkage-performed}

\begin{itemize}
\tightlist
\item
  Category: \texttt{C)\ Data,\ sample,\ and\ measurement}
\item
  Confidence: \texttt{low\ (0.2555935989172092)}
\item
  Retrieval method: \texttt{local}
\item
  Evidence type: \texttt{methods\ description}
\item
  Data source: Nielsen Homescan, Nielsen RMS, Nielsen TDLinx, USDA FNDDS/Standard Reference, ZIP Code Business Patterns, Nielsen PanelViews
\item
  Assumption flag: \texttt{False}
\item
  Answer:
\end{itemize}

Yes. Several cross-source linkages are used: (1) USDA nutrition facts are matched to product UPCs via USDA-provided crosswalks; (2) TDLinx supermarket entries are matched to Nielsen Homescan retailer codes (with filters to drop spurious entries); (3) Area Health Indexes from Nielsen RMS are regression-adjusted using ZIP Code Business Patterns' store-type distributions; (4) Nielsen PanelViews survey responses are merged to Homescan households; (5) Household calorie needs combine U.S. Dietary Guidelines with Homescan household composition.

\begin{itemize}
\tightlist
\item
  Quote snippet:
\end{itemize}

\begin{quote}
We match these nutrition facts to UPCs using crosswalks developed by the USDA (Carlson et al. 2019).
\end{quote}

\begin{itemize}
\tightlist
\item
  Citation anchors:

  \begin{itemize}
  \tightlist
  \item
    page=10, words=0-349, section=II.B. Grocery Retail Establishments, note=TDLinx entries matched to Homescan retailer codes
  \item
    page=10, words=275-424, section=II.C. Nutrition Facts and the Health Index, note=USDA nutrition facts matched to UPCs via USDA crosswalks; TDLinx entry cleaning
  \item
    page=24, words=275-452, section=Event study methodology, note=RMS area Health Index regression-adjusted using ZBP store-type distributions
  \item
    page=8, words=0-349, section=Data description, note=PanelViews survey linked to Homescan; calorie needs combined with Homescan composition
  \item
    page=9, words=275-389, section=Data sources, note=ZBP establishment counts used for ZIP-by-year linkage context
  \end{itemize}
\end{itemize}

\subsubsection{C12: What summary statistics are reported for main variables?}\label{c12-what-summary-statistics-are-reported-for-main-variables}

\begin{itemize}
\tightlist
\item
  Category: \texttt{C)\ Data,\ sample,\ and\ measurement}
\item
  Confidence: \texttt{medium\ (0.35556736097602754)}
\item
  Retrieval method: \texttt{local}
\item
  Evidence type: \texttt{table}
\item
  Data source: Nielsen Homescan households and ZIP Code Business Patterns (2004--2016)
\item
  Table/Figure: Table I: Descriptive Statistics (Panels A and B)
\item
  Assumption flag: \texttt{False}
\item
  Answer:
\end{itemize}

Means and standard deviations are reported (Table I) for Homescan household variables---income, education, age, household size, race shares (White/Black), marital and employment status, weekly work hours, household daily calorie need, health importance, nutrition knowledge, BMI, diabetic---and for ZIP code establishment counts---grocery, large grocery, supercenters/club stores, drug stores, convenience stores, and meat/fish/produce stores.

\begin{itemize}
\tightlist
\item
  Quote snippet:
\end{itemize}

\begin{quote}
TABLE I DESCRIPTIVE STATISTICS Standard Variable Mean deviation
\end{quote}

\begin{itemize}
\tightlist
\item
  Citation anchors:

  \begin{itemize}
  \tightlist
  \item
    page=9, words=0-349, section=Table I (Panels A and B), note=Table lists variables with columns for Mean and Standard deviation, covering Homescan household variables and ZIP code establishment counts with data-source notes.
  \end{itemize}
\end{itemize}

\subsubsection{C13: Are there descriptive figures/maps that establish baseline patterns?}\label{c13-are-there-descriptive-figuresmaps-that-establish-baseline-patterns}

\begin{itemize}
\tightlist
\item
  Category: \texttt{C)\ Data,\ sample,\ and\ measurement}
\item
  Confidence: \texttt{low\ (0.3249774887968107)}
\item
  Retrieval method: \texttt{local}
\item
  Evidence type: \texttt{figures/maps\ (descriptive)}
\item
  Data source: Nielsen RMS (2006--2016)
\item
  Table/Figure: Figure II; Online Appendix Figures A11, A15
\item
  Assumption flag: \texttt{False}
\item
  Answer:
\end{itemize}

Yes. The paper includes descriptive figures and maps that establish baseline patterns---for example, Figure II plots store healthfulness against ZIP code median income, and Online Appendix Figure A11 maps the geographic presence of major retailers (with A15 illustrating geographic price variation).

\begin{itemize}
\tightlist
\item
  Quote snippet:
\end{itemize}

\begin{quote}
Online Appendix Figure A11 maps the geographic presence of the five largest retailers in RMS
\end{quote}

\begin{itemize}
\tightlist
\item
  Citation anchors:

  \begin{itemize}
  \tightlist
  \item
    page=14, words=0-230, section=III.B. Supply Disparities by Neighborhood Income, note=Description of Figure II showing relationships between income and healthfulness offered in stores.
  \item
    page=32, words=275-431, section=Appendix figures discussion, note=Mentions Online Appendix Figure A11 (maps retailer presence) and Figure A15 (geographic variation).
  \end{itemize}
\end{itemize}

\subsubsection{D01: What is the headline main effect estimate (sign and magnitude)?}\label{d01-what-is-the-headline-main-effect-estimate-sign-and-magnitude}

\begin{itemize}
\tightlist
\item
  Category: \texttt{D)\ Results,\ magnitudes,\ heterogeneity,\ robustness}
\item
  Confidence: \texttt{medium\ (0.39317042595932583)}
\item
  Retrieval method: \texttt{local}
\item
  Evidence type: \texttt{textual}
\item
  Data source: Nielsen Homescan (as referenced elsewhere for these estimates)
\item
  Table/Figure: Table V (text; referencing Table II, Panel B, col (2))
\item
  Assumption flag: \texttt{False}
\item
  Answer:
\end{itemize}

Negative 0.9\%: one supermarket entry within a 10-minute drive reduces the top--minus--bottom income Health Index gap by about 0.9\%.

\begin{itemize}
\tightlist
\item
  Quote snippet:
\end{itemize}

\begin{quote}
one supermarket entry within a 10-minute drive of a bottom income quartile household reduces the top minus bottom quartile Health Index difference by 0.005 0.56 ≈ 0.9\%.
\end{quote}

\begin{itemize}
\tightlist
\item
  Citation anchors:

  \begin{itemize}
  \tightlist
  \item
    page=45, words=275-394, note=States the point estimate of supermarket entry's impact on the Health Index gap (\textasciitilde0.9\% reduction).
  \end{itemize}
\end{itemize}

\subsubsection{D02: What is the preferred specification and why is it preferred?}\label{d02-what-is-the-preferred-specification-and-why-is-it-preferred}

\begin{itemize}
\tightlist
\item
  Category: \texttt{D)\ Results,\ magnitudes,\ heterogeneity,\ robustness}
\item
  Confidence: \texttt{low\ (0.22637865371004126)}
\item
  Retrieval method: \texttt{local}
\item
  Evidence type: \texttt{textual}
\item
  Data source: QJE article: Food Deserts and Nutritional Inequality (excerpts provided)
\item
  Assumption flag: \texttt{True}
\item
  Assumption notes: Interprets \textquotesingle preferred specification\textquotesingle{} as the IV model using the chain-based price instrument; the text does not explicitly label a \textquotesingle preferred specification\textquotesingle.
\item
  Answer:
\end{itemize}

The preferred specification is the IV demand model that instruments for product-group prices using grocery chains' comparative-advantage-driven price variation across markets (Pjmt). It is preferred because it addresses price endogeneity (E{[}ε, p{]} ≠ 0), relies on the identifying assumption E{[}ε, Pjmt{]} = 0, has a very strong first stage, and delivers estimates that are robust to adding geography/income and other preference controls.

\begin{itemize}
\tightlist
\item
  Quote snippet:
\end{itemize}

\begin{quote}
we introduce a new instrument using price variation from chains' comparative advantages... The instrument has a very strong first stage and is uncorrelated with observably predicted tastes.
\end{quote}

\begin{itemize}
\tightlist
\item
  Citation anchors:

  \begin{itemize}
  \tightlist
  \item
    page=5, words=0-349, section=Instrument introduction and assumptions, note=Introduces chain-based price instrument; key assumption; strong first stage; robustness to controls.
  \item
    page=30, words=275-393, section=VI.B. Price Endogeneity, note=Motivates need for IV due to E{[}ε, p{]} ≠ 0.
  \item
    page=33, words=275-397, section=Identification assumption, note=States E{[}εijt, Pjmt{]} = 0 and discusses exclusion logic.
  \item
    page=38, words=275-377, section=Robustness to controls, note=Estimates similar with added controls, supporting instrument validity.
  \end{itemize}
\end{itemize}

\subsubsection{D03: How economically meaningful is the effect (percent change, elasticity, dollars)?}\label{d03-how-economically-meaningful-is-the-effect-percent-change-elasticity-dollars}

\begin{itemize}
\tightlist
\item
  Category: \texttt{D)\ Results,\ magnitudes,\ heterogeneity,\ robustness}
\item
  Confidence: \texttt{medium\ (0.44032523075622043)}
\item
  Retrieval method: \texttt{local}
\item
  Evidence type: \texttt{Reported\ effect\ sizes\ and\ WTP\ estimates\ from\ main\ text,\ figures,\ and\ tables}
\item
  Data source: Nielsen Homescan household-by-quarter data; ZBP
\item
  Table/Figure: Figure IV; Table II, Panel B
\item
  Assumption flag: \texttt{False}
\item
  Answer:
\end{itemize}

Economically small. Supermarket entry raises entrant chains' expenditure share by about 2 percentage points (\textasciitilde\$2 per \$100; \textasciitilde3 pp in food deserts), but effects on healthy consumption are tiny: one entry increases the Health Index by no more than \textasciitilde0.02 SD and explains at most \textasciitilde1.5\% of the income--nutrition gap. Preferences translate to modest willingness to pay for healthier bundles: roughly \(0.43–\)1.14 per 1,000 calories (or \(0.20–\)0.63 with added controls).

\begin{itemize}
\tightlist
\item
  Quote snippet:
\end{itemize}

\begin{quote}
entrant expenditure share increases by about 2 percentage points one year after entry.
\end{quote}

\begin{itemize}
\tightlist
\item
  Citation anchors:

  \begin{itemize}
  \tightlist
  \item
    page=18, words=0-349, section=Main text (Figure IV discussion), note=Entrant expenditure share increases \textasciitilde2 pp; \textasciitilde\$2 per \$100; \textasciitilde3\% in food deserts.
  \item
    page=22, words=0-349, section=Main text (Table II, Panel B), note=Effects on Health Index economically small; upper bound \textasciitilde0.02 SD; access explains \textasciitilde1.5\% of gap.
  \item
    page=37, words=275-433, section=Main text (WTP for healthy bundle), note=WTP: \$0.43 (lowest income) to \$1.14 (highest) per 1,000 calories.
  \item
    page=38, words=0-349, section=Main text (WTP with added controls), note=With shelf-life/convenience controls, WTP: \$0.20 to \$0.63 per 1,000 calories.
  \end{itemize}
\end{itemize}

\subsubsection{D04: What are the key robustness checks and do results survive them?}\label{d04-what-are-the-key-robustness-checks-and-do-results-survive-them}

\begin{itemize}
\tightlist
\item
  Category: \texttt{D)\ Results,\ magnitudes,\ heterogeneity,\ robustness}
\item
  Confidence: \texttt{low\ (0.33043848420519606)}
\item
  Retrieval method: \texttt{local}
\item
  Evidence type: \texttt{textual\ evidence\ from\ article\ excerpts}
\item
  Data source: QJE article excerpts (pages 21, 27, 34, 38)
\item
  Table/Figure: Online Appendix Tables A5, A9, A11
\item
  Assumption flag: \texttt{False}
\item
  Answer:
\end{itemize}

Key checks include: (1) redefining entry to only supercenters---expenditure-share effects get larger but Health Index effects remain economically small and mostly insignificant; (2) adding county-by-quarter fixed effects and controlling for convenience store entry---results do not qualitatively change; (3) in mover/event-study regressions, adding demographic controls has very little impact on estimates; (4) instrument validity tests---no correlation with tastes predicted by demographics and no systematic differences across high- vs low-income counties' high-Health Index categories; (5) adding/removing preference controls (e.g., county median income above national) yields very similar estimates. Overall, the main conclusions survive these checks.

\begin{itemize}
\tightlist
\item
  Quote snippet:
\end{itemize}

\begin{quote}
Adding county-by-quarter fixed effects or a control for convenience store entry does not qualitatively change the results.
\end{quote}

\begin{itemize}
\tightlist
\item
  Citation anchors:

  \begin{itemize}
  \tightlist
  \item
    page=21, words=275-503, section=Entry effects robustness checks, note=Supercenter-only entry; adding county-by-quarter fixed effects; controlling for convenience store entry; Health Index effects small.
  \item
    page=27, words=275-400, section=Event-study controls, note=Including demographics has very little impact; R\^{}2 only slightly increases.
  \item
    page=34, words=0-349, section=Instrument validity tests, note=Instrument not associated with predicted purchases; not systematically different across income-linked health categories.
  \item
    page=38, words=275-377, section=Exclusion restriction robustness, note=Estimates very similar when adding/removing controls like county median income above national.
  \end{itemize}
\end{itemize}

\subsubsection{D05: What placebo tests are run and what do they show?}\label{d05-what-placebo-tests-are-run-and-what-do-they-show}

\begin{itemize}
\tightlist
\item
  Category: \texttt{D)\ Results,\ magnitudes,\ heterogeneity,\ robustness}
\item
  Confidence: \texttt{low\ (0.2010298943364759)}
\item
  Retrieval method: \texttt{local}
\item
  Evidence type: \texttt{text\_quote}
\item
  Assumption flag: \texttt{True}
\item
  Assumption notes: The text does not label these as \textquotesingle placebo\textquotesingle{} explicitly; we interpret the pre-trend check and the instrument--tastes null-correlation tests as placebo-style validations.
\item
  Answer:
\end{itemize}

They run two main placebo-style checks. First, an event-study pre-trend test around household moves: before a move, there is no trend in healthy eating; after moving, purchases converge only slightly toward the new area's pattern. Second, two instrument-validity placebos: (i) using demographics to predict purchases at the household/county level and testing whether the price instrument correlates with those predicted tastes---there is no association; (ii) testing whether the instrument is systematically different in high--Health Index product groups between low- and high-income counties---again, no systematic difference. As an additional validation, they replicate the known immediate Coke-vs-Pepsi brand-choice effect using the same strategy, confirming the design detects expected effects.

\begin{itemize}
\tightlist
\item
  Quote snippet:
\end{itemize}

\begin{quote}
We find that the instrument is not associated with predicted purchases conditional on our standard set of county, product group, and year fixed effects.
\end{quote}

\begin{itemize}
\tightlist
\item
  Citation anchors:

  \begin{itemize}
  \tightlist
  \item
    page=4, words=275-423, section=Event study around moves, note=No pre-move trend in healthy eating; small post-move convergence.
  \item
    page=34, words=0-349, section=Instrument validity tests, note=Instrument not associated with predicted purchases; not systematically different across income in high-Health Index groups.
  \item
    page=27, words=275-482, section=Validation via Coke vs. Pepsi effect, note=Replicates known brand-choice effect as a design check.
  \end{itemize}
\end{itemize}

\subsubsection{D06: What falsification outcomes are tested (unaffected outcomes)?}\label{d06-what-falsification-outcomes-are-tested-unaffected-outcomes}

\begin{itemize}
\tightlist
\item
  Category: \texttt{D)\ Results,\ magnitudes,\ heterogeneity,\ robustness}
\item
  Confidence: \texttt{low\ (0.25114935550222117)}
\item
  Retrieval method: \texttt{local}
\item
  Evidence type: \texttt{Placebo/falsification\ (instrument\ validity)\ tests}
\item
  Table/Figure: Online Appendix Table A9
\item
  Answer:
\end{itemize}

They run placebo-style validity checks showing the instrument is unrelated to outcomes it should not affect: (1) predicted purchases/tastes based on household and county demographics, and (2) whether the instrument is systematically different in high--Health Index product groups across low- versus high-income counties. In both cases, there is no relationship.

\begin{itemize}
\tightlist
\item
  Quote snippet:
\end{itemize}

\begin{quote}
We find that the instrument is not associated with predicted purchases... We find that the instrument is not systematically different in high--Health Index product groups in low- versus high-income counties.
\end{quote}

\begin{itemize}
\tightlist
\item
  Citation anchors:

  \begin{itemize}
  \tightlist
  \item
    page=34, words=0-349, section=Instrument validity tests, note=Two tests: instrument unrelated to predicted purchases from demographics; not systematically different in high--Health Index product groups across income levels.
  \end{itemize}
\end{itemize}

\subsubsection{D07: What heterogeneity results are reported (by income, size, baseline exposure, region)?}\label{d07-what-heterogeneity-results-are-reported-by-income-size-baseline-exposure-region}

\begin{itemize}
\tightlist
\item
  Category: \texttt{D)\ Results,\ magnitudes,\ heterogeneity,\ robustness}
\item
  Confidence: \texttt{medium\ (0.4340082483189159)}
\item
  Retrieval method: \texttt{local}
\item
  Evidence type: \texttt{text}
\item
  Data source: Nielsen Homescan (2004--2016)
\item
  Table/Figure: Figure VII
\item
  Assumption flag: \texttt{False}
\item
  Answer:
\end{itemize}

Heterogeneity shows: (1) by income---expenditure share changes are generally larger for low-income households; (2) by size---when focusing on supercenter entry (larger stores), expenditure share effects are larger, while Health Index effects remain economically small and mostly insignificant; (3) by baseline exposure---expenditure share changes are larger for households in food deserts; (4) by region---adding census-region-by--product-group (and related) controls leaves estimates very similar, and the role of Census divisions is accounted for in the decomposition.

\begin{itemize}
\tightlist
\item
  Quote snippet:
\end{itemize}

\begin{quote}
the expenditure share changes are generally larger for low-income households and households in food deserts.
\end{quote}

\begin{itemize}
\tightlist
\item
  Citation anchors:

  \begin{itemize}
  \tightlist
  \item
    page=21, words=275-430, section=Entry effects and heterogeneity, note=Larger expenditure share changes for low-income households and those in food deserts.
  \item
    page=21, words=430-503, section=Entry effects and heterogeneity (supercenters), note=Supercenter-only entry yields larger expenditure share effects; Health Index effects small/mostly insignificant.
  \item
    page=38, words=0-220, section=Regional controls robustness, note=Adding census-region-by--product group and interactions yields very similar estimates.
  \item
    page=44, words=0-150, section=Decomposition figure note, note=Figure VII includes a Census division bar reflecting regional contribution.
  \end{itemize}
\end{itemize}

\subsubsection{D08: What mechanism tests are performed and what do they imply?}\label{d08-what-mechanism-tests-are-performed-and-what-do-they-imply}

\begin{itemize}
\tightlist
\item
  Category: \texttt{D)\ Results,\ magnitudes,\ heterogeneity,\ robustness}
\item
  Confidence: \texttt{low\ (0.16578383654914944)}
\item
  Retrieval method: \texttt{local}
\item
  Evidence type: \texttt{textual}
\item
  Data source: Article text (pages 33--34)
\item
  Table/Figure: Online Appendix Table A9
\item
  Assumption flag: \texttt{False}
\item
  Answer:
\end{itemize}

They run two predictive-taste tests of the price instrument: (1) predict product-group purchases using household and county demographics and test whether the instrument correlates with these predicted purchases---finding no association; and (2) use the nutrition--income gradient to test whether the instrument is systematically different for high--Health Index product groups in low- vs. high-income counties---finding no systematic differences. These results imply the instrument is not correlated with predictable tastes, supporting the exclusion restriction and suggesting it captures supply-side comparative pricing rather than demand-driven preferences.

\begin{itemize}
\tightlist
\item
  Quote snippet:
\end{itemize}

\begin{quote}
We implement two tests, each of which predicts tastes in a different way... We find that the instrument is not associated with predicted purchases... not systematically different in high--Health Index product groups...
\end{quote}

\begin{itemize}
\tightlist
\item
  Citation anchors:

  \begin{itemize}
  \tightlist
  \item
    page=34, words=0-349, note=Describes two tests and null findings on association with predicted tastes and income-based healthy product groups.
  \item
    page=33, words=275-397, note=States the identifying assumption and motivation for tests against predictable tastes.
  \end{itemize}
\end{itemize}

\subsubsection{D09: How sensitive are results to alternative samples/bandwidths/controls?}\label{d09-how-sensitive-are-results-to-alternative-samplesbandwidthscontrols}

\begin{itemize}
\tightlist
\item
  Category: \texttt{D)\ Results,\ magnitudes,\ heterogeneity,\ robustness}
\item
  Confidence: \texttt{medium\ (0.37156554303415845)}
\item
  Retrieval method: \texttt{local}
\item
  Evidence type: \texttt{text}
\item
  Data source: Provided QJE article excerpts (pp. 21, 27, 34, 38)
\item
  Table/Figure: Online Appendix Table A5; Online Appendix Table A11
\item
  Assumption flag: \texttt{False}
\item
  Answer:
\end{itemize}

Results appear robust to alternative controls and samples. Adding county-by-quarter fixed effects and a convenience-store-entry control does not qualitatively change results; adding household demographic controls has very little impact (only slightly increases R\^{}2). Restricting to supercenter-only entry yields larger expenditure-share effects but Health Index effects remain small and mostly insignificant. Instrument-validity checks show no association with predictable tastes, and estimates are very similar when adding/removing preference controls. The provided excerpts do not report bandwidth-sensitivity checks.

\begin{itemize}
\tightlist
\item
  Quote snippet:
\end{itemize}

\begin{quote}
Adding county-by-quarter fixed effects or a control for convenience store entry does not qualitatively change the results.
\end{quote}

\begin{itemize}
\tightlist
\item
  Citation anchors:

  \begin{itemize}
  \tightlist
  \item
    page=21, words=275-503, section=Entry effects and robustness, note=County-by-quarter FE and convenience-store-entry control; supercenter-only robustness; main effects unchanged
  \item
    page=27, words=275-482, section=Place-effects regressions, note=Including demographic controls has very little impact; R\^{}2 increases only slightly
  \item
    page=34, words=0-349, section=Instrument validity tests, note=Instrument unrelated to predictable tastes based on demographics and income-linked Health Index patterns
  \item
    page=38, words=275-377, section=Additional instrument robustness, note=Estimates very similar with alternative controls (Online Appendix Table A11)
  \end{itemize}
\end{itemize}

\subsubsection{D10: What are the main takeaways in the conclusion (bullet summary)?}\label{d10-what-are-the-main-takeaways-in-the-conclusion-bullet-summary}

\begin{itemize}
\item
  Category: \texttt{D)\ Results,\ magnitudes,\ heterogeneity,\ robustness}
\item
  Confidence: \texttt{low\ (0.2529475933271664)}
\item
  Retrieval method: \texttt{local}
\item
  Evidence type: \texttt{Direct\ quotations\ and\ paraphrase\ from\ the\ article\textquotesingle{}s\ conclusion\ and\ supporting\ results}
\item
  Assumption flag: \texttt{False}
\item
  Answer:
\item
  Supermarket entry has economically small effects on healthy grocery purchases; differences in local supermarket density explain no more than about 1.5\% of the income--nutrition gap.
\item
  Consumers travel far for groceries; even residents of food deserts get most groceries from supermarkets, so new nearby stores mostly divert purchases from other supermarkets.
\item
  The common "food desert" framing is misleading if it ignores shoppers\textquotesingle{} willingness to travel.
\item
  Broader place effects (e.g., neighborhood influences observed via moves) explain no more than about 3\% of the gap over several years.
\item
  Equalizing supply conditions (prices/availability) would reduce the gap by no more than about 10\%; most of the gap reflects differences in demand/preferences.
\item
  A demand-side, means-tested subsidy for healthy groceries could raise low-income households\textquotesingle{} healthy eating to high-income levels at a cost of about 15\% of the current SNAP budget.
\item
  Therefore, policies focused on eliminating food deserts are unlikely to substantially reduce nutritional inequality; policy attention should shift toward demand-side tools (e.g., targeted subsidies), with further welfare analysis and testing.
\item
  Quote snippet:
\end{itemize}

\begin{quote}
policies aimed at eliminating food deserts likely generate little progress toward a goal of reducing nutritional inequality.
\end{quote}

\begin{itemize}
\tightlist
\item
  Citation anchors:

  \begin{itemize}
  \tightlist
  \item
    page=47, words=0-349, section=VIII. CONCLUSION, note=Small impact of supermarket entry; ≤1.5\% explained; travel behavior and diversion; reframing food deserts
  \item
    page=48, words=0-323, section=VIII. CONCLUSION (cont.), note=Place effects ≤3\%; equalizing supply ≤10\%; means-tested subsidy \textasciitilde15\% of SNAP; limited gains from eliminating food deserts
  \item
    page=22, words=0-349, section=Event study results, note=Bound showing supermarket access explains ≈1.5\% of Health Index gap
  \item
    page=41, words=0-349, section=Decomposition of supply vs. demand, note=Supply explains \textasciitilde7--12\% while demand explains \textasciitilde88--93\% of the relationship
  \end{itemize}
\end{itemize}

\subsubsection{E01: What are the most important prior papers cited and why are they central here?}\label{e01-what-are-the-most-important-prior-papers-cited-and-why-are-they-central-here}

\begin{itemize}
\tightlist
\item
  Category: \texttt{E)\ Citations\ and\ related\ literature}
\item
  Confidence: \texttt{low\ (0.29819296174365445)}
\item
  Retrieval method: \texttt{local}
\item
  Evidence type: \texttt{direct\ quote\ and\ cited\ references}
\item
  Data source: Allcott et al., QJE 2019, Food Deserts and the Causes of Nutritional Inequality
\item
  Assumption flag: \texttt{False}
\item
  Answer:
\end{itemize}

The pivotal prior work is Bitler and Haider (2011), which argues that much existing research assumes supply-side (retail access) causes food deserts and concludes there is insufficient evidence that food deserts systematically cause unhealthy diets. This frames the central question the paper tackles by testing place effects and supply-side explanations. In addition, a core set of earlier food-access studies (Alwitt \& Donley 1997; Horowitz et al. 2004; Jetter \& Cassady 2005; Algert, Agrawal \& Lewis 2006; Baker et al. 2006; Powell et al. 2007; Larson, Story \& Nelson 2009; Sharkey, Horel \& Dean 2010) are central as they embody the supply-side perspective that this paper evaluates and ultimately finds limited for explaining nutritional inequality.

\begin{itemize}
\tightlist
\item
  Quote snippet:
\end{itemize}

\begin{quote}
``it appears that much of the existing research implicitly assumes that supply-side factors cause any food deserts that exist.''
\end{quote}

\begin{itemize}
\tightlist
\item
  Citation anchors:

  \begin{itemize}
  \tightlist
  \item
    page=2, words=300-415, note=Bitler and Haider (2011) critique: assumption of supply-side causality and insufficient evidence
  \item
    page=2, words=275-305, note=List of prominent prior food-access studies often cited in the literature
  \item
    page=48, words=0-150, note=Authors' conclusion that place/supply-side policies (eliminating food deserts) yield little progress
  \end{itemize}
\end{itemize}

\subsubsection{E02: Which papers does this work most directly build on or extend?}\label{e02-which-papers-does-this-work-most-directly-build-on-or-extend}

\begin{itemize}
\tightlist
\item
  Category: \texttt{E)\ Citations\ and\ related\ literature}
\item
  Confidence: \texttt{low\ (0.26231349582309116)}
\item
  Retrieval method: \texttt{local}
\item
  Evidence type: \texttt{direct\ textual\ statements\ in\ the\ article}
\item
  Data source: Quarterly Journal of Economics article text (provided context)
\item
  Assumption flag: \texttt{False}
\item
  Answer:
\end{itemize}

Most directly, it subsumes and replaces Handbury, Rahkovsky, and Schnell (2015) and Allcott, Diamond, and Dubé (2017); it also extends migration-based studies such as Bronnenberg, Dubé, and Gentzkow (2012), Atkin (2016), Eid et al. (2008), and Finkelstein, Gentzkow, and Williams (2016, 2018a, 2018b) and Molitor (2018).

\begin{itemize}
\tightlist
\item
  Quote snippet:
\end{itemize}

\begin{quote}
This article subsumes and replaces our previous work, Handbury, Rahkovsky, and Schnell (2015) and Allcott, Diamond, and Dubé (2017).
\end{quote}

\begin{itemize}
\tightlist
\item
  Citation anchors:

  \begin{itemize}
  \tightlist
  \item
    page=1, words=275-483, section=Front matter/notes, note=States the article subsumes and replaces Handbury et al. (2015) and Allcott et al. (2017).
  \item
    page=6, words=275-429, section=Related literature/migration event study, note=Says the migration event study adds a nutritional aspect to prior migration-based research (Bronnenberg et al. 2012; Atkin 2016; Eid et al. 2008; Finkelstein et al. 2016, 2018a,b; Molitor 2018).
  \end{itemize}
\end{itemize}

\subsubsection{E03: Which papers are used as benchmarks or comparisons in the results?}\label{e03-which-papers-are-used-as-benchmarks-or-comparisons-in-the-results}

\begin{itemize}
\tightlist
\item
  Category: \texttt{E)\ Citations\ and\ related\ literature}
\item
  Confidence: \texttt{low\ (0.2675972693069443)}
\item
  Retrieval method: \texttt{local}
\item
  Evidence type: \texttt{textual}
\item
  Answer:
\end{itemize}

DellaVigna and Gentzkow (2019) and Hausman (1996).

\begin{itemize}
\tightlist
\item
  Quote snippet:
\end{itemize}

\begin{quote}
DellaVigna and Gentzkow (2019), for example, use price variation from individual stores' short-term promotions... Hausman (1996) uses variation in prices over time in other markets
\end{quote}

\begin{itemize}
\tightlist
\item
  Citation anchors:

  \begin{itemize}
  \tightlist
  \item
    page=34, words=0-349, section=Instrument validity and comparisons, note=Paragraph contrasts their instrument with approaches in DellaVigna \& Gentzkow (2019) and Hausman (1996).
  \end{itemize}
\end{itemize}

\subsubsection{E04: What data sources or datasets are cited and how are they used?}\label{e04-what-data-sources-or-datasets-are-cited-and-how-are-they-used}

\begin{itemize}
\item
  Category: \texttt{E)\ Citations\ and\ related\ literature}
\item
  Confidence: \texttt{low\ (0.318231478069409)}
\item
  Retrieval method: \texttt{local}
\item
  Evidence type: \texttt{textual\ evidence\ from\ article\ body}
\item
  Data source: QJE article text (provided excerpts: pp. 1, 8--9)
\item
  Assumption flag: \texttt{False}
\item
  Answer:
\item
  Nielsen Homescan household panel: used to measure household grocery purchases and demographics; focus on packaged goods, with a 2004--2006 8,000-household ``magnet'' subsample recording nonpackaged items to validate representativeness of packaged produce.
\item
  Nielsen PanelViews (2017 add-on): collected health attitudes, a nutrition knowledge quiz, BMI, and diabetes status for Homescan households.
\item
  U.S. Dietary Guidelines: used to map age- and gender-specific calorie needs to households and construct adult-equivalent household size and daily calorie need.
\item
  Nielsen Retail Measurement Services (RMS): weekly UPC-level prices and sales for \textasciitilde42,000 stores in 160 chains (2006--2016); used for store-by-year analyses; liquor stores excluded.
\item
  ZIP Code Business Patterns (2004--2016): used to construct ZIP code establishment counts (508,951 ZIP-year observations).
\item
  General Nutrition Knowledge Questionnaire (Kliemann et al. 2016): source of 28 questions used in the PanelViews survey.
\item
  InfoUSA and National Establishment Time Series (NETS): noted as potentially useful for cross-sections but not reliable for store opening dates needed for entry analyses.
\item
  Nielsen Company datasets via the Kilts Center for Marketing (UChicago Booth): primary source of scanner and marketing data underlying analyses.
\item
  Quote snippet:
\end{itemize}

\begin{quote}
The Nielsen RMS data consist of weekly prices and sales volumes for each UPC sold at approximately 42,000 unique stores from 160 retail chains for 2006--2016
\end{quote}

\begin{itemize}
\tightlist
\item
  Citation anchors:

  \begin{itemize}
  \tightlist
  \item
    page=8, words=0-349, section=Data description (Homescan, PanelViews, Dietary Guidelines, RMS), note=Describes Homescan, magnet subsample, demographics, calorie needs via Dietary Guidelines, PanelViews survey, and RMS coverage/period.
  \item
    page=9, words=275-389, section=Additional data sources, note=Mentions 28 GNKQ questions, ZIP Code Business Patterns counts, data collapse, and comments on InfoUSA/NETS not recording openings.
  \item
    page=1, words=275-483, section=Acknowledgments/data provenance, note=States analyses are based in part on Nielsen Company data via the Kilts Center marketing databases.
  \end{itemize}
\end{itemize}

\subsubsection{E05: What methodological or econometric references are cited (e.g., DiD, IV, RDD methods)?}\label{e05-what-methodological-or-econometric-references-are-cited-eg-did-iv-rdd-methods}

\begin{itemize}
\tightlist
\item
  Category: \texttt{E)\ Citations\ and\ related\ literature}
\item
  Confidence: \texttt{medium\ (0.436374008181675)}
\item
  Retrieval method: \texttt{local}
\item
  Evidence type: \texttt{direct\_citation}
\item
  Assumption flag: \texttt{False}
\item
  Answer:
\end{itemize}

Methodological/econometric references cited include: (1) DellaVigna and Gentzkow (2019) on using store-level price variation (residual demand vs consumer demand identification); (2) Hausman (1996) on cross-market price variation as an instrumental variables approach; (3) Method of Moments Estimation (VI.C.); (4) Gelbach (2016) on covariate contribution/decomposition; (5) Dubois, Griffith, and Nevo (2014) on structural demand using prices and attributes; (6) Kling, Liebman, and Katz (2007) on experimental analysis of neighborhood effects.

\begin{itemize}
\tightlist
\item
  Quote snippet:
\end{itemize}

\begin{quote}
DellaVigna and Gentzkow (2019)\ldots{} identify a store's residual demand elasticity\ldots{} Hausman (1996) uses variation in prices over time in other markets. By contrast, our instrument generates cross-sectional identification\ldots{}
\end{quote}

\begin{itemize}
\tightlist
\item
  Citation anchors:

  \begin{itemize}
  \tightlist
  \item
    page=34, words=0-349, section=Instrument validity and comparisons, note=Main-text comparison to DellaVigna \& Gentzkow (2019) and Hausman (1996), and discussion of instrument
  \item
    page=34, words=275-385, section=VI.C. Method of Moments Estimation, note=Section header indicating Method of Moments Estimation
  \item
    page=50, words=0-349, section=References, note=Lists DellaVigna \& Gentzkow (2019); Dubois, Griffith, and Nevo (2014)
  \item
    page=51, words=0-349, section=References, note=Lists Gelbach (2016); Hausman (1996); Kling, Liebman, and Katz (2007)
  \end{itemize}
\end{itemize}

\subsubsection{E06: Are there any seminal or classic references the paper positions itself against?}\label{e06-are-there-any-seminal-or-classic-references-the-paper-positions-itself-against}

\begin{itemize}
\tightlist
\item
  Category: \texttt{E)\ Citations\ and\ related\ literature}
\item
  Confidence: \texttt{low\ (0.2711114475094952)}
\item
  Retrieval method: \texttt{local}
\item
  Evidence type: \texttt{textual}
\item
  Assumption flag: \texttt{False}
\item
  Answer:
\end{itemize}

Yes. The paper pushes back against the classic ``food desert'' causal narrative---i.e., that supply-side access drives unhealthy eating---associated with early, frequently cited studies documenting access disparities (e.g., Alwitt \& Donley 1997; Horowitz et al. 2004; Jetter \& Cassady 2005; Algert, Agrawal, \& Lewis 2006; Baker et al. 2006; Powell et al. 2007; Larson, Story, \& Nelson 2009; Sharkey, Horel, \& Dean 2010) and related policy arguments that increasing healthy-food supply would meaningfully reduce nutritional inequality. The authors explicitly state their findings counter that argument.

\begin{itemize}
\tightlist
\item
  Quote snippet:
\end{itemize}

\begin{quote}
These findings counter the argument that policies to increase the supply of healthy groceries could play an important role in reducing nutritional inequality.
\end{quote}

\begin{itemize}
\tightlist
\item
  Citation anchors:

  \begin{itemize}
  \tightlist
  \item
    page=2, words=0-349, section=Introduction, note=States findings counter the argument that increasing healthy-food supply would reduce nutritional inequality and references the large literature on food deserts.
  \item
    page=2, words=275-415, section=Introduction (footnotes/references), note=Lists classic food-desert studies (Alwitt \& Donley 1997; Horowitz et al. 2004; Jetter \& Cassady 2005; Algert et al. 2006; Baker et al. 2006; Powell et al. 2007; Larson et al. 2009; Sharkey et al. 2010) and contrasts with Bitler \& Haider (2011).
  \item
    page=48, words=0-323, section=Conclusion, note=Concludes that eliminating food deserts would generate little progress toward reducing nutritional inequality.
  \end{itemize}
\end{itemize}

\subsubsection{E07: Are there citations to code, data repositories, or appendices that are essential to the claims?}\label{e07-are-there-citations-to-code-data-repositories-or-appendices-that-are-essential-to-the-claims}

\begin{itemize}
\tightlist
\item
  Category: \texttt{E)\ Citations\ and\ related\ literature}
\item
  Confidence: \texttt{low\ (0.3281837968846545)}
\item
  Retrieval method: \texttt{local}
\item
  Evidence type: \texttt{supplementary\ material\ and\ references}
\item
  Data source: Harvard Dataverse (doi:10.7910/DVN/MSOBYI); Online Appendix; Nielsen Datasets at the Kilts Center
\item
  Assumption flag: \texttt{False}
\item
  Answer:
\end{itemize}

Yes. The article cites an Online Appendix and a Harvard Dataverse repository with replication code/data (doi:10.7910/DVN/MSOBYI), and notes use of Nielsen datasets via the Kilts Center.

\begin{itemize}
\tightlist
\item
  Quote snippet:
\end{itemize}

\begin{quote}
Code replicating tables and fig- ures in this article can be found in Allcott et al. (2019), in the Harvard Dataverse, doi: 10.7910/DVN/MSOBYI.
\end{quote}

\begin{itemize}
\tightlist
\item
  Citation anchors:

  \begin{itemize}
  \tightlist
  \item
    page=48, words=0-323, section=Supplementary Material, note=Mentions Online Appendix and replication code in Harvard Dataverse with DOI.
  \item
    page=49, words=0-349, section=References, note=Lists the Harvard Dataverse replication data entry with DOI.
  \item
    page=1, words=275-483, section=Acknowledgments/Data, note=Describes use of Nielsen datasets via the Kilts Center data repository.
  \end{itemize}
\end{itemize}

\subsubsection{E08: What gaps in the literature do the authors say these citations leave open?}\label{e08-what-gaps-in-the-literature-do-the-authors-say-these-citations-leave-open}

\begin{itemize}
\tightlist
\item
  Category: \texttt{E)\ Citations\ and\ related\ literature}
\item
  Confidence: \texttt{low\ (0.3164521631976586)}
\item
  Retrieval method: \texttt{local}
\item
  Evidence type: \texttt{direct\ quote\ from\ source}
\item
  Data source: QJE article text (Allcott et al. 2019)
\item
  Assumption flag: \texttt{False}
\item
  Answer:
\end{itemize}

The cited literature largely assumes supply-side access drives outcomes but leaves a key gap: there is not sufficient causal evidence that food deserts (neighborhood food environments) are systematically the cause of unhealthy eating among low-income households.

\begin{itemize}
\tightlist
\item
  Quote snippet:
\end{itemize}

\begin{quote}
``we do not have sufficient evidence to determine whether food deserts are systematically the cause'' of unhealthy eating by low-income people.
\end{quote}

\begin{itemize}
\tightlist
\item
  Citation anchors:

  \begin{itemize}
  \tightlist
  \item
    page=2, words=275-415, note=Bitler and Haider (2011) noted insufficient evidence that food deserts systematically cause unhealthy eating; prior work often assumes supply-side causation.
  \end{itemize}
\end{itemize}

\subsubsection{F01: Are replication files or code provided? If so, where?}\label{f01-are-replication-files-or-code-provided-if-so-where}

\begin{itemize}
\tightlist
\item
  Category: \texttt{F)\ Replication\ and\ transparency}
\item
  Confidence: \texttt{low\ (0.19556468707292782)}
\item
  Retrieval method: \texttt{local}
\item
  Evidence type: \texttt{text}
\item
  Data source: Harvard Dataverse DOI:10.7910/DVN/MSOBYI
\item
  Assumption flag: \texttt{False}
\item
  Answer:
\end{itemize}

Yes. Replication code and data are provided in the Harvard Dataverse (Allcott et al., 2019), DOI 10.7910/DVN/MSOBYI.

\begin{itemize}
\tightlist
\item
  Quote snippet:
\end{itemize}

\begin{quote}
Code replicating tables and figures in this article can be found in Allcott et al. (2019), in the Harvard Dataverse, doi: 10.7910/DVN/MSOBYI.
\end{quote}

\begin{itemize}
\tightlist
\item
  Citation anchors:

  \begin{itemize}
  \tightlist
  \item
    page=48, words=0-323, section=SUPPLEMENTARY MATERIAL, note=States that code replicating tables and figures is available with DOI.
  \item
    page=49, words=0-349, section=References, note=Reference entry titled \textquotesingle Replication Data...\textquotesingle{} with the same Harvard Dataverse DOI.
  \end{itemize}
\end{itemize}

\subsubsection{F02: Is there a pre-analysis plan or registered trial? Provide details if mentioned.}\label{f02-is-there-a-pre-analysis-plan-or-registered-trial-provide-details-if-mentioned}

\begin{itemize}
\tightlist
\item
  Category: \texttt{F)\ Replication\ and\ transparency}
\item
  Confidence: \texttt{low\ (0.24517154549579873)}
\item
  Retrieval method: \texttt{local}
\item
  Evidence type: \texttt{textual\ (no\ mention\ in\ provided\ excerpts)}
\item
  Data source: QJE (2020) Food Deserts and Nutritional Inequality -- provided excerpts
\item
  Assumption flag: \texttt{False}
\item
  Answer:
\end{itemize}

No. The provided excerpts do not mention any pre-analysis plan or registered trial.

\begin{itemize}
\tightlist
\item
  Quote snippet:
\end{itemize}

\begin{quote}
An Online Appendix for this article can be found at The Quar- terly Journal of Economics online. Code replicating tables and fig- ures in this article can be found in Allcott et al. (2019)
\end{quote}

\begin{itemize}
\tightlist
\item
  Citation anchors:

  \begin{itemize}
  \tightlist
  \item
    page=48, words=200-320, section=SUPPLEMENTARY MATERIAL, note=Mentions online appendix and replication code; no preregistration or pre-analysis plan is indicated.
  \end{itemize}
\end{itemize}

\subsubsection{F03: Are data access constraints disclosed (restricted access, proprietary data, NDAs)?}\label{f03-are-data-access-constraints-disclosed-restricted-access-proprietary-data-ndas}

\begin{itemize}
\tightlist
\item
  Category: \texttt{F)\ Replication\ and\ transparency}
\item
  Confidence: \texttt{low\ (0.2872801877515374)}
\item
  Retrieval method: \texttt{local}
\item
  Evidence type: \texttt{article\ text\ (acknowledgments\ and\ data\ description)}
\item
  Data source: Nielsen datasets via Kilts Center (Homescan/RMS, TDLinx); USDA nutrition databases; Harvard Dataverse replication DOI:10.7910/DVN/MSOBYI
\item
  Assumption flag: \texttt{True}
\item
  Assumption notes: Proprietary/restricted access is inferred from the Nielsen/Kilts Center sourcing and standard disclaimer language; no explicit mention of access restrictions or NDAs in the provided text.
\item
  Answer:
\end{itemize}

Partially. The paper discloses use of proprietary Nielsen datasets (Homescan/RMS and TDLinx) provided via the Kilts Center and includes required non-involvement disclaimers; it also cites a Harvard Dataverse replication package. However, explicit statements about restricted access terms or NDAs are not detailed.

\begin{itemize}
\tightlist
\item
  Quote snippet:
\end{itemize}

\begin{quote}
data from the Nielsen Company (US), LLC, and marketing databases provided through the Nielsen Datasets at the Kilts Center for Marketing Data Center at the University of Chicago Booth
\end{quote}

\begin{itemize}
\tightlist
\item
  Citation anchors:

  \begin{itemize}
  \tightlist
  \item
    page=1, words=275-483, section=Acknowledgments/Disclosure, note=Nielsen data source via Kilts Center; non-involvement disclaimer; USDA dissemination note.
  \item
    page=10, words=0-200, section=Data description, note=Use of Nielsen's proprietary TDLinx data set.
  \item
    page=49, words=0-349, section=References, note=Harvard Dataverse replication data citation (DOI provided).
  \end{itemize}
\end{itemize}

\subsubsection{F04: Are key steps in data cleaning and construction documented?}\label{f04-are-key-steps-in-data-cleaning-and-construction-documented}

\begin{itemize}
\tightlist
\item
  Category: \texttt{F)\ Replication\ and\ transparency}
\item
  Confidence: \texttt{low\ (0.2245506611258558)}
\item
  Retrieval method: \texttt{local}
\item
  Evidence type: \texttt{textual}
\item
  Data source: Nielsen Homescan and RMS; Nielsen TDLinx; USDA FNDDS and SR; ZIP Code Business Patterns
\item
  Table/Figure: Table I (Panels A and B)
\item
  Assumption flag: \texttt{False}
\item
  Answer:
\end{itemize}

Yes. The article documents multiple key steps in data cleaning and construction, including excluding liquor stores and nonpackaged items, deflating prices/incomes to 2010 dollars, collapsing data to household/store-by-year, validating and restricting supermarket entries (and dropping spurious ones), and detailing variable construction and weighting.

\begin{itemize}
\tightlist
\item
  Quote snippet:
\end{itemize}

\begin{quote}
we use a list of specific TDLinx stores transferred through mergers and acquisitions to exclude spurious ``entrants'' ... and we further drop potentially spurious entries
\end{quote}

\begin{itemize}
\tightlist
\item
  Citation anchors:

  \begin{itemize}
  \tightlist
  \item
    page=8, words=275-393, section=Data description (RMS/Homescan), note=Excludes liquor stores; excludes bulk/nonpackaged items; deflates to 2010 dollars.
  \item
    page=9, words=0-349, section=Descriptive statistics and construction notes, note=Weights for representativeness; construction of household size and normalized variables; collapsing to household/store-by-year.
  \item
    page=10, words=0-349, section=Store entry measurement (TDLinx), note=Validation of opening dates; inclusion/exclusion criteria for store types.
  \item
    page=10, words=275-424, section=Entry data cleaning, note=Excludes spurious entrants from M\&A transfers; drops entries with same subtype in same census block in prior year.
  \end{itemize}
\end{itemize}

\subsubsection{F05: Are robustness and sensitivity analyses fully reported or partially omitted?}\label{f05-are-robustness-and-sensitivity-analyses-fully-reported-or-partially-omitted}

\begin{itemize}
\tightlist
\item
  Category: \texttt{F)\ Replication\ and\ transparency}
\item
  Confidence: \texttt{low\ (0.28247126071816303)}
\item
  Retrieval method: \texttt{local}
\item
  Evidence type: \texttt{direct\_text}
\item
  Data source: Quarterly Journal of Economics article text (provided context)
\item
  Table/Figure: Online Appendix Tables A4, A5, A9, A11
\item
  Assumption flag: \texttt{False}
\item
  Answer:
\end{itemize}

Fully reported

\begin{itemize}
\tightlist
\item
  Quote snippet:
\end{itemize}

\begin{quote}
As a robustness check, Online Appendix Table A11 repeats this decomposition using our alternative estimates...
\end{quote}

\begin{itemize}
\tightlist
\item
  Citation anchors:

  \begin{itemize}
  \tightlist
  \item
    page=41, words=275-411, section=Main text, note=Explicit robustness check using alternative specifications (Table A11).
  \item
    page=34, words=0-349, section=Main text, note=Two validity tests for the instrument; details in Online Appendix Table A9.
  \item
    page=21, words=275-503, section=Main text, note=Additional estimates and sensitivity (Online Appendix Tables A4, A5); discusses anomaly and added controls.
  \item
    page=38, words=275-377, section=Main text, note=Results robust across alternative controls; references Table A11.
  \item
    page=7, words=0-349, section=Data/Overview, note=States all appendix material is available in the Online Appendix.
  \end{itemize}
\end{itemize}

\subsubsection{G01: What populations or settings are most likely to generalize from this study?}\label{g01-what-populations-or-settings-are-most-likely-to-generalize-from-this-study}

\begin{itemize}
\tightlist
\item
  Category: \texttt{G)\ External\ validity\ and\ generalization}
\item
  Confidence: \texttt{low\ (0.29237732075827916)}
\item
  Retrieval method: \texttt{local}
\item
  Evidence type: \texttt{Study\ population\ and\ design\ scope}
\item
  Data source: Nielsen Homescan Panel (2004--2016) and Nielsen Retail Scanner (RMS) data
\item
  Table/Figure: Figure V
\item
  Assumption flag: \texttt{False}
\item
  Answer:
\end{itemize}

Findings most directly generalize to U.S. households' at-home, packaged grocery purchases in retail scanner--covered markets (RMS stores), analyzed at the ZIP code or county level over a medium/several-year horizon. They are less applicable to away-from-home food consumption, nonpackaged foods, or nationally representative mover populations.

\begin{itemize}
\tightlist
\item
  Quote snippet:
\end{itemize}

\begin{quote}
We therefore focus on explaining income-related differences in the take-home market (i.e., grocery purchases) instead of overall diets.
\end{quote}

\begin{itemize}
\tightlist
\item
  Citation anchors:

  \begin{itemize}
  \tightlist
  \item
    page=7, words=0-349, section=II.A. Nielsen Homescan and Retail Scanner Data, note=Homescan scope; packaged UPC foods; excludes away-from-home; focus on take-home market
  \item
    page=24, words=0-349, section=II. DATA / Event study setup, note=Health Index defined for packaged groceries in RMS stores; geography is ZIP/county; movers not nationally representative; medium-term estimation
  \item
    page=48, words=0-323, section=Conclusion, note=Effects assessed over the several-year time horizon allowed by the data
  \end{itemize}
\end{itemize}

\subsubsection{G02: What populations or settings are least likely to generalize?}\label{g02-what-populations-or-settings-are-least-likely-to-generalize}

\begin{itemize}
\tightlist
\item
  Category: \texttt{G)\ External\ validity\ and\ generalization}
\item
  Confidence: \texttt{low\ (0.24059967921698125)}
\item
  Retrieval method: \texttt{local}
\item
  Evidence type: \texttt{textual\ evidence\ from\ article\ excerpts}
\item
  Data source: Nielsen Homescan Panel (2004--2016)
\item
  Assumption flag: \texttt{False}
\item
  Answer:
\end{itemize}

The study's findings are least likely to generalize to: (1) away-from-home eating settings (e.g., restaurants), (2) nonpackaged foods without UPCs (e.g., bulk/fresh items), (3) nationally representative inferences without weighting, and (4) contexts involving very recent store entries or randomized neighborhood assignment (given imputation issues and nonrandom migration).

\begin{itemize}
\tightlist
\item
  Quote snippet:
\end{itemize}

\begin{quote}
Homescan does not include data on food purchased away from home in establishments like restaurants. We therefore focus on explaining income-related differences in the take-home market
\end{quote}

\begin{itemize}
\tightlist
\item
  Citation anchors:

  \begin{itemize}
  \tightlist
  \item
    page=7, words=0-349, section=II.A. Nielsen Homescan and Retail Scanner Data, note=Homescan excludes away-from-home food and mostly nonpackaged items.
  \item
    page=21, words=275-503, section=Estimation notes, note=Observations are not weighted for national representativeness.
  \item
    page=20, words=275-432, section=Store entry analysis, note=Imputation is unreliable for recently entered stores.
  \item
    page=23, words=275-408, section=Migration/place effects, note=Moves are not randomly assigned; endogeneity concerns limit generalization to randomized settings.
  \end{itemize}
\end{itemize}

\subsubsection{G03: Do the authors discuss boundary conditions or scope limits?}\label{g03-do-the-authors-discuss-boundary-conditions-or-scope-limits}

\begin{itemize}
\tightlist
\item
  Category: \texttt{G)\ External\ validity\ and\ generalization}
\item
  Confidence: \texttt{low\ (0.2423924986550409)}
\item
  Retrieval method: \texttt{local}
\item
  Evidence type: \texttt{direct\ textual\ evidence}
\item
  Data source: Nielsen Homescan panel; Google Maps API (driving times)
\item
  Assumption flag: \texttt{True}
\item
  Assumption notes: Partial-equilibrium framework; possible upward bias from mover endogeneity; effects measured only over several years; supermarket exposure defined by 0--10/10--15 minute drive bands assuming no congestion; annual-only location reporting.
\item
  Answer:
\end{itemize}

Yes. They bound their analysis to medium-term, partial-equilibrium effects over a several-year horizon; interpret estimates as likely upper bounds due to potential mover endogeneity; restrict exposure to supermarket entries within 0--10 and 10--15 minutes (assuming no congestion); and use annual location data because panel geography is only reported at year-end.

\begin{itemize}
\tightlist
\item
  Quote snippet:
\end{itemize}

\begin{quote}
We therefore interpret the results in this section as likely upper bounds on true place effects.
\end{quote}

\begin{itemize}
\tightlist
\item
  Citation anchors:

  \begin{itemize}
  \tightlist
  \item
    page=4, words=275-423, section=Movers event study; scope, note=Panel not long enough; bounds medium-term, partial-equilibrium effects of place.
  \item
    page=24, words=0-349, section=Movers event study; limitations, note=Interpret results as likely upper bounds; annual location reporting; medium-term only.
  \item
    page=17, words=0-349, section=Supermarket entry design, note=Define exposure within 0--10/10--15 minutes and assume no congestion delay.
  \item
    page=48, words=0-323, section=Conclusion/scope, note=Effects assessed over the several-year time horizon allowed by the data; partial-equilibrium attribution.
  \end{itemize}
\end{itemize}

\subsubsection{G04: How might the results change in different time periods or markets?}\label{g04-how-might-the-results-change-in-different-time-periods-or-markets}

\begin{itemize}
\tightlist
\item
  Category: \texttt{G)\ External\ validity\ and\ generalization}
\item
  Confidence: \texttt{medium\ (0.35904641697863765)}
\item
  Retrieval method: \texttt{local}
\item
  Evidence type: \texttt{direct\ text}
\item
  Data source: Nielsen Homescan and RMS
\item
  Table/Figure: Figure IV (Event Study of Supermarket Entry)
\item
  Assumption flag: \texttt{False}
\item
  Answer:
\end{itemize}

They are largely identified within market--year cells, so period-wide shocks are differenced out, while cross-market differences can matter. In particular, results can differ by geography because retailers' presence and relative pricing vary across markets (e.g., produce cheaper on the West Coast, costlier on the East Coast). Effects also differ by market type: expenditure share changes are larger for low-income and food-desert areas. For timing, the movers design captures medium-term changes (a few years after a move) and is likely an upper bound on true place effects.

\begin{itemize}
\tightlist
\item
  Quote snippet:
\end{itemize}

\begin{quote}
identification comes only from variation in the relative prices across product groups within a market in a given year.
\end{quote}

\begin{itemize}
\tightlist
\item
  Citation anchors:

  \begin{itemize}
  \tightlist
  \item
    page=32, words=0-349, section=Instrument and identification, note=Identification comes from relative prices across product groups within a market-year.
  \item
    page=32, words=275-431, section=Instrument and identification, note=Produce predicted cheap on West Coast and expensive on East Coast, implying geographic differences.
  \item
    page=24, words=0-349, section=Movers event study, note=Estimates reflect medium-term effects after moves; authors view them as likely upper bounds due to possible endogeneity.
  \item
    page=21, words=275-503, section=Heterogeneity in effects, note=Expenditure share changes are larger for low-income households and in food deserts.
  \item
    page=19, words=0-310, section=Event Study of Supermarket Entry, note=2004--2016 event study; separate full-sample vs. food-desert subsample panels.
  \end{itemize}
\end{itemize}

\subsubsection{H01: Are key variables measured directly or via proxies?}\label{h01-are-key-variables-measured-directly-or-via-proxies}

\begin{itemize}
\tightlist
\item
  Category: \texttt{H)\ Measurement\ validity}
\item
  Confidence: \texttt{medium\ (0.3645774780007952)}
\item
  Retrieval method: \texttt{local}
\item
  Evidence type: \texttt{mixed\ (direct\ measures\ and\ constructed\ proxies/instruments)}
\item
  Data source: Nielsen Homescan and RMS
\item
  Assumption flag: \texttt{False}
\item
  Answer:
\end{itemize}

Mixed. Many variables are directly observed or self-reported (e.g., demographics, income bins, purchases/prices, survey measures), while key constructs are proxies/constructed (e.g., adult-equivalent household size and daily calorie need from guidelines; an IV based on out-of-market chain pricing to proxy cost advantages; residualized income quartiles; unobserved product characteristics).

\begin{itemize}
\tightlist
\item
  Quote snippet:
\end{itemize}

\begin{quote}
Our household size variable measures the number of adult ``equivalents'' in the household, where children are scaled into adults by their calorie needs.
\end{quote}

\begin{itemize}
\tightlist
\item
  Citation anchors:

  \begin{itemize}
  \tightlist
  \item
    page=8, words=0-349, section=Data description (Homescan), note=Direct self-reports of demographics; construction of daily calorie need and adult-equivalent household size; additional survey measures.
  \item
    page=30, words=0-349, section=Demand model, note=Constructed variables (price per calorie, characteristics per calorie) and inclusion of an unobserved product characteristic ξ.
  \item
    page=30, words=275-393, section=Price endogeneity, note=Motivates need for an instrument (proxy) due to correlation between prices and preferences.
  \item
    page=31, words=275-432, section=IV construction (part 1), note=Defines IV components using out-of-market chain prices to capture cost advantages, excluding local demand.
  \item
    page=32, words=0-349, section=IV construction (part 2), note=Formal definition of the price instrument Pjmt as a weighted average cost advantage across UPCs.
  \item
    page=34, words=0-349, section=Instrument validity checks, note=Validates IV is unrelated to predictable tastes; describes use across applications.
  \item
    page=34, words=275-385, section=Estimation setup, note=Constructs income quartiles after residualizing by household size/age/year---another constructed measure.
  \item
    page=42, words=275-356, section=Controls and survey measures, note=Direct measures of demographics and additional survey variables used as controls.
  \end{itemize}
\end{itemize}

\subsubsection{H02: What measurement error risks are acknowledged or likely?}\label{h02-what-measurement-error-risks-are-acknowledged-or-likely}

\begin{itemize}
\tightlist
\item
  Category: \texttt{H)\ Measurement\ validity}
\item
  Confidence: \texttt{low\ (0.2590133017456877)}
\item
  Retrieval method: \texttt{local}
\item
  Evidence type: \texttt{textual\ evidence\ with\ minor\ inference}
\item
  Data source: Nielsen Homescan and RMS; ZBP
\item
  Assumption flag: \texttt{True}
\item
  Assumption notes: Self-reported BMI/diabetes/attitudes/knowledge likely contain reporting error; income in 16 bins implies coarseness. These risks are not explicitly labeled as `measurement error' but follow from the described data features.
\item
  Answer:
\end{itemize}

Acknowledged/likely measurement error risks include: (1) Undercoverage of nonpackaged/bulk items, since both Homescan and RMS exclude bulk produce and other nonpackaged goods (mitigated partly by evidence that \textasciitilde60\% of produce calories among magnet households are packaged). (2) RMS is a sample, not a census; the store-type mix in RMS may not match a county's true distribution, requiring regression adjustment with ZBP and leaving scope for residual area-level measurement error in the Health Index. (3) Household residence is observed only at year-end, so exposure around moves is measured annually; within-year moves and local shopping patterns can cause misclassification, mitigated via a 50\% local-shopping restriction. (4) Self-reported health-related variables (e.g., BMI, diabetes, health importance, nutrition knowledge) are likely subject to reporting/recall error. (5) Income is recorded in 16 bins, implying coarse measurement of income.

\begin{itemize}
\tightlist
\item
  Quote snippet:
\end{itemize}

\begin{quote}
RMS does not include sales of bulk produce and other nonpackaged items.
\end{quote}

\begin{itemize}
\tightlist
\item
  Citation anchors:

  \begin{itemize}
  \tightlist
  \item
    page=8, words=0-349, section=Data coverage and demographics, note=Homescan excludes nonpackaged items; magnet sample shows 60\% of produce calories packaged; income recorded in 16 bins; presence of self-reports.
  \item
    page=8, words=275-393, section=RMS coverage, note=RMS excludes bulk produce and other nonpackaged items.
  \item
    page=24, words=275-452, section=RMS store-type composition vs true distribution, note=RMS is not a complete census; adjust area Health Index for store-type distribution using ZBP.
  \item
    page=24, words=0-349, section=Location timing, note=Household location observed only at end of each year; event study uses annual data around moves.
  \item
    page=25, words=275-468, section=Local shopping restriction, note=50\% local shopping restriction to mitigate misclassification around moves.
  \end{itemize}
\end{itemize}

\subsubsection{H03: Are there validation checks for key measures?}\label{h03-are-there-validation-checks-for-key-measures}

\begin{itemize}
\tightlist
\item
  Category: \texttt{H)\ Measurement\ validity}
\item
  Confidence: \texttt{low\ (0.26944750917915244)}
\item
  Retrieval method: \texttt{local}
\item
  Evidence type: \texttt{reported\ statistical\ validation\ and\ falsification\ tests}
\item
  Data source: Nielsen Homescan
\item
  Assumption flag: \texttt{False}
\item
  Answer:
\end{itemize}

Yes. They validate the key dietary-quality measure by showing the linearized Health Index closely matches the standard HEI (r=0.91) and that both correlate with BMI and diabetes. They also conduct falsification tests for their instrument, finding no association with predictable tastes or differential patterns across healthy product groups by income.

\begin{itemize}
\tightlist
\item
  Quote snippet:
\end{itemize}

\begin{quote}
the correlation between ``true'' HEI and linearized HEI is 0.91 in our household-by-year Homescan data.
\end{quote}

\begin{itemize}
\tightlist
\item
  Citation anchors:

  \begin{itemize}
  \tightlist
  \item
    page=11, words=0-349, section=Methods for Health Index (HEI linearization), note=Reports r=0.91 between true HEI and linearized HEI; correlations with BMI and diabetes
  \item
    page=34, words=0-349, section=Instrument validity checks, note=Instrument not associated with predicted purchases; no systematic differences across healthy categories by income
  \end{itemize}
\end{itemize}

\subsubsection{H04: Do the authors discuss construct validity for core outcomes?}\label{h04-do-the-authors-discuss-construct-validity-for-core-outcomes}

\begin{itemize}
\tightlist
\item
  Category: \texttt{H)\ Measurement\ validity}
\item
  Confidence: \texttt{low\ (0.2371564135213259)}
\item
  Retrieval method: \texttt{local}
\item
  Evidence type: \texttt{Within-study\ correlational\ (convergent)\ validity\ evidence}
\item
  Data source: Nielsen Homescan household-by-year data
\item
  Table/Figure: Online Appendix Table A2
\item
  Assumption flag: \texttt{False}
\item
  Answer:
\end{itemize}

Yes. They provide construct validity evidence for their core outcome (the Health Index) by showing it strongly correlates with the original HEI (r = 0.91) and that both the ``true'' and linearized HEI are highly correlated with BMI and diabetes status.

\begin{itemize}
\tightlist
\item
  Quote snippet:
\end{itemize}

\begin{quote}
the correlation between ``true'' HEI and linearized HEI is 0.91\ldots{} both the ``true'' HEI and our linearized HEI are highly correlated with Homescan panelists' BMI and diabetes status.
\end{quote}

\begin{itemize}
\tightlist
\item
  Citation anchors:

  \begin{itemize}
  \tightlist
  \item
    page=11, words=0-349, section=Description/validation of Health Index (HEI), note=Reports 0.91 correlation with true HEI; notes correlations with BMI and diabetes.
  \item
    page=11, words=275-399, section=III.A. Purchase Disparities: The Nutrition--Income Relationship, note=States both true and linearized HEI are highly correlated with BMI and diabetes status.
  \end{itemize}
\end{itemize}

\subsubsection{I01: What policy counterfactuals are considered or implied?}\label{i01-what-policy-counterfactuals-are-considered-or-implied}

\begin{itemize}
\tightlist
\item
  Category: \texttt{I)\ Policy\ counterfactuals\ and\ welfare}
\item
  Confidence: \texttt{low\ (0.3292589015742387)}
\item
  Retrieval method: \texttt{local}
\item
  Evidence type: \texttt{text}
\item
  Table/Figure: Figure VI
\item
  Assumption flag: \texttt{False}
\item
  Answer:
\end{itemize}

The paper considers or implies several policy counterfactuals: (1) a partial‑equilibrium `place effect' move---shifting a household from a low‑ to a high‑income retail environment; (2) supply‑side equalization---setting all groups' prices to those of high‑income areas and also equalizing product nutrient characteristics (eliminating food‑desert supply differences); (3) demand‑side preference counterfactuals---imposing high‑income nutrient and product‑group preferences on other groups; (4) encouraging supermarket entry (supply‑side policy); and (5) subsidies for healthy foods---implemented as a simple subsidy that scales with product healthfulness, including a means‑tested healthy‑grocery subsidy (estimated to close the gap at about 15\% of the SNAP budget).

\begin{itemize}
\tightlist
\item
  Quote snippet:
\end{itemize}

\begin{quote}
We focus on a simple subsidy that scales in a product's healthfulness.
\end{quote}

\begin{itemize}
\tightlist
\item
  Citation anchors:

  \begin{itemize}
  \tightlist
  \item
    page=27, words=275-482, section=Partial-equilibrium place effect thought experiment, note=Move from low- to high-income retail environment
  \item
    page=40, words=0-285, section=Figure VI counterfactuals, note=Equalizing prices and product nutrient characteristics; setting preferences to high-income group
  \item
    page=43, words=275-387, section=VII.C Using Subsidies, note=Supply-side policy of supermarket entry; healthy-food subsidies scaling with healthfulness
  \item
    page=48, words=0-323, section=Conclusion, note=Means-tested subsidy could close gap at \textasciitilde15\% of SNAP; equalizing supply has limited effect
  \end{itemize}
\end{itemize}

\subsubsection{I02: What are the main welfare tradeoffs or distributional impacts discussed?}\label{i02-what-are-the-main-welfare-tradeoffs-or-distributional-impacts-discussed}

\begin{itemize}
\tightlist
\item
  Category: \texttt{I)\ Policy\ counterfactuals\ and\ welfare}
\item
  Confidence: \texttt{medium\ (0.3951936988208884)}
\item
  Retrieval method: \texttt{local}
\item
  Evidence type: \texttt{Empirical\ event-study\ plus\ structural\ model\ counterfactuals\ and\ cost\ calculations}
\item
  Data source: Allcott et al. (2019), The Quarterly Journal of Economics
\item
  Table/Figure: Table V: Impacts of Means-Tested Healthy Grocery Subsidies
\item
  Answer:
\end{itemize}

The paper highlights a key welfare tradeoff between supply-side store-entry policies and targeted consumer subsidies. Equalizing supply (e.g., eliminating food deserts) reduces nutritional inequality by at most about 10\%, implying limited welfare gains from costly store subsidies. In contrast, a means-tested subsidy tied to product healthfulness and targeted to bottom-quartile households can achieve the same or greater improvements at transparent costs: about \$84 million to match the effect of one nearby supermarket entry, \$830 million to match full supply equalization, and \$11 billion (≈15\% of the SNAP budget) to close the entire gap. The authors caution that healthy eating is not the sole social objective and government outlays are not a complete measure of social costs, so a full welfare framework is needed, but distributionally, targeted subsidies are more cost-effective at improving low-income households' healthy grocery consumption than supply-side interventions.

\begin{itemize}
\tightlist
\item
  Quote snippet:
\end{itemize}

\begin{quote}
an annual subsidy of \$84 million would increase healthy eating by the same amount as one additional supermarket entry
\end{quote}

\begin{itemize}
\tightlist
\item
  Citation anchors:

  \begin{itemize}
  \tightlist
  \item
    page=48, words=0-323, section=Conclusion, note=Equalizing supply reduces gap by no more than \textasciitilde10\%; means-tested subsidy could close gap at \textasciitilde15\% of SNAP
  \item
    page=46, words=0-349, section=VII.C, note=\$84M matches supermarket entry; \$830M matches full supply equalization; \$11B closes gap
  \item
    page=45, words=0-349, section=VII.C, Table V, note=Table V quantifies subsidy parameters and total payments to bottom-quartile households
  \item
    page=43, words=275-387, section=VII.C, note=Introduces using subsidies for healthy foods as alternative to supply-side policies
  \item
    page=2, words=0-349, section=Introduction, note=Counterfactual: equalizing products/prices cuts inequality \textasciitilde10\%; 90\% due to demand, challenging supply-side focus
  \end{itemize}
\end{itemize}

\subsubsection{I03: Are cost-benefit or incidence analyses provided?}\label{i03-are-cost-benefit-or-incidence-analyses-provided}

\begin{itemize}
\tightlist
\item
  Category: \texttt{I)\ Policy\ counterfactuals\ and\ welfare}
\item
  Confidence: \texttt{low\ (0.29904970035413087)}
\item
  Retrieval method: \texttt{local}
\item
  Evidence type: \texttt{text}
\item
  Data source: Allcott et al. (2020), The Quarterly Journal of Economics
\item
  Assumption flag: \texttt{False}
\item
  Answer:
\end{itemize}

No. The excerpt does not provide a full cost-benefit or incidence analysis; it offers cost-effectiveness comparisons (e.g., required subsidy amounts) and explicitly notes that a welfare/optimal policy analysis would be needed before advocating policies.

\begin{itemize}
\tightlist
\item
  Quote snippet:
\end{itemize}

\begin{quote}
Before advocating for or against such a subsidy, one would need to measure the relevant market failures and study optimal policy in a principled welfare maximization framework.
\end{quote}

\begin{itemize}
\tightlist
\item
  Citation anchors:

  \begin{itemize}
  \tightlist
  \item
    page=48, words=0-323, section=Main text, note=States that optimal policy/welfare analysis would be needed before advocating a subsidy, implying no full cost-benefit/incidence analysis is provided.
  \item
    page=46, words=0-349, section=Main text, note=Provides cost-effectiveness comparisons (e.g., \$84M, \$830M, \$11B; statements about being cost-effective), not a full cost-benefit or incidence analysis.
  \end{itemize}
\end{itemize}

\subsubsection{I04: What policy recommendations are stated or implied?}\label{i04-what-policy-recommendations-are-stated-or-implied}

\begin{itemize}
\tightlist
\item
  Category: \texttt{I)\ Policy\ counterfactuals\ and\ welfare}
\item
  Confidence: \texttt{low\ (0.3321727108993598)}
\item
  Retrieval method: \texttt{local}
\item
  Evidence type: \texttt{direct\ textual\ evidence}
\item
  Assumption flag: \texttt{True}
\item
  Assumption notes: Model-based subsidy results are suggestive and depend on assumptions; authors recommend randomized demonstration projects to assess effects and note economic/political considerations around SNAP modifications.
\item
  Answer:
\end{itemize}

Shift policy away from supply-side efforts (e.g., subsidizing supermarket entry or focusing on local access) toward demand-side, means-tested subsidies for healthy foods---potentially scaled by product healthfulness---and test such approaches (including SNAP-related designs) via randomized demonstration projects. Additionally, improving education and nutrition information is suggested as a way to reduce nutritional inequality.

\begin{itemize}
\tightlist
\item
  Quote snippet:
\end{itemize}

\begin{quote}
policy makers interested in reducing nutritional inequality might redirect efforts away from promoting access to healthy groceries and toward means-tested subsidies.
\end{quote}

\begin{itemize}
\tightlist
\item
  Citation anchors:

  \begin{itemize}
  \tightlist
  \item
    page=47, words=0-349, section=Main text, note=Recommends redirecting efforts toward means-tested subsidies; calls for demonstration projects and notes SNAP considerations.
  \item
    page=43, words=0-349, section=Main text, note=Education and nutrition knowledge explain part of inequality; suggests roles for improving education and nutrition information.
  \item
    page=43, words=275-387, section=Main text, note=Introduces alternative policy: subsidies for healthy foods, scaled by product healthfulness.
  \item
    page=46, words=0-349, section=Main text, note=Indicates means-tested subsidies are more cost-effective than supermarket entry subsidies at increasing healthy purchases.
  \end{itemize}
\end{itemize}

\subsubsection{J01: What missingness or attrition patterns are reported?}\label{j01-what-missingness-or-attrition-patterns-are-reported}

\begin{itemize}
\tightlist
\item
  Category: \texttt{J)\ Data\ quality\ and\ integrity}
\item
  Confidence: \texttt{low\ (0.3176327986938804)}
\item
  Retrieval method: \texttt{local}
\item
  Evidence type: \texttt{text}
\item
  Data source: Nielsen Homescan Panel (2004--2016)
\item
  Assumption flag: \texttt{False}
\item
  Answer:
\end{itemize}

Two patterns are reported: (1) 10.6\% of household--product group--year observations are dropped due to zero purchases; ``Baby food'' has the most missing, and the missingness shares across income groups are not correlated with product-group characteristics. (2) The Homescan data omit away-from-home food and mostly exclude nonpackaged/bulk groceries, so those items are missing from coverage.

\begin{itemize}
\tightlist
\item
  Quote snippet:
\end{itemize}

\begin{quote}
We drop 10.6\% of observations at the household-by--product group--by-year level because they have zero purchases. ``Baby food'' is the product group with the most missing observations.
\end{quote}

\begin{itemize}
\tightlist
\item
  Citation anchors:

  \begin{itemize}
  \tightlist
  \item
    page=35, words=275-428, section=VI.C. Method of Moments Estimation, note=Reports 10.6\% dropped observations, baby food most missing, and lack of correlation of missingness with characteristics across income groups.
  \item
    page=7, words=0-349, section=II.A. Nielsen Homescan and Retail Scanner Data, note=Describes missing coverage for away-from-home consumption and mostly nonpackaged/bulk groceries.
  \end{itemize}
\end{itemize}

\subsubsection{J02: How are outliers handled (winsorization, trimming, exclusions)?}\label{j02-how-are-outliers-handled-winsorization-trimming-exclusions}

\begin{itemize}
\tightlist
\item
  Category: \texttt{J)\ Data\ quality\ and\ integrity}
\item
  Confidence: \texttt{low\ (0.33104689525725783)}
\item
  Retrieval method: \texttt{local}
\item
  Evidence type: \texttt{textual}
\item
  Data source: Nielsen Homescan and RMS, as described in the provided context
\item
  Assumption flag: \texttt{False}
\item
  Answer:
\end{itemize}

Outlier handling via winsorization or trimming is not described in the provided context. The only stated handling is sample exclusions/restrictions: they exclude liquor stores and nonpackaged/bulk items from RMS/Homescan coverage, and for mover analyses they use balanced panels and a 50\% local shopping restriction. No explicit winsorizing or trimming of variables is mentioned.

\begin{itemize}
\tightlist
\item
  Quote snippet:
\end{itemize}

\begin{quote}
We exclude liquor stores... RMS does not include sales of bulk produce and other nonpackaged items.
\end{quote}

\begin{itemize}
\tightlist
\item
  Citation anchors:

  \begin{itemize}
  \tightlist
  \item
    page=8, words=275-393, section=Data description, note=Excludes liquor stores; RMS does not include bulk/nonpackaged items
  \item
    page=24, words=275-452, section=Mover design setup, note=Defines balanced panel around a move
  \item
    page=25, words=275-468, section=Mover design restriction, note=50\% local shopping restriction to exclude atypical movers
  \end{itemize}
\end{itemize}

\subsubsection{J03: Are there data audits or validation steps described?}\label{j03-are-there-data-audits-or-validation-steps-described}

\begin{itemize}
\tightlist
\item
  Category: \texttt{J)\ Data\ quality\ and\ integrity}
\item
  Confidence: \texttt{low\ (0.23155706377695187)}
\item
  Retrieval method: \texttt{local}
\item
  Evidence type: \texttt{textual\ evidence}
\item
  Data source: Nielsen TDLinx validated against retailer administrative records; Nielsen Homescan magnet subsample
\item
  Table/Figure: Online Appendix Figure A1 (mentioned)
\item
  Assumption flag: \texttt{False}
\item
  Answer:
\end{itemize}

Yes. They validate TDLinx store openings against retailer administrative records, exclude spurious `entrants' using M\&A transfer lists and same-block checks, assess packaged-goods representativeness with a magnet subsample, and run instrument validity tests.

\begin{itemize}
\tightlist
\item
  Quote snippet:
\end{itemize}

\begin{quote}
In validation checks, we found that TDLinx data closely match entry dates and loca- tions for 1,914 supermarkets ... from four retailers' administrative records.
\end{quote}

\begin{itemize}
\tightlist
\item
  Citation anchors:

  \begin{itemize}
  \tightlist
  \item
    page=10, words=0-349, section=II.B. Grocery Retail Establishments, note=Validation of TDLinx opening dates/locations against retailer administrative records
  \item
    page=10, words=275-424, section=II.B. Grocery Retail Establishments, note=Procedures to exclude spurious entrants (M\&A transfers; same-block prior-year stores)
  \item
    page=8, words=0-349, section=Data and representativeness, note=Magnet subsample used to check representativeness of focusing on packaged groceries
  \item
    page=34, words=0-349, section=Instrument validity tests, note=Two tests showing instrument unrelated to predictable tastes
  \end{itemize}
\end{itemize}

\subsubsection{J04: Is there evidence of reporting bias or selective sample inclusion?}\label{j04-is-there-evidence-of-reporting-bias-or-selective-sample-inclusion}

\begin{itemize}
\tightlist
\item
  Category: \texttt{J)\ Data\ quality\ and\ integrity}
\item
  Confidence: \texttt{low\ (0.295750837001939)}
\item
  Retrieval method: \texttt{local}
\item
  Evidence type: \texttt{Explicit\ methodological\ limitations\ and\ sample\ representativeness\ statements\ in\ the\ text}
\item
  Data source: Allcott et al. (2019), The Quarterly Journal of Economics: Food Deserts and Nutritional Inequality
\item
  Table/Figure: Figure V
\item
  Assumption flag: \texttt{False}
\item
  Answer:
\end{itemize}

Yes. The authors explicitly note reporting and selection issues: Homescan largely captures only packaged UPC items and excludes away-from-home food; some analyses use movers who are not nationally representative and are not weighted; and remaining endogeneity may bias results upward.

\begin{itemize}
\tightlist
\item
  Quote snippet:
\end{itemize}

\begin{quote}
Because the set of movers is not nationally representative, we again do not use the Homescan sample weights for this analysis.
\end{quote}

\begin{itemize}
\tightlist
\item
  Citation anchors:

  \begin{itemize}
  \tightlist
  \item
    page=7, words=120-260, section=II.A. Nielsen Homescan and Retail Scanner Data, note=Homescan excludes away-from-home purchases and mostly nonpackaged groceries; potential reporting limitation.
  \item
    page=13, words=275-405, note=Results are similar in a `magnet' subsample that includes bulk purchases, addressing concerns about missing nonpackaged items.
  \item
    page=21, words=275-320, note=Observations are not weighted for national representativeness.
  \item
    page=24, words=0-200, note=Movers are not nationally representative; any remaining endogeneity likely biases estimates toward finding place effects.
  \end{itemize}
\end{itemize}

\subsubsection{K01: What goodness-of-fit or diagnostic metrics are reported?}\label{k01-what-goodness-of-fit-or-diagnostic-metrics-are-reported}

\begin{itemize}
\tightlist
\item
  Category: \texttt{K)\ Model\ fit\ and\ diagnostics}
\item
  Confidence: \texttt{low\ (0.2633273068183583)}
\item
  Retrieval method: \texttt{local}
\item
  Evidence type: \texttt{Reported\ statistics\ in\ tables\ and\ text\ (correlations,\ SEs,\ CIs,\ significance,\ R2\ comment)}
\item
  Data source: Nielsen Homescan 2004--2016 (as analyzed in the article's tables/notes)
\item
  Table/Figure: Table III; Table IV
\item
  Answer:
\end{itemize}

They report correlation coefficients (ρ) for HEI components, robust clustered standard errors, significance stars (10\%, 5\%, 1\%), 95\% confidence interval upper bounds, and note that regression R2 changes only slightly with added controls.

\begin{itemize}
\tightlist
\item
  Quote snippet:
\end{itemize}

\begin{quote}
95\% confidence interval upper bound 0.100 0.097 0.217 0.220 ... Robust standard errors, clustered by household and local area (ZIP code or county), are in parentheses.
\end{quote}

\begin{itemize}
\tightlist
\item
  Citation anchors:

  \begin{itemize}
  \tightlist
  \item
    page=11, words=275-399, section=Section III.A text, note=Reports correlation coefficients (ρ) for dietary components and HEI validity.
  \item
    page=27, words=0-349, section=Table III notes and surrounding text, note=Shows 95\% CI upper bounds, clustered robust SEs, significance stars; mentions R2 changes slightly.
  \item
    page=21, words=275-503, section=Table II notes, note=States robust clustered SEs and significance star levels.
  \end{itemize}
\end{itemize}

\subsubsection{K02: Are functional form choices tested (logs, levels, nonlinearities)?}\label{k02-are-functional-form-choices-tested-logs-levels-nonlinearities}

\begin{itemize}
\tightlist
\item
  Category: \texttt{K)\ Model\ fit\ and\ diagnostics}
\item
  Confidence: \texttt{low\ (0.2986677962411813)}
\item
  Retrieval method: \texttt{local}
\item
  Evidence type: \texttt{textual\ (model\ description)}
\item
  Data source: Article text (pp. 30, Sections VI.A--VI.B)
\item
  Answer:
\end{itemize}

Not in the provided context. The paper specifies a log-linear demand equation derived from Cobb--Douglas/CES preferences but does not report tests comparing logs vs levels or other nonlinear forms.

\begin{itemize}
\tightlist
\item
  Quote snippet:
\end{itemize}

\begin{quote}
ln Yijt = − ln p̃ijt − β̃c ãijct − ξ + δj + φm + φt + εijt
\end{quote}

\begin{itemize}
\tightlist
\item
  Citation anchors:

  \begin{itemize}
  \tightlist
  \item
    page=30, words=0-349, section=VI.A Empirical Model, note=Presents log-linear demand equation (7); no mention of functional-form tests.
  \item
    page=30, words=275-393, section=VI.A/VI.B, note=Discusses implications of Cobb--Douglas form; does not indicate testing alternative forms.
  \end{itemize}
\end{itemize}

\subsubsection{K03: Are residual checks or specification tests reported?}\label{k03-are-residual-checks-or-specification-tests-reported}

\begin{itemize}
\tightlist
\item
  Category: \texttt{K)\ Model\ fit\ and\ diagnostics}
\item
  Confidence: \texttt{low\ (0.2418967000364113)}
\item
  Retrieval method: \texttt{local}
\item
  Evidence type: \texttt{textual\ evidence}
\item
  Table/Figure: Online Appendix Table A9; Online Appendix Table A11
\item
  Assumption flag: \texttt{False}
\item
  Answer:
\end{itemize}

Yes. The paper reports multiple specification and robustness checks, including tests of instrument validity, checks for pre-trends around moves, and robustness to adding alternative controls (e.g., county-by-quarter fixed effects, convenience-store entry).

\begin{itemize}
\tightlist
\item
  Quote snippet:
\end{itemize}

\begin{quote}
Adding county-by-quarter fixed effects or a control for convenience store entry does not qualitatively change the results.
\end{quote}

\begin{itemize}
\tightlist
\item
  Citation anchors:

  \begin{itemize}
  \tightlist
  \item
    page=21, words=275-503, section=Entry effects robustness, note=Adding county-by-quarter fixed effects or a control for convenience store entry does not change results.
  \item
    page=25, words=275-468, section=Move analysis pre-trend check, note=No statistically significant evidence of premove trends in Health Index after moves.
  \item
    page=34, words=0-349, section=Instrument validity tests, note=Two tests show the instrument is not associated with predicted purchases and not systematically different across income-related health groups.
  \item
    page=38, words=275-377, section=Additional robustness, note=Estimates are very similar with alternative controls; supports exclusion restriction robustness.
  \end{itemize}
\end{itemize}

\subsubsection{K04: How sensitive are results to alternative specifications or estimators?}\label{k04-how-sensitive-are-results-to-alternative-specifications-or-estimators}

\begin{itemize}
\tightlist
\item
  Category: \texttt{K)\ Model\ fit\ and\ diagnostics}
\item
  Confidence: \texttt{medium\ (0.4193826056196361)}
\item
  Retrieval method: \texttt{local}
\item
  Evidence type: \texttt{text}
\item
  Data source: Provided context excerpts (pages 21, 27, 38)
\item
  Assumption flag: \texttt{False}
\item
  Answer:
\end{itemize}

Results appear robust. Including demographic controls has very little impact; adding county-by-quarter fixed effects or a control for convenience-store entry does not qualitatively change results; re-estimating with extensive census-region-by--product-group (and urban/rural, high-income) fixed effects yields very similar estimates. Restricting to supercenter entry increases expenditure-share effects, but Health Index effects remain economically small and mostly insignificant. Adding shelf-life and convenience characteristics lowers WTP magnitudes but preserves the qualitative preference patterns.

\begin{itemize}
\tightlist
\item
  Quote snippet:
\end{itemize}

\begin{quote}
Adding county-by-quarter fixed effects or a control for convenience store entry does not qualitatively change the results.
\end{quote}

\begin{itemize}
\tightlist
\item
  Citation anchors:

  \begin{itemize}
  \tightlist
  \item
    page=27, words=275-482, section=Robustness (demographic controls), note=Including household demographics has very little impact on results.
  \item
    page=21, words=275-503, section=Robustness (fixed effects, alternative entry definition), note=County-by-quarter FE and convenience-store entry control do not change results; supercenter-only effects larger for shares but Health Index effects small/insignificant.
  \item
    page=38, words=275-377, section=Robustness (additional fixed effects), note=Re-estimating with added region-by--product-group and interactions yields very similar estimates.
  \item
    page=38, words=0-349, section=Robustness (adding characteristics), note=Including shelf life and convenience lowers WTP levels but leaves preference patterns similar.
  \end{itemize}
\end{itemize}

\subsubsection{P01: ResponseTextConfig(format=ResponseFormatText(type=\textquotesingle text\textquotesingle), verbosity=\textquotesingle medium\textquotesingle)}\label{p01-responsetextconfigformatresponseformattexttypetext-verbositymedium}

\begin{itemize}
\item
  Category: \texttt{P)\ Previous\ questions}
\item
  Confidence: \texttt{n/a}
\item
  Retrieval method: \texttt{n/a}
\item
  Evidence type: \texttt{n/a}
\item
  Answer:
\item
  Data sources and coverage

  \begin{itemize}
  \tightlist
  \item
    Homescan records packaged groceries; in 2004--2006, an 8,000-household ``magnet'' subsample also recorded nonpackaged items. About 60\% of magnet households' produce calories are from packaged goods, with no statistical income differences, suggesting packaged focus is reasonably representative for produce and that produce is a small share of groceries overall (page 8, words 0--349).
  \item
    Households report income (16 bins), composition, demographics; researchers compute daily calorie needs using Dietary Guidelines and scale children to adult equivalents. PanelViews add-on (2017) provides health importance, nutrition quiz, BMI, diabetes; results are nationally weighted (page 8, words 0--349).
  \item
    RMS: weekly UPC-level prices/volumes at \textasciitilde42,000 stores from 160 chains (≈35,000/year observed) for 2006--2016; excludes liquor and bulk produce. Channel coverage: 53\% grocery, 32\% mass merchandiser, 55\% drug, 2\% convenience. Prices/incomes deflated to 2010 dollars using CPI-U (page 8, words 275--393).
  \end{itemize}
\item
  Identification and market structure facts

  \begin{itemize}
  \tightlist
  \item
    Substantial geographic variation in the presence of the five largest retailers; these chains set different relative prices across product groups, yielding geographic price instruments. Example: produce predicted cheap on West Coast and expensive on East Coast, likely due to California production; market fixed effects address endogeneity from demand elasticity differences across markets (page 32, words 275--431).
  \item
    Within-household regressions: adding demographic controls changes results little and increases R\^{}2 only slightly, implying possible unobserved within-household changes (Oster 2019). A migration-style thought experiment considers moving a household from low- to high-income retail environments (partial equilibrium) (page 27, words 275--482).
  \item
    Validation: replicates immediate brand choice migration effect for Coke vs. Pepsi with τ̂ ≈ 0.16; a 10 pp higher county Coke share raises a household's Coke share by 1.6 pp (page 27, words 275--482).
  \end{itemize}
\item
  Preference estimates (Table IV) and WTP for health

  \begin{itemize}
  \tightlist
  \item
    GMM estimates of nutrient preference parameters by income quartile show:

    \begin{itemize}
    \tightlist
    \item
      Higher-income quartiles have more negative WTP for sodium and added sugar, and more positive WTP for whole fruit and whole grains; patterns are monotonic across quartiles (page 36, words 0--349).
    \item
      ``WTP for Health Index'' rises with income: Q1 ≈ 0.429, Q2 ≈ 0.631, Q3 ≈ 0.820, Q4 ≈ 1.141 (SEs shown in table) (page 36, words 0--349).
    \end{itemize}
  \item
    Units: sodium (g), whole/other fruit and dairy (cups), whole/refined grains and proteins (oz), added sugar (tsp), solid fats (calories). WTP for Health Index defined as \(\sum_c \hat{\beta}_c s_c r_c\) (page 36, words 0--349).
  \end{itemize}
\item
  Decomposition: supply vs. demand in nutrition--income gaps

  \begin{itemize}
  \tightlist
  \item
    Across alternative specifications (including census-region-by--product group fixed effects and interactions with urban/rural and above-median income), supply explains 7\%--12\% and demand explains 88\%--93\% of the nutrition--income relationship; summary: about 90\% demand, 10\% supply (page 41, words 275--411; page 38, words 275--377).
  \item
    The unobserved characteristic \(\hat{\xi}\) is attributed to demand because it primarily determines demand elasticity with respect to price (page 41, words 275--411).
  \end{itemize}
\item
  Demographic controls used in outcomes regressions

  \begin{itemize}
  \tightlist
  \item
    Controls include household size, age indicators, log years of education, race, marital status, employment, weekly work hours, plus add-on variables such as the self-reported importance of staying healthy (page 42, words 275--356; page 8, words 0--349).
  \end{itemize}
\end{itemize}

\end{document}
