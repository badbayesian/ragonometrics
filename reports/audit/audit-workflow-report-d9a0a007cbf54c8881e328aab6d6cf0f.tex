% Options for packages loaded elsewhere
\PassOptionsToPackage{unicode}{hyperref}
\PassOptionsToPackage{hyphens}{url}
\documentclass[
]{article}
\usepackage{xcolor}
\usepackage{amsmath,amssymb}
\setcounter{secnumdepth}{-\maxdimen} % remove section numbering
\usepackage{iftex}
\ifPDFTeX
  \usepackage[T1]{fontenc}
  \usepackage[utf8]{inputenc}
  \usepackage{textcomp} % provide euro and other symbols
\else % if luatex or xetex
  \usepackage{unicode-math} % this also loads fontspec
  \defaultfontfeatures{Scale=MatchLowercase}
  \defaultfontfeatures[\rmfamily]{Ligatures=TeX,Scale=1}
\fi
\usepackage{lmodern}
\ifPDFTeX\else
  % xetex/luatex font selection
\fi
% Use upquote if available, for straight quotes in verbatim environments
\IfFileExists{upquote.sty}{\usepackage{upquote}}{}
\IfFileExists{microtype.sty}{% use microtype if available
  \usepackage[]{microtype}
  \UseMicrotypeSet[protrusion]{basicmath} % disable protrusion for tt fonts
}{}
\makeatletter
\@ifundefined{KOMAClassName}{% if non-KOMA class
  \IfFileExists{parskip.sty}{%
    \usepackage{parskip}
  }{% else
    \setlength{\parindent}{0pt}
    \setlength{\parskip}{6pt plus 2pt minus 1pt}}
}{% if KOMA class
  \KOMAoptions{parskip=half}}
\makeatother
\setlength{\emergencystretch}{3em} % prevent overfull lines
\providecommand{\tightlist}{%
  \setlength{\itemsep}{0pt}\setlength{\parskip}{0pt}}
\usepackage{bookmark}
\IfFileExists{xurl.sty}{\usepackage{xurl}}{} % add URL line breaks if available
\urlstyle{same}
\hypersetup{
  hidelinks,
  pdfcreator={LaTeX via pandoc}}

\author{}
\date{}

\begin{document}

\section{\texorpdfstring{Audit Report: Workflow \texttt{d9a0a007cbf54c8881e328aab6d6cf0f}}{Audit Report: Workflow d9a0a007cbf54c8881e328aab6d6cf0f}}\label{audit-report-workflow-d9a0a007cbf54c8881e328aab6d6cf0f}

\subsection{Overview}\label{overview}

\begin{itemize}
\tightlist
\item
  Source JSON: \texttt{reports\textbackslash{}workflow-report-d9a0a007cbf54c8881e328aab6d6cf0f.json}
\item
  Run ID: \texttt{d9a0a007cbf54c8881e328aab6d6cf0f}
\item
  Papers input: \texttt{papers\textbackslash{}Calorie\ Posting\ in\ Chain\ Restaurants\ -\ Bollinger\ et\ al.\ (2011).pdf}
\item
  Started at: \texttt{2026-02-16T04:22:51.161738+00:00}
\item
  Finished at: \texttt{2026-02-16T04:31:33.697368+00:00}
\item
  Duration: \texttt{0:08:42.535630}
\end{itemize}

\subsection{Effective Configuration}\label{effective-configuration}

\begin{itemize}
\tightlist
\item
  Chat model: \texttt{gpt-5-nano}
\item
  Embedding model: \texttt{text-embedding-3-large}
\item
  Top K: \texttt{10}
\item
  Chunk words / overlap: \texttt{350} / \texttt{75}
\item
  Batch size: \texttt{64}
\item
  Database URL configured: \texttt{True}
\end{itemize}

\subsection{Step Outcomes}\label{step-outcomes}

\begin{itemize}
\tightlist
\item
  \texttt{prep}: \texttt{completed}
\item
  \texttt{ingest}: \texttt{num\_pdfs=1,\ num\_papers=1}
\item
  \texttt{enrich}: \texttt{openalex=0,\ citec=0}
\item
  \texttt{econ\_data}: \texttt{fetched}
\item
  \texttt{agentic}: \texttt{completed}
\item
  \texttt{index}: \texttt{indexed}
\item
  \texttt{report\_store}: \texttt{pending}
\end{itemize}

\subsection{Agentic Summary}\label{agentic-summary}

\begin{itemize}
\tightlist
\item
  Status: \texttt{completed}
\item
  Main question: What are the paper\textquotesingle s main contribution, identification strategy, key results, and limitations?
\item
  Report question set: \texttt{both}
\item
  Structured questions generated: \texttt{84}
\item
  Confidence mean/median: \texttt{0.27970699443533575} / \texttt{0.28419734314792333}
\item
  Confidence labels: low=74, medium=9, high=0
\end{itemize}

\subsubsection{Final Answer}\label{final-answer}

\begin{itemize}
\item
  Main contributions

  \begin{itemize}
  \tightlist
  \item
    Empirically assesses NYC's mandatory calorie posting on chain restaurants using two datasets (store-day transaction data and cardholder data) in NYC vs control cities (Boston/Philadelphia); identification via a post-policy indicator and fixed effects/controls. (page 13, 275-528)
  \item
    Decomposes effects into extensive vs. intensive margins, showing calories drop primarily through reduced food purchases (extensive margin) and, to a lesser extent, lower calories per item. (page 18, 0-349)
  \item
    Provides distributional and heterogeneity analyses, showing larger reductions for higher-calorie consumers and higher quantiles; differential effects by income, education, etc. (pages 21-22, 0-349)
  \item
    Includes external validation via a Seattle placebo, suggesting beverage-focused effects and limited noise from the policy elsewhere. (page 15, 0-254)
  \end{itemize}
\item
  Identification strategy

  \begin{itemize}
  \tightlist
  \item
    Difference-in-differences design comparing NYC to control cities with a POST indicator (1 after April 1, 2008), using fixed effects and additional controls; evidence drawn from store-day transaction data and cardholder data. (page 13, 275-528)
  \end{itemize}
\item
  Key results

  \begin{itemize}
  \tightlist
  \item
    Overall effect: calories per store-day fall by about 4.6; calories per transaction fall less, consistent with more total transactions after posting. (page 13, 275-528)
  \item
    Item- and per-item effects (Table 4): beverages per transaction up modestly (0.005***); food items per transaction down (−0.029***); calories per beverage and per food item both decline (log beverage calories per beverage: −0.008***; log calories per food item: −0.039***). (page 18, 0-349)
  \item
    Mechanism: the calorie reduction is driven by both extensive margin (fewer food items purchased) and intensive margin (lower calories per item), with food items contributing notably. (page 18, 0-349)
  \item
    Main channel: extensive margin dominates---about three-quarters of the total calorie reduction comes from not purchasing food items. (page 18, 0-349)
  \item
    Distributional and heterogeneity patterns: larger reductions at higher quantiles; percent change roughly 5--6\% from 75th to 99th percentile; effects differ by income, education, age, gender, and customer type (Posting × interactions). (pages 21-22, 0-349)
  \item
    Summary takeaway: mandatory posting reduces calories per transaction mainly via reduced food purchases (extensive margin) and, to a lesser extent, lower calories per item; effects stronger for higher-calorie consumers and at higher consumption quantiles. (page 18, 0-349; pages 21-22, 0-349)
  \end{itemize}
\item
  Limitations

  \begin{itemize}
  \tightlist
  \item
    Limitations are not explicitly enumerated in the provided excerpts; robustness checks include a Seattle placebo, which supports external validity to some extent but does not fully generalize beyond the studied contexts. (Seattle placebo: page 15, 0-254)
  \end{itemize}
\end{itemize}

\subsubsection{Sub-Answers}\label{sub-answers}

\paragraph{Sub-answer 1}\label{sub-answer-1}

\begin{itemize}
\item
  Question: ResponseTextConfig(format=ResponseFormatText(type=\textquotesingle text\textquotesingle), verbosity=\textquotesingle medium\textquotesingle)
\item
  Question tokens estimate: \texttt{19}
\item
  Answer:
\item
  Objective and data

  \begin{itemize}
  \tightlist
  \item
    Analyzes the impact of NYC's mandatory calorie posting on chain restaurants using two datasets (store-day transaction data and cardholder data) in NYC vs control cities (Boston/Philadelphia). POST = 1 after April 1, 2008; results rely on NYC vs controls and fixed effects/controls. (page 13, 275-528)
  \end{itemize}
\item
  Overall impact on calories

  \begin{itemize}
  \tightlist
  \item
    Calorie posting reduces average calories per store-day by about 4.6 calories, with the reduction in calories per transaction smaller, consistent with an increase in total transactions after posting. (page 13, 275-528)
  \end{itemize}
\item
  Seattle placebo comparison (external validation)

  \begin{itemize}
  \tightlist
  \item
    In Seattle, beverage calories per transaction fell by 4.6 calories, while food calories per transaction rose by 0.8 calories, suggesting beverage-focused effects and little noise from the posting policy elsewhere. (page 15, 0-254)
  \end{itemize}
\item
  Item- and per-item effects

  \begin{itemize}
  \tightlist
  \item
    Table 4 (transaction data) shows: number of beverages per transaction increases modestly (0.005***); number of food items per transaction falls (−0.029***); calories per beverage and per food item also decline (log beverage calories per beverage: −0.008***; log food calories per food item: −0.039***). (page 18, 0-349)
  \item
    Interpretation: the reduction in calories is driven both by fewer items purchased (extensive margin with food items) and by lower calories per item (intensive margin). Food items contribute notably to the effect. (page 18, 0-349)
  \end{itemize}
\item
  Main channel: extensive margin dominates

  \begin{itemize}
  \tightlist
  \item
    About three quarters of the total calorie reduction is attributed to people opting not to buy food items (extensive margin of food demand). (page 18, 0-349)
  \end{itemize}
\item
  Distributional and heterogeneity patterns

  \begin{itemize}
  \tightlist
  \item
    Quantile evidence: calorie reductions are larger for higher quantiles; the percent change is roughly stable around 5--6 percent from 75th to 99th percentile, though absolute decreases are larger at higher quantiles. (page 22, 0-349)
  \item
    Heterogeneity: effects differ by characteristics (e.g., income, education, age, gender, customer type); e.g., higher-income and more-educated groups show additional differential effects (Posting × median income, Posting × percent with college degree, etc.). (page 21, 0-349)
  \end{itemize}
\item
  Summary takeaway

  \begin{itemize}
  \tightlist
  \item
    Mandatory calorie posting reduces calories per transaction primarily through reduced food purchases (extensive margin) and, to a lesser extent, lower calories per item; effects are larger for higher-calorie consumers and more pronounced at higher consumption quantiles. (page 18, 0-349; page 21-22, 0-349)
  \end{itemize}
\end{itemize}

Sources used:

\begin{itemize}
\tightlist
\item
  NYC calorie posting effects and overall magnitude: page 13, words 275-528
\item
  Seattle placebo example: page 15, words 0-254
\item
  Table 4 item-level and margin findings: page 18, words 0-349
\item
  Extensive-margin interpretation: page 18, words 0-349
\item
  Distributional/heterogeneity results: pages 21-22, words 0-349
\end{itemize}

\subsection{Structured Q\&A Appendix}\label{structured-qa-appendix}

This section mirrors \texttt{agentic.report\_questions} for audit traceability.

\subsubsection{A01: What is the main research question of the paper?}\label{a01-what-is-the-main-research-question-of-the-paper}

\begin{itemize}
\tightlist
\item
  Category: \texttt{A)\ Research\ question\ /\ contribution}
\item
  Confidence: \texttt{low\ (0.3160348019347735)}
\item
  Retrieval method: \texttt{local}
\item
  Evidence type: \texttt{explicit\ research\ question\ stated\ in\ the\ introduction}
\item
  Data source: Transaction data and cardholder data from Starbucks in NYC, Boston, and Philadelphia
\item
  Answer:
\end{itemize}

Whether mandatory calorie posting affects consumers' behavior, specifically calories per transaction, at Starbucks.

\begin{itemize}
\tightlist
\item
  Quote snippet:
\end{itemize}

\begin{quote}
Ultimately, whether calorie posting affects people's behavior is an empirical question.
\end{quote}

\begin{itemize}
\tightlist
\item
  Citation anchors:

  \begin{itemize}
  \tightlist
  \item
    page=3, words=275-296, section=Introduction, note=Location of the explicit empirical question about calorie posting effects.
  \end{itemize}
\end{itemize}

\subsubsection{A02: What is the paper\textquotesingle s primary contribution relative to the existing literature?}\label{a02-what-is-the-papers-primary-contribution-relative-to-the-existing-literature}

\begin{itemize}
\tightlist
\item
  Category: \texttt{A)\ Research\ question\ /\ contribution}
\item
  Confidence: \texttt{low\ (0.29760137826352995)}
\item
  Retrieval method: \texttt{local}
\item
  Evidence type: \texttt{textual\_evidence}
\item
  Data source: Starbucks transaction data; cardholder data; in-store consumer surveys
\item
  Answer:
\end{itemize}

The paper\textquotesingle s primary contribution is providing a large‑scale, long‑horizon analysis of calorie posting, using a dataset far larger and broader than prior work (over 100 million Starbucks transactions across NYC, Boston, and Philadelphia, plus 1.5 million cardholder observations and surveys), enabling examination of the time path, product substitution, heterogeneity in responses, and profits, and clarifying learning and salience as mechanisms.

\begin{itemize}
\tightlist
\item
  Quote snippet:
\end{itemize}

\begin{quote}
the dataset we study is much larger and broader---the universe of over 100 million transactions at Starbucks in Boston, NYC, and Philadelphia over a 14-month period. We also analyze individual-level data (1.5 million transactions of anonymous customers over time), as well as a survey that focuses on testing consumers' knowledge).
\end{quote}

\begin{itemize}
\tightlist
\item
  Citation anchors:

  \begin{itemize}
  \tightlist
  \item
    page=8, words=0-349, section=Comparison with prior research and data contributions, note=States dataset is much larger/broader and includes multiple data sources, enabling long-run analysis and examination of substitution, heterogeneity, and profits.
  \end{itemize}
\end{itemize}

\subsubsection{A03: What is the central hypothesis being tested?}\label{a03-what-is-the-central-hypothesis-being-tested}

\begin{itemize}
\tightlist
\item
  Category: \texttt{A)\ Research\ question\ /\ contribution}
\item
  Confidence: \texttt{low\ (0.2901430171019807)}
\item
  Retrieval method: \texttt{local}
\item
  Evidence type: \texttt{empirical\ analysis\ of\ transaction/cardholder\ data}
\item
  Data source: Starbucks transaction data and cardholder data
\item
  Answer:
\end{itemize}

The central hypothesis is that mandatory calorie posting at chain restaurants reduces calories per transaction (i.e., lowers calories purchased), and that this effect may operate through learning about calories or through increased salience of calorie information.

\begin{itemize}
\tightlist
\item
  Quote snippet:
\end{itemize}

\begin{quote}
Ultimately, whether calorie posting affects people's behavior is an empirical question.
\end{quote}

\begin{itemize}
\tightlist
\item
  Citation anchors:

  \begin{itemize}
  \tightlist
  \item
    page=3, words=275-564, section=Introduction -- empirical question and mechanisms, note=States that the effect of calorie posting is an empirical question and discusses learning vs salience.
  \item
    page=29, words=0-349, section=Results -- commuters and NYC effects, note=Presents results showing changes in calories per transaction for commuters and non-commuters.
  \item
    page=30, words=0-349, section=Results -- learning vs salience conclusions, note=Discusses interpretation: both learning and salience contribute.
  \end{itemize}
\end{itemize}

\subsubsection{A04: What are the main outcomes of interest (dependent variables)?}\label{a04-what-are-the-main-outcomes-of-interest-dependent-variables}

\begin{itemize}
\tightlist
\item
  Category: \texttt{A)\ Research\ question\ /\ contribution}
\item
  Confidence: \texttt{low\ (0.3047339608395138)}
\item
  Retrieval method: \texttt{local}
\item
  Evidence type: \texttt{definition\ of\ dependent\ variables\ from\ results\ section}
\item
  Data source: transaction data and cardholder data
\item
  Answer:
\end{itemize}

The main outcomes of interest are calories per transaction measures: beverage calories per transaction, food calories per transaction, and total calories per transaction (log forms).

\begin{itemize}
\tightlist
\item
  Quote snippet:
\end{itemize}

\begin{quote}
the dependent variable is log(beverage calories per transaction). Based on the transaction data we estimate that calorie posting caused a trivial decrease in beverage calories per transaction.
\end{quote}

\begin{itemize}
\tightlist
\item
  Citation anchors:

  \begin{itemize}
  \tightlist
  \item
    page=12, words=0-349, section=Table 3 -- Regression results, note=Defines dependent variables: log(beverage calories per transaction), log(food calories per transaction), and log(beverage + food calories) per transaction.
  \end{itemize}
\end{itemize}

\subsubsection{A05: What are the key treatment/exposure variables (independent variables)?}\label{a05-what-are-the-key-treatmentexposure-variables-independent-variables}

\begin{itemize}
\tightlist
\item
  Category: \texttt{A)\ Research\ question\ /\ contribution}
\item
  Confidence: \texttt{low\ (0.24409435786648528)}
\item
  Retrieval method: \texttt{local}
\item
  Evidence type: \texttt{regression\ specification\ /\ treatment\ variable\ definition}
\item
  Data source: Transaction data and cardholder data (Starbucks)
\item
  Answer:
\end{itemize}

POST, the calorie-posting dummy variable equal to one for NYC stores after April 1, 2008, is the key treatment/exposure variable.

\begin{itemize}
\tightlist
\item
  Quote snippet:
\end{itemize}

\begin{quote}
y\_sct = x\_sct β + γ POS Tc t + ε\_sct, where y\_sct is a measure of calories per transaction at store s in city c on day t, POS Tc t is a dummy equal to one if calories were posted (i.e., NYC stores after April 1, 2008)
\end{quote}

\begin{itemize}
\tightlist
\item
  Citation anchors:

  \begin{itemize}
  \tightlist
  \item
    page=11, words=0-349, section=Regression specification / Methods, note=Defines the POST variable as a dummy equal to one if calories were posted (NYC stores after April 1, 2008).
  \end{itemize}
\end{itemize}

\subsubsection{A06: What setting/context does the paper study (country, market, period)?}\label{a06-what-settingcontext-does-the-paper-study-country-market-period}

\begin{itemize}
\tightlist
\item
  Category: \texttt{A)\ Research\ question\ /\ contribution}
\item
  Confidence: \texttt{medium\ (0.3533244294259882)}
\item
  Retrieval method: \texttt{local}
\item
  Evidence type: \texttt{textual}
\item
  Data source: Starbucks transaction data and cardholder data from NYC, Boston, and Philadelphia (USA).
\item
  Table/Figure: Table 1---Summary Statistics for Transaction Data and Cardholder Data
\item
  Answer:
\end{itemize}

United States (New York City as the primary setting) in the chain-restaurant sector (Starbucks), examining the NYC calorie-posting law during 2008--2009 (with NYC stores compared to Boston and Philadelphia).

\begin{itemize}
\tightlist
\item
  Quote snippet:
\end{itemize}

\begin{quote}
Our transaction data cover all 222 Starbucks locations in NYC, and all 94 Starbucks locations in Boston and Philadelphia.
\end{quote}

\begin{itemize}
\tightlist
\item
  Citation anchors:

  \begin{itemize}
  \tightlist
  \item
    page=5, words=0-40, section=A. Data Summary, note=Describes NYC/Boston/Philadelphia coverage and the 2008--2009 period
  \end{itemize}
\end{itemize}

\subsubsection{A07: What is the main mechanism proposed by the authors?}\label{a07-what-is-the-main-mechanism-proposed-by-the-authors}

\begin{itemize}
\tightlist
\item
  Category: \texttt{A)\ Research\ question\ /\ contribution}
\item
  Confidence: \texttt{low\ (0.28587559983424177)}
\item
  Retrieval method: \texttt{local}
\item
  Evidence type: \texttt{textual\ claim\ from\ article}
\item
  Data source: Starbucks transaction and cardholder data
\item
  Answer:
\end{itemize}

A combination of learning and salience effects.

\begin{itemize}
\tightlist
\item
  Quote snippet:
\end{itemize}

\begin{quote}
a combination of learning and salience effects.
\end{quote}

\begin{itemize}
\tightlist
\item
  Citation anchors:

  \begin{itemize}
  \tightlist
  \item
    page=page 25, words=0-349, section=A. Why is There an Effect?, note=Authors state that behavior may be driven by a combination of learning and salience effects.
  \end{itemize}
\end{itemize}

\subsubsection{A08: What alternative mechanisms are discussed?}\label{a08-what-alternative-mechanisms-are-discussed}

\begin{itemize}
\tightlist
\item
  Category: \texttt{A)\ Research\ question\ /\ contribution}
\item
  Confidence: \texttt{low\ (0.22576387223949634)}
\item
  Retrieval method: \texttt{local}
\item
  Evidence type: \texttt{textual}
\item
  Data source: Bollinger et al., Calorie Posting in Chain Restaurants, American Economic Journal: Economic Policy
\item
  Answer:
\end{itemize}

The document discusses two alternative mechanisms: a learning effect and a salience effect (often in combination).

\begin{itemize}
\tightlist
\item
  Quote snippet:
\end{itemize}

\begin{quote}
One reason why calorie posting may affect consumer choice is a learning effect. Another possible explanation for the observed reduction in calories per transaction is a salience effect.
\end{quote}

\begin{itemize}
\tightlist
\item
  Citation anchors:

  \begin{itemize}
  \tightlist
  \item
    page=25, words=One-effects., section=A. Why is There an Effect?, note=Mentions learning and salience as explanations; notes behavior may be driven by a combination.
  \item
    page=25, words=Another-salience effects., section=A. Why is There an Effect?, note=Explicit mention of salience effect as an alternative mechanism.
  \end{itemize}
\end{itemize}

\subsubsection{A09: What are the main policy implications claimed by the paper?}\label{a09-what-are-the-main-policy-implications-claimed-by-the-paper}

\begin{itemize}
\tightlist
\item
  Category: \texttt{A)\ Research\ question\ /\ contribution}
\item
  Confidence: \texttt{medium\ (0.3767458184348189)}
\item
  Retrieval method: \texttt{local}
\item
  Evidence type: \texttt{Discussion\ and\ Conclusion\ (policy\ implications)\ from\ the\ paper}
\item
  Data source: Starbucks NYC transaction data, cardholder data, and in-store surveys
\item
  Answer:
\end{itemize}

Mandatory calorie posting across chain restaurants reduces calories per transaction (mainly by lowering food calories) with little effect on beverage calories or visit frequency; while the direct obesity benefits may be small, the costs are very low and the policy could yield long-run benefits such as menu innovation toward lower-calorie options and public education, making it potentially worthwhile, especially since both learning and salience appear to drive effects.

\begin{itemize}
\tightlist
\item
  Quote snippet:
\end{itemize}

\begin{quote}
mandatory calorie posting caused food calories per transaction to fall by 14 percent, on average, but had a negligible impact on beverage calories per transaction.
\end{quote}

\begin{itemize}
\tightlist
\item
  Citation anchors:

  \begin{itemize}
  \tightlist
  \item
    page=24, words=275-564, section=IV. Discussion, note=Contains the main policy implications and discussion of learning/salience, costs, and potential benefits.
  \item
    page=34, words=0-349, section=IV. Discussion, note=Concludes with the summary of findings and policy relevance, including low costs and potential long-run effects.
  \end{itemize}
\end{itemize}

\subsubsection{A10: What is the welfare interpretation (if any) of the results?}\label{a10-what-is-the-welfare-interpretation-if-any-of-the-results}

\begin{itemize}
\tightlist
\item
  Category: \texttt{A)\ Research\ question\ /\ contribution}
\item
  Confidence: \texttt{low\ (0.31027249337452484)}
\item
  Retrieval method: \texttt{local}
\item
  Evidence type: \texttt{policy\ implications\ /\ welfare\ interpretation\ based\ on\ results}
\item
  Data source: Starbucks calorie posting study; transaction and cardholder data in NYC vs control cities
\item
  Table/Figure: Table 9
\item
  Assumption flag: \texttt{True}
\item
  Assumption notes: Assumes reductions in calories translate into health/obesity benefits; obesity outcomes not measured in the study
\item
  Answer:
\end{itemize}

The welfare interpretation is that calorie posting may yield a welfare gain through informing consumers and reducing calorie intake (especially from high-calorie foods) with little impact on visit frequency or profits, suggesting health-oriented benefits with minimal consumer or producer distortion. However, the paper emphasizes that obesity outcomes are not measured, so the welfare impact remains uncertain and contingent on whether reduced calories translate into meaningful health gains.

\begin{itemize}
\tightlist
\item
  Quote snippet:
\end{itemize}

\begin{quote}
IV. Discussion To summarize briefly, the analyses above show that mandatory calorie posting caused food calories per transaction to fall by 14 percent, on average, but had a negligible impact on beverage calories per transaction.
\end{quote}

\begin{itemize}
\tightlist
\item
  Citation anchors:

  \begin{itemize}
  \tightlist
  \item
    page=24, words=275-564, section=IV. Discussion, note=Summary of welfare-relevant results: calories per transaction fell; beverage calories largely unaffected; discussion of potential welfare implications and need for obesity data
  \end{itemize}
\end{itemize}

\subsubsection{A11: What are the main limitations acknowledged by the authors?}\label{a11-what-are-the-main-limitations-acknowledged-by-the-authors}

\begin{itemize}
\tightlist
\item
  Category: \texttt{A)\ Research\ question\ /\ contribution}
\item
  Confidence: \texttt{low\ (0.27592370260423454)}
\item
  Retrieval method: \texttt{local}
\item
  Evidence type: \texttt{textual\ limitation}
\item
  Answer:
\end{itemize}

The main limitations are (1) obesity effects are not directly measured because BMI data are not yet available, so they cannot link calorie posting to obesity outcomes, and (2) the analysis uses data from only one chain (Starbucks), making it unclear whether the findings generalize to other chains; plus the possibility that some effects could be offset by changes in eating at home.

\begin{itemize}
\tightlist
\item
  Quote snippet:
\end{itemize}

\begin{quote}
There are two important limitations to this research. First, we do not directly measure the effect of calorie posting on obesity itself. A second limitation is that we have data for only one chain (Starbucks).
\end{quote}

\begin{itemize}
\tightlist
\item
  Citation anchors:

  \begin{itemize}
  \tightlist
  \item
    page=3, words=275-564, section=limitations, note=Two main limitations: no direct obesity measurement and data for only one chain; potential offset by home eating.
  \end{itemize}
\end{itemize}

\subsubsection{A12: What does the paper claim is novel about its data or identification?}\label{a12-what-does-the-paper-claim-is-novel-about-its-data-or-identification}

\begin{itemize}
\tightlist
\item
  Category: \texttt{A)\ Research\ question\ /\ contribution}
\item
  Confidence: \texttt{low\ (0.3097688817473264)}
\item
  Retrieval method: \texttt{local}
\item
  Evidence type: \texttt{data/identification\ novelty\ claim}
\item
  Data source: Starbucks transaction data and anonymous cardholder data
\item
  Answer:
\end{itemize}

The paper claims its novelty lies in using cardholder-level data to identify how calorie information affects individual behavior and to separate intensive and extensive margins from changes in consumer composition, i.e., novel data and identification.

\begin{itemize}
\tightlist
\item
  Quote snippet:
\end{itemize}

\begin{quote}
The advantage of the cardholder data is that we can assess how the calorie information causes particular individuals to change behavior, isolating intensive and extensive margins from changes in consumer composition.
\end{quote}

\begin{itemize}
\tightlist
\item
  Citation anchors:

  \begin{itemize}
  \tightlist
  \item
    page=5, words=0-349, section=Data and identification novelty, note=States that the advantage of the cardholder data allows isolating effects on intensive and extensive margins from consumer composition.
  \item
    page=3, words=275-564, section=Data suitability, note=Notes that the detailed transaction data are uniquely well-suited to answering the question.
  \end{itemize}
\end{itemize}

\subsubsection{B01: What is the identification strategy (in one sentence)?}\label{b01-what-is-the-identification-strategy-in-one-sentence}

\begin{itemize}
\tightlist
\item
  Category: \texttt{B)\ Identification\ strategy\ /\ causal\ design}
\item
  Confidence: \texttt{low\ (0.27842383323913433)}
\item
  Retrieval method: \texttt{local}
\item
  Evidence type: \texttt{regression-based\ identification\ using\ within-city\ variation\ (difference-in-differences)}
\item
  Data source: Transaction data and cardholder data from Starbucks in NYC, Boston, and Philadelphia.
\item
  Assumption flag: \texttt{True}
\item
  Assumption notes: Parallel trends assumption: pre-treatment trends in NYC and control cities are similar; identification relies on NYC changes following posting being exogenous relative to controls.
\item
  Answer:
\end{itemize}

The identification uses a within-city variation over time (difference-in-differences) by comparing NYC before and after calorie posting to Boston/Philadelphia controls, attributing changes to posting.

\begin{itemize}
\tightlist
\item
  Quote snippet:
\end{itemize}

\begin{quote}
identification of the effect of calorie posting stems from within-city variation over time.
\end{quote}

\begin{itemize}
\tightlist
\item
  Citation anchors:

  \begin{itemize}
  \tightlist
  \item
    page=11, words=275-522, section=Estimation/Identification, note=Statement that identification stems from within-city variation over time.
  \end{itemize}
\end{itemize}

\subsubsection{B02: Is the design experimental, quasi-experimental, or observational?}\label{b02-is-the-design-experimental-quasi-experimental-or-observational}

\begin{itemize}
\tightlist
\item
  Category: \texttt{B)\ Identification\ strategy\ /\ causal\ design}
\item
  Confidence: \texttt{low\ (0.3052100415762042)}
\item
  Retrieval method: \texttt{local}
\item
  Evidence type: \texttt{Describes\ treatment/control\ groups\ and\ post-treatment\ indicator\ in\ the\ methods/regression\ specification.}
\item
  Data source: Transaction data and cardholder data from Starbucks across NYC, Boston, Philadelphia, and Seattle
\item
  Table/Figure: Table 2
\item
  Assumption flag: \texttt{False}
\item
  Answer:
\end{itemize}

Quasi-experimental design (a natural experiment using treatment and control cities with before/after comparison, estimated via regression with a post-treatment indicator).

\begin{itemize}
\tightlist
\item
  Quote snippet:
\end{itemize}

\begin{quote}
POS\_Tc t is a dummy equal to one if calories were posted (i.e., NYC stores after April 1, 2008)
\end{quote}

\begin{itemize}
\tightlist
\item
  Citation anchors:

  \begin{itemize}
  \tightlist
  \item
    page=11, words=0-349, section=II. Effect of Mandatory Calorie Posting on Calorie Consumption, A. Calories Per Transaction, note=Shows the regression specification including POS\_Tc t and the treatment/control framework.
  \item
    page=7, words=275-566, section=Methods/Design, note=Describes Seattle vs NYC setup with control cities and the differential posting law.
  \end{itemize}
\end{itemize}

\subsubsection{B03: What is the source of exogenous variation used for identification?}\label{b03-what-is-the-source-of-exogenous-variation-used-for-identification}

\begin{itemize}
\tightlist
\item
  Category: \texttt{B)\ Identification\ strategy\ /\ causal\ design}
\item
  Confidence: \texttt{low\ (0.2985466131536938)}
\item
  Retrieval method: \texttt{local}
\item
  Evidence type: \texttt{Identification\ strategy\ description\ (policy-induced\ exogenous\ shock\ with\ within-city,\ city-week\ variation;\ difference-in-differences\ using\ NYC\ as\ treated\ and\ Boston/Philadelphia\ as\ controls).}
\item
  Data source: Starbucks transaction data and anonymized cardholder data for NYC, Boston, and Philadelphia (with Seattle as robustness check).
\item
  Answer:
\end{itemize}

The source of exogenous variation is the timing of calorie posting in New York City (NYC) starting April 1, 2008, which creates a treatment (NYC) versus control (Boston and Philadelphia) setup. Identification relies on within-city variation over time (city-week level variation) to estimate the effect of posting.

\begin{itemize}
\tightlist
\item
  Quote snippet:
\end{itemize}

\begin{quote}
In both cases (transaction data and cardholder data), identification of the effect of calorie posting stems from within-city variation over time.
\end{quote}

\begin{itemize}
\tightlist
\item
  Citation anchors:

  \begin{itemize}
  \tightlist
  \item
    page=11, words=275-522, section=Data/Identification, note=NYC calorie posting commenced on April 1, 2008; treated NYC vs controls for identification.
  \item
    page=12, words=0-349, section=Identification, note=Identification stems from within-city variation over time; policy variation at the city-week level.
  \end{itemize}
\end{itemize}

\subsubsection{B04: What is the treatment definition and timing?}\label{b04-what-is-the-treatment-definition-and-timing}

\begin{itemize}
\tightlist
\item
  Category: \texttt{B)\ Identification\ strategy\ /\ causal\ design}
\item
  Confidence: \texttt{low\ (0.13190735780502477)}
\item
  Retrieval method: \texttt{local}
\item
  Evidence type: \texttt{textual\ evidence\ (methods:\ treatment\ definition\ and\ timing)}
\item
  Data source: Starbucks transaction and cardholder data (NYC, Boston, Philadelphia)
\item
  Answer:
\end{itemize}

Treatment is the NYC calorie-posting intervention, defined by a post-period indicator that equals one for NYC stores after the posting date (April 1, 2008).

\begin{itemize}
\tightlist
\item
  Quote snippet:
\end{itemize}

\begin{quote}
POS Tc t is a dummy equal to one if calories were posted (i.e., NYC stores after April 1, 2008)
\end{quote}

\begin{itemize}
\tightlist
\item
  Citation anchors:

  \begin{itemize}
  \tightlist
  \item
    page=11, words=0-349, section=II.A - Calories Per Transaction, note=Definition of treatment: POS is a dummy equal to one if calories were posted (i.e., NYC stores after April 1, 2008).
  \end{itemize}
\end{itemize}

\subsubsection{B05: What is the control/comparison group definition?}\label{b05-what-is-the-controlcomparison-group-definition}

\begin{itemize}
\tightlist
\item
  Category: \texttt{B)\ Identification\ strategy\ /\ causal\ design}
\item
  Confidence: \texttt{low\ (0.1800548992326989)}
\item
  Retrieval method: \texttt{local}
\item
  Evidence type: \texttt{Textual\ evidence\ from\ the\ study\ describing\ the\ control\ group.}
\item
  Data source: NYC vs Boston/Philadelphia transaction and cardholder data; Seattle robustness check with Portland/San Francisco as controls.
\item
  Answer:
\end{itemize}

The control/comparison group consists of Boston and Philadelphia, used as controls for NYC.

\begin{itemize}
\tightlist
\item
  Quote snippet:
\end{itemize}

\begin{quote}
Qualitatively, however, it appears that Boston and Philadelphia are reasonable controls for NYC.
\end{quote}

\begin{itemize}
\tightlist
\item
  Citation anchors:

  \begin{itemize}
  \tightlist
  \item
    page=6, words=0-349, section=Table 1---Summary Statistics; Control Group, note=States that Boston and Philadelphia are used as controls for NYC.
  \item
    page=14, words=275-590, section=Robustness checks with Seattle; control cities, note=Mentions Seattle\textquotesingle s control cities Portland and San Francisco used in robustness checks.
  \end{itemize}
\end{itemize}

\subsubsection{B06: What is the estimating equation / baseline regression specification?}\label{b06-what-is-the-estimating-equation--baseline-regression-specification}

\begin{itemize}
\tightlist
\item
  Category: \texttt{B)\ Identification\ strategy\ /\ causal\ design}
\item
  Confidence: \texttt{low\ (0.30863926179799794)}
\item
  Retrieval method: \texttt{local}
\item
  Evidence type: \texttt{Baseline\ regression\ specification\ (Equation\ 1)\ from\ the\ paper}
\item
  Data source: Bollinger et al. 2011, American Economic Journal: Economic Policy
\item
  Answer:
\end{itemize}

The baseline estimating equation is y\_sct = x\_sct β + γ POS\_Tc t + ε\_sct, where y\_sct is a measure of calories per transaction at store s in city c on day t, POS\_Tc t is a dummy equal to one if calories were posted (i.e., NYC stores after April 1, 2008), and x\_sct includes week fixed effects (to control for seasonality), day-of-week dummies, holiday dummies, temperature and temperature squared, and precipitation and precipitation squared. In the transaction data, store fixed effects are included; in the cardholder data, individual fixed effects are included.

\begin{itemize}
\tightlist
\item
  Quote snippet:
\end{itemize}

\begin{quote}
y\_sct = x\_sct β + γ POS Tc t + ε\_sct, where y\_sct is a measure of calories per transaction at store s in city c on day t
\end{quote}

\begin{itemize}
\tightlist
\item
  Citation anchors:

  \begin{itemize}
  \tightlist
  \item
    page=11, words=275-522, section=Estimation framework / Regression specification, note=Equation (1): y\_sct = x\_sct β + γ POS Tc t + ε\_sct; y\_sct is a measure of calories per transaction; POS Tc t is a dummy equal to one if calories were posted (NYC stores after April 1, 2008); x\_sct includes week fixed effects (to control for seasonality), day-of-week dummies, holiday dummies, temperature and temperature\^{}2, precipitation and precipitation\^{}2; store fixed effects used with transaction data; individual fixed effects used with cardholder data.
  \end{itemize}
\end{itemize}

\subsubsection{B07: What fixed effects are included (unit, time, two-way, higher dimensional)?}\label{b07-what-fixed-effects-are-included-unit-time-two-way-higher-dimensional}

\begin{itemize}
\tightlist
\item
  Category: \texttt{B)\ Identification\ strategy\ /\ causal\ design}
\item
  Confidence: \texttt{low\ (0.25447351634256876)}
\item
  Retrieval method: \texttt{local}
\item
  Evidence type: \texttt{Description\ of\ fixed-effects\ structure\ from\ regression\ specifications\ across\ the\ cited\ pages.}
\item
  Data source: American Economic Journal: Economic Policy, February 2011
\item
  Answer:
\end{itemize}

Unit fixed effects: store fixed effects (transaction data) and individual fixed effects (cardholder data). Time fixed effects: week fixed effects, day-of-week dummies, and holiday dummies; robustness check adds date fixed effects. Two-way fixed effects: store fixed effects with week fixed effects. Higher-dimensional fixed effects: date fixed effects in robustness checks (drop weekly/daily effects).

\begin{itemize}
\tightlist
\item
  Quote snippet:
\end{itemize}

\begin{quote}
Store fixed effects to control for all time-invariant, store-specific heterogeneity. Store fixed effects also control for time-invariant city characteristics. In the cardholder data we include individual consumer fixed effects (and drop the store fixed effects).
\end{quote}

\begin{itemize}
\tightlist
\item
  Citation anchors:

  \begin{itemize}
  \tightlist
  \item
    page=11, words=275-330, section=Regression specification and time controls, note=Mentions week fixed effects to control for seasonality, day-of-week dummies, holiday dummies, and weather controls.
  \item
    page=12, words=0-60, section=Fixed effects in store vs cardholder specifications, note=Notes store fixed effects for time-invariant, store-specific heterogeneity; cardholder data use individual fixed effects (and drop store fixed effects).
  \item
    page=12, words=60-120, section=Robustness checks and additional fixed effects, note=Robustness check includes date fixed effects (instead of day-of-week/week fixed effects).
  \end{itemize}
\end{itemize}

\subsubsection{B08: What standard errors are used (robust, clustered; at what level)?}\label{b08-what-standard-errors-are-used-robust-clustered-at-what-level}

\begin{itemize}
\tightlist
\item
  Category: \texttt{B)\ Identification\ strategy\ /\ causal\ design}
\item
  Confidence: \texttt{low\ (0.2807844345302521)}
\item
  Retrieval method: \texttt{local}
\item
  Evidence type: \texttt{textual}
\item
  Data source: cardholder data (Table 9)
\item
  Table/Figure: Table 9
\item
  Answer:
\end{itemize}

They used robust (heteroskedasticity-robust) standard errors, reported in parentheses; the regression is at the transaction (observation) level (each observation is a transaction). The notes do not indicate any clustering.

\begin{itemize}
\tightlist
\item
  Quote snippet:
\end{itemize}

\begin{quote}
Robust standard errors in parentheses.
\end{quote}

\begin{itemize}
\tightlist
\item
  Citation anchors:

  \begin{itemize}
  \tightlist
  \item
    page=29, words=0-349, section=Table 9 notes, note=Robust standard errors in parentheses; observation is a transaction
  \end{itemize}
\end{itemize}

\subsubsection{B09: What is the key identifying assumption (parallel trends, exclusion restriction, ignorability)?}\label{b09-what-is-the-key-identifying-assumption-parallel-trends-exclusion-restriction-ignorability}

\begin{itemize}
\tightlist
\item
  Category: \texttt{B)\ Identification\ strategy\ /\ causal\ design}
\item
  Confidence: \texttt{low\ (0.2723451096565549)}
\item
  Retrieval method: \texttt{local}
\item
  Evidence type: \texttt{parallel\_trends}
\item
  Data source: Starbucks transaction data and cardholder data
\item
  Table/Figure: Figure 2
\item
  Assumption flag: \texttt{True}
\item
  Assumption notes: The validity relies on parallel trends between NYC and the control cities; pre-trend similarity is reported, and results are corroborated across data sources and with within-individual identification.
\item
  Answer:
\end{itemize}

Parallel trends assumption (differences-in-differences): the treated NYC stores would have followed the same time trend in calories per transaction as the control cities (Boston/Philadelphia) in the absence of calorie posting.

\begin{itemize}
\tightlist
\item
  Quote snippet:
\end{itemize}

\begin{quote}
First, with both datasets we see no evidence of pre-trend differences between NYC and Boston/Philadelphia.
\end{quote}

\begin{itemize}
\tightlist
\item
  Citation anchors:

  \begin{itemize}
  \tightlist
  \item
    page=14, words=0-60, section=Pre-trend checks, note=No evidence of pre-trend differences between NYC and control cities (Boston/Philadelphia).
  \item
    page=14, words=275-590, section=Identification \& robustness checks, note=Within-individual variation; robustness checks across datasets; pre-trends discussed.
  \end{itemize}
\end{itemize}

\subsubsection{B10: What evidence is provided to support the identifying assumption?}\label{b10-what-evidence-is-provided-to-support-the-identifying-assumption}

\begin{itemize}
\tightlist
\item
  Category: \texttt{B)\ Identification\ strategy\ /\ causal\ design}
\item
  Confidence: \texttt{low\ (0.24037184619592758)}
\item
  Retrieval method: \texttt{local}
\item
  Evidence type: \texttt{parallel-trends\ check;\ within-individual\ variation\ design}
\item
  Data source: Starbucks transaction data and cardholder data
\item
  Table/Figure: Figure 2
\item
  Assumption flag: \texttt{True}
\item
  Assumption notes: Pre-trend equivalence across NYC and control cities; identification supported by within-individual variation in cardholder data to isolate intensive-margin effects.
\item
  Answer:
\end{itemize}

Evidence supports the identifying assumption by showing no pre-trend differences between NYC and control cities (parallel trends) in both transaction and cardholder analyses, and by using cardholder data that isolates intensive-margin changes via within-individual variation over time.

\begin{itemize}
\tightlist
\item
  Quote snippet:
\end{itemize}

\begin{quote}
no evidence of pre-trend differences between NYC and Boston/Philadelphia
\end{quote}

\begin{itemize}
\tightlist
\item
  Citation anchors:

  \begin{itemize}
  \tightlist
  \item
    page=14, words=0-349, section=Figure 2 results, note=No pre-trend differences between NYC and controls; supports parallel trends.
  \item
    page=12, words=275-624, section=Cardholder vs Transaction analysis, note=Cardholder data isolate intensive-margin changes; tracks same individuals; identification via within-individual variation.
  \item
    page=14, words=275-590, section=Cardholder data identification, note=Within-individual variation over time identifies effects.
  \end{itemize}
\end{itemize}

\subsubsection{B11: Are there event-study or pre-trend tests? What do they show?}\label{b11-are-there-event-study-or-pre-trend-tests-what-do-they-show}

\begin{itemize}
\tightlist
\item
  Category: \texttt{B)\ Identification\ strategy\ /\ causal\ design}
\item
  Confidence: \texttt{low\ (0.3177309306712598)}
\item
  Retrieval method: \texttt{local}
\item
  Evidence type: \texttt{Event-study/pre-trend\ test\ comparing\ NYC\ vs\ control\ cities;\ Figure\ 2\ shows\ no\ pre-trend\ differences;\ post-change\ effects\ observed\ starting\ Apr\ 1,\ 2008.}
\item
  Data source: Transaction and cardholder data from Starbucks in NYC, Boston, and Philadelphia (pre/post calorie posting).
\item
  Table/Figure: Figure 2
\item
  Assumption flag: \texttt{True}
\item
  Assumption notes: Support for parallel trends assumption is evidenced by no pre-trend differences between NYC and control cities prior to calorie posting.
\item
  Answer:
\end{itemize}

Yes. The study conducts event-study style pre-trend checks by comparing NYC stores to control cities (Boston/Philadelphia). It finds no evidence of pre-trend differences before calorie posting, and the post-policy drop in calories per transaction begins around April 1, 2008. Cardholder data show no pre-trend differences; any Christmas/New Year deviations appear only in transaction data.

\begin{itemize}
\tightlist
\item
  Quote snippet:
\end{itemize}

\begin{quote}
First, with both datasets we see no evidence of pre-trend differences between NYC and Boston/Philadelphia.
\end{quote}

\begin{itemize}
\tightlist
\item
  Citation anchors:

  \begin{itemize}
  \tightlist
  \item
    page=9, words=0-349, section=Results: Pre-trend checks and Figure 2, note=No evidence of pre-trend differences; drop begins around Apr 1, 2008; transaction vs cardholder differences noted.
  \item
    page=14, words=0-349, section=Figure 2 and pre-trend discussion, note=Explicit statement of no pre-trend differences; post-period pattern follows posting date.
  \end{itemize}
\end{itemize}

\subsubsection{B12: What instruments are used (if IV)? Define instrument and first stage.}\label{b12-what-instruments-are-used-if-iv-define-instrument-and-first-stage}

\begin{itemize}
\tightlist
\item
  Category: \texttt{B)\ Identification\ strategy\ /\ causal\ design}
\item
  Confidence: \texttt{low\ (0.09999694849378597)}
\item
  Retrieval method: \texttt{local}
\item
  Evidence type: \texttt{text}
\item
  Data source: Transaction data; Cardholder data
\item
  Answer:
\end{itemize}

No instrumental variables are used. The study relies on a post-treatment indicator for NYC after April 1, 2008 and city controls with fixed effects and other controls; there is no IV and no first-stage regression reported.

\begin{itemize}
\tightlist
\item
  Quote snippet:
\end{itemize}

\begin{quote}
The results shown in Table 3 are based on specifications in which calorie posting is binary---i.e., the POST variable is simply a dummy equal to one at NYC stores on every day after April 1, 2008.
\end{quote}

\begin{itemize}
\tightlist
\item
  Citation anchors:

  \begin{itemize}
  \tightlist
  \item
    page=13, words=275-528, section=Methods (Table 3 specification), note=POST variable is simply a dummy equal to one at NYC stores on every day after April 1, 2008.
  \item
    page=15, words=0-254, section=Figure 2 regression specification, note=Regression of log(calories per transaction) on separate week effects for NYC and control regions with day-of-week effects and weather controls.
  \end{itemize}
\end{itemize}

\subsubsection{B13: What is the first-stage strength (F-stat, partial R\^{}2, relevance evidence)?}\label{b13-what-is-the-first-stage-strength-f-stat-partial-r2-relevance-evidence}

\begin{itemize}
\tightlist
\item
  Category: \texttt{B)\ Identification\ strategy\ /\ causal\ design}
\item
  Confidence: \texttt{low\ (0.24514799960328948)}
\item
  Retrieval method: \texttt{local}
\item
  Evidence type: \texttt{not\_reported}
\item
  Data source: Starbucks transaction data and cardholder data
\item
  Answer:
\end{itemize}

The provided text does not report any first-stage statistics (F-stat, partial R\^{}2) or explicit evidence of first-stage strength.

\begin{itemize}
\tightlist
\item
  Quote snippet:
\end{itemize}

\begin{quote}
we estimate regressions of the following form: (1) \hspace{0pt}y\hspace{0pt}sct\hspace{0pt} = \hspace{0pt}xs\hspace{0pt}ct\hspace{0pt}β + γPOS\hspace{0pt}Tc\hspace{0pt}t\hspace{0pt} + \hspace{0pt}εs\hspace{0pt}ct, where \hspace{0pt}y\hspace{0pt}sct\hspace{0pt}is a measure of calories per transaction
\end{quote}

\begin{itemize}
\tightlist
\item
  Citation anchors:

  \begin{itemize}
  \tightlist
  \item
    page=11, words=275-522, section=Methods/Empirical specification, note=Describes regression form with POST indicator; no instrument or first-stage statistics mentioned.
  \item
    page=3, words=275-564, section=Limitations/Context, note=Discusses study limitations and data, with no mention of first-stage strength.
  \item
    page=30, words=0-349, section=Discussion/Conclusion, note=Notes learning and salience; no first-stage statistics are discussed.
  \end{itemize}
\end{itemize}

\subsubsection{B14: If RDD: what is the running variable and cutoff? bandwidth choice?}\label{b14-if-rdd-what-is-the-running-variable-and-cutoff-bandwidth-choice}

\begin{itemize}
\tightlist
\item
  Category: \texttt{B)\ Identification\ strategy\ /\ causal\ design}
\item
  Confidence: \texttt{low\ (0.22339790055511907)}
\item
  Retrieval method: \texttt{local}
\item
  Evidence type: \texttt{design\ description;\ not\ an\ RDD}
\item
  Answer:
\end{itemize}

The study does not implement an RDD. It uses a post-treatment indicator for NYC stores after April 1, 2008 within a regression framework with fixed effects, which is a difference‑in‑differences style design rather than an RDD. Therefore there is no running variable, cutoff, or bandwidth specification for an RDD.

\begin{itemize}
\tightlist
\item
  Quote snippet:
\end{itemize}

\begin{quote}
POS Tc\_t is a dummy equal to one if calories were posted (i.e., NYC stores after April 1, 2008)
\end{quote}

\begin{itemize}
\tightlist
\item
  Citation anchors:

  \begin{itemize}
  \tightlist
  \item
    page=11, words=0-349, section=Methods/Identification, note=Definition of treatment: POS\_t is a dummy equal to one if calories were posted (NYC stores after April 1, 2008) and the regression specification.
  \end{itemize}
\end{itemize}

\subsubsection{B15: If DiD: what is the timing variation (staggered adoption)? estimator used?}\label{b15-if-did-what-is-the-timing-variation-staggered-adoption-estimator-used}

\begin{itemize}
\tightlist
\item
  Category: \texttt{B)\ Identification\ strategy\ /\ causal\ design}
\item
  Confidence: \texttt{low\ (0.2889126741456382)}
\item
  Retrieval method: \texttt{local}
\item
  Evidence type: \texttt{textual\ description\ of\ the\ estimation\ strategy\ with\ post\ dummy\ and\ event-study\ variant}
\item
  Data source: Transaction and cardholder data from Starbucks stores in NYC, Boston, and Philadelphia; Seattle data used for robustness
\item
  Table/Figure: Figure 2 and Table 3 (DiD with post indicator and timing analysis)
\item
  Assumption flag: \texttt{True}
\item
  Assumption notes: Parallel trends between NYC and control cities prior to posting; pre-trend differences examined and found not evidence of divergence (Figure 2).
\item
  Answer:
\end{itemize}

The paper uses a standard difference-in-differences setup with a post indicator for NYC after the calorie-posting date (April 1, 2008). They also implement an event-study style variation by using separate week dummies for NYC and the control cities to examine timing/dynamics of the effect. Additionally, they reference a staggered adoption across cities by noting Seattle's post-law date (January 1, 2009) as a robustness check.

\begin{itemize}
\tightlist
\item
  Quote snippet:
\end{itemize}

\begin{quote}
y\_sct = x\_sct β + γ\_POST Tc\_t + ε\_sct, where y\_sct is a measure of calories per transaction at store s in city c on day t, POST Tc\_t is a dummy equal to one if calories were posted (i.e., NYC stores after April 1, 2008)
\end{quote}

\begin{itemize}
\tightlist
\item
  Citation anchors:

  \begin{itemize}
  \tightlist
  \item
    page=11, words=275-522, section=Estimation, note=Defines the regression y\_sct = x\_sct β + γ\_POST Tc\_t + ε\_sct; POST is a dummy for NYC stores after April 1, 2008; includes week fixed effects and other controls.
  \item
    page=14, words=0-349, section=Results/Timing, note=Figure 2 and description of plotting the difference between NYC weekly fixed effects and control weekly effects (event-study style timing evidence).
  \item
    page=16, words=275-558, section=Robustness / Additional Adoption, note=Notes Seattle's post-law date (Jan 1, 2009) as an additional adoption date and robustness check.
  \end{itemize}
\end{itemize}

\subsubsection{C01: What dataset(s) are used? (name sources explicitly)}\label{c01-what-datasets-are-used-name-sources-explicitly}

\begin{itemize}
\tightlist
\item
  Category: \texttt{C)\ Data,\ sample,\ and\ measurement}
\item
  Confidence: \texttt{low\ (0.2803849335009675)}
\item
  Retrieval method: \texttt{local}
\item
  Evidence type: \texttt{dataset\_description}
\item
  Data source: Starbucks transaction data; Starbucks cardholder data
\item
  Answer:
\end{itemize}

Two datasets: the Starbucks transaction data and the Starbucks cardholder data.

\begin{itemize}
\tightlist
\item
  Quote snippet:
\end{itemize}

\begin{quote}
Our transaction data cover all 222 Starbucks locations in NYC, and all 94 Starbucks locations in Boston and Philadelphia.
\end{quote}

\begin{itemize}
\tightlist
\item
  Citation anchors:

  \begin{itemize}
  \tightlist
  \item
    page=5, words=0-349, section=Data Summary, note=Describes the transaction data: 222 NYC locations and 94 Boston/Philadelphia locations.
  \item
    page=5, words=275-591, section=Data Summary, note=Explicitly states dataset naming: \textquotesingle first dataset as the transaction data and the second dataset as the cardholder data\textquotesingle.
  \end{itemize}
\end{itemize}

\subsubsection{C02: What is the unit of observation (individual, household, firm, county, transaction, product)?}\label{c02-what-is-the-unit-of-observation-individual-household-firm-county-transaction-product}

\begin{itemize}
\tightlist
\item
  Category: \texttt{C)\ Data,\ sample,\ and\ measurement}
\item
  Confidence: \texttt{low\ (0.32252574139462475)}
\item
  Retrieval method: \texttt{local}
\item
  Evidence type: \texttt{text}
\item
  Data source: Starbucks transaction data and cardholder data
\item
  Answer:
\end{itemize}

transaction

\begin{itemize}
\tightlist
\item
  Quote snippet:
\end{itemize}

\begin{quote}
For each transaction we observe the time and date, store location, items purchased, and price of each item.
\end{quote}

\begin{itemize}
\tightlist
\item
  Citation anchors:

  \begin{itemize}
  \tightlist
  \item
    page=5, words=0-349, section=Data Summary, note=Unit of observation is a transaction; data include time/date, store, items, price per transaction.
  \end{itemize}
\end{itemize}

\subsubsection{C03: What is the sample period and geographic coverage?}\label{c03-what-is-the-sample-period-and-geographic-coverage}

\begin{itemize}
\tightlist
\item
  Category: \texttt{C)\ Data,\ sample,\ and\ measurement}
\item
  Confidence: \texttt{low\ (0.2306843139246988)}
\item
  Retrieval method: \texttt{local}
\item
  Evidence type: \texttt{text}
\item
  Data source: Starbucks transaction data and cardholder data (NYC, Boston, Philadelphia).
\item
  Answer:
\end{itemize}

Sample period: January 1, 2008 to February 28, 2009; geographic coverage: Starbucks locations in New York City, Boston, and Philadelphia.

\begin{itemize}
\tightlist
\item
  Quote snippet:
\end{itemize}

\begin{quote}
Our transaction data cover all 222 Starbucks locations in NYC, and all 94 Starbucks locations in Boston and Philadelphia.
\end{quote}

\begin{itemize}
\tightlist
\item
  Citation anchors:

  \begin{itemize}
  \tightlist
  \item
    page=5, words=0-349, section=Data Summary, note=Geographic coverage (NYC; Boston \& Philadelphia) and time window around calorie posting.
  \end{itemize}
\end{itemize}

\subsubsection{C04: What are the sample restrictions / inclusion criteria?}\label{c04-what-are-the-sample-restrictions--inclusion-criteria}

\begin{itemize}
\tightlist
\item
  Category: \texttt{C)\ Data,\ sample,\ and\ measurement}
\item
  Confidence: \texttt{low\ (0.24538555087080954)}
\item
  Retrieval method: \texttt{local}
\item
  Evidence type: \texttt{direct\ textual\ evidence\ from\ the\ study\textquotesingle{}s\ data\ and\ methods\ sections}
\item
  Data source: Starbucks transaction and cardholder data used in Bollinger et al., American Economic Journal: Economic Policy, 2011
\item
  Answer:
\end{itemize}

Sample restrictions include: (1) Cardholder analysis uses a subsample of anonymous cardholders who averaged at least one transaction per week in one of NYC, Boston, or Philadelphia during the pre-period before calorie posting in NYC; this yields 7,520 NYC and 3,772 Boston/Philadelphia individuals in the cardholder sample. (2) Transaction data include all transactions from 222 NYC stores and 94 stores in Boston and Philadelphia, but exclude stores not open for the entire period and exclude any transactions with more than four units of a single item (balanced panel).

\begin{itemize}
\tightlist
\item
  Quote snippet:
\end{itemize}

\begin{quote}
We define a subsample containing any individual that averaged at least one transaction per week in one of NYC, or Boston and Philadelphia, in the period before calorie posting in NYC.
\end{quote}

\begin{itemize}
\tightlist
\item
  Citation anchors:

  \begin{itemize}
  \tightlist
  \item
    page=5, words=0-349, section=Data and sample description (cardholder subsample), note=Definition of cardholder subsample: averaged at least one transaction per week in NYC, Boston, or Philadelphia during the pre-period before calorie posting in NYC.
  \item
    page=6, words=275-517, section=Sample restrictions (transaction data), note=Exclude stores not open for the entire period (balanced panel) and exclude transactions with more than four units of any one item.
  \end{itemize}
\end{itemize}

\subsubsection{C05: What is the sample size (N) in the main analysis?}\label{c05-what-is-the-sample-size-n-in-the-main-analysis}

\begin{itemize}
\tightlist
\item
  Category: \texttt{C)\ Data,\ sample,\ and\ measurement}
\item
  Confidence: \texttt{low\ (0.29791024856995396)}
\item
  Retrieval method: \texttt{local}
\item
  Evidence type: \texttt{Data\ summary\ from\ the\ Cardholder\ dataset\ describing\ sample\ size.}
\item
  Data source: Starbucks transaction data (universe of transactions) and cardholder data (subsample with ≥1 transaction per week in NYC/Boston/Philadelphia).
\item
  Answer:
\end{itemize}

1.51 million transactions (cardholder data).

\begin{itemize}
\tightlist
\item
  Quote snippet:
\end{itemize}

\begin{quote}
There are 7,520 such individuals in NYC and 3,772 such individuals in Boston and Philadelphia, generating a combined 1.51 million transactions for us to study.
\end{quote}

\begin{itemize}
\tightlist
\item
  Citation anchors:

  \begin{itemize}
  \tightlist
  \item
    page=5, words=0-349, section=Data Summary -- Cardholder sample size, note=Reports 7,520 NYC cardholders and 3,772 in Boston/Philadelphia, generating a combined 1.51 million transactions.
  \item
    page=5, words=275-591, section=Data Summary -- Cardholder sample size, note=Contains the same statement about the cardholder subsample size and total transactions (1.51 million).
  \end{itemize}
\end{itemize}

\subsubsection{C06: How is the key outcome measured? Any transformations (logs, z-scores, indices)?}\label{c06-how-is-the-key-outcome-measured-any-transformations-logs-z-scores-indices}

\begin{itemize}
\tightlist
\item
  Category: \texttt{C)\ Data,\ sample,\ and\ measurement}
\item
  Confidence: \texttt{low\ (0.3267221472357983)}
\item
  Retrieval method: \texttt{local}
\item
  Evidence type: \texttt{textual}
\item
  Data source: Transaction data; Cardholder data
\item
  Answer:
\end{itemize}

The key outcome is calories per transaction, measured as calories per transaction at a store, city, and day (with breakdowns for beverage calories per transaction, food calories per transaction, and total calories per transaction). Transformations include log specifications: for example, log(beverage + food calories) and log(calories per transaction) are used in regressions, with coefficients interpreted as percent changes (e.g., a −0.119 coefficient implying an 11.2\% decrease). The analysis also examines quantiles of calories per transaction and uses both level and log forms to decompose intensive vs. extensive-margin effects.

\begin{itemize}
\tightlist
\item
  Quote snippet:
\end{itemize}

\begin{quote}
y\_sct is a measure of calories per transaction at store s in city c on day t
\end{quote}

\begin{itemize}
\tightlist
\item
  Citation anchors:

  \begin{itemize}
  \tightlist
  \item
    page=11, words=275-522, section=Methods/Empirical specification, note=Defines y\_sct as calories per transaction and presents the regression form including the POST dummy.
  \item
    page=12, words=275-624, section=Results/Specifications, note=Mentions log(beverage + food calories) and presents a −0.063 coefficient; discusses cardholder data.
  \item
    page=12, words=275-624, section=Results/Interpretation, note=Reports a 11.2 percent decrease in food calories per transaction (coefficient −0.119) and a 5.0 percent total calories change using cardholder data.
  \item
    page=16, words=275-558, section=Tables/Decomposition, note=Discusses margins and decomposition into number of beverages/food items per transaction.
  \item
    page=22, words=0-349, section=Tables/Quantiles, note=Describes quantile-based analysis of calories per transaction (Table 6) and percent changes across quantiles.
  \end{itemize}
\end{itemize}

\subsubsection{C07: How is treatment/exposure measured? Any constructed variables?}\label{c07-how-is-treatmentexposure-measured-any-constructed-variables}

\begin{itemize}
\tightlist
\item
  Category: \texttt{C)\ Data,\ sample,\ and\ measurement}
\item
  Confidence: \texttt{low\ (0.2924329921257721)}
\item
  Retrieval method: \texttt{local}
\item
  Evidence type: \texttt{Describes\ treatment\ construction\ (POST\ dummy)\ and\ alternative\ specifications;\ lists\ constructed\ outcomes\ and\ fixed-effects\ approach.}
\item
  Data source: Transaction dataset and cardholder dataset
\item
  Assumption flag: \texttt{True}
\item
  Assumption notes: Assumes identification comes from within-city variation over time (city-week level) with NYC post-implementation as the treatment; assumes no other concurrent shocks differentially affecting NYC stores during the post period.
\item
  Answer:
\end{itemize}

Treatment/exposure is measured with a binary post variable (POST) that equals 1 for NYC stores on every day after April 1, 2008, indicating calorie posting. Identification relies on within-city variation over time (city-week level). A robustness alternative uses separate week dummies for NYC and control cities, excluding the POST indicator. Constructed variables include the treatment indicator itself and various calorie-related outcome metrics (e.g., log calories per transaction for beverages/food, total calories per transaction) as well as other derived measures (number of items per transaction, calories per item, etc.).

\begin{itemize}
\tightlist
\item
  Quote snippet:
\end{itemize}

\begin{quote}
POS Tc is a dummy equal to one if calories were posted (i.e., NYC stores after April 1, 2008)
\end{quote}

\begin{itemize}
\tightlist
\item
  Citation anchors:

  \begin{itemize}
  \tightlist
  \item
    page=11, words=275-522, section=Estimation specification, note=POS Tc is a dummy equal to one if calories were posted (i.e., NYC stores after April 1, 2008).
  \item
    page=11, words=0-349, section=Identification strategy, note=Identification stems from within-city variation over time; policy variation is at the city-week level.
  \item
    page=12, words=0-349, section=Methods, note=Store fixed effects for transaction data and individual fixed effects for cardholder data; identification from within-city variation over time.
  \item
    page=12, words=275-624, section=Robustness / Alternative specification, note=An alternative approach uses separate week dummies for NYC and control cities, excluding the POST variable.
  \end{itemize}
\end{itemize}

\subsubsection{C08: Are there key covariates/controls? Which ones are always included?}\label{c08-are-there-key-covariatescontrols-which-ones-are-always-included}

\begin{itemize}
\tightlist
\item
  Category: \texttt{C)\ Data,\ sample,\ and\ measurement}
\item
  Confidence: \texttt{low\ (0.28724934374498645)}
\item
  Retrieval method: \texttt{local}
\item
  Evidence type: \texttt{text}
\item
  Data source: Transaction data and cardholder data from the Starbucks calorie posting study
\item
  Answer:
\end{itemize}

Yes. The key covariates/controls are week fixed effects, day-of-week dummies, holiday dummies, and weather controls (temperature and precipitation and their squares). These are included in the baseline regression specification for both the transaction data and the cardholder data; in the cardholder analysis, there are also individual fixed effects. Thus, the controls that are always included across specifications are week fixed effects, day-of-week dummies, holiday dummies, and weather controls. Individual fixed effects are added in the cardholder dataset but not necessarily in the transaction dataset.

\begin{itemize}
\tightlist
\item
  Quote snippet:
\end{itemize}

\begin{quote}
x\_sct includes week fixed effects (to control for seasonality), day-of-week dummies, holiday dummies, temperature and temperature squared, and precipitation and precipitation squared.
\end{quote}

\begin{itemize}
\tightlist
\item
  Citation anchors:

  \begin{itemize}
  \tightlist
  \item
    page=11, words=0-349, section=Methods/Regression controls, note=Baseline controls include week fixed effects, day-of-week dummies, holiday dummies, temperature and temperature squared, and precipitation and precipitation squared.
  \item
    page=29, words=0-349, section=Cardholder data regression, note=Cardholder regressions include individual, week, and day-of-week fixed effects, and weather controls.
  \end{itemize}
\end{itemize}

\subsubsection{C09: How are missing data handled (dropping, imputation, weighting)?}\label{c09-how-are-missing-data-handled-dropping-imputation-weighting}

\begin{itemize}
\tightlist
\item
  Category: \texttt{C)\ Data,\ sample,\ and\ measurement}
\item
  Confidence: \texttt{low\ (0.2782624872026636)}
\item
  Retrieval method: \texttt{local}
\item
  Evidence type: \texttt{data\ cleaning\ /\ exclusion\ criteria}
\item
  Data source: Transaction data and cardholder data from Starbucks locations in NYC, Boston, and Philadelphia (2008--2009).
\item
  Answer:
\end{itemize}

Missing data are handled by dropping incomplete observations: the analysis uses a balanced panel by excluding transactions that were not open for the full data period and excluding transactions with more than four units of any item. The paper notes that the excluded transactions represent 2.2 percent of all transactions; there is no indication of imputation or weighting.

\begin{itemize}
\tightlist
\item
  Quote snippet:
\end{itemize}

\begin{quote}
The excluded transactions represent only 2.2 percent of all transactions.
\end{quote}

\begin{itemize}
\tightlist
\item
  Citation anchors:

  \begin{itemize}
  \tightlist
  \item
    page=6, words=0-349, section=Data cleaning / exclusions, note=Excludes transactions not observed during the entire data period (balanced panel) and those with more than four units; 2.2\% of transactions are excluded.
  \end{itemize}
\end{itemize}

\subsubsection{C10: Are weights used (survey weights, propensity weights)? How?}\label{c10-are-weights-used-survey-weights-propensity-weights-how}

\begin{itemize}
\tightlist
\item
  Category: \texttt{C)\ Data,\ sample,\ and\ measurement}
\item
  Confidence: \texttt{low\ (0.2957772185028331)}
\item
  Retrieval method: \texttt{local}
\item
  Evidence type: \texttt{textual}
\item
  Assumption flag: \texttt{False}
\item
  Answer:
\end{itemize}

No. The provided context does not indicate the use of survey weights or propensity weights. The analysis relies on regression specifications with fixed effects and other controls, but weights are not described.

\begin{itemize}
\tightlist
\item
  Quote snippet:
\end{itemize}

\begin{quote}
to examine the effects of calorie posting while controlling for seasonality and other influences, we estimate regressions of the following form: (1) y\_sct = x\_sct β + γPOST Tc t + ε\_sct
\end{quote}

\begin{itemize}
\tightlist
\item
  Citation anchors:

  \begin{itemize}
  \tightlist
  \item
    page=5, words=0-349, section=Data Summary, note=Data description; no weighting discussed.
  \item
    page=11, words=0-349, section=Methods/Empirical strategy, note=Regression specification cited; includes fixed effects and weather controls; no weighting described.
  \item
    page=11, words=275-522, section=Methods/Empirical strategy, note=Continuation of regression approach; no weighting described.
  \item
    page=35, words=0-161, section=Appendix A, note=Survey appendix; no weighting described.
  \end{itemize}
\end{itemize}

\subsubsection{C11: Are data linked/merged across sources? How is linkage performed?}\label{c11-are-data-linkedmerged-across-sources-how-is-linkage-performed}

\begin{itemize}
\tightlist
\item
  Category: \texttt{C)\ Data,\ sample,\ and\ measurement}
\item
  Confidence: \texttt{low\ (0.22255016300986244)}
\item
  Retrieval method: \texttt{local}
\item
  Evidence type: \texttt{text}
\item
  Data source: Starbucks calorie posting study data (transaction data and anonymous cardholder data; NYC, Boston, Philadelphia; Seattle robustness check)
\item
  Answer:
\end{itemize}

No---data are not merged across sources. The study uses two separate data sources: the transaction data and the anonymous cardholder data; the analysis then compares the separately estimated effects from each dataset rather than merging them into a single linked file.

\begin{itemize}
\tightlist
\item
  Quote snippet:
\end{itemize}

\begin{quote}
We refer to the first dataset as the transaction data and the second dataset as the cardholder data. In the analysis we compare the separately estimated effects of calorie posting on the cardholder data with transaction data.
\end{quote}

\begin{itemize}
\tightlist
\item
  Citation anchors:

  \begin{itemize}
  \tightlist
  \item
    page=5, words=0-349, section=Data sources and linkage, note=Defines two datasets (transaction data and cardholder data) and states the analysis compares them rather than merging.
  \item
    page=5, words=275-591, section=Data sources and linkage, note=Further description of cardholder vs. transaction data and their separate analyses compared in the study.
  \end{itemize}
\end{itemize}

\subsubsection{C12: What summary statistics are reported for main variables?}\label{c12-what-summary-statistics-are-reported-for-main-variables}

\begin{itemize}
\tightlist
\item
  Category: \texttt{C)\ Data,\ sample,\ and\ measurement}
\item
  Confidence: \texttt{low\ (0.3270386890203643)}
\item
  Retrieval method: \texttt{local}
\item
  Evidence type: \texttt{Table\ 1}
\item
  Data source: Table 1: Summary Statistics for Transaction Data and Cardholder Data (prior to calorie posting in NYC)
\item
  Table/Figure: Table 1
\item
  Answer:
\end{itemize}

The main variables\textquotesingle{} summary statistics are reported in Table 1 for two datasets (Transaction data and Cardholder data). The table lists: average weekly transactions per store; average weekly revenue per store; percent of transactions with brewed coffee; percent with beverage; percent with food; average number of items per transaction; average number of drink items per transaction; average number of food items per transaction; food attach rate; average dollars per transaction; average calories per transaction; average drink calories per transaction; and average food calories per transaction. The statistics are shown separately for Boston, New York City, and Philadelphia within each dataset, and are normalized to 1.00 to preserve confidentiality. All statistics are based on data prior to calorie posting in NYC (April 1, 2008).

\begin{itemize}
\tightlist
\item
  Quote snippet:
\end{itemize}

\begin{quote}
Table 1---Summary Statistics for Transaction Data and Cardholder Data (Prior to policy change)
\end{quote}

\begin{itemize}
\tightlist
\item
  Citation anchors:

  \begin{itemize}
  \tightlist
  \item
    page=6, words=0-349, section=Table 1, note=Summary statistics for transaction data and cardholder data; normalization to 1.00; data prior to calorie posting in NYC.
  \end{itemize}
\end{itemize}

\subsubsection{C13: Are there descriptive figures/maps that establish baseline patterns?}\label{c13-are-there-descriptive-figuresmaps-that-establish-baseline-patterns}

\begin{itemize}
\tightlist
\item
  Category: \texttt{C)\ Data,\ sample,\ and\ measurement}
\item
  Confidence: \texttt{low\ (0.2833682878412908)}
\item
  Retrieval method: \texttt{local}
\item
  Evidence type: \texttt{descriptive\_figures}
\item
  Data source: Transaction data and cardholder data from Starbucks locations in NYC, Boston, and Philadelphia
\item
  Table/Figure: Figure 1 and Figure 2
\item
  Answer:
\end{itemize}

Yes. The paper includes descriptive figures that establish baseline patterns, notably Figure 1 showing average calories per transaction by city, and Figure 2 showing pre-trend checks (no pre-trend differences between NYC and control cities).

\begin{itemize}
\tightlist
\item
  Quote snippet:
\end{itemize}

\begin{quote}
Figure 2 depicts the results for each dataset. There are no pre-trend differences between NYC and Boston/Philadelphia.
\end{quote}

\begin{itemize}
\tightlist
\item
  Citation anchors:

  \begin{itemize}
  \tightlist
  \item
    page=page 9, words=0-349, section=Calories Per Transaction (Figure 1), note=Figure 1 shows average calories per transaction per week for NYC vs controls.
  \item
    page=page 14, words=0-349, section=Baseline trend check (Figure 2), note=Figure 2 depicts results and shows no pre-trend differences between NYC and controls.
  \end{itemize}
\end{itemize}

\subsubsection{D01: What is the headline main effect estimate (sign and magnitude)?}\label{d01-what-is-the-headline-main-effect-estimate-sign-and-magnitude}

\begin{itemize}
\tightlist
\item
  Category: \texttt{D)\ Results,\ magnitudes,\ heterogeneity,\ robustness}
\item
  Confidence: \texttt{medium\ (0.4277382164232411)}
\item
  Retrieval method: \texttt{local}
\item
  Evidence type: \texttt{regression\ result\ reporting\ effect\ on\ calories\ per\ transaction\ (Table\ 3)}
\item
  Data source: American Economic Journal: Economic Policy, calorie posting study using Starbucks data
\item
  Table/Figure: Table 3
\item
  Answer:
\end{itemize}

-5.8\% decrease in average calories per transaction.

\begin{itemize}
\tightlist
\item
  Quote snippet:
\end{itemize}

\begin{quote}
finding a 5.8 percent decrease in average calories per transaction, equivalent to 14.4 calories.
\end{quote}

\begin{itemize}
\tightlist
\item
  Citation anchors:

  \begin{itemize}
  \tightlist
  \item
    page=12, words=0-349, section=Table 3 results, note=Calorie posting reduces average calories per transaction by 5.8\% (14.4 calories).
  \end{itemize}
\end{itemize}

\subsubsection{D02: What is the preferred specification and why is it preferred?}\label{d02-what-is-the-preferred-specification-and-why-is-it-preferred}

\begin{itemize}
\tightlist
\item
  Category: \texttt{D)\ Results,\ magnitudes,\ heterogeneity,\ robustness}
\item
  Confidence: \texttt{low\ (0.18139544853577588)}
\item
  Retrieval method: \texttt{local}
\item
  Evidence type: \texttt{Regression\ specification\ with\ fixed\ effects\ and\ weather\ controls\ used\ to\ identify\ the\ effect\ of\ calorie\ posting.}
\item
  Data source: Transaction data and cardholder data for Starbucks (NYC calorie posting study)
\item
  Answer:
\end{itemize}

The preferred specification is the regression in (1): y\_sct = x\_sct β + γ\_POST Tc t + ε\_sct, where y\_sct is calories per transaction, POST Tc t is a dummy for NYC stores after April 1, 2008, and x\_sct includes week fixed effects, day-of-week dummies, holiday dummies, temperature and temperature squared, and precipitation and precipitation squared. This specification controls for seasonality and other influences (via fixed effects and weather controls) to isolate the effect of calorie posting, and is estimated separately for transaction data and cardholder data.

\begin{itemize}
\tightlist
\item
  Quote snippet:
\end{itemize}

\begin{quote}
To examine the effects of calorie posting while controlling for seasonality and other influences, we estimate regressions of the following form: (1) y\_sct = x\_sct β + γ\_POST Tc t + ε\_sct, where y\_sct is a measure of calories per transaction...
\end{quote}

\begin{itemize}
\tightlist
\item
  Citation anchors:

  \begin{itemize}
  \tightlist
  \item
    page=11, words=0-349, section=Regression specification, note=Describes the preferred specification and variables in the regression.
  \item
    page=11, words=275-522, section=Regression specification, note=Continues with weather controls and sample usage.
  \end{itemize}
\end{itemize}

\subsubsection{D03: How economically meaningful is the effect (percent change, elasticity, dollars)?}\label{d03-how-economically-meaningful-is-the-effect-percent-change-elasticity-dollars}

\begin{itemize}
\tightlist
\item
  Category: \texttt{D)\ Results,\ magnitudes,\ heterogeneity,\ robustness}
\item
  Confidence: \texttt{medium\ (0.45859206251221274)}
\item
  Retrieval method: \texttt{local}
\item
  Evidence type: \texttt{Empirical\ findings\ from\ the\ Starbucks\ calorie-posting\ study\ (transaction\ and\ cardholder\ data),\ including\ robustness\ checks\ and\ heterogeneity\ analyses.}
\item
  Data source: Starbucks transaction data (NYC) and anonymous cardholder data with fixed effects; control cities for seasonal variation.
\item
  Answer:
\end{itemize}

The effects are economically meaningful but modest. Food calories per transaction fell about 14\% on average after calorie posting (with roughly 75\% of the reduction due to fewer food items purchased and about 25\% from substitution to lower-calorie items). Total calories per transaction declined around 5--6\% (about 5.8\% in transaction data and 5.0\% in cardholder data), translating to roughly 14.4 fewer calories per transaction. Beverage calories were largely unchanged. The impact was larger for high-calorie purchasers (e.g., a 26\% reduction for those who tended to make high-calorie purchases). There was some evidence of a price-per-item increase, and profits were not significantly affected. In a back-of-the-envelope macro calculation, the authors suggest that even a 6\% reduction in calories at all chain restaurants could have meaningful population-level implications.

\begin{itemize}
\tightlist
\item
  Quote snippet:
\end{itemize}

\begin{quote}
calorie posting caused food calories per transaction to fall by 14 percent, on average
\end{quote}

\begin{itemize}
\tightlist
\item
  Citation anchors:

  \begin{itemize}
  \tightlist
  \item
    page=24, words=275-564, section=IV. Discussion, note=Food calories per transaction fall by 14\% on average; three-quarters of the reduction from not purchasing a food item (extensive margin), one-quarter from substitution (intensive margin).
  \item
    page=12, words=0-349, section=Results (Table 3) and cardholder data, note=Transaction data show −14\% reduction in food calories per transaction; beverage calories −0.3\%; total calories −5.8\% (14.4 calories). Cardholder data show beverage calories not significantly different; food calories −11.2\%; total −5.0\%.
  \item
    page=22, words=0-349, section=Table 6, note=High-calorie purchasers exhibit larger reductions (e.g., 26\% reduction for high-calorie purchases).
  \item
    page=30, words=0-349, section=Is the Effect Big Enough to Matter?, note=Crude macro estimate: a 6\% reduction in calories across all chain restaurants could have meaningful population-level effects.
  \end{itemize}
\end{itemize}

\subsubsection{D04: What are the key robustness checks and do results survive them?}\label{d04-what-are-the-key-robustness-checks-and-do-results-survive-them}

\begin{itemize}
\tightlist
\item
  Category: \texttt{D)\ Results,\ magnitudes,\ heterogeneity,\ robustness}
\item
  Confidence: \texttt{low\ (0.2836801199350919)}
\item
  Retrieval method: \texttt{local}
\item
  Evidence type: \texttt{Robustness\ checks\ in\ regression\ analyses:\ date\ fixed\ effects,\ alternative\ dependent\ variable,\ serial-correlation\ handling,\ and\ cross-dataset\ validation}
\item
  Data source: American Economic Journal: Economic Policy (Bollinger et al., calorie posting in chain restaurants), 2011
\item
  Table/Figure: Table 3
\item
  Answer:
\end{itemize}

Key robustness checks include (1) adding date fixed effects as a robustness check (replacing week/day controls); results barely change (beverage calories: −0.004, food calories: −0.152, total calories: −0.063); (2) re-running with absolute calories as the dependent variable; results are almost identical to the main specification; (3) addressing potential serial correlation by aggregating pre/post posting data and testing differences; results are broadly consistent with the main findings; and (4) cross-dataset validation using cardholder data vs transaction data to isolate intensive-margin effects; results show the pattern persists, with beverage calories not significantly different from zero in cardholder data, but food calories and total calories still show reductions, indicating robustness of the calorie-posting effect though magnitudes vary by dataset.

\begin{itemize}
\tightlist
\item
  Quote snippet:
\end{itemize}

\begin{quote}
As a robustness check, we include date fixed effects in the above specification (and therefore drop the day-of-week and week fixed effects). The estimates based on the transaction data are barely changed.
\end{quote}

\begin{itemize}
\tightlist
\item
  Citation anchors:

  \begin{itemize}
  \tightlist
  \item
    page=12, words=0-40, section=robustness\_date\_fixed\_effects, note=As a robustness check, we include date fixed effects; results barely change.
  \item
    page=12, words=41-90, section=robustness\_absolute\_calories, note=We repeated the analysis using absolute calories as the dependent variable; results are almost identical.
  \item
    page=12, words=91-140, section=robustness\_serial\_correlation, note=To address serial correlation, we aggregate pre/post calorie posting data and test the difference; findings are broadly consistent.
  \item
    page=12, words=141-190, section=robustness\_cross\_dataset, note=Cardholder data vs transaction data to isolate intensive-margin effects; beverage calories not significantly different from zero, but food calories and total calories show reductions.
  \end{itemize}
\end{itemize}

\subsubsection{D05: What placebo tests are run and what do they show?}\label{d05-what-placebo-tests-are-run-and-what-do-they-show}

\begin{itemize}
\tightlist
\item
  Category: \texttt{D)\ Results,\ magnitudes,\ heterogeneity,\ robustness}
\item
  Confidence: \texttt{low\ (0.21220639848696363)}
\item
  Retrieval method: \texttt{local}
\item
  Evidence type: \texttt{placebo\ tests}
\item
  Data source: cardholder data and transaction data
\item
  Answer:
\end{itemize}

Two placebo tests are reported: (1) a Seattle transaction-data placebo showing no significant effect of calorie posting on calories per transaction, and (2) a parallel-trends placebo check showing no pre-trend differences between NYC and the control cities (Boston/Philadelphia) before posting.

\begin{itemize}
\tightlist
\item
  Quote snippet:
\end{itemize}

\begin{quote}
As noted in Section III, regression analysis of the transaction data for Seattle (and controls) shows no significant impact of calorie posting on either food or beverage calories per transaction.
\end{quote}

\begin{itemize}
\tightlist
\item
  Citation anchors:

  \begin{itemize}
  \tightlist
  \item
    page=25, words=275-599, section=Seattle placebo test (Seattle transaction data), note=Seattle regression shows no significant impact of calorie posting.
  \item
    page=14, words=0-349, section=Parallel-trends placebo test, note=No evidence of pre-trend differences between NYC and Boston/Philadelphia.
  \end{itemize}
\end{itemize}

\subsubsection{D06: What falsification outcomes are tested (unaffected outcomes)?}\label{d06-what-falsification-outcomes-are-tested-unaffected-outcomes}

\begin{itemize}
\tightlist
\item
  Category: \texttt{D)\ Results,\ magnitudes,\ heterogeneity,\ robustness}
\item
  Confidence: \texttt{low\ (0.25705277603599697)}
\item
  Retrieval method: \texttt{local}
\item
  Evidence type: \texttt{falsification\_outcome}
\item
  Data source: Milk order data; cardholder and transaction data
\item
  Answer:
\end{itemize}

Milk usage by type and the frequency of cardholder purchases were tested as falsification/unaffected outcomes.

\begin{itemize}
\tightlist
\item
  Quote snippet:
\end{itemize}

\begin{quote}
In all cases, there was no statistically significant impact of calorie posting.
\end{quote}

\begin{itemize}
\tightlist
\item
  Citation anchors:

  \begin{itemize}
  \tightlist
  \item
    page=page 20, words=0-349, section=B. Substitution Effects, note=Milk usage data tested; no significant impact found.
  \end{itemize}
\end{itemize}

\subsubsection{D07: What heterogeneity results are reported (by income, size, baseline exposure, region)?}\label{d07-what-heterogeneity-results-are-reported-by-income-size-baseline-exposure-region}

\begin{itemize}
\tightlist
\item
  Category: \texttt{D)\ Results,\ magnitudes,\ heterogeneity,\ robustness}
\item
  Confidence: \texttt{medium\ (0.37475415756821695)}
\item
  Retrieval method: \texttt{local}
\item
  Evidence type: \texttt{summary\ of\ heterogeneity\ findings\ from\ Tables\ 2\ and\ 5\ and\ related\ text}
\item
  Data source: American Economic Journal: Economic Policy, February 2011
\item
  Answer:
\end{itemize}

Heterogeneity results reported include: (i) by income/education: the decrease in calories per transaction is larger in zip codes with higher income and in areas with more residents with college degrees; (ii) by size: beverage-size responses show heterogeneity---some customers switch to smaller sizes with lower calories per ounce, while others switch to larger sizes with higher calories per ounce; (iii) by baseline exposure: high-calorie and high-frequency customers show different magnitudes of response (e.g., high-calorie customers see a larger drop, while high-frequency customers show a slightly smaller drop than low-frequency customers); (iv) by region: similar patterns are observed in control cities (Boston and Philadelphia), suggesting the effects are not driven by regional differences in NYC alone.

\begin{itemize}
\tightlist
\item
  Quote snippet:
\end{itemize}

\begin{quote}
"the decrease in calories per transaction was larger in zips with higher income and in zips with more education (i.e., more people with college degrees)."
\end{quote}

\begin{itemize}
\tightlist
\item
  Citation anchors:

  \begin{itemize}
  \tightlist
  \item
    page=20, words=0-349, section=Heterogeneity in the Effect of Mandatory Calorie Posting, note=Income and education heterogeneity (Table 5)
  \item
    page=11, words=0-349, section=Table 2, note=Size-based heterogeneity in beverage choices (Smaller/Same/Larger size)
  \item
    page=21, words=0-349, section=Heterogeneity in the Effect of Mandatory Calorie Posting, note=Baseline exposure heterogeneity: high-calorie vs other groups
  \item
    page=12, words=0-349, section=Control cities / Regional variation, note=Region: patterns similar in control cities (Boston/Philadelphia)
  \end{itemize}
\end{itemize}

\subsubsection{D08: What mechanism tests are performed and what do they imply?}\label{d08-what-mechanism-tests-are-performed-and-what-do-they-imply}

\begin{itemize}
\tightlist
\item
  Category: \texttt{D)\ Results,\ magnitudes,\ heterogeneity,\ robustness}
\item
  Confidence: \texttt{low\ (0.1427236921374667)}
\item
  Retrieval method: \texttt{local}
\item
  Evidence type: \texttt{Empirical\ mechanism\ tests\ using\ commuter\ vs\ non-commuter\ comparisons,\ knowledge\ surveys,\ margin\ decomposition\ (extensive\ vs\ intensive),\ heterogeneity,\ and\ control-city\ robustness.}
\item
  Data source: Starbucks transaction data and cardholder data; NYC posting; control cities (Boston, Philadelphia); Seattle data; commuter vs non-commuter comparisons.
\item
  Table/Figure: Figure 3
\item
  Answer:
\end{itemize}

Mechanism tests performed include: (1) a salience vs learning test by comparing NYC posting effects to nonposting markets using commuters (regular purchasers who shop outside NYC) to see if effects persist outside posting areas; they find effects in nonposting stores, implying learning rather than salience; (2) in-store surveys conducted before and after posting to gauge knowledge of calories and how posting affected that knowledge; (3) margin decomposition to separate extensive vs intensive effects, examining substitutions to lower-calorie items vs changes in purchase frequency/quantity; (4) heterogeneity analyses across quantiles of calories per transaction and across commuting status to assess who is most affected; (5) use of control cities (Boston, Philadelphia, Seattle controls) to distinguish time/seasonality from posting effects. Collectively, these tests imply a learning mechanism (information exposure changes choices beyond posting venues) with substitutions toward lower-calorie items and reductions in quantity-item purchases, rather than a pure salience effect or frequency change; higher-impact effects are seen among high-calorie purchasers and across markets, with commuter data suggesting cross-market learning.

\begin{itemize}
\tightlist
\item
  Quote snippet:
\end{itemize}

\begin{quote}
exposure to calorie information affects their choices even at nonposting (i.e., non-NYC) stores, which is consistent with a learning effect but inconsistent with the salience effect.
\end{quote}

\begin{itemize}
\tightlist
\item
  Citation anchors:

  \begin{itemize}
  \tightlist
  \item
    page=3, words=275-564, section=Mechanism tests, note=Salience vs learning test: commuters\textquotesingle{} exposure effects in nonposting stores imply learning rather than salience.
  \item
    page=7, words=275-566, section=Survey evidence, note=In-store surveys to measure knowledge of calories and impact of posting on knowledge.
  \item
    page=16, words=0-349, section=B. Substitution Effects, note=Margin decomposition testing extensive vs intensive effects (substitution vs fewer items).
  \item
    page=29, words=275-442, section=Commuter analysis, note=Commuters show calorie reductions in NYC and outside NYC; tests indicate cross-market learning patterns.
  \end{itemize}
\end{itemize}

\subsubsection{D09: How sensitive are results to alternative samples/bandwidths/controls?}\label{d09-how-sensitive-are-results-to-alternative-samplesbandwidthscontrols}

\begin{itemize}
\tightlist
\item
  Category: \texttt{D)\ Results,\ magnitudes,\ heterogeneity,\ robustness}
\item
  Confidence: \texttt{medium\ (0.3592342266694231)}
\item
  Retrieval method: \texttt{local}
\item
  Evidence type: \texttt{robustness\ checks\ across\ samples,\ bandwidths,\ and\ controls}
\item
  Data source: Starbucks transaction data and cardholder data (NYC, Boston, Philadelphia)
\item
  Table/Figure: Figure 2; Table 3; Table 9
\item
  Answer:
\end{itemize}

The results are robust to alternative samples, bandwidths, and controls. Using both transaction-level data and cardholder data yields consistent declines in calories per transaction after calorie posting. Robustness checks include comparing NYC against control cities, addressing potential serial correlation by aggregating pre/post data, and controlling for seasonality, holidays, and weather. Beverage calories show no significant change, while food calories per transaction fall by about 11.2\% (total calories down about 5\%), with similar patterns observed in transaction and cardholder analyses.

\begin{itemize}
\tightlist
\item
  Quote snippet:
\end{itemize}

\begin{quote}
To address any concern over serial correlation, we aggregate all transaction data before calorie posting, and all transaction data after calorie posting, then test the difference between average calories per transaction before versus after.
\end{quote}

\begin{itemize}
\tightlist
\item
  Citation anchors:

  \begin{itemize}
  \tightlist
  \item
    page=11, words=1-20, section=Methodology robustness, note=Estimates reported from both transaction data and cardholder data.
  \item
    page=12, words=0-60, section=Serial correlation robustness, note=Aggregate pre/post posting data to test differences; results remain similar.
  \item
    page=12, words=550-630, section=Cardholder data isolate intensive margin, note=Cardholder data isolates intensive-margin effects; beverage calories not significant; food calories −11.2\%.
  \item
    page=14, words=0-80, section=Figure 2 robustness, note=No pre-trend differences; drop around April 1, 2008; persists in both datasets.
  \item
    page=29, words=0-100, section=Commuter analysis robustness, note=Table 9 shows commuters vs non-commuters; effects observed in NYC and outside NYC.
  \end{itemize}
\end{itemize}

\subsubsection{D10: What are the main takeaways in the conclusion (bullet summary)?}\label{d10-what-are-the-main-takeaways-in-the-conclusion-bullet-summary}

\begin{itemize}
\item
  Category: \texttt{D)\ Results,\ magnitudes,\ heterogeneity,\ robustness}
\item
  Confidence: \texttt{low\ (0.2697093921016416)}
\item
  Retrieval method: \texttt{local}
\item
  Evidence type: \texttt{Conclusion\ bullets\ describing\ main\ findings\ and\ policy\ implications\ from\ the\ study.}
\item
  Data source: Starbucks transaction data (city comparisons) and cardholder data; surveys.
\item
  Answer:
\item
  Mandatory calorie posting reduces average calories per transaction by about 6\% at Starbucks, with the effect long-lasting.
\item
  The reduction is mainly due to changes in food purchases; beverage calories show little change.
\item
  The effect is larger among high-calorie consumers.
\item
  Both learning and salience contribute to the observed behavior: exposure to calorie information increases sensitivity (salience) and prior NYC exposure can lead to reductions even at non-posting stores (learning).
\item
  The overall impact on Starbucks profits is negligible on average; in some proximity to Dunkin Donuts, revenue may even rise.
\item
  Direct impacts on obesity are likely small, but longer-run effects could be more meaningful if restaurants add more low-calorie options and public education benefits accrue; posting costs are low, which supports potential policy benefits.
\item
  Quote snippet:
\end{itemize}

\begin{quote}
We find that mandatory calorie posting causes average calories per transaction to fall by 6 percent at Starbucks.
\end{quote}

\begin{itemize}
\tightlist
\item
  Citation anchors:

  \begin{itemize}
  \tightlist
  \item
    page=34, words=0-349, section=Conclusion, note=Main conclusion statements about 6\% decrease, long-lasting effects, food vs beverage, learning and salience.
  \item
    page=28, words=0-349, section=Conclusion/Discussion, note=Survey evidence suggests calorie posting reduces calories per transaction due to salience rather than learning.
  \item
    page=30, words=0-349, section=Commuters, note=Learning and salience both play a role; effects observed in non-posting stores after NYC exposure.
  \item
    page=31, words=0-349, section=Is the Effect Big Enough to Matter?, note=Back-of-the-envelope obesity impact discussion; small potential effect overall.
  \item
    page=24, words=275-564, section=IV. Discussion, note=Profit impact negligible on average; higher profit margins via higher-priced items; implications for policy.
  \end{itemize}
\end{itemize}

\subsubsection{E01: What are the most important prior papers cited and why are they central here?}\label{e01-what-are-the-most-important-prior-papers-cited-and-why-are-they-central-here}

\begin{itemize}
\tightlist
\item
  Category: \texttt{E)\ Citations\ and\ related\ literature}
\item
  Confidence: \texttt{low\ (0.27340975858791017)}
\item
  Retrieval method: \texttt{local}
\item
  Evidence type: \texttt{literature\ review}
\item
  Answer:
\end{itemize}

The most important prior papers cited are: 1) Jacoby, Chestnut, and Silberman (1977) and the broader nutrition-information literature (Ippolito \& Mathios 1990, 1995) showing that information about nutrition can influence consumer choices; 2) Bassett et al. (2008) and Downs, Loewenstein, and Wisdom (2009) and Wisdom, Downs, and Loewenstein (2010) providing empirical and strategic perspectives on calorie information at the point of purchase and how information design can promote healthier choices; 3) Currie et al. (2010) and related work (Anderson \& Matsa 2011) connecting fast-food exposure and labeling to obesity outcomes, which frames what calorie-posting studies aim to influence; 4) DellaVigna (2009) and Chetty, Looney, and Kroft (2009) offering a theoretical framework on learning versus salience to interpret posting effects; 5) Variyam \& Cawley (2006) and Jin \& Leslie (2003, 2009) addressing signaling and the signaling value of labels in restaurant and grocery contexts. Collectively, these works establish that information provision at the point of purchase can affect choices, that the magnitude and interpretation depend on learning versus salience, and that obesity outcomes are an important but complex downstream question.

\begin{itemize}
\tightlist
\item
  Quote snippet:
\end{itemize}

\begin{quote}
Two subsequent papers compare purchase data before and after calorie posting in NYC.
\end{quote}

\begin{itemize}
\tightlist
\item
  Citation anchors:

  \begin{itemize}
  \tightlist
  \item
    page=25, words=275-599, section=Narrative, note=Mentions: \textquotesingle Two subsequent papers compare purchase data before and after calorie posting in NYC\textquotesingle{} (Downs et al. 2009).
  \item
    page=37, words=0-349, section=References, note=Contains Jacoby, Chestnut, and Silberman (1977) and Ippolito \& Mathios (1990, 1995).
  \item
    page=37, words=275-624, section=References, note=Cites Currie et al. (2010); DellaVigna (2009); Wisdom et al. (2010); Downs et al. (2009).
  \item
    page=36, words=275-422, section=References, note=Includes Anderson \& Matsa (2011); Currie et al. (2010); Chetty et al. (2009) relevant to salience/learning framing.
  \item
    page=38, words=0-349, section=References, note=Cites Roberto et al. (2009); Variyam \& Cawley (2006); Jin \& Leslie (2003, 2009).
  \end{itemize}
\end{itemize}

\subsubsection{E02: Which papers does this work most directly build on or extend?}\label{e02-which-papers-does-this-work-most-directly-build-on-or-extend}

\begin{itemize}
\tightlist
\item
  Category: \texttt{E)\ Citations\ and\ related\ literature}
\item
  Confidence: \texttt{low\ (0.23771165855886442)}
\item
  Retrieval method: \texttt{local}
\item
  Evidence type: \texttt{textual\ citation\ from\ related\ work\ section}
\item
  Data source: NYC receipt data from fast-food patrons used in Downs, Loewenstein, and Wisdom (2009) and Elbel et al. (2009)
\item
  Answer:
\end{itemize}

The work most directly builds on or extends the NYC calorie-posting receipt studies by Downs, Loewenstein, and Wisdom (2009) and Elbel et al. (2009).

\begin{itemize}
\tightlist
\item
  Quote snippet:
\end{itemize}

\begin{quote}
Two subsequent papers compare purchase data before and after calorie posting in NYC.
\end{quote}

\begin{itemize}
\tightlist
\item
  Citation anchors:

  \begin{itemize}
  \tightlist
  \item
    page=7, words=275-566, section=B. Related Research, note=Mentions two NYC calorie-posting receipt studies that are directly relevant prior work
  \item
    page=8, words=0-349, section=Discussion of NYC calorie-posting and prior studies, note=Context describing prior NYC studies (Downs et al. 2009; Elbel et al. 2009) as the immediate predecessors
  \end{itemize}
\end{itemize}

\subsubsection{E03: Which papers are used as benchmarks or comparisons in the results?}\label{e03-which-papers-are-used-as-benchmarks-or-comparisons-in-the-results}

\begin{itemize}
\tightlist
\item
  Category: \texttt{E)\ Citations\ and\ related\ literature}
\item
  Confidence: \texttt{low\ (0.26010566557050246)}
\item
  Retrieval method: \texttt{local}
\item
  Evidence type: \texttt{literature\ references\ /\ benchmark\ studies\ cited\ in\ the\ results}
\item
  Answer:
\end{itemize}

The cited benchmarks/comparisons are the papers by Anderson \& Matsa (2011) and Currie et al. (2010) as related work, and the NYC pre-post calorie-posting studies by Downs, Loewenstein, \& Wisdom (2009) and Elbel et al. (2009).

\begin{itemize}
\tightlist
\item
  Quote snippet:
\end{itemize}

\begin{quote}
Two recent papers by Michael L. Anderson and David A. Matsa (2011) and Janet Currie et al. (2010). Neither paper finds that fast food restaurants have a significant effect on obesity in general.
\end{quote}

\begin{itemize}
\tightlist
\item
  Citation anchors:

  \begin{itemize}
  \tightlist
  \item
    page=8, words=Anderson-Matsa, section=Literature review / Background, note=Mentions benchmark papers Anderson \& Matsa (2011) and Currie et al. (2010).
  \item
    page=8, words=Currie-2010)., section=Literature review, note=Cites Currie et al. (2010) on obesity effects.
  \item
    page=8, words=Downs-posting., section=NYC pre-post studies, note=Downs, Loewenstein, Wisdom (2009) pre-post calorie posting study used as comparison.
  \item
    page=8, words=Elbel-NYC., section=NYC pre-post studies, note=Elbel et al. (2009) pre-post calorie posting study used as comparison.
  \end{itemize}
\end{itemize}

\subsubsection{E04: What data sources or datasets are cited and how are they used?}\label{e04-what-data-sources-or-datasets-are-cited-and-how-are-they-used}

\begin{itemize}
\tightlist
\item
  Category: \texttt{E)\ Citations\ and\ related\ literature}
\item
  Confidence: \texttt{low\ (0.289029838272936)}
\item
  Retrieval method: \texttt{local}
\item
  Evidence type: \texttt{data\ sources}
\item
  Data source: Starbucks transaction data; Starbucks cardholder data; milk order data; in-store customer surveys; Seattle/San Francisco survey data
\item
  Table/Figure: Table 1
\item
  Answer:
\end{itemize}

Cited datasets include: (1) Starbucks transaction data covering all 222 NYC stores and all 94 Boston/Philadelphia stores, with three months pre- and eleven months post-calorie posting; (2) Starbucks cardholder data (2.7 million anonymous individuals) used to analyze behavior changes and to separate intensive vs extensive margins; (3) Starbucks milk order data used to assess daily milk quantities replenished by location; (4) in-store customer surveys conducted in Seattle and San Francisco to test consumer knowledge of calories; and (5) related survey data and prior studies referenced for context and methodological comparison.

\begin{itemize}
\tightlist
\item
  Quote snippet:
\end{itemize}

\begin{quote}
"Our transaction data cover all 222 Starbucks locations in NYC, and all 94 Starbucks locations in Boston and Philadelphia."
\end{quote}

\begin{itemize}
\tightlist
\item
  Citation anchors:

  \begin{itemize}
  \tightlist
  \item
    page=5, words=0-349, section=Data Summary, note=Transaction data cover all Starbucks locations in NYC and Boston/Philadelphia, with pre- and post-period for calorie posting; includes item-level purchases and calories.
  \item
    page=5, words=275-591, section=Data Summary, note=Cardholder data: 2.7 million anonymous individuals; used to isolate intensive vs extensive margins and compare with transaction data.
  \item
    page=6, words=0-349, section=Table 1, note=Table 1 presents summary statistics for transaction data and cardholder data prior to policy change.
  \item
    page=6, words=275-517, section=Table 1, note=Further notes on dataset comparability and confidentiality; supports description of data sources.
  \item
    page=14, words=0-349, section=Results comparison, note=Figure 2 compares results for transaction data and cardholder data, showing NYC vs controls and pre/post patterns.
  \item
    page=25, words=275-599, section=Seattle/San Francisco surveys, note=In-store customer surveys conducted in Seattle and San Francisco to supplement data with consumer responses.
  \end{itemize}
\end{itemize}

\subsubsection{E05: What methodological or econometric references are cited (e.g., DiD, IV, RDD methods)?}\label{e05-what-methodological-or-econometric-references-are-cited-eg-did-iv-rdd-methods}

\begin{itemize}
\tightlist
\item
  Category: \texttt{E)\ Citations\ and\ related\ literature}
\item
  Confidence: \texttt{medium\ (0.3857005142588578)}
\item
  Retrieval method: \texttt{local}
\item
  Evidence type: \texttt{DiD/panel\ fixed-effects\ regression\ with\ within-city\ variation;\ references\ to\ randomized\ natural\ field\ experiments\ in\ the\ literature}
\item
  Data source: Cited methodological references and the study\textquotesingle s own transaction/cardholder data; bibliography includes works on field experiments and nutrition labeling
\item
  Answer:
\end{itemize}

The study uses a difference-in-differences--style identification with fixed effects (city-week, store, week, etc.) and within-city/time variation to estimate the effect of calorie posting. It also cites methodological literature on field experiments, notably randomized natural field experiments.

\begin{itemize}
\tightlist
\item
  Quote snippet:
\end{itemize}

\begin{quote}
In both cases (transaction data and cardholder data), identification of the effect of calorie posting stems from within-city variation over time.
\end{quote}

\begin{itemize}
\tightlist
\item
  Citation anchors:

  \begin{itemize}
  \tightlist
  \item
    page=11, words=275-522, section=II.A Calories Per Transaction, note=Presents the regression form and notes identification from within-city variation over time
  \item
    page=12, words=0-349, section=II.A Calories Per Transaction, note=Discusses store fixed effects and individual fixed effects for identification
  \item
    page=37, words=275-624, section=References, note=Cites Cai, Chen, and Fang (2009) Observational Learning: Evidence from a Randomized Natural Field Experiment
  \end{itemize}
\end{itemize}

\subsubsection{E06: Are there any seminal or classic references the paper positions itself against?}\label{e06-are-there-any-seminal-or-classic-references-the-paper-positions-itself-against}

\begin{itemize}
\tightlist
\item
  Category: \texttt{E)\ Citations\ and\ related\ literature}
\item
  Confidence: \texttt{low\ (0.2466639630086526)}
\item
  Retrieval method: \texttt{local}
\item
  Evidence type: \texttt{literature\ positioning\ against\ seminal\ works}
\item
  Answer:
\end{itemize}

Yes. It positions itself against classic nutrition-information literature, notably Jacoby, Chestnut, and Silberman (1977) and Ippolito \& Mathios (1990, 1995), treating them as seminal references whose claim that more nutrition information changes behavior is an old idea that the current study questions.

\begin{itemize}
\tightlist
\item
  Quote snippet:
\end{itemize}

\begin{quote}
The notion that increasing the provision of nutrition information may stimulate people to adopt healthier eating habits is an old idea
\end{quote}

\begin{itemize}
\tightlist
\item
  Citation anchors:

  \begin{itemize}
  \tightlist
  \item
    page=7, words=275-566, section=B. Related Research, note=Mentions classic nutrition-information literature; cites Jacoby et al. (1977) and Ippolito \& Mathios (1990, 1995).
  \item
    page=8, words=0-349, section=B. Related Research, note=Notes earlier studies finding demand sensitive to nutrition information (Ippolito \& Mathios; Kiesel \& Villas-Boas; Variyam \& Cawley).
  \end{itemize}
\end{itemize}

\subsubsection{E07: Are there citations to code, data repositories, or appendices that are essential to the claims?}\label{e07-are-there-citations-to-code-data-repositories-or-appendices-that-are-essential-to-the-claims}

\begin{itemize}
\tightlist
\item
  Category: \texttt{E)\ Citations\ and\ related\ literature}
\item
  Confidence: \texttt{low\ (0.29701273332227734)}
\item
  Retrieval method: \texttt{local}
\item
  Evidence type: \texttt{Appendix\ reference\ (survey\ questionnaire)}
\item
  Data source: Appendix (survey questionnaire)
\item
  Answer:
\end{itemize}

Yes. The text cites an Appendix that contains the in-store survey questionnaire (two-page) used to measure calorie knowledge.

\begin{itemize}
\tightlist
\item
  Quote snippet:
\end{itemize}

\begin{quote}
The questionnaire is shown in the Appendix.
\end{quote}

\begin{itemize}
\tightlist
\item
  Citation anchors:

  \begin{itemize}
  \tightlist
  \item
    page=7, words=275-566, section=Methods / Surveys, note=The questionnaire is shown in the Appendix.
  \end{itemize}
\end{itemize}

\subsubsection{E08: What gaps in the literature do the authors say these citations leave open?}\label{e08-what-gaps-in-the-literature-do-the-authors-say-these-citations-leave-open}

\begin{itemize}
\tightlist
\item
  Category: \texttt{E)\ Citations\ and\ related\ literature}
\item
  Confidence: \texttt{low\ (0.2924731048267649)}
\item
  Retrieval method: \texttt{local}
\item
  Evidence type: \texttt{Limitations\ and\ gaps\ in\ existing\ literature\ as\ acknowledged\ by\ the\ authors\ in\ the\ study}
\item
  Data source: Starbucks transaction data
\item
  Answer:
\end{itemize}

Gaps in the literature highlighted by the authors include (1) no direct measurement of calorie-posting effects on obesity due to lack of BMI data, (2) limited generalizability because the study uses data from only one chain (Starbucks), and (3) uncertainty about whether changes in Starbucks purchases are offset by other eating decisions at home.

\begin{itemize}
\tightlist
\item
  Quote snippet:
\end{itemize}

\begin{quote}
First, we do not directly measure the effect of calorie posting on obesity itself. A second limitation is that we have data for only one chain (Starbucks).
\end{quote}

\begin{itemize}
\tightlist
\item
  Citation anchors:

  \begin{itemize}
  \tightlist
  \item
    page=3, words=275-564, section=Limitations, note= authors note gaps: no obesity effect measurement; single-chain data; generalizability and potential home offset unobserved
  \item
    page=8, words=0-349, section=Literature gaps / measurement, note= prior literature often relies on survey responses; limited use of actual sales data, signaling measurement gaps
  \end{itemize}
\end{itemize}

\subsubsection{F01: Are replication files or code provided? If so, where?}\label{f01-are-replication-files-or-code-provided-if-so-where}

\begin{itemize}
\tightlist
\item
  Category: \texttt{F)\ Replication\ and\ transparency}
\item
  Confidence: \texttt{low\ (0.16797747380054723)}
\item
  Retrieval method: \texttt{local}
\item
  Evidence type: \texttt{not\_present}
\item
  Answer:
\end{itemize}

There is no indication of replication files or code being provided in the context; the excerpts reference an Appendix with a Stanford University survey but do not mention replication materials.

\begin{itemize}
\tightlist
\item
  Quote snippet:
\end{itemize}

\begin{quote}
Appendix A 1 STANFORD UNIVERSITY RESEARCH SURVEY How many times per week do you typically come to Starbucks?
\end{quote}

\begin{itemize}
\tightlist
\item
  Citation anchors:

  \begin{itemize}
  \tightlist
  \item
    page=35, words=0-161, section=Appendix A, note=Includes the Stanford University research survey; no replication files or code mentioned.
  \end{itemize}
\end{itemize}

\subsubsection{F02: Is there a pre-analysis plan or registered trial? Provide details if mentioned.}\label{f02-is-there-a-pre-analysis-plan-or-registered-trial-provide-details-if-mentioned}

\begin{itemize}
\tightlist
\item
  Category: \texttt{F)\ Replication\ and\ transparency}
\item
  Confidence: \texttt{low\ (0.2585968748290788)}
\item
  Retrieval method: \texttt{local}
\item
  Evidence type: \texttt{No\ pre-analysis\ plan\ or\ trial\ registration\ mentioned}
\item
  Data source: Transaction data for Starbucks locations ( NYC, Boston, Philadelphia ) and cardholder data; plus in-store survey data.
\item
  Answer:
\end{itemize}

No. The provided text does not mention a pre-analysis plan or registered trial. It describes data sources and a regression-based empirical strategy but does not reference preregistration or trial registration.

\begin{itemize}
\tightlist
\item
  Quote snippet:
\end{itemize}

\begin{quote}
The basic impact of mandatory calorie posting on calorie consumption is evident without any regression analysis (no controls of any kind).
\end{quote}

\begin{itemize}
\tightlist
\item
  Citation anchors:

  \begin{itemize}
  \tightlist
  \item
    page=11, words=350-420, section=II. Effect of Mandatory Calorie Posting on Calorie Consumption -- Empirical Strategy, note=Describes regression specification with POS\_Tc\_t dummy and fixed effects.
  \item
    page=14, words=0-349, section=Methods / Estimation, note=Figure 2 and regression-based analysis are presented.
  \item
    page=9, words=0-349, section=Data \& Approach, note=Outlines data sources (transaction and cardholder data) but no preregistration mention.
  \end{itemize}
\end{itemize}

\subsubsection{F03: Are data access constraints disclosed (restricted access, proprietary data, NDAs)?}\label{f03-are-data-access-constraints-disclosed-restricted-access-proprietary-data-ndas}

\begin{itemize}
\tightlist
\item
  Category: \texttt{F)\ Replication\ and\ transparency}
\item
  Confidence: \texttt{low\ (0.275037519980488)}
\item
  Retrieval method: \texttt{local}
\item
  Evidence type: \texttt{Confidentiality\ and\ restricted\ data}
\item
  Data source: Starbucks transaction data and cardholder data (confidential; restricted access)
\item
  Answer:
\end{itemize}

Yes. The context discloses confidentiality constraints and restricted access for the data (confidential, cannot reveal further details).

\begin{itemize}
\tightlist
\item
  Quote snippet:
\end{itemize}

\begin{quote}
Due to confidentiality requirements, we are unable to reveal any more details about these differences.
\end{quote}

\begin{itemize}
\tightlist
\item
  Citation anchors:

  \begin{itemize}
  \tightlist
  \item
    page=5, words=0-349, section=A. Data Summary, note=To preserve confidentiality of competitively sensitive information, for both datasets
  \item
    page=6, words=275-517, section=A. Data Summary, note=Due to confidentiality requirements, we are unable to reveal any more details about these differences.
  \end{itemize}
\end{itemize}

\subsubsection{F04: Are key steps in data cleaning and construction documented?}\label{f04-are-key-steps-in-data-cleaning-and-construction-documented}

\begin{itemize}
\tightlist
\item
  Category: \texttt{F)\ Replication\ and\ transparency}
\item
  Confidence: \texttt{low\ (0.21844437303641645)}
\item
  Retrieval method: \texttt{local}
\item
  Evidence type: \texttt{procedural\_description}
\item
  Data source: Starbucks transaction data for NYC, Boston, Philadelphia and Starbucks cardholder data
\item
  Table/Figure: Table 1---Summary Statistics for Transaction Data and Cardholder Data (Prior to policy change)
\item
  Answer:
\end{itemize}

Yes. The paper documents key data cleaning and construction steps, including excluding stores not open for the entire data period to form a balanced panel, removing transactions with more than four units of any item, and normalizing values (NYC) to preserve confidentiality. It also notes the use of two datasets (transaction and cardholder) and provides a table summarizing data.

\begin{itemize}
\tightlist
\item
  Quote snippet:
\end{itemize}

\begin{quote}
we exclude transactions at stores that were not open during the entire data period (i.e., we analyze the balanced panel), and we exclude transactions that included more than four units of any one item
\end{quote}

\begin{itemize}
\tightlist
\item
  Citation anchors:

  \begin{itemize}
  \tightlist
  \item
    page=6, words=0-40, section=Data cleaning, note=Excludes stores not open for the full period and excludes large-quantity transactions.
  \item
    page=6, words=275-320, section=Data cleaning, note=Normalizes NYC values to preserve confidentiality.
  \end{itemize}
\end{itemize}

\subsubsection{F05: Are robustness and sensitivity analyses fully reported or partially omitted?}\label{f05-are-robustness-and-sensitivity-analyses-fully-reported-or-partially-omitted}

\begin{itemize}
\tightlist
\item
  Category: \texttt{F)\ Replication\ and\ transparency}
\item
  Confidence: \texttt{low\ (0.28419734314792333)}
\item
  Retrieval method: \texttt{local}
\item
  Evidence type: \texttt{explicit\ robustness\ check\ described\ (exclusion\ of\ weather\ controls).}
\item
  Data source: Starbucks transaction data and cardholder data (NYC, Boston, Philadelphia).
\item
  Answer:
\end{itemize}

Partially omitted.

\begin{itemize}
\tightlist
\item
  Quote snippet:
\end{itemize}

\begin{quote}
Our findings are unchanged if we exclude the weather controls.
\end{quote}

\begin{itemize}
\tightlist
\item
  Citation anchors:

  \begin{itemize}
  \tightlist
  \item
    page=11, words=275-522, section=Robustness checks / Empirical strategy, note=Finding unchanged when weather controls are excluded; weather variables have an insignificant impact.
  \end{itemize}
\end{itemize}

\subsubsection{G01: What populations or settings are most likely to generalize from this study?}\label{g01-what-populations-or-settings-are-most-likely-to-generalize-from-this-study}

\begin{itemize}
\tightlist
\item
  Category: \texttt{G)\ External\ validity\ and\ generalization}
\item
  Confidence: \texttt{low\ (0.28509778243723166)}
\item
  Retrieval method: \texttt{local}
\item
  Evidence type: \texttt{Textual\ evidence\ from\ the\ study\ highlighting\ commuter\ effects\ and\ limitations\ on\ generalizability\ to\ other\ chains.}
\item
  Data source: Starbucks transaction and cardholder datasets (NYC, Seattle, Boston, Philadelphia).
\item
  Answer:
\end{itemize}

Generalizable populations/settings are urban adults who are regular customers of large chain restaurants exposed to calorie labeling, especially commuters who split their visits between NYC and other locations; results may extend to other chain restaurants with posted calories, though generalization beyond Starbucks should be treated with caution due to the study focusing on a single chain.

\begin{itemize}
\tightlist
\item
  Quote snippet:
\end{itemize}

\begin{quote}
However, these cardholders may not be representative of Starbucks customers more generally, as we expect these individuals are above average in their loyalty to Starbucks.
\end{quote}

\begin{itemize}
\tightlist
\item
  Citation anchors:

  \begin{itemize}
  \tightlist
  \item
    page=3, words=275-564, section=Main findings -- Commuters, note=Demonstrates that exposure to calorie information affects commuters\textquotesingle{} choices across NYC and non-NYC stores, implying learning effects.
  \item
    page=5, words=0-349, section=Limitations, note=Two important limitations are that there is data for only one chain (Starbucks), limiting generalization to other chains.
  \item
    page=29, words=0-349, section=Commuters, note=Defines commuters and presents results for commuters' post-posting behavior in NYC vs non-NYC stores.
  \end{itemize}
\end{itemize}

\subsubsection{G02: What populations or settings are least likely to generalize?}\label{g02-what-populations-or-settings-are-least-likely-to-generalize}

\begin{itemize}
\tightlist
\item
  Category: \texttt{G)\ External\ validity\ and\ generalization}
\item
  Confidence: \texttt{low\ (0.21427365593611017)}
\item
  Retrieval method: \texttt{local}
\item
  Evidence type: \texttt{explicit\ limitations\ on\ generalizability\ discussed\ in\ the\ study}
\item
  Data source: Bollinger et al., Calorie Posting in Chain Restaurants, American Economic Journal: Economic Policy (2011).
\item
  Answer:
\end{itemize}

The populations/settings least likely to generalize are (1) other restaurant chains beyond Starbucks and (2) the broader Starbucks customer base, since the cardholder sample is highly loyal and not representative.

\begin{itemize}
\tightlist
\item
  Quote snippet:
\end{itemize}

\begin{quote}
There are two important limitations to this research.
\end{quote}

\begin{itemize}
\tightlist
\item
  Citation anchors:

  \begin{itemize}
  \tightlist
  \item
    page=3, words=There-research., section=Limitations, note=Only one chain (Starbucks) studied; cannot generalize to other chains.
  \item
    page=5, words=However,-Starbucks., section=Data limitations, note=Cardholder sample may not reflect the broader Starbucks customer base; loyalty bias.
  \item
    page=14, words=Seattle-2009., section=Cross-city robustness, note=Policy context differs across cities (Seattle vs NYC); pastry exemption affects generalizability.
  \end{itemize}
\end{itemize}

\subsubsection{G03: Do the authors discuss boundary conditions or scope limits?}\label{g03-do-the-authors-discuss-boundary-conditions-or-scope-limits}

\begin{itemize}
\tightlist
\item
  Category: \texttt{G)\ External\ validity\ and\ generalization}
\item
  Confidence: \texttt{low\ (0.20192755605586474)}
\item
  Retrieval method: \texttt{local}
\item
  Evidence type: \texttt{limitations\ and\ boundary\ conditions\ discussion}
\item
  Data source: Starbucks cardholder data (transaction data)
\item
  Answer:
\end{itemize}

Yes. They discuss boundary conditions and scope limits, notably that obesity effects are not measured and the data come from only one chain (Starbucks), limiting generalizability to other chains.

\begin{itemize}
\tightlist
\item
  Quote snippet:
\end{itemize}

\begin{quote}
There are two important limitations to this research. First, we do not directly measure the effect of calorie posting on obesity itself. A second limitation is that we have data for only one chain (Starbucks).
\end{quote}

\begin{itemize}
\tightlist
\item
  Citation anchors:

  \begin{itemize}
  \tightlist
  \item
    page=3, words=275-564, section=Limitations and scope, note=Two important limitations: obesity not measured and data limited to Starbucks; generalizability to other chains is uncertain.
  \end{itemize}
\end{itemize}

\subsubsection{G04: How might the results change in different time periods or markets?}\label{g04-how-might-the-results-change-in-different-time-periods-or-markets}

\begin{itemize}
\tightlist
\item
  Category: \texttt{G)\ External\ validity\ and\ generalization}
\item
  Confidence: \texttt{medium\ (0.35759790340300607)}
\item
  Retrieval method: \texttt{local}
\item
  Evidence type: \texttt{Empirical\ study\ findings\ with\ cross-city\ analysis\ and\ discussion\ of\ limitations}
\item
  Data source: Starbucks transaction and cardholder data (NYC vs. control cities) around calorie posting starting April 1, 2008
\item
  Table/Figure: Figure 2
\item
  Assumption flag: \texttt{True}
\item
  Assumption notes: Assumes that learning effects observed in NYC/Starbucks could transfer to other markets or chains; acknowledges that generalization is unproven and dependent on similar exposure, consumer base, and competitive context.
\item
  Answer:
\end{itemize}

Results may differ in different time periods or markets due to seasonality, holidays, and changes in consumer composition, as well as differences in market structure and exposure. The study shows an immediate drop in calories per transaction after posting (April 1, 2008) that persisted through February 2009 in NYC data, with a temporary decline in effectiveness around Christmas/New Year in transaction data but not in cardholder data, suggesting time-varying dynamics. Generalizing beyond Starbucks to other chains or markets is uncertain because the analysis uses data from a single chain and there may be varying competitive environments, menus, and consumer bases. In commuter vs. non-commuter settings, effects vary by location and context, indicating potential learning transfer but not universal applicability.

\begin{itemize}
\tightlist
\item
  Quote snippet:
\end{itemize}

\begin{quote}
The drop in calories per transaction occurred right around April 1, 2008, and persisted through February 2009.
\end{quote}

\begin{itemize}
\tightlist
\item
  Citation anchors:

  \begin{itemize}
  \tightlist
  \item
    page=3, words=275-564, section=Limitations and salience vs learning, note=Two important limitations: data from one chain; uncertain generalizability to other markets.
  \item
    page=12, words=275-624, section=Temporal pattern of calorie posting effects, note=Drop in calories per transaction occurred around April 1, 2008 and persisted through February 2009.
  \item
    page=14, words=0-349, section=Time variation and holidays, note=Transaction data show reduced effectiveness around Christmas/New Year; cardholder data do not show this pattern.
  \item
    page=29, words=0-349, section=Commuters vs non-commuters in different markets, note=Table 9 analyzes commuters' vs non-commuters' responses; demonstrates outside NYC effects and differential impacts.
  \end{itemize}
\end{itemize}

\subsubsection{H01: Are key variables measured directly or via proxies?}\label{h01-are-key-variables-measured-directly-or-via-proxies}

\begin{itemize}
\tightlist
\item
  Category: \texttt{H)\ Measurement\ validity}
\item
  Confidence: \texttt{low\ (0.3080115570445991)}
\item
  Retrieval method: \texttt{local}
\item
  Evidence type: \texttt{Direct\ measurement\ from\ transaction/cardholder\ data;\ derived\ nutrition\ metrics;\ survey-based\ self-reports}
\item
  Data source: Transaction data and cardholder data
\item
  Table/Figure: Table 1
\item
  Answer:
\end{itemize}

Key variables are measured directly from the available data sources (transaction-level data and cardholder data). Calories per transaction and related calorie metrics are computed from observed purchases, and milk-order data provide direct measurements of milk choices; survey data contribute self-reported information on calories and perceptions.

\begin{itemize}
\tightlist
\item
  Quote snippet:
\end{itemize}

\begin{quote}
Based on the trans- action data, we compute that, prior to calorie posting, in NYC: average drink calories per transaction were 143; average total calories per transaction were 247.
\end{quote}

\begin{itemize}
\tightlist
\item
  Citation anchors:

  \begin{itemize}
  \tightlist
  \item
    page=5, words=275-591, section=Data overview, note=Introduces cardholder vs. transaction data and measurement of calorie-related outcomes
  \item
    page=6, words=0-349, section=Table 1, note=Table 1 presents summary statistics derived from transaction/cardholder data prior to posting
  \item
    page=11, words=275-522, section=Empirical strategy, note=Describes regressions for calories per transaction using data sources
  \item
    page=12, words=275-624, section=Empirical strategy, note=Notes that cardholder data isolate intensive-margin changes; data sources for calorie measures
  \end{itemize}
\end{itemize}

\subsubsection{H02: What measurement error risks are acknowledged or likely?}\label{h02-what-measurement-error-risks-are-acknowledged-or-likely}

\begin{itemize}
\tightlist
\item
  Category: \texttt{H)\ Measurement\ validity}
\item
  Confidence: \texttt{low\ (0.3323464745797265)}
\item
  Retrieval method: \texttt{local}
\item
  Evidence type: \texttt{Limitations\ and\ measurement\ error\ risks\ in\ data\ collection\ and\ analysis.}
\item
  Data source: Transaction data, cardholder data, and in-store surveys from Seattle, San Francisco, NYC (Starbucks).
\item
  Answer:
\end{itemize}

Key measurement-error risks acknowledged or likely include: (1) reliance on data from a single chain (Starbucks), limiting generalizability and potentially biasing calorie estimates if Starbucks differs from other chains; (2) incomplete measurement of calories per transaction due to missing data on milk additions, customization, and items like soy milk, which can alter calorie counts; (3) inability to capture changes in the extensive margin (non-transactions) when measuring calories per transaction; (4) survey-based sampling biases in the in-store surveys (e.g., under-representation of regular coffee consumers, potential wave-specific sampling differences, and timing biases); (5) potential nonrepresentativeness of cardholders, which could bias learning vs. salience interpretations; (6) potential confounds such as time trends or January effects affecting survey responses, making attribution to posting uncertain.

\begin{itemize}
\tightlist
\item
  Quote snippet:
\end{itemize}

\begin{quote}
There are two important limitations to this research.
\end{quote}

\begin{itemize}
\tightlist
\item
  Citation anchors:

  \begin{itemize}
  \tightlist
  \item
    page=3, words=275-290, section=Limitations, note=Two important limitations are discussed, including lack of obesity measurement and data from only one chain.
  \item
    page=25, words=275-320, section=Survey methods, note=In-store surveys with potential sampling bias; under-representation of regular coffee consumers; possible wave-specific sampling bias.
  \item
    page=12, words=275-320, section=Data/Methods, note=Cardholder data isolates intensive-margin effects but does not capture extensive-margin changes; measurement limitations acknowledged.
  \item
    page=27, words=0-60, section=Survey results, note=Survey results may reflect sampling biases; second wave potentially sampled more informed consumers.
  \item
    page=6, words=275-300, section=Data limitations, note=Confidentiality limits detail about differences in data; incomplete disclosure relevant to measurement interpretation.
  \end{itemize}
\end{itemize}

\subsubsection{H03: Are there validation checks for key measures?}\label{h03-are-there-validation-checks-for-key-measures}

\begin{itemize}
\tightlist
\item
  Category: \texttt{H)\ Measurement\ validity}
\item
  Confidence: \texttt{low\ (0.24878415846921556)}
\item
  Retrieval method: \texttt{local}
\item
  Evidence type: \texttt{Robustness\ checks\ /\ validation\ checks}
\item
  Data source: transaction data and cardholder data
\item
  Answer:
\end{itemize}

Yes. The paper reports several validation/robustness checks for key measures, including weather controls robustness, holiday/day-of-week fixed effects, and cross-dataset validation across transaction and cardholder data (and Seattle post-law data) to ensure results are not driven by particular assumptions or data sources.

\begin{itemize}
\tightlist
\item
  Quote snippet:
\end{itemize}

\begin{quote}
we include day-of-week dummies and holiday dummies that differ for NYC and the controls, finding no difference in the estimates.
\end{quote}

\begin{itemize}
\tightlist
\item
  Citation anchors:

  \begin{itemize}
  \tightlist
  \item
    page=11, words=275-522, section=Weather controls robustness, note=Weather variables insignificant; results unchanged if weather controls are excluded.
  \item
    page=14, words=275-590, section=Fixed effects and holidays robustness, note=Day-of-week and holiday dummies included; estimates unchanged when holidays are dropped.
  \item
    page=14, words=275-590, section=Cross-dataset validation, note=Similar calorie-posting effects observed in both transaction and cardholder data.
  \item
    page=14, words=275-590, section=Seattle robustness check, note=Seattle post-law data used as an additional robustness check.
  \end{itemize}
\end{itemize}

\subsubsection{H04: Do the authors discuss construct validity for core outcomes?}\label{h04-do-the-authors-discuss-construct-validity-for-core-outcomes}

\begin{itemize}
\tightlist
\item
  Category: \texttt{H)\ Measurement\ validity}
\item
  Confidence: \texttt{low\ (0.20302754159074737)}
\item
  Retrieval method: \texttt{local}
\item
  Evidence type: \texttt{explicit\ discussion\ of\ limitations\ but\ no\ construct\ validity\ discussion}
\item
  Answer:
\end{itemize}

No. The authors do not discuss construct validity for core outcomes in the provided context.

\begin{itemize}
\tightlist
\item
  Quote snippet:
\end{itemize}

\begin{quote}
However, there are two important limitations to this research. First, we do not directly measure the effect of calorie posting on obesity itself.
\end{quote}

\begin{itemize}
\tightlist
\item
  Citation anchors:

  \begin{itemize}
  \tightlist
  \item
    page=3, words=275-564, section=Limitations, note=Limitations section notes that obesity effects are not directly measured; no mention of construct validity discussion.
  \end{itemize}
\end{itemize}

\subsubsection{I01: What policy counterfactuals are considered or implied?}\label{i01-what-policy-counterfactuals-are-considered-or-implied}

\begin{itemize}
\tightlist
\item
  Category: \texttt{I)\ Policy\ counterfactuals\ and\ welfare}
\item
  Confidence: \texttt{low\ (0.3421445345053721)}
\item
  Retrieval method: \texttt{local}
\item
  Evidence type: \texttt{textual\ evidence}
\item
  Data source: American Economic Journal: Economic Policy, 2011; Starbucks calorie posting study
\item
  Table/Figure: Table 3; Figure 3
\item
  Answer:
\end{itemize}

Policy counterfactuals considered or implied: (1) The within-city counterfactual using NYC stores with posting vs control cities (Boston and Philadelphia) where posting did not occur, to identify what would have happened absent posting. (2) A nationwide policy counterfactual: if calorie posting were implemented across all chain restaurants, with a crude assumption of a 6\% reduction in calories, what impact could that have on obesity. (3) Generalizability counterfactual: whether the observed Starbucks effects would hold for other chains, since the study uses data from a single chain and cannot confirm effects elsewhere. (4) Timing/menu-change counterfactual: whether posting would drive changes in menu offerings or simply shift consumer choices, noting pre-existing trends toward low-calorie options and potential lags in menu changes.

\begin{itemize}
\tightlist
\item
  Quote snippet:
\end{itemize}

\begin{quote}
To control for other factors affecting transactions, we also observe every transaction at Starbucks company stores in Boston and Philadelphia, where there was no calorie posting.
\end{quote}

\begin{itemize}
\tightlist
\item
  Citation anchors:

  \begin{itemize}
  \tightlist
  \item
    page=2, words=0-40, section=Dataset description / Counterfactual control cities, note=Control cities Boston and Philadelphia used as counterfactual for NYC.
  \item
    page=3, words=275-564, section=Limitations, note=Only one chain; cannot know if effects generalize to other chains.
  \item
    page=32, words=0-80, section=B. Is the Effect Big Enough to Matter?, note=Policy counterfactual: 6 percent reduction at all chain restaurants.
  \end{itemize}
\end{itemize}

\subsubsection{I02: What are the main welfare tradeoffs or distributional impacts discussed?}\label{i02-what-are-the-main-welfare-tradeoffs-or-distributional-impacts-discussed}

\begin{itemize}
\tightlist
\item
  Category: \texttt{I)\ Policy\ counterfactuals\ and\ welfare}
\item
  Confidence: \texttt{low\ (0.31018180399339024)}
\item
  Retrieval method: \texttt{local}
\item
  Evidence type: \texttt{empirical\ findings\ from\ a\ field\ study\ of\ mandatory\ calorie\ posting\ in\ NYC\ Starbucks}
\item
  Data source: American Economic Journal: Economic Policy (Feb 2011) -- Calorie Posting in NYC Starbucks
\item
  Assumption flag: \texttt{False}
\item
  Answer:
\end{itemize}

Mandatory calorie posting reduced calories per transaction, mainly by discouraging food purchases (extensive margin) with little impact on visit frequency or beverage calories. About 14\% fewer food calories per transaction overall, and roughly three-quarters of the total calorie reduction comes from not buying food items. Reductions are larger among higher-income, more-educated areas and among women; higher-calorie purchasers saw larger absolute reductions, and reductions tend to be larger at the upper end of the calorie distribution. There are some costs to policy implementation (menu-board changes, potential legal costs), and profits were largely unaffected overall (prices per item rose slightly, possibly shifting demand toward higher-margin items).

\begin{itemize}
\tightlist
\item
  Quote snippet:
\end{itemize}

\begin{quote}
three quarters of the total calorie reduction can be attributed to people opting not to buy food items (i.e., the extensive margin of food demand).
\end{quote}

\begin{itemize}
\tightlist
\item
  Citation anchors:

  \begin{itemize}
  \tightlist
  \item
    page=18, words=275-541, section=Table 4 bottom-panel findings on calories per item vs items per transaction, note=Notes that nearly three quarters of the total calorie reduction is due to not buying food items (extensive margin).
  \item
    page=24, words=275-564, section=Discussion of overall calorie effects and profits, note=States that calorie posting reduced food calories per transaction by 14\% and discusses profit implications and price per item changes.
  \item
    page=20, words=0-349, section=Frequency of cardholder purchases, note=Reports no statistically significant change in transaction frequency; purchases frequency largely unaffected.
  \item
    page=20, words=0-349, section=Heterogeneity by zip socioeconomic status, note=Finds larger calorie reductions in higher-income and more-educated zips.
  \item
    page=20, words=275-551, section=Demographic heterogeneity in cardholder data, note=Female cardholders more responsive to calorie posting than males.
  \item
    page=22, words=0-349, section=Distributional pattern of calorie reductions, note=Calorie reductions tend to be larger in the top half of the distribution; percent reductions are roughly similar from 75th to 99th percentile.
  \item
    page=22, words=0-349, section=Cost considerations, note=Notes direct costs of changing menu boards and potential higher legal costs.
  \end{itemize}
\end{itemize}

\subsubsection{I03: Are cost-benefit or incidence analyses provided?}\label{i03-are-cost-benefit-or-incidence-analyses-provided}

\begin{itemize}
\tightlist
\item
  Category: \texttt{I)\ Policy\ counterfactuals\ and\ welfare}
\item
  Confidence: \texttt{low\ (0.31306873732199253)}
\item
  Retrieval method: \texttt{local}
\item
  Evidence type: \texttt{Descriptive\ discussion\ of\ costs\ and\ absence\ of\ formal\ cost-benefit/incidence\ analysis\ in\ the\ article.}
\item
  Data source: Starbucks transaction data in NYC and control cities; in-store surveys; Seattle control data
\item
  Assumption flag: \texttt{False}
\item
  Answer:
\end{itemize}

No formal cost-benefit or incidence analyses are provided. The paper discusses costs (e.g., a cited \$2,000 per store compliance cost for Wendy\textquotesingle s and other potential indirect costs) but does not perform a formal cost-benefit or incidence analysis; it mainly reviews revenue/profit implications and discusses potential public benefits.

\begin{itemize}
\tightlist
\item
  Quote snippet:
\end{itemize}

\begin{quote}
One news report indicated the cost of compliance for the Wendy's chain was about \$2,000 per store.
\end{quote}

\begin{itemize}
\tightlist
\item
  Citation anchors:

  \begin{itemize}
  \tightlist
  \item
    page=4, words=275-576, section=Costs/Implementation discussion, note=Mentions costs of posting and cites Wendy\textquotesingle s compliance cost (\textasciitilde\$2,000 per store) and indirect costs.
  \item
    page=34, words=275-351, section=IV. Discussion, note=Concludes with cost-benefit considerations and potential benefits; references costs and potential outweighing benefits.
  \end{itemize}
\end{itemize}

\subsubsection{I04: What policy recommendations are stated or implied?}\label{i04-what-policy-recommendations-are-stated-or-implied}

\begin{itemize}
\tightlist
\item
  Category: \texttt{I)\ Policy\ counterfactuals\ and\ welfare}
\item
  Confidence: \texttt{low\ (0.3366218290825945)}
\item
  Retrieval method: \texttt{local}
\item
  Evidence type: \texttt{Policy\ recommendation}
\item
  Data source: Starbucks NYC transaction data (with controls in Boston and Philadelphia), Starbucks cardholder data, and in-store surveys.
\item
  Assumption flag: \texttt{True}
\item
  Assumption notes: Assumes that the benefits of calorie posting (calorie reductions, potential obesity impact, and public education) outweigh costs; assumes low costs of posting; assumes effects generalize beyond NYC and could have long-run effects.
\item
  Answer:
\end{itemize}

Implement or maintain mandatory calorie posting on restaurant menus; benefits likely outweigh costs due to low posting costs, potential modest calorie reductions, public-education gains, and possible long-run menu innovation; expansion nationwide is plausible.

\begin{itemize}
\tightlist
\item
  Quote snippet:
\end{itemize}

\begin{quote}
We find that mandatory calorie posting causes average calories per transaction to fall by 6 percent at Starbucks... the costs of calorie posting are very low---so even these small benefits could outweigh the costs.
\end{quote}

\begin{itemize}
\tightlist
\item
  Citation anchors:

  \begin{itemize}
  \tightlist
  \item
    page=33, words=275-320, section=V. Conclusion, note=Policy recommendation that mandatory calorie posting is beneficial if benefits outweigh costs; low costs; potential for innovation and education; calls for expansion.
  \item
    page=34, words=0-60, section=Conclusion, note=Empirical finding: 6\% reduction in calories; costs are very low; long-run effects possible; policy implications.
  \end{itemize}
\end{itemize}

\subsubsection{J01: What missingness or attrition patterns are reported?}\label{j01-what-missingness-or-attrition-patterns-are-reported}

\begin{itemize}
\tightlist
\item
  Category: \texttt{J)\ Data\ quality\ and\ integrity}
\item
  Confidence: \texttt{low\ (0.3156909266007003)}
\item
  Retrieval method: \texttt{local}
\item
  Evidence type: \texttt{Data\ exclusions\ and\ sampling\ patterns\ (balanced\ panel,\ bulk-purchase\ exclusions,\ percentage\ exclusions,\ sample\ representativeness).}
\item
  Data source: Starbucks transaction data and cardholder (anonymous) purchase data.
\item
  Answer:
\end{itemize}

The study reports several missingness/attrition patterns via data exclusions and sample selection: (1) they exclude transactions from stores not open during the entire data period to maintain a balanced panel; (2) they exclude transactions that included more than four units of any one item (bulk purchases); (3) these exclusions amount to 2.2\% of all transactions; (4) the fraction of excluded transactions is unknown but believed to be well under 5\%; and (5) the cardholder sample may not be representative of Starbucks customers more generally, which could affect generalizability.

\begin{itemize}
\tightlist
\item
  Quote snippet:
\end{itemize}

\begin{quote}
We exclude transactions at stores that were not open during the entire data period (i.e., we analyze the balanced panel).
\end{quote}

\begin{itemize}
\tightlist
\item
  Citation anchors:

  \begin{itemize}
  \tightlist
  \item
    page=5, words=0-349, section=Data Summary, note=Exclude transactions at stores not open during the entire data period to maintain a balanced panel.
  \item
    page=5, words=0-349, section=Data Summary, note=Exclude transactions that included more than four units of any one item (bulk purchases).
  \item
    page=5, words=0-349, section=Notes, note=The excluded transactions represent 2.2 percent of all transactions.
  \item
    page=5, words=0-349, section=Notes, note=The fraction of excluded transactions is unknown, but we believe it to be well under 5 percent.
  \item
    page=5, words=275-591, section=Data Summary, note=Cardholders may not be representative of Starbucks customers more generally.
  \end{itemize}
\end{itemize}

\subsubsection{J02: How are outliers handled (winsorization, trimming, exclusions)?}\label{j02-how-are-outliers-handled-winsorization-trimming-exclusions}

\begin{itemize}
\tightlist
\item
  Category: \texttt{J)\ Data\ quality\ and\ integrity}
\item
  Confidence: \texttt{low\ (0.31128811353338065)}
\item
  Retrieval method: \texttt{local}
\item
  Evidence type: \texttt{Data\ filtering\ /\ exclusions\ for\ outliers}
\item
  Data source: Transaction data and Cardholder data (Starbucks dataset)
\item
  Answer:
\end{itemize}

Outliers are handled by excluding certain transactions and stores rather than applying statistical winsorization. Specifically, the authors exclude transactions that include more than four units of any one item (to remove bulk purchases) and exclude stores that were not open for the entire data period (balanced panel). These exclusions account for about 2.2\% of all transactions.

\begin{itemize}
\tightlist
\item
  Quote snippet:
\end{itemize}

\begin{quote}
We exclude transactions at stores that were not open during the entire data period (i.e., we analyze the balanced panel), and we exclude transactions that included more than four units of any one item because we consider these purchases to be driven by fundamentally different processes (bulk purchases for an office, say). The excluded transactions represent only 2.2 percent of all transactions.
\end{quote}

\begin{itemize}
\tightlist
\item
  Citation anchors:

  \begin{itemize}
  \tightlist
  \item
    page=5, words=275-591, section=Data filtering and exclusions, note=Exclusion criteria: balanced panel stores and purchases with more than four units; 2.2\% of transactions excluded.
  \end{itemize}
\end{itemize}

\subsubsection{J03: Are there data audits or validation steps described?}\label{j03-are-there-data-audits-or-validation-steps-described}

\begin{itemize}
\tightlist
\item
  Category: \texttt{J)\ Data\ quality\ and\ integrity}
\item
  Confidence: \texttt{low\ (0.24336161630175762)}
\item
  Retrieval method: \texttt{local}
\item
  Evidence type: \texttt{Data\ validation\ and\ robustness\ checks}
\item
  Data source: Transaction data; Cardholder data; Milk order data; Seattle data
\item
  Table/Figure: Figure 1 and Figure 2; robustness checks referenced in text
\item
  Answer:
\end{itemize}

Yes. The paper describes multiple data validation and robustness steps, including (1) creating a balanced panel and excluding incomplete or extreme transactions, (2) pre-trend checks showing no differential trends between NYC and controls, (3) robustness checks using day-of-week and holiday dummies, (4) analysis with two data sources (transaction data and cardholder data) for cross-validation, and (5) external checks such as Seattle\textquotesingle s post-law data and control comparisons.

\begin{itemize}
\tightlist
\item
  Quote snippet:
\end{itemize}

\begin{quote}
as a robustness check, we include day-of-week dummies and holiday dummies that differ for NYC and the controls, finding no difference in the estimates.
\end{quote}

\begin{itemize}
\tightlist
\item
  Citation anchors:

  \begin{itemize}
  \tightlist
  \item
    page=14, words=0-349, section=Robustness checks, note=Robustness checks with day-of-week/holiday dummies; pre-trend checks; within-individual variation.
  \item
    page=6, words=0-349, section=Data cleaning, note=2.2\% of transactions excluded; balanced panel.
  \item
    page=5, words=275-591, section=Data sources / Balance, note=Cardholder vs. transaction data; balanced panel.
  \item
    page=14, words=275-590, section=Placebo / external validation, note=Seattle placebo test with pastry exemption.
  \item
    page=20, words=275-551, section=Heterogeneity analyses, note=Gender, frequency, and other demographics effects.
  \item
    page=25, words=275-599, section=Survey / Additional data, note=In-store surveys; data collection methods.
  \end{itemize}
\end{itemize}

\subsubsection{J04: Is there evidence of reporting bias or selective sample inclusion?}\label{j04-is-there-evidence-of-reporting-bias-or-selective-sample-inclusion}

\begin{itemize}
\tightlist
\item
  Category: \texttt{J)\ Data\ quality\ and\ integrity}
\item
  Confidence: \texttt{low\ (0.3048953057287519)}
\item
  Retrieval method: \texttt{local}
\item
  Evidence type: \texttt{sampling\ bias\ /\ selective\ sample\ acknowledgment\ in\ methods}
\item
  Data source: Starbucks transaction data (NYC, Boston, Philadelphia) and anonymous cardholder data; Seattle survey data
\item
  Assumption flag: \texttt{False}
\item
  Answer:
\end{itemize}

Yes. The authors acknowledge selective sample inclusion and potential reporting bias in their data sources, noting that the cardholder sample may not be representative of Starbucks customers overall (likely more loyal) and that the survey sample may under-represent certain consumer types (e.g., those ordering regular coffee).

\begin{itemize}
\tightlist
\item
  Quote snippet:
\end{itemize}

\begin{quote}
However, these cardholders may not be representative of Starbucks customers more generally, as we expect these individuals are above average in their loyalty to Starbucks.
\end{quote}

\begin{itemize}
\tightlist
\item
  Citation anchors:

  \begin{itemize}
  \tightlist
  \item
    page=5, words=275-591, section=A. Data Summary, note=Cardholder data may not be representative of Starbucks customers more generally.
  \end{itemize}
\end{itemize}

\subsubsection{K01: What goodness-of-fit or diagnostic metrics are reported?}\label{k01-what-goodness-of-fit-or-diagnostic-metrics-are-reported}

\begin{itemize}
\tightlist
\item
  Category: \texttt{K)\ Model\ fit\ and\ diagnostics}
\item
  Confidence: \texttt{low\ (0.2471733819564274)}
\item
  Retrieval method: \texttt{local}
\item
  Evidence type: \texttt{Tables\ and\ figures\ notes\ (Table\ 4,\ Figure\ 2,\ Table\ 2)}
\item
  Table/Figure: Table 4 (R\^{}2 values) and Figure 2 (95\% confidence intervals); Table 2 (p-value 0.11).
\item
  Answer:
\end{itemize}

Reported goodness-of-fit and diagnostic metrics include R-squared values for the regression specifications (e.g., R\^{}2's ranging from 0.27 to 0.82; 0.26--0.37 for cardholder data), standard errors in parentheses for coefficient estimates, and 95\% confidence intervals (Figure 2 notes). A Pearson chi-square p-value (0.11) is also reported for a related test of equality of proportions.

\begin{itemize}
\tightlist
\item
  Quote snippet:
\end{itemize}

\begin{quote}
R\^{}2\textquotesingle s ranging from 0.27 to 0.82.
\end{quote}

\begin{itemize}
\tightlist
\item
  Citation anchors:

  \begin{itemize}
  \tightlist
  \item
    page=page 18, words=0-349, section=Table 4 notes, note=R\^{}2 values reported for regression specifications (R\^{}2's ranging from 0.27 to 0.82; cardholder data 0.26--0.37).
  \item
    page=page 15, words=0-254, section=Figure 2 notes, note=95 percent confidence intervals (dashed lines) shown in regression figures.
  \item
    page=page 11, words=0-349, section=Table 2 notes, note=Pearson's chi-square test with p-value 0.11.
  \end{itemize}
\end{itemize}

\subsubsection{K02: Are functional form choices tested (logs, levels, nonlinearities)?}\label{k02-are-functional-form-choices-tested-logs-levels-nonlinearities}

\begin{itemize}
\tightlist
\item
  Category: \texttt{K)\ Model\ fit\ and\ diagnostics}
\item
  Confidence: \texttt{low\ (0.2680523417435153)}
\item
  Retrieval method: \texttt{local}
\item
  Evidence type: \texttt{Regression\ specifications\ incorporating\ logs,\ levels,\ and\ nonlinear\ weather\ term}
\item
  Table/Figure: Table 9; Table 2; Figure 3
\item
  Answer:
\end{itemize}

Yes. The authors test both log and level functional forms: they regress with logged dependent variables (e.g., log daily sales; log(calories per item purchased); log(calories + 1)) and also use level outcomes (e.g., number of items per transaction). They also include nonlinear elements in controls (temperature and temperature squared) rather than only linear specifications, though they do not report testing other nonlinear functional forms for the main outcomes.

\begin{itemize}
\tightlist
\item
  Quote snippet:
\end{itemize}

\begin{quote}
"we regress log daily sales on an indicator for calorie posting"
\end{quote}

\begin{itemize}
\tightlist
\item
  Citation anchors:

  \begin{itemize}
  \tightlist
  \item
    page=page 16, words=0-349, section=Regression specifications (logs and levels), note=regress log daily sales
  \item
    page=page 16, words=275-558, section=Regression specifications (logs and levels), note=dependent variables include number of items per transaction and log(calories per item purchased)
  \item
    page=page 29, words=0-349, section=Table 9 description, note=The dependent variable is log(calories + 1)
  \item
    page=page 11, words=0-349, section=Model specification / Weather controls, note=temperature and temperature squared (nonlinear term) included
  \end{itemize}
\end{itemize}

\subsubsection{K03: Are residual checks or specification tests reported?}\label{k03-are-residual-checks-or-specification-tests-reported}

\begin{itemize}
\tightlist
\item
  Category: \texttt{K)\ Model\ fit\ and\ diagnostics}
\item
  Confidence: \texttt{low\ (0.23556498944397872)}
\item
  Retrieval method: \texttt{local}
\item
  Evidence type: \texttt{textual\_evidence}
\item
  Data source: Transaction data and Cardholder data (calorie posting study)
\item
  Answer:
\end{itemize}

No. The text describes regression specifications and robustness checks (alternative specifications, controls, etc.) and reports robust standard errors, but it does not present residual checks or formal specification tests.

\begin{itemize}
\tightlist
\item
  Quote snippet:
\end{itemize}

\begin{quote}
An alternative approach is to modify equation (1) to include separate week dummies for NYC and the control cities, and to exclude the POST variable.
\end{quote}

\begin{itemize}
\tightlist
\item
  Citation anchors:

  \begin{itemize}
  \tightlist
  \item
    page=page 11, words=0-349, section=Variations in specification (alternative specification approach), note=Describes an alternative specification with separate week dummies and exclusion of POST
  \end{itemize}
\end{itemize}

\subsubsection{K04: How sensitive are results to alternative specifications or estimators?}\label{k04-how-sensitive-are-results-to-alternative-specifications-or-estimators}

\begin{itemize}
\tightlist
\item
  Category: \texttt{K)\ Model\ fit\ and\ diagnostics}
\item
  Confidence: \texttt{medium\ (0.37660521918656825)}
\item
  Retrieval method: \texttt{local}
\item
  Evidence type: \texttt{Robustness\ checks\ and\ alternative\ estimators}
\item
  Data source: Transaction data and cardholder data
\item
  Answer:
\end{itemize}

Results are robust to alternative specifications and estimators. Robustness checks show that including date fixed effects, dropping some controls, and using a conservative serial-correlation approach do not materially change the estimated effects; results are unchanged when weather controls are removed; using absolute calories as the dependent variable yields almost identical results; and findings from transaction data are corroborated by cardholder-data analyses.

\begin{itemize}
\tightlist
\item
  Quote snippet:
\end{itemize}

\begin{quote}
To address any concern over serial correlation, we aggregate all transaction data before calorie posting, and all transaction data after calorie posting, then test the difference between average calories per transaction before versus after.
\end{quote}

\begin{itemize}
\tightlist
\item
  Citation anchors:

  \begin{itemize}
  \tightlist
  \item
    page=page 12, words=0-349, section=Robustness checks (date fixed effects; serial correlation; weather controls), note=Date fixed effects included; results barely changed; weather controls do not drive conclusions; serial-correlation robustness test.
  \item
    page=page 12, words=0-349, section=Robustness checks (alternative dependent variable), note=Repeating the analysis using absolute calories as the dependent variable yields almost identical results.
  \item
    page=page 12, words=275-624, section=Data-source robustness, note=Cardholder-data results corroborate transaction-data findings, isolating intensive-margin effects.
  \end{itemize}
\end{itemize}

\subsubsection{P01: ResponseTextConfig(format=ResponseFormatText(type=\textquotesingle text\textquotesingle), verbosity=\textquotesingle medium\textquotesingle)}\label{p01-responsetextconfigformatresponseformattexttypetext-verbositymedium}

\begin{itemize}
\item
  Category: \texttt{P)\ Previous\ questions}
\item
  Confidence: \texttt{n/a}
\item
  Retrieval method: \texttt{n/a}
\item
  Evidence type: \texttt{n/a}
\item
  Answer:
\item
  Objective and data

  \begin{itemize}
  \tightlist
  \item
    Analyzes the impact of NYC's mandatory calorie posting on chain restaurants using two datasets (store-day transaction data and cardholder data) in NYC vs control cities (Boston/Philadelphia). POST = 1 after April 1, 2008; results rely on NYC vs controls and fixed effects/controls. (page 13, 275-528)
  \end{itemize}
\item
  Overall impact on calories

  \begin{itemize}
  \tightlist
  \item
    Calorie posting reduces average calories per store-day by about 4.6 calories, with the reduction in calories per transaction smaller, consistent with an increase in total transactions after posting. (page 13, 275-528)
  \end{itemize}
\item
  Seattle placebo comparison (external validation)

  \begin{itemize}
  \tightlist
  \item
    In Seattle, beverage calories per transaction fell by 4.6 calories, while food calories per transaction rose by 0.8 calories, suggesting beverage-focused effects and little noise from the posting policy elsewhere. (page 15, 0-254)
  \end{itemize}
\item
  Item- and per-item effects

  \begin{itemize}
  \tightlist
  \item
    Table 4 (transaction data) shows: number of beverages per transaction increases modestly (0.005***); number of food items per transaction falls (−0.029***); calories per beverage and per food item also decline (log beverage calories per beverage: −0.008***; log food calories per food item: −0.039***). (page 18, 0-349)
  \item
    Interpretation: the reduction in calories is driven both by fewer items purchased (extensive margin with food items) and by lower calories per item (intensive margin). Food items contribute notably to the effect. (page 18, 0-349)
  \end{itemize}
\item
  Main channel: extensive margin dominates

  \begin{itemize}
  \tightlist
  \item
    About three quarters of the total calorie reduction is attributed to people opting not to buy food items (extensive margin of food demand). (page 18, 0-349)
  \end{itemize}
\item
  Distributional and heterogeneity patterns

  \begin{itemize}
  \tightlist
  \item
    Quantile evidence: calorie reductions are larger for higher quantiles; the percent change is roughly stable around 5--6 percent from 75th to 99th percentile, though absolute decreases are larger at higher quantiles. (page 22, 0-349)
  \item
    Heterogeneity: effects differ by characteristics (e.g., income, education, age, gender, customer type); e.g., higher-income and more-educated groups show additional differential effects (Posting × median income, Posting × percent with college degree, etc.). (page 21, 0-349)
  \end{itemize}
\item
  Summary takeaway

  \begin{itemize}
  \tightlist
  \item
    Mandatory calorie posting reduces calories per transaction primarily through reduced food purchases (extensive margin) and, to a lesser extent, lower calories per item; effects are larger for higher-calorie consumers and more pronounced at higher consumption quantiles. (page 18, 0-349; page 21-22, 0-349)
  \end{itemize}
\end{itemize}

Sources used:

\begin{itemize}
\tightlist
\item
  NYC calorie posting effects and overall magnitude: page 13, words 275-528
\item
  Seattle placebo example: page 15, words 0-254
\item
  Table 4 item-level and margin findings: page 18, words 0-349
\item
  Extensive-margin interpretation: page 18, words 0-349
\item
  Distributional/heterogeneity results: pages 21-22, words 0-349
\end{itemize}

\end{document}
